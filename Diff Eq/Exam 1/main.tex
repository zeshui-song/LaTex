\documentclass[12pt]{article}
\usepackage{amssymb}
\usepackage{geometry}
\usepackage{amsmath, amsfonts, bm, graphicx}
\usepackage{float}
\geometry{margin=1in}
\title{}
\date{}
\author{}

\begin{document}
\begin{center}
    \Large \textbf{Differential Equations Review Sheet} \\
    \normalsize Exam I
\end{center}
\section*{1.1 Definitions and Terminology}
\textbf{Linear Equations}\\
n $n$th-order ODE is said to be \textbf{linear} if it is linear in $y, y', . . . , y^{(n)}$. This means that an $n$th-order ODE is linear when it is represented as 
\[a_n(x)y^{(n)}+ a_{n-1}(x)y^{(n-1)}+\cdots+a_1(x)y'+a_0(x)y=g(x)\]
If not the ODE is said to be \textbf{nonlinear}.\\\\
\textbf{Solution of a Differential Equation}\\
A \textbf{solution} of a differential equation is a function (or set of functions) that has at least as many continuous derivatives as the order of the differential equation and satisfies the equation on a given interval.\\\\
That interval $I$ is called the \textbf{interval of validity}, \textbf{interval of definition}, or \textbf{interval of existence} for the solution.\\\\
\underline{\textbf{Thm 1.2.1 Existence of a Unique Solution}}\\
For a rectangular region in the $xy$-plane defined by $x \in [a,b], \, y \in [c,d]$ that contains the initial point $(x_0, y_0)$ in its interior. \textbf{If $f(x, y)$ and $\frac{\partial f}{\partial y}$ are continuous in the region}, then there exists a unique solution $y(x)$, defined on $I_0$ where $I_0$ is within $[a,b]$, for the initial value problem:
\begin{align*} \frac{dy}{dx} &=f(x,y)\\y(x_0)&=y_0\end{align*}
\section*{1.3 Differential Equations as Mathematical Models}
\textbf{Population Dynamics}\\
The rate of population growth is proportional to the current population:
\[\frac{dP}{dt} = kP(t)\] 
where the proportionality constant $k>0$ (for growth) or $k<0$ (for decay).
\newpage\noindent
\textbf{Spread of Disease}\\
 A contagious disease is spread throughout a community by people coming into contact with other people.\\\\
 Let $x(t)$ be the number of people who have contracted the disease and $y(t)$ be the number of people who have not yet been exposed. Assuming that the rate $dx/dt$ at which the disease spreads is proportional to the number of interactions between these two groups of people. Then if we assume that the number of interactions is jointly proportional to $x(t)$ and $y(t)$ then we can describe this as the model 
 \[\frac{dx}{dt} = kxy\]
\textbf{Mixtures}\\
Determine the rate of change of the amount of substance in the tank as two solutions of differing concentrations mix.\\\\
Suppose a large tank initially contains 20 g of salt dissolved in 200 liters of water. Starting at $t_0=0$, water that contains $1/4$ g of salt per liter is poured into the tank at the rate of 4 liters/min and the mixture is drained from the tank at the rate of 2 liters/min.\\\\  If $A(t)$ denotes the amount of salt (measured in grams) in the tank at time $t$, then the rate at which $A(t)$ changes is a net rate:
\begin{align}\frac{dA(t)}{dt}=&(\textrm{input rate of salt})- (\textrm{output rate of salt})\\=&R_{in}-R_{out}\end{align}
where 
$$R_{in} = 1/4\textrm{ g/liter}*4\textrm{ liter/min} = 1\textrm{g/min}$$
To find the output rate, we need to determine the concentration of salt in the tank at time $t$. The volume of solution in the tank at time $t$ is
\[V(t) = 200 + (4-2)t = 200 + 2t \textrm{ liters}\]
Thus the concentration of salt in the tank at time $t$ is
\[\frac{A(t)}{V(t)}=\frac{A(t)}{200 + 2t}\textrm{ g/liter}\]
and the output rate of salt is
\[R_{out} = \frac{A(t)}{200 + 2t} \cdot 2 \textrm{ liters/min} = \frac{A(t)}{100 + t} \textrm{ g/min}\]
Putting it all together, we have the initial value problem
\[\frac{dA}{dt} = 1 - \frac{A}{100 + t}, \quad A(0) = 20\]
\newpage\noindent
\section*{2.1 Solution Curves Without a Solution}
Without explicitly solving a differential equation, we can learn about its solutions by analyzing its \textbf{slope field} and \textbf{isoclines}.\\\\
\textbf{Slope Field}\\
Suppose a differential equation of the form
\[\frac{dy}{dx} = f(x,y)\]
The derivative $f(x,y)$ gives the slope of the solution curve $y(x)$ at the point $(x,y)$. A \textbf{slope field} or \textbf{direction field} is a graphical representation of the slopes of solution curves of a first-order differential equation at all the points in the $xy$-plane.\\\\
\textbf{Isoclines}\\
If the derivative $f(x,y)$ is constant, then it forms a straight line solution curve called an \textbf{isocline}. If the derivative is zero, then it is called a \textbf{nullcline}.\\\\
\underline{Note:} By the Existence and Uniqueness Theorem, through each point in the \(xy\)-plane there can be only \textbf{one} solution curve. Therefore, since isoclines and nullclines themselves are solutions, no other solution curve can intersect them without violating the uniqueness condition. As a result, solution curves are effectively \textit{bounded} by the isoclines and nullclines and are asymptotic to them.\\\\
\textbf{Autonomous Differential Equations}\\
A differential equation where the independent variable does not explicitly appear is called an \textbf{autonomous} differential equation. An autonomous differential equation has the form
\[\frac{dy}{dx} = f(y)\]
The nullclines of an autonomous differential equation are the horizontal lines where $f(y) = 0$. These lines become \textbf{equilibrium points} or \textbf{critical points}.
\newpage\noindent
We can analyze the stability by looking at the sign of the derivative to see if the solution curves move toward or away from the equilibrium points.
\begin{itemize}
    \item If 2 arrows point toward the equilibrium point, it is \textbf{stable} and is known as an \textbf{attractor} or \textbf{sink}.
    \item If 2 arrows point away from the equilibrium point, it is \textbf{unstable} and is known as a \textbf{repellor} or \textbf{source}.
    \item If 1 arrow points toward and 1 arrow points away from the equilibrium point, it is \textbf{semistable} and known as a \textbf{saddle}.
\end{itemize}
\begin{figure}[H]
    \centering
    \includegraphics[width=0.8\textwidth]{Phase.png}
\end{figure}
Next we will look at methods of solving first-order differential equations.
\newpage
\section*{2.2 Separable Equations}
A first-order differential equation is \textbf{separable} if it has the form:
\[\frac{dy}{dx} = g(x)h(y)\]
\fbox{%
\begin{minipage}{0.95\textwidth}
\textbf{Method:}
$$\frac{dy}{dx} = g(x)h(y)$$
Separate and integrate:
\[\frac{1}{h(y)}\,dy = g(x)\,dx\]
$$\boxed{\int \frac{1}{h(y)} dy = \int g(x)dx}$$
\end{minipage}%
}\\\\
\underline{Note:} For cases where there are absolute values or square roots, consider the initial conditions to determine if the solution is positive or negative.\\\\
Also take note to specify the interval of validity for the solution.
\newpage
\section*{2.3 Linear Equations}
For a first-order linear differential equation of the form:
\[\frac{dy}{dx} + p(t)y = g(t)\]
\fbox{%
\begin{minipage}{0.95\textwidth}
\textbf{Method:}\\
Find the integrating factor:
\[\boxed{\mu(t) = e^{\int p(t) dt}}\]
The solution is given by:
\[\boxed{y(t) = \frac{1}{\mu(t)}\int \mu(t)g(t) dt}\]
\end{minipage}%
}\\\\
\underline{Note:} We found this by assuming that there is a $\mu(t)$ such that
\begin{align*}
\frac{d(\mu y)}{dt} &= \mu\frac{dy}{dt}+\mu p(t)y\\
&= \mu g(t)
\end{align*}
$$\implies \mu y = \int \mu g(t) dt$$
and then solving for $y$.
\[y = \frac{1}{\mu}\int \mu g(t) \ dt\]
To determine the integrating factor \(\mu\), we use the product rule
\[\frac{d(\mu y)}{dt}=\frac{d\mu}{dt}y+\mu\frac{dy}{dt}, \quad \frac{d(\mu y)}{dt} = \mu\frac{dy}{dt}+\mu p(t)y\]
Matching coefficients gives the differential equation
\[\frac{d\mu}{dt}=\mu p(t)\]
which is separable. Solving yields the integrating factor
\[\mu(t)=e^{\int p(t)\,dt}\]
\underline{Note:} Transient terms in the solution are terms that approach zero as \(t \to \infty\). Steady-state terms are terms that remain as \(t \to \infty\).
\newpage
\section*{2.4 Exact Equations}
For first-order differential equations of the form:
\[M(x,y)dx + N(x,y)dy = 0\]
\fbox{%
\begin{minipage}{0.95\textwidth}
\textbf{Method:}\\
Check if the equation is exact:
\[\frac{\partial M}{\partial y} = \frac{\partial N}{\partial x}\]
\[M_y = N_x\]
If exact, there exists a function \(\Psi(x,y)\) such that
\[\frac{\partial \Psi}{\partial x} = M(x,y), \quad \frac{\partial \Psi}{\partial y} = N(x,y)\]
Thus, the implicit solution is
\[\Psi(x,y) = C\]
where $C$ is a constant of integration given from solving initial conditions.\\\\
To find \(\Psi(x,y)\):
\begin{itemize}
    \item Integrate \(M(x,y)\) with respect to \(x\):
    \[\Psi(x,y) = \int M(x,y) \, dx + h(y)\]
    where \(h(y)\) is a function of \(y\) only.
    \item Differentiate \(\Psi(x,y)\) with respect to \(y\):
    \[\frac{\partial \Psi}{\partial y} = \frac{\partial}{\partial y}\left(\int M(x,y) \, dx + h(y)\right)\]
    \item Solve for \(h(y)\) by comparing with \(N(x,y)\).
    \[\frac{\partial \Psi}{\partial y} = N(x,y)\]
\end{itemize}
\underline{Note:} The order of integrating for \(\Psi(x,y)\) can be switched.
\end{minipage}%
}
\newpage\noindent
\underline{Ex:}
\[2x+y^2+2xyy'=0\]
\[\implies \frac{dy}{dx} = \frac{-(2x+y^2)}{2xy}, \qquad \left(\frac{dy}{dx}=\frac{-M(x,y)}{N(x,y)}\right)\]
Check for exact:
\[M_y=2y,\quad N_x=2y\]
Solve for \(\Psi(x,y)\):
\[\Psi_x = M(x,y)\]
\begin{align*}
    \Psi &= \int M(x,y) \, dx \\
    &= \int (2x + y^2) \, dx\\ 
    &= x^2 + y^2x + h(y)
\end{align*}
Solve for \(h(y)\):
\[\Psi_y = N(x,y)\]
\begin{align*}
    \Psi_y &= \frac{\partial}{\partial y}(x^2 + y^2x + h(y)) \\
    &= 2xy + h'(y)
\end{align*}
Set equal to \(N(x,y)\):
\[2xy + h'(y) = 2xy\]
\[\implies h'(y)=0 \implies h(y)=C\]
Thus the solution is
\[x^2 + y^2x +C=0\]
If the differential equation is not exact, we can sometimes find an \textbf{integrating factor} \(\mu(x,y)\) such that multiplying the entire equation by \(\mu\) makes it exact.\\\\
\fbox{%
\begin{minipage}{0.95\textwidth}
\textbf{Method:}\\
Find an integrating factor \(\mu(x,y)\) such that
$$\mu(x,y)M(x, y) dx + \mu(x,y)N(x, y) dy = 0$$
To be exact,
\[(\mu M)_y = (\mu N)_x\]
To find $\mu$, it must either be a function of $x$ only or $y$ only.
\begin{itemize}
    \item If $\mu$ is ONLY in terms of $x$:
\[\frac{d\mu}{dx}=\mu\frac{M_y-N_x}{N}\]
    \item If $\mu$ is ONLY in terms of $y$:
\[\frac{d\mu}{dy}=\mu\frac{N_x-M_y}{M}\]
\end{itemize}
Then we solve for $\mu$ by separating and integrating.
\end{minipage}%
}\\\\
\underline{Ex:}
$$ y(x+y+1)dx+(x+2y)dy=0$$
\begin{eqnarray*}
M &=& y(x+y+1)=xy+y^2+y\\
N &=& (x+2y)
\end{eqnarray*}
Check for exact:
\[M_y = x+2y+1, \quad N_x = 1\]
Try for integrating factor:
\[\frac{M_y - N_x}{N} = \frac{(x+2y+1)-1}{x+2y} = \frac{x+2y}{x+2y} = 1\]
\[\implies \mu = e^{\int 1 \, dx} = e^x\]
Multiply the entire equation by $\mu$:
$$ e^xy(x+y+1)dx+e^x(x+2y)dy=0$$
Now it is exact:
\[M = e^xy(x+y+1)=e^x(xy+y^2+y), \quad N = e^x(x+2y)\]
\[M_y = xe^x+2ye^x+e^x,\quad N_x = xe^x+e^x+2ye^x\]
Finally solve for \(\Psi(x,y)\):
\[\Psi_y = N(x,y)\]
\begin{align*}
    \Psi &= \int N(x,y) \, dy \\
    &= \int e^x(x+2y) \, dy\\ 
    &= xe^xy+y^2e^x+ h(x)
\end{align*}
Solve for \(h(x)\):
\[\Psi_x = e^xy + xe^xy + y^2e^x+ h'(x)\]
Set equal to \(M(x,y)\):
\[e^x(xy+y^2+y) = e^xy + xe^xy + y^2e^x+ h'(x)\]
\[\implies h'(x)=0 \implies h(x)=C\]
Thus the solution is
\[xe^xy+y^2e^x+C=0\]
\section*{2.5 Solutions by Substitution}
Some first-order differential equations can be simplified by an appropriate substitution. Here are some common types:
\begin{enumerate}
    \item \textbf{Homogeneous Equations:} If a first-order differential equation can be expressed in the form
\end{enumerate}



\end{document}