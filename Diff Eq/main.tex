\documentclass[12pt]{article}
\usepackage{amssymb}
\usepackage{geometry}
\usepackage{amsmath, amsfonts, bm, graphicx}
\usepackage{float}
\geometry{margin=1in}
\title{}
\date{}
\author{}

\begin{document}
\begin{center}
    \Large \textbf{Differential Equations Review Sheet} \\
    \normalsize Exam II
\end{center}
\section*{4.1 Linear Equations}
General form of a nth order linear differential equation:
\[a_n(x)y^{(n)}+ a_{n-1}(x)y^{(n-1)}+\cdots+a_1(x)y'+a_0(x)y=g(x)\]
\underline{\textbf{Thm: 4.1.1 Existence of a Unique Solution:}}\\
For an interval $I: a<x<b$, if the functions $a_n(x), a_{n-1}(x), \ldots, a_0(x)$ and $g(x)$ are continuous on $I$ and $a_n(x)\neq 0$ for all $x$ in $I$, then there exists a unique solution $y=y(x)$ of the differential equation.\\\\
If $x=x_o$ is in $I$, then a solution that satisfies the initial conditions exists on the interval and is unique.\\
\underline{Note:} If $a_n(x)=0$ for some $x$ in the interval $I$ then there \textbf{may not} exist a unique solution.\\\\
\textbf{Boundary Value Problems}\\
If the constraints of a linear differential equation are at \textit{different points} instead of using derivatives at the \textit{same} point then it is known as a \textbf{boundary value problem (BVP)} and the constraints are known as \textbf{boundary conditions}.\\\\
\textbf{Homogeneous vs Nonhomogeneous nth order ODE}\\
Homogeneous: $a_n(x)y^{(n)}+ a_{n-1}(x)y^{(n-1)}+\cdots+a_1(x)y'+a_0(x)y=0$\\\\
Nonhomogeneous: $a_n(x)y^{(n)}+ a_{n-1}(x)y^{(n-1)}+\cdots+a_1(x)y'+a_0(x)y=g(x)$, where $g(x)\neq 0$\\\\
\underline{\textbf{Thm 4.1.2 Superposition Principle - Homogeneous Equations}}\\
If $y_1, y_2,\ldots, y_k$ are solutions to the \textit{homogeneous} nth-order linear differential equation on an interval $I$, then the linear combination where $c_1, c_2, \ldots, c_n$ are arbitrary constants.
\[y =c_1 y_1(x)+ c_2y_2(x)+\cdots+c_k y_k(x)\]
is \textit{also} a solution on the interval.\\\\
\newpage\noindent
\textbf{Fundamental Set of Solutions}\\
There exists a fundamental set of solutions $\{y_1, y_2, \ldots, y_n\}$ to the homogeneous nth-order linear differential equation. The fundamental set of solutions has to be \textit{linearly independent}.\\\\
We test for linear independence using the \textbf{Wronskian}:\\\\
The solutions $\{y_1, y_2, \ldots, y_n\}$ are linearly independent on the interval $I$ if and only if
\[W(y_1, y_2,..., y_n)=\begin{vmatrix}
y_1&y_2&\cdots &y_n\\
Dy_1&Dy_2&\cdots &Dy_n\\
\vdots&\vdots&&\vdots\\
D^{n-1}y_1&D^{n-1}y_2&\cdots &D^{n-1}y_n\\
\end{vmatrix}\neq0\]
for every $x$ in the interval.\\\\
\underline{Note:} $D$ is the differentiation operator, i.e. $Dy=\frac{dy}{dx}$, $D^2y=\frac{d^2y}{dx^2}$, and so on.\\\\
\underline{\textbf{Thm 4.1.5 General Solution — Homogeneous Equations}}\\
If $y_1, y_2, \ldots, y_n$ is a fundamental set of solutions to the homogeneous nth-order linear differential equation on an interval $I$, then the general solution is given by
\[y=c_1 y_1(x)+ c_2y_2(x)+\cdots+c_n y_n(x)\]
where $c_1, c_2, \ldots, c_n$ are arbitrary constants.\\\\
\underline{\textbf{Thm 4.1.6 General Solution — Nonhomogeneous Equations}}\\
If $y_p$ is a particular solution to the nonhomogeneous nth-order linear differential equation on an interval $I$, and if $y_h$ is the general solution to the corresponding homogeneous equation, then the general solution to the nonhomogeneous equation is given by
\[y=y_h + y_p\]
\newpage\noindent
\section*{4.2 Reduction of Order}
We know that the general solution of a homogeneous linear second-order differential equation
\[a_2(x)y''+ a_1(x)y'+ a_0(x)y =0 \]
is a linear combination 
\[y = c_1y_1 +c_2y_2\]
where $y_1$ and $y_2$ are solutions that constitute a linearly independent set on some interval $I$.\\\\
Given that the differential equation only has constant coefficients, if we know one solution $y_1$, we can find a second solution $y_2$ using the method of \textbf{reduction of order}.\\\\
\textbf{Method:}\\
Since $y_1$ and $y_2$ are linearly independent, then their quotient $y_2/y_1$ hat to be nonconstant on $I$. In other words,
\[\frac{y_2(x)}{y_1(x)}=u(x)\]
\[\implies y_2(x) = u(x) y_1(x)\]
We can find the function $u(x)$ by substituting $y_2(x) = u(x) y_1(x)$ into the original differential equation and solving for $u(x)$.\\\\
\underline{Note:} This method reduces the differential equation from second-order to first-order in terms of $w=u'(x)$. But, it only works for second order ODE.\\\\
\textbf{Solving for $u(x)$ for a General Case:}\\
For a second-order linear differential equation with constant coefficients:
\[a_2(x)y''+a_1(x)y'+a_0(x)y=0\]
We put into general form by dividing through by $a_2(x)$:
\[y''+P(x)y'+Q(x)y=0\]
Assuming we know one solution $y_1$, we let $y_2 = u(x)y_1(x)$. Then we compute the first and second derivatives of $y_2$:
\[y_2' = u'y_1 + uy_1'\]
\[y_2'' = u''y_1 + 2u'y_1' + uy_1''\]
Substituting $y_2$, $y_2'$, and $y_2''$ into the original differential equation gives:
\[u\left[y_1'' + P(x)y_1' + Q(x)y_1\right]+ u''y_1 + u'(2y_1' + P(x)y_1) = 0\]
Since $y_1$ is a solution to the original equation, $y_1'' + P(x)y_1' + Q(x)y_1=0$.
\[\implies u''y_1 + u'(2y_1' + P(x)y_1) = 0\]
Letting $w = u'$, we have a linear first order differential equation in $w$:
\[w'y_1 + w(2y_1' + P(x)y_1) = 0\]
We can use an integrating factor to solve for $w$:\\\\
\fbox{%
\begin{minipage}{0.95\textwidth}
\underline{Recall:} Given a first-order linear ODE of the form
\[\frac{dy}{dx}+p(x)y=g(x)\]
the integrating factor is given by
\[\mu(x) = e^{\textstyle\int p(x) \, dx}\]
and the solution is 
\[y(x)=\frac{1}{\mu}\int \mu g(x) \, dx\]
\end{minipage}%
}\\\\
Thus,
\[w'+ w\frac{2y_1' + P(x)y_1}{y_1} = 0\]
\[\mu(x) = e^{\textstyle \int \frac{2y_1' + P(x)y_1}{y_1} \, dx}\]
\[\int \frac{2y_1' + P(x)y_1}{y_1} \, dx = \int 2 \frac{y_1'}{y_1} \,dx + \int P(x) \, dx, \quad u=y_1, du=y'_1dx\]
\[2\int \frac{1}{u}\,dx+\int P(x) \, dx\]
\[2ln(y_1)+\int P(x) \, dx\]
\[\implies \mu(x) = e^{2\ln(y_1) + \int P(x) \, dx} = y_1^2 e^{\int P(x) \, dx}\]
Finally, we can solve for $w$:
\[w(x) = \frac{1}{\mu(x)}\int \mu(x) \cdot 0 \, dx = \frac{c_1}{\mu(x)} = c_1\frac{e^{-\int P(x) \, dx}}{y_1^2}\]
Then, we can find $u(x)$ by integrating $w$:
\[u(x) = \int w(x) \, dx =  c_1\int\frac{e^{-\int P(x) \, dx}}{y_1^2} \, dx+c_2\]
Choosing $c_1=1$ and $c_2=0$ for the fundamental set:
\[\boxed{u(x)=\int\frac{e^{-\int P(x) \, dx}}{y_1^2} \, dx}\]
Where $y_2(x) = u(x)y_1(x)$.
\newpage\noindent
\section*{4.3 Homogeneous Linear Equations with Constant Coefficients}
For higher order homogeneous linear differential equations with constant coefficients of the form, we can use the characteristic equation to find the general solution.\\\\
\textbf{Method:}\\
Given a homogeneous linear differential equation with constant coefficients:
\[ay'' + by' + cy = 0\]
We assume the solution of the form:
\vspace{-1em}
\begin{eqnarray*}
y(t) &=& e^{rt}\\
y'(t) &=& re^{rt}\\
y''(t) &=& r^{2}e^{rt}\\
\end{eqnarray*}
Then,  
\vspace{-1em}
\begin{eqnarray*}
ay''+by'+cy&=&0\\
ar^{2}e^{rt}+bre^{rt}+ce^{rt}&=&0\\
e^{rt}(ar^{2}+br+c)&=&0\\
ar^{2}+br+c&=&0
\end{eqnarray*}
We can solve for $r$ using the quadratic formula:
\[r=\frac{-b\pm\sqrt{b^2-4ac}}{2a}\]
\underline{Note:} For higher order differential equations, we would get a polynomial of degree $n$, and we would solve for the roots by factoring.\\\\
\textbf{Case 1: Real and Unique Roots}\\
For roots $r_1$ and $r_2$, the general solution is:
\[y(t) = c_1 e^{r_1 t} + c_2 e^{r_2 t}\]
\textbf{Case 2: Complex Roots}\\
For roots $\alpha \pm \beta i$, the general solution is:
\[y(t) =   c_1 e^{\alpha t}\cos \beta t + c_2 e^{\alpha t} \sin \beta t\]
\textbf{Case 3: Repeated Roots}\\
For repeated root $r$, the general solution is:
\[y(t) = c_1 e^{r t} + c_2 t e^{r t}\]
\newpage\noindent
\section*{4.4 Undetermined Coefficients (Superposition Approach)}
To find a particular solution $Y_p$ to the higher order nonhomogeneous differential equation with constant coefficients, we will guess the form of $Y_p$ based on the form of $g(x)$.\\\\
\textbf{Method:}
\begin{itemize}
    \item Guess a form of $Y_{P}(t)$ leaving the coefficient(s) undetermined (and hence the name of the method).
    \item Plug the guess into the differential equation and see if we can determine values of the coefficients.
    \item If we can determine values for the coefficients then we guessed correctly, if we can't find values for the coefficients then we guessed incorrectly.
\end{itemize}
In general, we will guess

\[\begin{array}{c|c}
g(t) & Y_{P}(t) \\
\hline\hline
ae^{bt} & Ae^{bt} \\
a\cos(\beta t) & A\cos(\beta t)+B\sin(\beta t) \\
a\sin(\beta t) & A\cos(\beta t)+B\sin(\beta t) \\
a\cos(\beta t)+b\sin(\beta t) & A\cos(\beta t)+B\sin(\beta t) \\
\text{n-th degree polynomial} & A_{n}t^{n}+\cdots +A_{1}t+A_{0}
\end{array}\]\\
\underline{Note:} If any term of $Y_{P}$ is duplicated in $y_{h}$ then the duplicated term must be multiplied by the minimum $+$ integer power of $t$ required to make all terms in general solution linearly independent. (So find the general homogeneous solution first before guessing the particular solution!)\\\\
Then we will substitute our guess for $Y_{P}(t)$ into the original differential equation and solve for the undetermined coefficients.
\section*{4.5 Undetermined Coefficients (Annihilator Approach)}
Another way to solve for a particular solution $Y_p$ to the higher order nonhomogeneous differential equation with constant coefficients is to use the \textbf{differential operator} $D$ to create an \textbf{annihilator operator} to reduce the ODE.\\\\
\textbf{Annihilator}\\
If $L$ is a differential operator such that 
\[L(g(x))=0\]
then $L$ is called an \textbf{annihilator} of $g(x)$.\\\\
\newpage\noindent
\underline{Ex:} 
\[y''+y'-6y=0\]
\[D^2y + Dy - 6y = 0\implies (D^2 + D - 6)y = 0\]
Here, the differential operator is $D^2 + D - 6$.\\\\
\textbf{Method:}\\
Given a nonhomogeneous differential equation:
\[a_n y^{(n)} + a_{n-1} y^{(n-1)} + \cdots + a_1 y' + a_0 y = g(x)\]
We can rewrite it using the differential operator $L$ on $y$ such that:
\[L(y) = g(x)\]
Next, we find an annihilator $L_1$ of $g(x)$ such that:
\[L_1(g(x)) = 0\]
Then, we apply $L_1$ to both sides of the equation:
\[L_1(L(y)) = L_1(g(x))\]
\[\implies L_1 L(y) = 0\]
So by solving the \textit{homogeneous higher-order equation} $L_1L(y)=0$ we can discover the form of a particular solution $y_p$ for the original \textit{nonhomogeneous equation} $L(y)=g(x)$.\\\\
\textbf{List of Annihilators:}
\begin{itemize}
    \item $D^n$ annihilates polynomials of degree $n-1$ or less. Thus, it annihilates each of the following:
    \[1,x,x^2,\ldots,x^{n-1}\]
    \item $(D-\alpha)^n$ annihilates each of the following:
    \[e^{\alpha x},xe^{\alpha x},x^2e^{\alpha x},\ldots,x^{n-1}e^{\alpha x}\]
    \item $[D^2-2\alpha D+(\alpha^2+\beta^2)]^n$ annihilates each of the following:
    \[e^{\alpha x}\cos \beta x, xe^{\alpha x}\cos \beta x, \ldots, x^{n-1}e^{\alpha x}\cos \beta x\]
    \[e^{\alpha x}\sin \beta x, xe^{\alpha x}\sin \beta x, \ldots, x^{n-1}e^{\alpha x}\sin \beta x\]
\end{itemize}
\underline{Ex:} Finding the particular solution for $y''+y'-6y=e^{4t}$\\
Using $y=e^{rt}$, we find the homogeneous solution:
\[y_h=c_1 e^{-3t} + c_2 e^{2t}\]
The annihilator for $e^{4t}$ is $(D-4)$. Thus, we apply the annihilator to both sides of the equation:
\[(D-4)(D^2 + D - 6)y = 0\]
Solving the characteristic equation:
\[(D-4)(D-2)(D+3)y=0\]
The general solution to the higher order homogeneous equation is (matching each annihilator factor to its corresponding $g(t)$ term):
\[y = c_1 e^{4t} + c_2 e^{2t} + c_3 e^{-3t}\]
Thus, we can choose the particular solution to be:
\[y_p = A e^{4t}\]
Substituting $y_p$ into the original differential equation we can solve for $A$.
\section*{4.6 Variation of Parameters}
For nonhomogeneous linear differential equations where the method of undetermined coefficients does NOT work (ie. $g(x)$ is not of the right form, and NONconstant coefficients), we can use the method of \textbf{variation of parameters} to find a particular solution.\\\\
\underline{Note:} Variation of parameters works for higher order ODEs, but the formulas get more complicated so most examples will be second order ODEs. \\
IF we are given $y_h$, the homogeneous solution, we can find $y_p$, the particular solution, using variation of parameters. However, if we are not given $y_h$, we must find it first using the characteristic equation method. (Thus, unless given $y_h$ the problems we do will still only be constant coefficients)\\\\
\textbf{Method:}\\
\vspace{-1em}
\begin{itemize}
    \item Obtain general homogeneous solution:
    \[y_h=c_1y_1 + c_2 y_2\]
    For the second order ODE:
    \[y'' + p(x)y' + q(x)y = g(x)\]
    \underline{Note:} the coefficient of $y''$ \textbf{must} be 1, so divide through if necessary.
    \item The particular solution is of the form:
    \[y_p = u_1(x)y_1 + u_2(x)y_2\]
    where we need to solve for $u_1(x)$ and $u_2(x)$.
\end{itemize}
Where $W(y_1,y_2)$ is the Wronskian of $y_1$ and $y_2$:
\[\boxed{{u_1=-\int{\frac{y_2g(t)}{W(y_{1},y_{2})}} \ dt, \ \  u_2=\int{\frac{y_1g(t)}{W(y_{1},y_{2})}} \ dt}}\]
Finally, the particular solution is:
\[\boxed{Y_{P} = -y_{1}\int{\frac{y_2g(t)}{W(y_{1},y_{2})}} \ dt+y_{2}\int{\frac{y_1g(t)}{W(y_{1},y_{2})}} \ dt}\]
\section*{4.7 Cauchy-Euler Equation}
Nonhomogeneous differential equations with variable coefficients of the form:
\[a_nx^ny^{(n)}+ a_{n-1}x^{n-1}y^{(n-1)}+\cdots+a_1xy'+a_0y=g(x)\]
where $a_n,a_{n-1},\ldots, a_1, a_0$ are constants, are known as \textbf{Cauchy-Euler equations}.\\\\
The method for solving Cauchy-Euler equations can be applied to higher order ODEs, but we will focus on second order ODEs.
\[ax^2y''+ bxy'+cy=g(x)\]
\underline{Note:} The coefficient $ax^2$ is zero at $x=0$.  To guarantee uniqueness we will assume we are looking for general solutions defined on the interval $(0,\infty)$.\\\\
\textbf{Method to solve for homogeneous solution:}\\
Given:
\[ax^2y''+ bxy'+cy=0\]
We assume the solution of the form:
\vspace{-1em}
\begin{eqnarray*}
y(t) &=& x^m\\
y'(t) &=& mx^{m-1}\\
y''(t) &=& m(m-1)x^{m-2}\\
\end{eqnarray*}
Plugging into the differential equation:
\vspace{-1em}
\begin{eqnarray*}
ax^2y''+ bxy'+cy&=&0\\
ax^2m(m-1)x^{m-2} + bxmx^{m-1} + cx^m &=& 0\\
x^m (am(m-1) + bm + c) &=& 0\\
am(m-1) + bm + c &=& 0
\end{eqnarray*}
Thus, the characteristic equation for $m$ is:
\[am^2 + (b-a)m + c = 0\]
\newpage\noindent
\textbf{Case 1: Real and Unique Roots}\\
For roots $m_1$ and $m_2$, the general solution is:
\[y_h = c_1 x^{m_1} + c_2 x^{m_2}\]
\textbf{Case 2: Repeated Roots}\\
If we have a repeated root then we know the discriminant of the quadratic formula must be zero.  Thus, the root is of the form $m_1=-(b-a)/(2a)$  To get the second solution we can do a \textbf{reduction of order}.\\\\
Let $y_1=x^{m_1}$.\\\\


\end{document}