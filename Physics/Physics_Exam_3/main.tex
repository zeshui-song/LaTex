\documentclass[12pt]{article}
\usepackage{amssymb}
\usepackage{geometry}
\usepackage{amsmath, amsfonts, bm, graphicx}
\usepackage{float,multicol}
\geometry{margin=1in}
\title{}
\date{}
\author{}

\begin{document}

\section*{Mathematical Methods}

\begin{enumerate}
    \item Use the Taylor series expansion to find approximations. The ones for sin, cos, tan, and $(1 + x)^n$ are especially useful.
\begin{flalign*}
\sin x &= \sum_{n=0}^{\infty} (-1)^n \frac{x^{2n+1}}{(2n+1)!} 
       = x - \frac{x^3}{3!} + \frac{x^5}{5!} - \frac{x^7}{7!} + \cdots & \\[6pt]
\cos x &= \sum_{n=0}^{\infty} (-1)^n \frac{x^{2n}}{(2n)!} 
       = 1 - \frac{x^2}{2!} + \frac{x^4}{4!} - \frac{x^6}{6!} + \cdots & \\[6pt]
\tan x &= \sum_{n=1}^{\infty} \frac{B_{2n} (-4)^n (1-4^n)}{(2n)!}\,x^{2n-1} 
       = x + \frac{x^3}{3} + \frac{2x^5}{15} + \frac{17x^7}{315} + \cdots & \\[6pt]
(1+x)^m &= \sum_{n=0}^{\infty} \binom{m}{n} x^n 
       = 1 + mx + \frac{m(m-1)}{2}x^2 + \frac{m(m-1)(m-2)}{6}x^3 + \cdots &
\end{flalign*}
\underline{Note:} For small x, higher order terms reduce to zero

    \item Use complex exponentials to manipulate complicated trig functions.
\[e^{ix} = \cos x + i \sin x\]
    \item Solve differential equations by substituting in trial solutions. Especially you should recognize the differential equation for a simple harmonic oscillator and be able to come up with solutions to that ODE that satisfy any initial conditions you are given.

\[\frac{d^2x}{dt^2} + \omega^2 x = 0\]

\[x(t) = A \cos(\omega t) + B \sin(\omega t)\]
Wave equation: 
\[v^2 \frac{\partial ^2 y}{\partial x^2}=\frac{\partial^2 y}{\partial t^2}\]

\item Useful integration formulas:
\[\int\frac{1}{(x^2+a^2)^\frac{3}{2}}dx=\frac{x}{a^2\sqrt{x^2+a^2}}+C\]
\[\int\frac{x}{(x^2+a^2)^\frac{3}{2}}dx=-\frac{1}{\sqrt{x^2+a^2}}+C\]
\end{enumerate}
\newpage
\section*{Chapter 28 - Magnetic Fields}
\textbf{Types of magnets}
\begin{itemize}
    \item \textbf{Current loop:} a current carrying loop of wire creates an electromagnet.
    \item \textbf{Permanent Magnet:} the magnetic fields of the electrons within the material do not cancel out, resulting in a net magnetic field.
\end{itemize}
\textit{All} magnets are \textbf{magnetic dipoles} with a \textbf{north} and \textbf{south} pole (the magnetic monopole doesn't exist, sadly). Opposite magnetic poles attract each other, and like magnetic poles repel each other. Magnetic field lines are \textit{closed loops} that exit through the North pole and enter through the South pole.
\begin{multicols}{2}
    \begin{figure}[H]
    \centering
    \includegraphics[width=0.6\textwidth]{Magnetic Field.png}
\end{figure}
\begin{figure}[H]
    \centering
    \includegraphics[width=0.34\textwidth]{Horseshoe.png}
\end{figure}
\end{multicols}\noindent
\underline{Note:} Inside the bar magnet, the magnetic field lines point from south to north, completing the closed loop.\\\\
\underline{Magnetic field lines and the magnetic field are related by:}
\begin{itemize}
    \item The direction of the magnetic field is tangent to the field lines.
    \item The spacing of the field lines represents the strength (magnitude) of the magnetic field. Closer lines = stronger field.
\end{itemize}
Also, analogous to Gauss's law for electric fields, we have \textbf{Gauss's law for magnetism:}
\[\int \vec{B} \cdot \hat{n}\,d\vec{A} = 0\]
Since there are no magnetic monopoles, the net magnetic flux through any closed surface is zero (there are no sources or sinks of magnetic field lines).\\\\\newpage
\begin{enumerate}
    \item Solve Newton's second law to determine the motion of charged particles acting under the influence of a magnetic field and any other forces (e.g., gravity, electric fields...).\\\\
Stationary charges do not interact with the magnetic field. Moving charges with a component of velocity perpendicular to the magnetic field experiences a force:
\[\boxed{\vec{F}_B = q\vec{v} \times \vec{B}}\]
\underline{Note:} This force is \textit{always} perpendicular to the velocity of the particle, so it does \textbf{no work} on the particle and cannot change its speed, only its direction.\\\\
\underline{Note:} The magnetic force is zero when the velocity is along the magnetic field lines (i.e., parallel or antiparallel) or when stationary.\\\\
 The unit for the magnetic field $\vec{B}$ is the Tesla (T):
 \[1\, \text{T} = 1\, \frac{\text{N}}{\text{C}\cdot \text{m/s}} = 1\, \frac{\text{N}}{\text{A}\cdot \text{m}}\]
\underline{Recall:} \textbf{Right hand rule}
\begin{multicols}{2}
\begin{figure}[H]
    \centering
    \includegraphics[width=0.3\textwidth]{RHR.png}
\end{figure}
\columnbreak
\begin{itemize}
    \item Point fingers in the direction of the velocity $\vec{v}$.
    \item Curl fingers toward the direction of the magnetic field $\vec{B}$, sweeping through the smaller angle.
    \item Thumb points in the direction of the force $\vec{F}_B$ for \textbf{a positive charge}. For a negative charge, the force is in the opposite direction.
\end{itemize}
\end{multicols}
\underline{Note:} When $\vec{B}$ and $\vec{v}$ are orthogonal, we can just multiply the magnitudes to find the force and use the right hand rule to find the direction.\\\\
\underline{Note:} A magnetic force exists even if there is \textit{relative velocity} between charges and a magnetic field. For example, a moving magnet will exert a magnetic force on stationary charges.\\\\
\underline{Note to self:} Bring dynamics formula sheet for kinematics equations.\\\\
The total electromagnetic force on a charged particle in both electric and magnetic fields is given by the \textbf{Lorentz force:}
\[\boxed{\vec{F} = q\vec{E} + q\vec{v} \times \vec{B}}\]
\newpage
    \item Explain the Hall effect and describe its applications.\\\\
Here are several interesting applications where both the magnetic field and electric field acts on a moving charge.\\\\
\textbf{Wien Filter (Velocity Selector)}
\begin{figure}[H]
    \centering
    \includegraphics[width=0.7\textwidth]{Wien.jpg}
\end{figure}
\vspace{-1em}
Suppose we have a source charged particles ($+q$) with random velocities. If it passes through a region with \textit{only} an $\vec{E}$ field, it will be pushed onto the negative plate, following a parabolic trajectory. However, if there is a $\vec{B}$ field in addition to the $\vec{E}$ field, then the forces will cancel for particles with a specific velocity:
\[\sum \vec{F}_y=\vec{F}_B-\vec{F}_E=0 \implies \vec{F}_B=\vec{F}_E \]
\[q\vec{v}\times \vec{B}=q\vec{E}\]
\[\boxed{v=\frac{E}{B}}\]
Thus, only particles with velocity $v=E/B$ will pass straight through the filter. \\\\
By combining the Wien filter with another region of magnetic fields, we create a mass spectrometer that can separate particles based on their charge-to-mass ratio.\\\\
\textbf{Mass Spectrometer}
\begin{figure}[H]
    \centering
    \includegraphics[width=0.4\textwidth]{Mass Spec.jpg}
\end{figure}
\vspace{-1em}
Since we know both the charge and velocity entering the magnetic field region, we can find the particle's mass by measuring the radius of its circular path:
\[\sum \vec{F}_n=\vec{F}_B=\frac{mv^2}{R} \implies qvB=\frac{mv^2}{R}\]
\[\boxed{m=\frac{BRq}{v}}\]
\newpage
\underline{Note:} Typically, the charges are accelerated through a potential difference $V$ before entering the velocity selector, so we can find their velocity using energy conservation:
\[W_{nc}=\Delta E = 0 \implies -\Delta U = \Delta K\]
\[\Delta K = -q\Delta V\]
For a positive charge to gain kinetic energy, it must be accelerated through a drop in electric potential (from high to low potential), thus $\Delta V = V_f-V_i = -V$:
\[qV=\frac{1}{2}mv^2 \implies v=\sqrt{\frac{2qV}{m}}\]
The full set up looks like this:
\begin{figure}[H]
    \centering
    \includegraphics[width=0.9\textwidth]{Mass Spec full.jpg}
\end{figure}
Also note that since the magnetic force is opposite for negative charges, they will curve in the opposite direction in the magnetic field region:
\begin{figure}[H]
    \centering
    \includegraphics[width=0.35\textwidth]{Diff Charges.jpg}
\end{figure}
\newpage
Let's consider a setup similar to a Wien filter, but where the parallel plates are designed to deflect the particle beam rather than selectively filter it.\\\\
\textbf{Cathode Ray Tube}
\begin{figure}[H]
    \centering
    \includegraphics[width=0.6\textwidth]{Deflection.jpg}
\end{figure}
\vspace{-1em}
First, without a $\vec{B}$ field, the particles will be deflected by the $\vec{E}$ field:
\[\sum \vec{F}=-qE\hat{\jmath} \implies |a_y|=\frac{|q|E}{m}\]
The time spent in the field is:
\[\Delta x = vt = L \implies t = \frac{L}{v}\]
The vertical displacement upon exiting the plates is:
\[\Delta y = v_{oy}t + \frac{1}{2}a_yt^2 = 0 + \frac{1}{2}\left(\frac{|q|E}{m}\right)\left(\frac{L}{v}\right)^2\]
\[\boxed{\Delta y = \frac{|q|EL^2}{2mv^2}}\]
Now consider adding a magnetic field like this:
\begin{figure}[H]
    \centering
    \includegraphics[width=0.6\textwidth]{CRT.png}
\end{figure}
\vspace{-1em}
We know from the Wien filter that the forces will cancel when:
\[v=\frac{E}{B}\]
Thus, plugging this into our previous equation for vertical displacement:
\[\frac{m}{|q|}=\frac{BL^2}{2\Delta y E}\]
\newpage
Finally, let's talk about the Hall effect!\\\\
\textbf{Hall Effect}
\begin{figure}[H]
    \centering
    \includegraphics[width=0.6\textwidth]{Hall Effect.jpg}
\end{figure}
\vspace{-1em}
Consider a conductor with a current $I$ flowing through it in a region with a $\vec{B}$ field. The moving charges will be pushed to one side of the conductor by the magnetic force, creating a \textbf{Hall potential difference} ($\Delta V$) and an electric field ($\vec{E}$) inside the conductor. 
\[\Delta V=Ed\]
Eventually, when the electric force balances the magnetic force, the charges stop accumulating.
\[\sum \vec{F} = \vec{F}_E-\vec{F}_B=0 \implies qE=qv_dB\]
Thus, by measuring the Hall potential difference, we can find the magnetic field strength:
\[\boxed{B=\frac{\Delta V}{v_d d}}\]
We can also find the number of charge carriers per unit volume ($n$) in the conductor, letting $q=e$ for electrons and plugging in for $v_d$ from before:
\[I=nev_d A ,\quad \text{(A is cross-sectional area of conductor)}\]
\[n=\frac{IBd}{eA\Delta V}\]
\underline{Note:} It is also possible to determine the drift velocity using the Hall effect, by mechanically moving the conductor such that there is no relative velocity between the charges and the magnetic field. Therefore, there will be \textbf{zero} Hall potential difference (since there is no magnetic force).
\newpage
    \item Explain the principle of operation of a cyclotron\\\\
\textbf{Circulating Charged Particles}\\
We know that for a particle of charge $q$ moving with $\vec{v}$ in a uniform magnetic field, it will tend towards a circular path due to the magnetic force (no tangential force, only normal force).
\[\sum F_n = |q|vB=\frac{mv^2}{r} \implies r=\frac{mv}{|q|B} \quad \text{(radius)}\]
From which we can define the following quantities:
\[T=\frac{2\pi r}{v}=\frac{2 \pi}{v}\frac{mv}{|q|B}=\frac{2\pi m}{|q|B} \quad \text{(period)}\]
\[f=\frac{1}{T}=\frac{|q|B}{2\pi m} \quad \text{(frequency)}\]
\[\omega = \frac{2 \pi}{T}=2\pi f = \frac{|q|B}{m} \quad \text{(angular frequency)}\]
\underline{Note:} The quantities $T,f, \text{ and } \omega$ do not depend on the speed of the particle (as long as it isn't moving at relativistic speeds). Fast particles move in large circles and slow ones in small circles, but all particles with the same charge-to-mass ratio $|q|/m$ take the same time $T$ to complete one loop.\\\\
\textbf{Helical Paths}\\
If the velocity of the charged particle has a component parallel to the magnetic field, then the particle will follow a helical path:
\vspace{-1em}
\begin{figure}[H]
    \centering
    \includegraphics[width=0.6\textwidth]{Helical.png}
\end{figure}
\vspace{-1em}
Where the angle $\phi$ is the angle between $\vec{v}$ and $\vec{B}$.
\[v_{\parallel} = v \cos \phi, \quad v_{\perp} = v \sin \phi\]
The radius of the helical path is determined by $v_{\perp}$:
\[r=\frac{mv_{\perp}}{|q|B}=\frac{mv\sin \phi}{|q|B}\]
The pitch of the helix (distance between successive turns) is determined by $v_{\parallel}$ and the period $T$:
\[p=v_{\parallel}T=v\cos \phi \left(\frac{2\pi m}{|q|B}\right)\]
\newpage\noindent
\textbf{Cyclotron}\\
A cyclotron is a device that uses a combination of a constant magnetic field and an oscillating electric potential difference to accelerate charged particles. The magnetic field forces the particles to move in circular paths while the potential difference between the dees accelerates them each time they cross the gap.
\begin{figure}[H]
    \centering
    \includegraphics[width=0.35\textwidth]{Cyclotron.png}
\end{figure}
Suppose a proton is injected at source $S$. It will be accelerated toward the negatively charged dee and enter it. Once inside, there will be no electric field (shielded by the conducting walls of the dee), and it will move in a semicircular path due to the magnetic field. When it exits the dee, the potential difference is reversed to accelerate it again across the gap. Thus, the frequency $f$ at which the proton circulates (independent of speed) \textit{must} match the frequency of the oscillating potential difference $f_{osc}$:
\[f=f_{osc} \quad \text{(resonance condition)}\]
\[\frac{|q|B}{2\pi m}=f_{osc}\]
\underline{Note:} To find the speed of particles exiting the cyclotron, we can use the radius and centripetal force due to the magnetic field, or by using the frequency of oscillation:
\[f=\frac{1}{T}=\frac{v}{2\pi r}\]
\textbf{Synchrotron}\\
At relativistic speeds (above 10\% of $c$), the frequency of revolution is now no longer independent of the charged particle's speed. As the speed approaches the speed of light, the frequency of revolution decreases, and is no longer in sync to the fixed $f_{osc}$. Thus a \textbf{synchrotron} is used to vary both the magnetic field and $f_{osc}$ to keep the particle in resonance as it accelerates to higher speeds. The proton also follows a circular path instead of a spiral in a synchrotron.
\newpage
    \item Determine the forces and/or torques on various arrangements of current carrying wires (straight, circular loops, square loops, etc...) located in a given magnetic field.\\\\
\textbf{Magnetic Force on a Current Carrying Wire}\\
We know that moving charges experience a magnetic force in a magnetic field. Thus, a current-carrying wire (which has moving charges) will also experience a magnetic force when placed in a magnetic field.
\begin{figure}[H]
    \centering
    \includegraphics[width=0.5\textwidth]{Wire.png}
\end{figure}
\vspace{-1em}
\underline{Note:} The motion of electrons is opposite to the direction of conventional current. However, since both the charge and velocity are negative, the magnetic force ends up being in the same direction as if we considered positive charges moving with the current.\\\\
We know that the magnetic force on a single charge is:
\[\vec{F}_B = q\vec{v} \times \vec{B}\]
Thus, for $N$ charges in a wire segment of length $L$, the total magnetic force is:
\[\vec{F}_B = Nq\vec{v}_d \times \vec{B}\]
If we rewrite $N$ in terms of the number of charge carriers per unit volume $n$ and the volume of the wire segment $AL$ (where $A$ is the cross-sectional area), we get:
\[\vec{F}_B = (nAL)q\vec{v}_d \times \vec{B}\]
Recall that current is defined as $I=qnv_dA$, so we can rewrite the magnetic force as:
\[\boxed{\vec{F}_B = I \vec{L} \times \vec{B} \quad \text{(force on a straight wire)}}\]
Where $\vec{L}$ is a vector in the direction of the conventional current with magnitude $L$.\\\\
If the wire is not straight or the field is not uniform, we can find the differential force on a small current element $I dl$ and integrate over the length of the wire:
\[\boxed{d\vec{F}_B = I d\vec{L} \times \vec{B}}\]
\underline{Note:} There is no such thing as an isolated current-carrying wire, there must always be a way to introduce current into the wire and take it out at the other end.
\newpage
\textbf{Magnetic Torque on a Current Loop}\\
A motor converts current into rotation by using magnetic forces on a current-carrying loop to generate a torque. In this case, the direction of current is reversed every half turn to keep the torque in the same direction using a commutator (not shown).
\vspace{-1em}
\begin{figure}[H]
    \centering
    \includegraphics[width=0.4\textwidth]{Torque.png}
\end{figure}
\vspace{-1em}
Let's consider the following rectangular current loop in a uniform magnetic field:
\begin{figure}[H]
    \centering
    \includegraphics[width=0.9\textwidth]{Torque 1.jpg}
\end{figure}
\begin{multicols}{2}
\begin{figure}[H]
    \centering
    \includegraphics[width=0.34\textwidth]{RHR 2.png}
\end{figure}
The orientation of the loop is defined using a normal vector $\vec{n}$ that is perpendicular to the plane of the loop and follows the right-hand rule with respect to the current direction.\\\\
Curl fingers in the direction of the current and the thumb points in the direction of $\vec{n}$. The angle $\theta$ is defined as the angle between $\vec{n}$ and $\vec{B}$.
\end{multicols}
Finding the magnetic force on each side of the loop using $\vec{F}_B = I \vec{L} \times \vec{B}=ILBsin\phi$ where $\phi$ is the angle between $\vec{L}$ and $\vec{B}$:
\[||\vec{F}_1||=||\vec{F}_3||=iaB\]
\[||\vec{F}_2||=||\vec{F}_4||=ibBsin(90^\circ-\theta)=ibBcos\theta\]
\newpage
By symmetry, the forces act in opposite directions on each side, so the net force on the loop is zero. However, there is a torque about the center of the loop due to $\vec{F}_1$ and $\vec{F}_3$ (since their lines of action do not pass through the center):
\[\tau = \vec{r}\times\vec{F}\]
\[\tau = \left(iaB \frac{b}{2}sin\theta\right)+\left(iaB \frac{b}{2}sin\theta\right)=iabBsin\theta\]
Note that $A=ab$ is the area of the loop, so we can rewrite the torque as:
\[\boxed{\tau = iABsin\theta}\]
This relation holds for any shape of current loop, as long as $A$ is the area of the loop and $\theta$ is the angle between $\vec{n}$ and $\vec{B}$.\\\\
If we have a \textit{coil} with $N$ loops of wire, we can approximate them as $N$ identical current loops stacked together in the same plane. Thus, the total torque on the coil is:
\[\boxed{\sum \tau = NiABsin\theta}\]
\underline{Note:} The current-carrying coil will tend to rotate such that $\vec{n}$ is aligned with $\vec{B}$, minimizing the potential energy of the system.
\newpage
\textbf{Magnetic Dipole Moment}\\
Similar to a bar magnet, a current-carrying loop tends to align itself with an external magnetic field. Thus, the current loop is said to be a \textbf{magnetic dipole} with a \textbf{magnetic dipole moment} $\vec{\mu}$ defined as:
\[\boxed{\vec{\mu} = NIA\hat{n}}\]
Where $N$ is the number of loops, $I$ is the current, $A$ is the area of the loop, and $\hat{n}$ is the unit normal vector to the plane of the loop.It has units of Amphere-square meter ($\text{A}\cdot \text{m}^2$). \\\\
Using $\vec{\mu}$, we can rewrite the torque on the current loop as:
\[\boxed{\vec{\tau} = \vec{\mu} \times \vec{B}}\]
\underline{Note:} This is only for torque about an axis through the center of the loop.\\\\
Similar to the electric dipole in an electric field, the potential energy of a magnetic dipole in a magnetic field is given by integrating the work done by the magnetic torque as it rotates ($\Delta U = -W_c$):
\[\boxed{U = -\vec{\mu} \cdot \vec{B}}\]
The minimum potential energy ($-\mu B$) occurs when $\vec{\mu}$ is aligned with $\vec{B}$, and the maximum potential energy ($\mu B$) occurs when they are anti-aligned.
\begin{figure}[H]
    \centering
    \includegraphics[width=0.3\textwidth]{Potential.png}
\end{figure}
\underline{Note:} The work done by an external torque to rotate the dipole is $W_{ext}=\Delta U = U_f - U_i$. If the dipole is stationary before and after the rotation.\\\\
We will later see that there is an energy stored in the external magnetic field. The potential energy of the magnetic dipole is related to the change in the energy stored in the magnetic field when the dipole is rotated.\\\\
\underline{Note:} This idea of a change in energy will have to consider the system including the current loop that produces the external magnetic field. Thus, there will be mutual inductance effect as well.
\end{enumerate}
\newpage\noindent
A bar magnet and a rotating sphere of charge are magnetic dipoles as well, and we can approximate the Earth as a big magnetic dipole. Most subatomic particles (like electrons, protons, and neutrons) also have intrinsic magnetic dipole moments due to their spin and charge. Thus, we can model their interactions with magnetic fields using the same equations as above.\\\\
When a magnet exerts a magnetic force on a current-carrying wire, Newton’s third law requires that the wire exert an equal and opposite force back on the magnet. The only way the wire can interact with the magnet is through magnetic fields, so the wire must itself produce a magnetic field...
\section*{Chapter 29 - Magnetic Fields due to Currents}
\begin{enumerate}
    \item Use the Biot-Savart law to calculate the magnetic field due to a current-carrying wires of arbitrary (but tractable) geometry. e.g., a loop.
\begin{multicols}{2}
Like electric fields, magnetic fields obey superposition. Thus, we can find the magnetic field from a wire by summing up the $d\vec{B}$ at point $P$ produced by small current elements $Id\vec{l}$ along the wire.
\begin{figure}[H]
    \centering
    \includegraphics[width=0.5\textwidth]{Biot Savart.jpg}
\end{figure}
\end{multicols}
\vspace{-1em}
We can find $d\vec{B}$ by using the \textbf{Biot-Savart Law}:
\[\boxed{d\vec{B} = \frac{\mu_0}{4\pi} \frac{ I d\vec{l} \times \hat{r}}{r^2}\quad \text{(current carrying wire)}}\]
Where
\begin{itemize}
    \item $I d\vec{l}$ is the current element that produces the differential magnetic field $d\vec{B}$.
    \item $r$ is the distance from the current element to point $P$.
    \item $\hat{r}$ is the unit vector that points from the current element to point $P$.
    \item $\mu_0$ is the permeability of free space and has a value of:
\[\boxed{\mu_0 = 4\pi \times 10^{-7} \,\text{T}\cdot \text{m/A}}\]
\end{itemize}
\underline{Note:} $I d\vec{l} \times \hat{r} = I dl \sin \theta $ where $\theta$ is the angle between $d\vec{l}$ and $\vec{r}$.\\\\
For a moving point charge:
\[\boxed{d\vec{B} = \frac{\mu_0}{4\pi} \frac{ q \vec{v} \times \hat{r}}{r^2} \quad \text{(point charge)}}\]
\newpage\noindent
We can also find the direction of the magnetic field using the right-hand rule:
\begin{multicols}{2}
\begin{figure}[H]
    \centering
    \includegraphics[width=0.2\textwidth]{RHR 3.png}
\end{figure}
\begin{itemize}
    \item Grasp the current element and point thumb in the direction of the current.
    \item Your fingers will then naturally curl around in the direction of the magnetic field lines due to that element.
\end{itemize}
\end{multicols}
\textbf{Magnetic Field Due to a Current in a Straight Wire}
\begin{figure}[H]
    \centering
    \includegraphics[width=0.4\textwidth]{Long Wire.jpg}
\end{figure}
\vspace{-1em}
Biot-Savart law:
\[d\vec{B} = \frac{\mu_0}{4\pi} \frac{ I d\vec{l} \times \hat{r}}{r^2}\]
Vectors:
\[d\vec{l} = dy' \hat{\jmath}\]
\[\vec{r}=\left<x,0\right>-\left<0,y'\right>=\left<x,-y'\right>,\quad ||\vec{r}|| = \sqrt{x^2 + y'^{\,2}}\]
\[\hat{r}=\frac{\left<x,-y'\right>}{\sqrt{x^2 + y'^{\,2}}}\]
Cross product:
\begin{align*}
d\vec{B} &= \frac{\mu_0}{4\pi} \frac{ I dy'\hat{\jmath} \times \left<x,-y'\right>}{(x^2 + y'^{\,2})^{3/2}}\\
&= \frac{\mu_0}{4\pi} \frac{ -Ixdy'\hat{k}}{(x^2 + y'^{\,2})^{3/2}}
\end{align*}
Integrate:
\begin{align*}
\vec{B}&=-\frac{\mu_0 I}{2\pi}\int_{-L/2}^{L/2}\frac{xdy'}{(x^2 + y'^{\,2})^{3/2}} \hat{k} \\
&=-\frac{\mu_0 I}{2\pi} \frac{1}{x\sqrt{x^2+(L/2)^2}}\hat{k}
\end{align*}
Taking the limit as $L \to \infty$, letting $r=x$:
\[\boxed{||\vec{B}|| = \frac{\mu_0 I}{2\pi r} \quad \text{(infinite straight wire)}}\]
\underline{Note:} For a semi-infinite wire (from $0\to\infty$), the magnetic field is half that of an infinite wire at the same distance $r$ from the wire.
\[||\vec{B}|| = \frac{\mu_0 I}{4\pi r} \quad \text{(semi-infinite straight wire)}\]
\textbf{Magnetic Field Due to a Current in a Circular Loop of Wire}
\begin{figure}[H]
    \centering
    \includegraphics[width=0.5\textwidth]{Loop.jpg}
\end{figure}
Biot-Savart law:
\[d\vec{B} = \frac{\mu_0}{4\pi} \frac{ I d\vec{l} \times \hat{r}}{r^2}\]
Vectors:
\[d\vec{l} = Rd\theta \, \hat{\theta}\]
\[\vec{r}=\left<0,0,z\right>-\left<Rcos\theta,Rsin\theta,0\right>=\left<-Rcos\theta,-Rsin\theta,z\right>,\quad ||\vec{r}|| = \sqrt{R^2+z^2}\]
\[\hat{r}=\frac{\left<-Rcos\theta,-Rsin\theta,z\right>}{\sqrt{R^2+z^2}}\]
\newpage\noindent
Rewriting $\hat{\theta}$ in Cartesian, using a unit circle:
\begin{figure}[H]
    \centering
    \includegraphics[width=0.3\textwidth]{Unit circle.jpg}
\end{figure}
\vspace{-2em}
\[\hat{\theta}=\left<-sin\theta,cos\theta,0\right>\]
Cross product:
\begin{align*}
d\vec{l}\times\vec{r}&=
\begin{vmatrix}
\hat{\imath} & \hat{\jmath} & \hat{k} \\
-Rd\theta sin\theta & Rd\theta cos\theta & 0 \\
-Rcos\theta & -Rsin\theta & z
\end{vmatrix}\\
&=(Rzcos\theta d\theta)\hat{\imath}+(Rzsin\theta d\theta)\hat{\jmath}+(R^2 sin^2\theta d\theta+R^2cos^2 \theta d\theta)\hat{k}\\
&=(Rzcos\theta d\theta)\hat{\imath}+(Rzsin\theta d\theta)\hat{\jmath}+(R^2 d\theta)\hat{k}
\end{align*}
\vspace{-2em}
\begin{figure}[H]
    \centering
    \includegraphics[width=0.45\textwidth]{loop 2.png}
\end{figure}
By symmetry, the $\hat{\imath}$ and $\hat{\jmath}$ components will cancel out when integrating over the full loop, so we only need to consider the $\hat{k}$ component:
\[B_z=\int_{0}^{2\pi} \frac{\mu_0}{4\pi}\frac{IR^2 \, d\theta}{(R^2+z^2)^{3/2}}=\frac{\mu_0}{4\pi}\frac{2\pi IR^2}{(R^2+z^2)^{3/2}}\]
Notice that $\mu = NIA = I\pi R^2$, Thus
\[\boxed{\vec{B}=\frac{\mu_0}{4\pi}\frac{2\mu}{(R^2+z^2)^{3/2}} \hat{k}}\]
\textbf{Magnetic Field due to a Current in a Circular Arc of Wire}
\begin{figure}[H]
    \centering
    \includegraphics[width=0.5\textwidth]{Arc.jpg}
\end{figure}
\vspace{-1em}
Biot-Savart law:
\[d\vec{B} = \frac{\mu_0}{4\pi} \frac{ I d\vec{l} \times \hat{r}}{r^2}\]
The angle between $d\vec{l}$ and $\vec{r}$ is always $90^\circ$ for a circular arc, so:
\[d\vec{B}=\frac{\mu_0}{4\pi} \frac{ I dl}{R^2} \hat{k},\quad \text{(Direction found by RHR)}\]
\[\vec{B}=\int_{0}^{R \phi}\frac{\mu_0}{4\pi} \frac{ I dl}{R^2} \hat{k}\]
\[\boxed{\vec{B}=\frac{\mu_0 I\phi}{4\pi R} \hat{k} \quad \text{(circular arc of wire)}}\]
    \item Describe Ampere's law and the conditions under which it can be used to solve for a magnetic field from a given current distribution. Use Ampere's law to find the field in those situations with sufficient symmetry to apply it. e.g., a long wire.\\\\
\textbf{Ampere's Law}\\
Ampere's Law states that the circulation of the magnetic field $\vec{B}$ around a closed \textbf{Amperian loop} is proportional to the net current $I_{\text{enc}}$ piercing the surface bounded by that loop.
\[\boxed{\int \vec{B}\cdot d\vec{l}=\mu_0 I_{\text{enc}}}\]
We can arbitrarily choose a direction for integration around the Amperian loop and assume that the magnetic field $\vec{B}$ is along this direction. If the resulting value of $\vec{B}$ is negative, then the true direction of the magnetic field is opposite to the assumed direction.
\newpage\noindent
The sign for the current enclosed can be determined using the right-hand rule:
\begin{multicols}{2}
\begin{figure}[H]
    \centering
    \includegraphics[width=0.3\textwidth]{RHR4.png}
\end{figure}
\begin{itemize}
    \item Curl fingers around the Amperian loop, with fingers pointing in the direction of integration ($d\vec{l}$).
    \item Your thumb will point in the direction of positive current piercing the surface.
\end{itemize}
\end{multicols}
\underline{Note:} Generally, for a CCW integration direction, current coming out of the page is positive and current going into the page is negative.\\\\
The magnetic field determined using Ampere's law is the superposition of the fields produced by all currents, including those outside the Amperian loop. However, similar to Gauss's law, only the current enclosed by the Amperian loop contributes to the net magnetic circulation around the closed path, while the contribution from external currents results in zero net circulation.\\\\
Ampere's law is most useful for finding magnetic fields in situations with high symmetry, such as:
\begin{itemize}
    \item Infinite straight wire (cylindrical symmetry)
    \item Infinite solenoid (translational and cylindrical symmetry)
    \item Toroidal solenoid (cylindrical symmetry)
    \item Infinite sheet of current (planar symmetry)
\end{itemize}
In these cases, the magnetic field is uniform along the Amperian loop, allowing us to take $\vec{B}$ outside the integral.
\newpage\noindent
\textbf{Magnetic Field of a Long Straight Wire with Current}\\
\textbf{Outside wire:}
\begin{figure}[H]
    \centering
    \includegraphics[width=0.4\textwidth]{Ampere long wire.jpg}
\end{figure}
\vspace{-1em}
Ampere's law:
\[\int \vec{B}\cdot d\vec{l}=\mu_0 I_{\text{enc}}\]
\[B\int dl = \mu_0 I\]
\[\boxed{B=\frac{\mu_0 I}{2\pi r} \quad \text{(outside wire)}}\]
\textbf{Inside wire (with uniform current density):}
\begin{figure}[H]
    \centering
    \includegraphics[width=0.4\textwidth]{Ampere long wire 1.jpg}
\end{figure}
\vspace{-1em}
Current density:
\[J=\frac{I}{\pi R^2}\]
\[\implies I_{\text{enc}} = J \pi r^2 = I \frac{r^2}{R^2}\]
Ampere's law:
\[\int \vec{B}\cdot d\vec{l}=\mu_0 I_{\text{enc}}\]
\[B\int dl = \mu_0 I \frac{{r^2}}{R^2}\]
\[\boxed{B=\frac{\mu_0 I r}{2\pi R^2} \quad \text{(inside wire)}}\]
\newpage\noindent
\textbf{Magnetic Field of an Infinite Solenoid}\\
A solenoid is to magnetostatics as capacitors is to electrostatics. An ideal infinite solenoid produces a uniform magnetic field inside and \textbf{zero} magnetic field outside.\\\\
\textit{Why is it zero outside the infinite solenoid?}\\
Consider the magnetic field at a point off axis from a single current loop. When it is close, the current loop looks more like an infinite wire, magnetic field pointing left.
\vspace{-1em}
\begin{figure}[H]
    \centering
    \includegraphics[width=0.3\textwidth]{Solenoid near.jpg}
\end{figure}
\vspace{-1em}
However, as point $P$ moves further away, the current loop starts to look more like a magnetic dipole, magnetic field pointing right.
\vspace{-1em}
\begin{figure}[H]
    \centering
    \includegraphics[width=0.6\textwidth]{Solenoid far.jpg}
\end{figure}
\vspace{-1em}
The key idea is that, for an infinite solenoid, only a \textit{finite} set of nearby current loops contributes a strong leftward magnetic field at a given external point, while an \textit{infinite} set of more distant loops contributes a much weaker rightward field. The cumulative effect of these infinitely many weak contributions exactly cancels the finite strong contribution, yielding zero net magnetic field outside the solenoid. The full mathematical proof is given by Pathak (\textit{An elementary argument for the magnetic field outside a solenoid}).
\begin{multicols}{2}
    Note that in a \textbf{finite} solenoid, it is not long enough for complete cancellation to occur, so there will be a small magnetic field outside the solenoid (similar to the fringe electric fields in a finite capacitor).
    \begin{figure}[H]
    \centering
    \includegraphics[width=0.29\textwidth]{Ideal Solenoid.png}
\end{figure}
\end{multicols}
\newpage
Calculating the magnetic field inside an infinite solenoid using Ampere's law:
\begin{figure}[H]
    \centering
    \includegraphics[width=0.5\textwidth]{Solenoid.jpg}
\end{figure}
Suppose an Amperian loop with side length $L$ that encloses $N$ loops of wire, each loop carries current $I$
\[\int \vec{B}\cdot d\vec{l}=\mu_0 I_{\text{enc}}\]
\[\int_b B dl = \mu_0 NI \quad \text{(Zero magnetic circulation along sides $a,c,d$)}\]
\[B_{\text{inside}}=\frac{\mu_0 NI}{L}\]
\[\boxed{B_{\text{inside}}=\mu_0 n I \quad \text{(infinite solenoid)}}\]
Where $n=\frac{N}{L}$ is the number of loops per unit length.\\\\
\underline{Note:} If the Amperian loop encloses both ends of the loop, the net current enclosed is zero, so the magnetic field outside the solenoid is zero as well.\\\\
We can also use the right-hand rule to find the direction of the magnetic field inside the solenoid. Curl fingers around the solenoid in the direction of the current, and your thumb will point in the direction of the magnetic field inside the solenoid (it will also point towards the North pole of the solenoid if it were a bar magnet).
\newpage\noindent
\textbf{Magnetic Field of a Toroidal Solenoid}
\vspace{-1em}
\begin{figure}[H]
    \centering
    \includegraphics[width=0.7\textwidth]{Toroid.png}
\end{figure}
\vspace{-1em}
Suppose the toroid has $N$ loops of wire, each carrying current $I$, we will integrate clockwise around an Amperian loop of radius $r$:
\[\int \vec{B}\cdot d\vec{l}=\mu_0 I_{\text{enc}}\]
\[B(2\pi r)=\mu_0 I N\]
\[\boxed{B=\frac{\mu_0 IN}{2\pi r} \quad \text{(toroidal solenoid)}}\]
\textbf{Infinite Sheet of Current}
\vspace{-1em}
\begin{figure}[H]
    \centering
    \includegraphics[width=0.5\textwidth]{Inf Sheet 1.jpg}
\end{figure}
\vspace{-1em}
Suppose the sheet carries a current density $\rho$ (current per unit length). We will integrate around a rectangular Amperian loop of width $L$ and height $H$:
\begin{multicols}{2}
\begin{figure}[H]
    \includegraphics[width=0.4\textwidth]{Inf Sheet.jpg}
\end{figure}
\[\int \vec{B}\cdot d\vec{l}=\mu_0 I_{\text{enc}}, \quad I_{\text{enc}} = \rho L\]
\[B(2L)=\mu_0 \rho L\]
\[\boxed{B=\frac{\mu_0 \rho}{2} \quad \text{(infinite sheet of current)}}\]
\underline{Note:} Due to superposition, the magnetic field above and below are uniform and point in opposite directions.
\end{multicols}
\newpage
    \item Find the magnetic force on a current carrying wire due to another current carrying wire.\\\\
\textbf{Force Between Two Parallel Current-Carrying Wires}
\begin{figure}[H]
    \centering
    \includegraphics[width=0.4\textwidth]{parallel wire.png}
\end{figure}
\vspace{-1em}
To find the force on wire b due to wire a, we first find the magnetic field produced by wire a at the location of wire b:
\[B_a = \frac{\mu_0 i_a}{2\pi d}\]
Since the distance between the wires is constant, this magnetic field is unifrom along wire b. Now we can find the force on wire b using 
\[\vec{F}_{ba} = i_b \vec{L} \times \vec{B}_a\]
\[F_{ba}=i_bLB_asin(90^\circ)=\frac{\mu_0Li_ai_b}{2\pi d}, \quad \text{($\vec{L}$ and $\vec{B}_a$ are perpendicular)}\]
Using the right-hand rule, we find that the force is toward wire a. By Newton's third law, the force on wire a due to wire b is equal in magnitude and opposite in direction.\\\\
\underline{Note:} Parallel currents attract, and anti-parallel currents repel.
    \item Describe the forces and torques on magnetic dipoles in terms of their magnetic moment.\\\\
Recall that the torque on a magnetic dipole in a magnetic field is given by:
\[\vec{\tau} = \vec{\mu} \times \vec{B} \implies U = -\vec{\mu} \cdot \vec{B}, \qquad \text{where } \vec{\mu} = NIA\hat{n}\]
We know that potential energy is related to force by:
\[\boxed{\vec{F} = -\nabla U}\]
\newpage\noindent
\underline{Ex:} Force between a bar magnet and a circular loop of current\\\\
Since both the bar magnet and the current loop are magnetic dipoles, we can imagine that they will behave like two bar magnets whose like poles repel and opposite poles attract.
\vspace{-1em}
\begin{figure}[H]
    \centering
    \includegraphics[width=0.9\textwidth]{dipole.jpg}
\end{figure}
\vspace{-1em}
Recall that the magnetic field along the axis of a current loop of radius $R$ is: 
\[B_z=\frac{\mu_0}{4\pi}\frac{2\mu}{(R^2+z^2)^{3/2}}\]
Taking the limit $R \ll z$ (far field approximation):
\[B_{z}=\frac{\mu_0}{4\pi}\frac{2\mu}{z^3}\]
Thus, the potential energy of the current loop in the magnetic field of the bar magnet is:
\[U=-\vec{\mu}_{loop} \cdot \vec{B}_{bar,z} = -\mu_{loop}\frac{\mu_0}{4\pi}\frac{2\mu_{bar}}{z^3}\]
Now let's find the magnitude of the force on the current loop due to the bar magnet:
\[\boxed{||\vec{F}_z||=\left|-\frac{dU}{dz}\right|=\frac{\mu_0}{4 \pi}\frac{6\mu_{bar}\,\mu_{loop}}{z^4}}\]
\begin{multicols}{2}
    We can find the direction of the force using the right-hand rule (also by intuition by looking at the poles). Note that this force is equal and opposite for the force on the bar magnet due to the current loop.
\begin{figure}[H]
    \centering
    \includegraphics[width=0.5\textwidth]{dipole 1.jpg}
\end{figure}
\end{multicols}
\textbf{Force from a Magnetic Mirror}\\
A charged particle can be confined to a region of space by a magnetic field that is stronger at the ends than in the middle. As the particle approaches the stronger field (shown by the more closely spaced field lines), a component of the magnetic force pushes it back toward the center of the region, reflecting it back and forth between the two ends.\\\\
In a magnetic mirror, a charged particle moves in a helical path with
\[\vec{v} = v_{\parallel} \hat{B} + v_{\perp} \hat{\perp}\]
With a component along the magnetic field and a component perpendicular to it.
\vspace{-1em}
\begin{figure}[H]
    \centering
    \includegraphics[width=0.5\textwidth]{Fields_in_magnetic_bottles.jpg}
\end{figure}
Magnetic moment of a positive charged particle (with an axis of rotation tangent to the local magnetic field line):
\[T=\frac{2\pi r_L}{v_{\perp}}=\frac{2\pi}{\omega},
\qquad \omega=\frac{qB}{m}\]
For a single loop \(N=1\), the magnetic dipole moment magnitude is
\[\|\mu\| = IA, \qquad I=\frac{q}{T}=\frac{q\omega}{2\pi}, \qquad A=\pi r_L^2, \qquad
r_L=\frac{m v_{\perp}}{qB}\]
Substituting,
\[\|\mu\|
= \frac{q\omega}{2\pi}\,\pi r_L^2
= \frac{q\omega}{2}\,r_L^2
= \frac{q}{2}\left(\frac{qB}{m}\right)\left(\frac{m v_{\perp}}{qB}\right)^2\]
\[\boxed{\|\mu\|=\frac{m v_{\perp}^2}{2B}}\]
Since the magnetic force does no work on a charged particle and there is no electric field, the total kinetic energy is conserved:
\[K=\frac{1}{2}m v_{\perp}^{2}+\frac{1}{2}m v_{\parallel}^{2} = \text{constant}\]
Rewriting the perpendicular kinetic energy in terms of the magnetic moment,
\[\frac{1}{2}m v_{\perp}^{2} = \mu B\]
Thus the total kinetic energy becomes
\[K = \mu B + \frac{1}{2}m v_{\parallel}^{2}
= \text{constant}\]
Taking the derivative along the magnetic field lines \(s\),
\[\frac{dK}{ds}
= \mu \frac{dB}{ds}
+ m v_{\parallel} \frac{d v_{\parallel}}{ds}=0\]
Rearranging,
\[m v_{\parallel} \frac{d v_{\parallel}}{ds}
= -\mu \frac{dB}{ds}\]
Since \( \dfrac{dv_{\parallel}}{ds}=\dfrac{dv_{\parallel}}{dt} \dfrac{dt}{ds} = \dfrac{1}{v_{\parallel}} \dfrac{dv_{\parallel}}{dt} \), this gives
\[m \frac{d v_{\parallel}}{dt}
= -\mu \frac{dB}{ds}\]
Therefore, the parallel force on the particle is
\[\boxed{F_{\parallel}= -\mu \frac{dB}{ds}}\]
\underline{Note:} The magnetic force only reflects the particle as a result of the changing magnetic field strength (shown by the curvature of the field lines), which creates a component of the magnetic force along the direction of motion of the particle.
\begin{figure}[H]
    \centering
    \includegraphics[width=0.5\textwidth]{Magnetic bottle.png}
\end{figure}
\end{enumerate}
\newpage
\section*{Chapter 30 \& 31.11 - Induction and Inductance}
\begin{enumerate}
    \item Understand the meaning Faraday's law of induction and be able to use it to determine the electromotive force. Explain and apply the equivalence of Faraday's law and the Lorentz force law for motional emfs.\\\\
\textbf{Faraday's Law of Induction}\\
Faraday's law states that if an arbitrary closed loop encloses a region with changing magnetic flux, then an electromotive force (emf) will be induced in the loop. If there happens to be a conducting wire along the loop, then an induced current will exist in the wire.
\[\boxed{\mathcal{E} = -\frac{d\Phi_B}{dt}}\]
The units for magnetic flux $\Phi_B$ are Weber (Wb), where
\[1\, \text{Wb} = 1\, \text{T}\cdot \text{m}^2\]
It can also be written in integral form as:
\[\boxed{\mathcal{E} = \int \vec{E} \cdot d\vec{l} = -\frac{d}{dt}\int \vec{B}\cdot\hat{n}dA}\]
Where
\begin{itemize}
    \item $\mathcal{E}$ is the \textbf{induced emf} (in volts), representing the work done per unit charge by the induced electric field (\(\vec{E}\)) around the faradian loop (in a CCW direction).
    \item $\hat{n}$ is the unit normal vector to the surface bounded by the faradian loop.
    \item $dA$ is the differential area element of that surface.
\end{itemize}
Another way to think about Faraday's law is that there will be circulating electric fields induced in regions of changing magnetic flux.\\\\
The induced electric field has the following properties:
\begin{itemize}
    \item Do not terminate and originate on charges
    \item Form closed loops!
    \item Is present whenever the magnetic flux is changing, even in regions where there are no conductors
    \item Is a non-conservative field, meaning that the work done by the field around a closed loop is non-zero (and thus cannot be described by a potential energy function)
\end{itemize}
In the case where there is a coil of wire with $N$ turns, and the coil is tightly wound so that each turn experiences the same magnetic flux, then the total emf induced in the coil is:
\[\boxed{\mathcal{E}=-N\frac{d\phi_B}{dt} \quad \text{(coil of N turns)}}\]
\underline{Note:} Typically, we don't need to think too hard about the direction of the induced electric field, and just need to consider the magnitude of the induced emf. The direction can be determined using Lenz's law.\\\\
By examining the terms in Faraday's law, we can see that there are a few ways to induce an emf in a loop:
\begin{itemize}
    \item Change the magnitude of the magnetic field $B$ within the coil.
    \item Change either the total area of the coil or the portion of that area that lies within the magnetic field (ex: by expanding the coil or moving the coil in/out of the magnetic field).
    \item Change the orientation of the loop (angle between $\vec{B}$ and $\hat{n}$). For example, by rotating the coil within the magnetic field.
\end{itemize}
\textbf{Motional emf}\\
A motional emf is induced when there is relative motion between a conductor and a magnetic field. This can be understood using either Faraday's law or the Lorentz force law.
\begin{figure}[H]
    \centering
    \includegraphics[width=0.5\textwidth]{Motional emf.jpg}
\end{figure}
\vspace{-1em}
Consider the following circuit formed by a conductor sliding on two stationary rails in the presence of a uniform magnetic field $\vec{B}$ pointing out of the page. By the Lorentz force law, we can find the emf induced in the circuit as follows:
\[\vec{F}_B=q\vec{v}\times\vec{B} \implies F_B = qvB\]
\[F_E= qE\]
The magnetic force pushes positive charges down, creating a potential difference and electric field in the conductor. The electrons stop moving when the magnetic force is balanced by the electric force (Lorentz force equalling zero):
\[F_B = F_E\]
\[qvB = qE \implies E = vB\]
The emf induced in the circuit is then:
\[\boxed{\mathcal{E} = EL = vBL}\]
\newpage\noindent
We can also find the emf using Faraday's law:
\[\mathcal{E} = \int \vec{E} \cdot d\vec{l} = -\frac{d}{dt}\int \vec{B}\cdot\hat{n}dA\]
Only the area is changing as the conductor slides, so:
\[\mathcal{E} = -\frac{d}{dt}(BLx) = -BL\frac{dx}{dt} = -BLv\]
Taking the magnitude, we get the same result as before:
\[\boxed{|\mathcal{E}| = BLv}\]
\textbf{Power and Energy}\\
Considering the same circuit as before, when there is a current $I$ flowing through the circuit, there is a magnetic force slowing down the conductor:
\[\vec{F}_B = I\vec{L} \times \vec{B} \implies F_B = ILB\]
We can find the current using Ohm's law and the emf:
\[I = \frac{\mathcal{E}}{R} = \frac{BLv}{R}\]
\[\implies F_B =\frac{BLv}{R}LB = \frac{B^2L^2v}{R}\]
If we want the conductor to move at a constant velocity then we need to apply an external force equal in magnitude and opposite in direction to the magnetic force:
\[F_{\text{app}} =F_B= \frac{B^2L^2v}{R}\]
The power supplied by this external force is:
\[P_{\text{app}} = F_{\text{app}} v = \frac{B^2L^2v^2}{R}\]
There is no change in kinetic energy of the conductor so all this power must be dissipated as heat in the resistor. We can verify this by calculating the power dissipated in the resistor:
\[P_R = I^2 R = \left(\frac{BLv}{R}\right)^2 R = \frac{B^2L^2v^2}{R}\]
\underline{Note:} Even though there is also current in the upper and lower rails, and thus a magnetic force, these forces cancel out since they point in opposite directions, so they do not contribute to the net force on the moving conductor.\\\\
Also note that as a result of Lenz's law, regardless of the direction of motion of the conductor, there will always be a magnetic force opposing the motion and thus requiring the external force to do \textit{positive} work to keep the conductor moving at a constant velocity. 
\newpage\noindent
\textbf{Eddy Currents}\\
Suppose that we move a solid conducting plate out of a region with a uniform magnetic field $\vec{B}$ pointing into the page. Similar to the motional emf example, the relative motion of the field and the conductor will induce a current and thus a magnetic force that opposes the motion. However, since the conductor solid, the conduction electrons will not follow a single path, but rather will swirl about within the plate, forming eddy currents. However, we can represent the effect of these eddy currents \textit{as if} it were a single induced current loop around the edge of the plate.
\vspace{-1em}
\begin{figure}[H]
    \centering
    \includegraphics[width=0.3\textwidth]{Eddy.png}
\end{figure}
\vspace{-1em}
\begin{multicols}{2}
If the plate in the example was longer, it will enclose 2 regions of changing magnetic flux (one where the flux is increasing as the plate enters the field, and one where the flux is decreasing as the plate exits the field). Thus, there will be two sets of eddy currents induced in the plate, each opposing the change in flux in their respective regions.
\columnbreak 
\begin{figure}[H]
    \centering
    \includegraphics[width=0.4\textwidth]{Eddy 2.png}
\end{figure}
\end{multicols}
\begin{multicols}{2}
From in the motional emf example, we know that the interaction between the magnetic field and the eddy currents produces a magnetic force that opposes the motion of the plate. This causes energy to be lost as heat in the conductor due to the resistance of the material, thereby slowing down the motion of the conducting plate. This effect is used in electromagnetic braking systems.
\columnbreak
\begin{figure}[H]
    \centering
    \includegraphics[width=0.4\textwidth]{Eddy Brake.png}
\end{figure}
\end{multicols}
\newpage\noindent
Sometimes we want to minimize the energy loss from eddy current such as in transformers or electric motors. This can be done by using materials with low conductivity (high resistivity) or by using thin sheets of material with insulating layers in between (laminations) to restrict the flow of eddy currents. Similar to the Hall effect, charges build up on the surfaces of the laminations, creating an electric field that opposes the motion of the charges and thus reducing the eddy currents.
\vspace{-1em}
\begin{figure}[H]
    \centering
    \includegraphics[width=0.4\textwidth]{Lam.png}
\end{figure} 
\textbf{Revisiting Electric Potential}\\
In electrostatics, we were able to define an electric potential $V$ as a scalar field from the electric field produced by \textit{static charges}, however, we cannot define a unique electric potential from the circulating electric fields induced by a changing magnetic flux. This is because the induced electric fields are non-conservative as the work done by the field around a closed loop is non-zero ($\oint \vec{E}\cdot d\vec{l}\neq0$).\\\\
However, it is still valid to consider the induced emf as a type of \textit{potential difference}, similar to a battery that adds or removes energy from charges as they move around a circuit. Thus, equations such as Ohm's law, $V=Ed$, $P=IV$, etc. are applicable in circuits with induced emfs.
    \item Apply Lenz's law to determine the sign of induced currents or electromotive forces.\\\\
\textbf{Lenz's Law}\\
Lenz's law states that the induced current around a loop will flow in a direction that produces a secondary magnetic field that \textit{opposes} the \textit{change} in the magnetic flux in through the loop. Note that Lenz's law still holds even if there is no conducting wire to actually produce the secondary magnetic field, the direction is the same regardless.
\newpage
    \item Define inductance and be able to calculate the self or mutual inductance for various (usually simple) current distributions. Know and use the reciprocity relation for mutual inductances.\\\\
\textbf{Inductance}\\
Inductance measures the ability of a conductor to induce an emf in itself (self-inductance) or in another conductor (mutual inductance) as a result of a changing current. The inductance depends on only geometry and other constants. For an inductor carrying current $I$ with $N$ turns, the \textbf{inductance} is defined as:
\[\boxed{L=N\frac{\Phi_B}{I}}\]
Where $\phi_B$ is the magnetic flux through a single turn of the inductor. The units for inductance are Henry (H), where
\[1\, \text{H} = 1\,\text{T}\cdot \text{m}^2/\text{A}\]
\textbf{Self Inductance of a Solenoid}\\
Consider a solenoid with $N$ turns, length $l$, cross-sectional area $A$, carrying current $I$. The magnetic field inside the solenoid is:
\[B=\mu_0 n I, \quad n=\frac{N}{l}\]
The magnetic flux through a single turn of the solenoid is:
\[\Phi_B = B A = \mu_0 n I A\]
Thus, the inductance of the solenoid is:
\[L = \frac{N \Phi_B}{I} = \frac{N (\mu_0 n I A)}{I} = \mu_0 n^2 A l\]
\[\boxed{L = \mu_0 \frac{N^2}{l} A \quad \text{(solenoid)}}\]
\textbf{Self Induced emf}\\
When the current in an inductor changes, the changing magnetic flux induces an emf in the inductor itself, opposing the change in current (Lenz's law). This is called \textbf{self-induction} and the \textbf{self-induced emf} can be found using Faraday's law:\\\\
We can rewrite emf in terms of inductance:
\[N\Phi_B = LI\]
\[\mathcal{E} = -\frac{d(N\Phi_B)}{dt}=-\frac{d(LI)}{dt}\]
\[\boxed{\mathcal{E} = -L \frac{dI}{dt}}\]
This self-induced emf opposes changes in current, thus an increasing current induces a negative emf (opposing the increase), and this is sometimes called the \textit{back emf}.
\newpage\noindent
We can also define a self-induced potential difference $V_L$ across the inductor (outside the region of changing magnetic flux). For an \textit{ideal} inductor, the magnitude of $V_L$ is equal to the magnitude of the self-induced emf. For \textit{non-ideal} inductors, we can model it as an ideal inductor in series with a resistor (to account for energy loss due to resistance).\\\\
\textbf{Mutual Induction}\\
Suppose we have two inductors (coils of wire) placed close to each other, such that the current in one coil will produce a magnetic flux that passes through the other coil. If the current in coil 1 changes, the changing magnetic flux through coil 2 will induce an emf in coil 2. This is called \textbf{mutual induction} and is defined by the \textbf{mutual inductance} $M_{21}$ of coil 2 with respect to coil 1 as:
\[\boxed{M_{21} = \frac{N_2 \Phi_{B2}}{I_1}}\]
Note that the mutual inductance is \textit{symmetric}, meaning that
\[M_{21} = M_{12}\]
This is very useful in cases where it is easier to find the flux through one coil than the other.\\\\
Similar to self-induction, the emf induced in coil 2 due to a changing current in coil 1:
\[\boxed{\mathcal{E}_2 = -M_{21} \frac{dI_1}{dt}}\]
If the current in coil 2 changes, it will induce an emf in coil 1:
\[\boxed{\mathcal{E}_1 = -M_{12} \frac{dI_2}{dt}}\]
\newpage\noindent
\textbf{Mutual Induction of a Transformer}\\
Suppose there are 2 concentric coils of wire, coil 1 with $N_1$ turns, length $l_1$, and area $A_1$, and coil 2 with $N_2$ turns, length $l_2$, and area $A_2$. Coil 1 carries a current $I_1$, and coil 2 is an open circuit (not carrying any current).
\begin{figure}[H]
    \centering
    \includegraphics[width=0.5\textwidth]{Transformer.jpg}
\end{figure}
The magnetic field within coil 1 is the same as that of a solenoid:
\[B_1 = \mu_0 \frac{N_1}{l_1} I_1\]
The magnetic flux through a single turn of coil 2 is:
\[\Phi_{B2} = B_1 A_2 = \mu_0 \frac{N_1}{l_1} I_1 A_2\]
Thus, the mutual inductance of coil 2 with respect to coil 1 is:
\[M_{21} = \frac{N_2 \Phi_{B2}}{I_1} = \frac{N_2 (\mu_0 \frac{N_1}{l_1} I_1 A_2)}{I_1} = \mu_0 \frac{N_1 N_2 A_2}{l_1} \]
The voltage across coil 2 when the current in coil 1 is changing is then:
\[V_2 = |\mathcal{E}_2| = M_{21} \frac{dI_1}{dt}\]
\[V_2 =\mu_0 \frac{N_1 N_2 A_2}{l_1} \frac{dI_1}{dt}\]
Since coil 2 doesn't induce a magnetic flux in coil 1 (open circuit), the voltage across coil 1 is just its self-induced emf:
\[V_1 = |\mathcal{E}_1|= L_1 \frac{dI_1}{dt}, \quad L_1 = \mu_0 \frac{N_1^2 A_1}{l_1}\]
\[V_1 = \mu_0 \frac{N_1^2 A_1}{l_1} \frac{dI_1}{dt}\]
The ratio of the voltages across the two coils is then:
\[\frac{V_2}{V_1} = \frac{N_2}{N_1}\quad \text{(transformer equation)}\]
    \item Be able to calculate the energy stored in magnetic fields for a given inductor using either the formula involving inductance or the energy density of the magnetic field.\\\\
\textbf{Energy Stored in an Inductor}\\
When current flows through an inductor, energy is stored in the magnetic field created by the current. The energy stored in the magnetic field of an inductor can be found by integrating the power supplied to the inductor as the current increases from 0 to $I$:
\[|\mathcal{E}|=\frac{d\Phi_B}{dt}=L\frac{dI}{dt}\]
\[P = IV\]
This equation is valid because it is derived using potential difference and work. Thus, the power supplied to the inductor is:
\[P = I |\mathcal{E}| = I L \frac{dI}{dt}\]
The energy stored in the inductor is then:
\[P=\frac{dU}{dt} \implies \int dU = \int_0^I I L dI = \frac{1}{2} L I^2\]
\[\boxed{U = \frac{1}{2} L I^2}\]
\textbf{Energy Density of a Magnetic Field}\\
Similar to finding the energy density of an electric field, we can find the energy density of a magnetic field by dividing the total energy stored in the magnetic field by the volume over which the field exists. Assuming that the magnetic field is uniform over that volume.\\\\
For a solenoid with length $l$ and cross-sectional area $A$, its volume is $V = Al$, and it stores an energy of:
\[U = \frac{1}{2} L I^2, \quad L_{\text{solenoid}}=\mu_0 \frac{N^2}{l}A=\mu_0 n^2 A l\]
The energy density of the magnetic field is then:
\[u_B = \frac{U}{V} = \frac{\frac{1}{2} (\mu_0 n^2 A l) I^2}{A l} = \frac{1}{2} \mu_0 n^2 I^2\]
Since the magnetic field inside the solenoid is $B = \mu_0 n I$, we can rewrite the energy density as:
\[\boxed{u_B = \frac{1}{2\mu_0} B^2}\]
Like how we found the general electric field energy density using a parallel plate capacitor. This equation is valid for any magnetic field, not just the uniform field inside a solenoid.\\\\
Using the magnetic field energy density, we can find the total energy stored in any magnetic field by integrating the energy density over the volume of the field.
\newpage
    \item Describe the principle of operation of a generator and a motor.\\\\
By Faraday's law, we can induce an emf in a coil of wire by changing its orientation in a magnetic field. We can solve for the emf induced in a rotating coil as follows:
\begin{figure}[H]
    \centering
    \includegraphics[width=0.8\textwidth]{Motor.jpg}
\end{figure}
As the coil rotates, the magnetic flux through the coil will decrease, thus inducing a ccw current in the coil. First we can define how the orientation of the coil changes by:
\[\hat{n}=\cos(\omega t)\hat{k}+\sin(\omega t)\hat{\imath}\]
The magnetic field is constant and points in the $\hat{k}$ direction $\vec{B}=B\hat{k}$, so the magnetic flux through the coil is:
\[\Phi_B = \vec{B}\cdot\hat{n}A = BA\cos(\omega t)\]
The emf induced in the coil is then:
\[\mathcal{E} = -N\frac{d\Phi_B}{dt}, \quad N=1\]
\[\mathcal{E} = -\frac{d}{dt}(AB\cos(\omega t)) = AB\omega \sin(\omega t)\]
\[\boxed{\mathcal{E} = AB\omega \sin(\omega t)}\]
This is the principle of operation of an AC generator, where mechanical energy is converted into electrical energy by rotating a coil in a magnetic field. It is AC because the direction of the induced current changes from ccw to cw as the coil continues to rotate.\\\\
Similarly, a motor works in reverse by supplying an AC current to a coil in a magnetic field, which produces a torque on the coil that causes it to rotate, converting electrical energy into mechanical energy.
\end{enumerate}
\newpage
\section*{Chapter 32 - Maxwell's Equations}
\makeatletter
\@fleqntrue
\makeatother
\textbf{In integral form:}
\begin{enumerate}
    \item \textbf{Gauss's Law for Electricity}\\
    Relates the electric flux through a closed 3D Gaussian surface to the total charge enclosed within that surface.
    \[\int \vec{E}\cdot\hat{n}\,dA=\frac{Q_{\text{enc}}}{\varepsilon_0}\]
    \item \textbf{Gauss's Law for Magnetism}\\
    Any 3D Gaussian surface will have zero net magnetic flux (no magnetic monopoles).
    \[\int \vec{B}\cdot\hat{n}dA=0\]
    \item \textbf{Ampere-Maxwell Law}\\
    Relates the magnetic circulation around a closed Amperian loop to the enclosed current and changing electric flux through the surface bounded by the loop.
    \[\oint \vec{B}\cdot d\vec{l}=\mu_0 \left[I_{\text{enc}}+\varepsilon_0 \frac{d\Phi_E}{dt}\right],\quad \Phi_E=\int \vec{E}\cdot \hat{n}\,dA\]
    \item \textbf{Faraday's Law of Induction}\\
    Relates the electric circulation around a closed Faradian loop to the changing magnetic flux through the surface bounded by the loop.    
    \[\oint \vec{E}\cdot d\vec{l}=-\frac{d\Phi_B}{dt}, \quad \Phi_B = \int \vec{B}\cdot \hat{n}\,dA\]
    \underline{Note:} For coils with $N$ turns, multiply the flux by $N$. 
\end{enumerate}
\textbf{In differential form:}
\begin{enumerate}
    \item \textbf{Gauss's Law for Electricity}
    \[\vec{\nabla}\cdot\vec{E}=\frac{\rho}{\varepsilon_0}\]
    \item \textbf{Gauss's Law for Magnetism}
    \[\vec{\nabla}\cdot\vec{B}=0\]
    \item \textbf{Ampere-Maxwell Law}
    \[\vec{\nabla}\times\vec{B}=\mu_0 \left(\vec{J}+\varepsilon_0 \frac{\partial \vec{E}}{\partial t}\right), \quad I_{\text{enc}}=\int \vec{J}\cdot \hat{n}dA\]
    \item \textbf{Faraday's Law of Induction}
    \[\vec{\nabla}\times\vec{E}=-\frac{\partial \vec{B}}{\partial t}\]
\end{enumerate}
\makeatletter
\@fleqnfalse
\makeatother
\newpage
\begin{enumerate}
    \item Describe Gauss's law for Magnetism and its applications.\\\\
Gauss's law for magnetism states that the net magnetic flux through any closed surface is zero. This is because there only exists magnetic dipoles, so no matter how much or how little magnetic material is enclosed, the net magnetic flux will be zero. \\\\
Even if you split a magnet into pieces, new north and south poles will form on each piece, so there are still no magnetic monopoles. Thus, even if you enclose a section of a magnet, Gauss's law for magnetism still holds.
    \item Describe the Ampere-Maxwell law, the displacement current, and calculate the magnetic field in a charging or discharging capacitor.\\\\
Consider the following charging capacitor with rectangular plates of area $A$ and separation distance $d$. Since it is not fully charged, there is current $I$ in the wires, inducing a magnetic field.
\begin{figure}[H]
    \centering
    \includegraphics[width=0.5\textwidth]{Charging cap.jpg}
\end{figure}
The magnetic field around the wire can be found using Ampere's law:
\[\oint \vec{B}\cdot d\vec{l}=\mu_0 I_{\text{enc}}\]
\[B(2\pi r) = \mu_0 I \implies B = \frac{\mu_0 I}{2\pi r}\]
However in the region between the plates, there is no current, so $I_{\text{enc}}=0$, and thus Ampere's law would give $B=0$. However, we can still draw a slanted amperian loop that encloses the wire and passes between the plates. Thus, this indicates that Ampere's law is incomplete.\\\\
To fix this, we have the Ampere-Maxwell law  with an additional term called the \textbf{displacement current}, which accounts for the changing electric flux between the plates of the capacitor. The displacement current is defined as:
\[I_d = \varepsilon_0 \frac{d\Phi_E}{dt}, \quad \Phi_E = \int \vec{E}\cdot \hat{n} dA\]
Thus, the Ampere-Maxwell law is:
\[\oint \vec{B}\cdot d\vec{l}=\mu_0 \left(I_{\text{enc}} + \varepsilon_0 \frac{d\Phi_E}{dt}\right)\]
\newpage\noindent
To find the magnetic field between the plates, we first know that the electric field between the plates is:
\[E = \frac{V}{d}, \qquad V = \frac{Q}{C}, \qquad C = \varepsilon_0 \frac{A}{d}\]
\[E = \frac{1}{d}Q\frac{d}{\varepsilon_0 A} = \frac{Q}{\varepsilon_0 A}\]
The electric flux through a circular amperian loop that fully encloses the plates is then:
\[\Phi_E = \int \vec{E}\cdot \hat{n} dA = EA = \frac{Q}{\varepsilon_0 A} A = \frac{Q}{\varepsilon_0}\]
The displacement current is then:
\[I_d = \varepsilon_0 \frac{d\Phi_E}{dt} = \varepsilon_0 \frac{d}{dt}\left(\frac{Q}{\varepsilon_0}\right) = \frac{dQ}{dt} = I\]
\underline{Note:} $Q$ refers to the charge on each plate of the capacitor. The rate at which this charge changes is equal to the current $I$ by conservation of charge.\\\\
Thus, the magnetic field around the plates can be found using the Ampere-Maxwell law:
\[\oint \vec{B}\cdot d\vec{l}=\mu_0 \varepsilon_0 \frac{d\Phi_E}{dt}\]
\[B(2\pi r) = \mu_0 I \implies \boxed{B = \frac{\mu_0 I}{2\pi r}}\]
This is the same magnetic field as outside the plates, showing that the magnetic field is continuous across the capacitor.\\\\
\textbf{Note:} The circulation of the magnetic field caused by a changing electric flux follows the right-hand rule. If the electric flux is increasing, the magnetic field circulates counterclockwise when viewed along the direction of the electric field, and clockwise if the flux is decreasing. Point your thumb in the direction of increasing $\vec{E}$ and your fingers curl in the direction of $\vec{B}$.
\end{enumerate}
\newpage
\section*{Chapter 33 - Electromagnetic Waves}
The key mechanism of an electromagnetic wave is given by Maxwell’s equations: a time-varying electric field induces a magnetic field through Faraday’s law, and a time-varying magnetic field induces an electric field through the Maxwell–Ampere law. This mutual induction allows the electric and magnetic fields to sustain each other and self-propagate through space as a wave.
\begin{figure}[H]
    \centering
    \includegraphics[width=0.6\textwidth]{EM.png}
\end{figure}
\textbf{Properties of electromagnetic waves}
\begin{itemize}
    \item Electromagnetic waves do not require a medium to propagate; they can travel through vacuum.
    \item The $\vec{E}$ and $\vec{B}$ fields are always perpendicular to the direction of wave propagation. Thus, electromagnetic waves are transverse waves, and $\vec{E}\times \vec{B}$ points in the direction of propagation.
    \item The $\vec{E}$ field is always perpendicular to the $\vec{B}$ field.
    \item The magnitudes of the fields are related by $E=cB$, where $c$ is the speed of light in vacuum.
    \item The speed of electromagnetic waves in vacuum is a constant $c=\dfrac{1}{\sqrt{\mu_0 \varepsilon_0}} \approx 3.00\times 10^8\, \text{m/s}$.
    \item The fields always vary sinusoidally and are in phase with each other with the same frequency $f$ and wavelength $\lambda$.    
\end{itemize}
\begin{enumerate}
    \item Test whether a given arrangement of propagating electromagnetic fields satisfies Maxwell's Equations.\\\\
Consider the following electromagnetic wave pulse traveling with velocity $\vec{v}$ in a vacuum, carrying a time-varying electric field $\vec{E}$ and magnetic field $\vec{B}$.
\begin{figure}[H]
    \centering
    \includegraphics[width=0.3\textwidth]{EM pulse.jpg}
\end{figure}
\vspace{-1em}
\textit{Does it satisfy Gauss's Law?}\\
The pulse is in a vacuum without any external charges, thus $Q_{\text{enc}}=0$ and $I_{\text{enc}}=0$.\\\\
\begin{minipage}{0.5\textwidth}
    \includegraphics[width=\linewidth]{EM Gauss.jpg}
\end{minipage}
\hfill
\begin{minipage}{0.4\textwidth}
\[\int \vec{E}\cdot \hat{n} dA = \frac{{Q_{\text{enc}}}}{\varepsilon_0} = 0\]
\[\int \vec{B}\cdot \hat{n} dA = 0\]
\end{minipage}\\\\
Therefore, Gauss's law for electricity and magnetism are satisfied.\\\\
\textit{Does it satisfy Faraday's Law?}
\begin{figure}[H]
    \centering
    \includegraphics[width=0.4\linewidth]{EM Faraday.jpg}
\end{figure}
\vspace{-2em}
\[\Phi_B = \int \vec{B}\cdot \hat{n} dA = Bhx \implies \frac{d\Phi_B}{dt}=-Bhv \quad \text{(flux is decreasing)}\]
\[\int \vec{E}\cdot d\vec{l}=-Eh \quad \text{(CCW is positive, $\vec{E}$ is CW)}\]
Thus,
\[-Eh=-Bhv \implies \boxed{E = Bv, \quad v=c}\]
\newpage\noindent
\textit{Does it satisfy Ampere-Maxwell Law?}
\begin{figure}[H]
    \centering
    \includegraphics[width=0.4\linewidth]{EM Ampere.jpg}
\end{figure}
\vspace{-1em}
There is no enclosed current, so $I_{\text{enc}}=0$.
\[\int \vec{B}\cdot d\vec{l}=\mu_0\varepsilon_0 \frac{d\Phi_E}{dt}\]
\[\Phi_E = \int \vec{E}\cdot \hat{n} dA\ = Ehx \implies \frac{d\Phi_E}{dt}=Ehv \quad \text{(flux is increasing)}\]
\[\int \vec{B}\cdot d\vec{l}=Bh \quad \text{(CCW is positive, $\vec{B}$ is CCW)}\]
Thus,
\[Bh=\mu_0 \varepsilon_0 Ehv \implies \boxed{B = \mu_0 \varepsilon_0 E v, \quad v=c}\]
To satisfy both Faraday's law and the Ampere-Maxwell law, the fields must satisfy:
\[E = Bc, \qquad B = \mu_0 \varepsilon_0 E c\]
\[\implies E = \mu_0 \varepsilon_0 c^2 E\]
Thus,
\[\boxed{c=\frac{1}{\sqrt{\mu_0\varepsilon_0}} \quad \text{(speed of light)}}\]
\underline{Note:} We can also derive the wave equation for electromagnetic waves using the differential form of Maxwell's equations in a charge-free vacuum ($\rho=0$, $\vec{J}=0$).
\[\vec{\nabla}\times\vec{E}=-\frac{d\vec{B}}{dt}\quad \text{(Faraday's law)}\]
\[\vec{\nabla}\times\left(\vec{\nabla}\times\vec{E}\right)=\vec{\nabla}\times\left(-\frac{d\vec{B}}{dt}\right)\]
Right hand side:
\[\vec{\nabla}\times\left(-\frac{d\vec{B}}{dt}\right) = -\frac{d}{dt}\left(\vec{\nabla}\times\vec{B}\right)=-\frac{d}{dt}\left(\mu_0\varepsilon_0\frac{d\vec{E}}{dt}\right)=-\frac{1}{c^2}\frac{d^2 \vec{E}}{dt^2}\]
Left hand side:
\[\vec{\nabla}\times\left(\vec{\nabla}\times\vec{E}\right) = \vec{\nabla}\left(\vec{\nabla}\cdot\vec{E}\right) - \nabla^2 \vec{E}\]
Since there are no enclosed charge, Gauss's law states $\vec{\nabla}\cdot\vec{E}=0$, so:
\[\vec{\nabla}\times\left(\vec{\nabla}\times\vec{E}\right) = -\nabla^2 \vec{E}\]
Thus we have:
\[-\nabla^2 \vec{E} = -\frac{1}{c^2}\frac{d^2 \vec{E}}{dt^2} \implies \boxed{\nabla^2 \vec{E} = \frac{1}{c^2}\frac{d^2 \vec{E}}{dt^2}}\]
    \item Understand the principles of electromagnetic waves and their basic properties, such as phase speed, relation between $\vec{k}$, $\vec{E}$, and $\vec{B}$.\\\\
If we consider a plane electromagnetic wave traveling in the $+(\hat{E}\times\hat{B})$ direction, the electric and magnetic fields can be represented as:
\[\vec{E}(x,t) = E_{m} \sin(kx - \omega t) \hat{E}\]
\[\vec{B}(x,t) = B_{m} \sin(kx - \omega t) \hat{B}\]
Where $k$ is the wave number, $\omega$ is the angular frequency, and $E_{m}$ and $B_{m}$ are the amplitudes of the fields, and $\hat{E}$ and $\hat{B}$ are unit vectors in the directions of the electric and magnetic fields, respectively.\\\\
Recall that the phase velocity of a transverse wave is:
\[v = \frac{\omega}{k}=\frac{\lambda}{T}=\lambda f\]
\[v=c\]
Also recall that the wavenumber is related to the wavelength by:
\[k = \frac{2\pi}{\lambda}\]
The electromagnetic wave vector $\vec{k}$ has magnitude $k$ and points in the direction of wave propagation. Thus,
\[\boxed{\vec{k} = k (\hat{E}\times\hat{B})}\]
\textbf{Relationship between $\vec{E}$ and $\vec{B}$}\\
Recall that the magnitudes of the electric and magnetic fields in an electromagnetic wave are related by $E=cB$.
\newpage
    \item Define the Poynting vector or Poynting flux and be able to use it to calculate the energy carried by electromagnetic fields.\\\\
The \textbf{Poynting vector} $\vec{S}$ describes the \textit{rate} of energy transfer (power) per unit area carried by an electromagnetic wave. We can define it by examining the energy density of the electric and magnetic fields in an electromagnetic wave.\\\\
\textit{What is the rate of energy passing through the face of the box?}
\begin{figure}[H]
    \centering
    \includegraphics[width=0.5\textwidth]{Poynting.jpg}
\end{figure}
\vspace{-1em}
The energy density of the electric and magnetic fields are:
\[u_E = \frac{1}{2} \varepsilon_0 E^2, \quad u_B = \frac{1}{2\mu_0} B^2\]
\[\implies u_{\text{Tot}} = u_E + u_B = \frac{1}{2} \varepsilon_0 E^2 + \frac{1}{2\mu_0} B^2\]
Using the relation $E=cB$, we can rewrite the magnetic energy density in terms of $E$:
\[u_{\text{Tot}}=\frac{1}{2}\varepsilon_0 E^2 + \frac{1}{2\mu_0}\frac{E^2}{c^2}, \quad \frac{1}{c^2}=\mu_0\varepsilon_0\]
\[u_{\text{Tot}}=\frac{1}{2}\varepsilon_0 E^2 + \frac{1}{2\mu_0} E^2 \mu_0 \varepsilon_0 = \varepsilon_0 E^2\]
In time $\Delta t$, the wave travels a distance $\Delta x = c \Delta t$. Thus, the amount of energy passing through area $A$ will be equal to the energy contained in a volume $A \Delta x$. 
\[\Delta U = u_{\text{Tot}} A (c \Delta t)\]
The rate of energy transfer per unit area is then:
\[S= \frac{\Delta U}{A \Delta t} = u_{\text{Tot}} c = \varepsilon_0 c E^2 \quad \left(\frac{\text{energy/time}}{\text{area}}\right)\]
We define the Poynting vector with magnitude equal to this rate of energy transfer per unit area, and direction given by the direction of wave propagation $\hat{S} = \hat{E}\times\hat{B}$. Thus,
\[S = \varepsilon_0 EBc^2 = \frac{1}{\mu_0}EB, \quad E=cB\]
\[\boxed{\vec{S} = \frac{1}{\mu_0} \vec{E} \times \vec{B}}\]
Its units are Watts per square meter (W/m$^2$).
\newpage\noindent
\textit{Example with Poynting vector:} Power transfer into a cylindrical resistor
\begin{figure}[H]
    \centering
    \includegraphics[width=0.7\textwidth]{Poynting ex.jpg}
\end{figure}
\vspace{-1em}
Finding the Poynting flux at the surface of the resistor:
\[\vec{S} = \frac{1}{\mu_0} \vec{E} \times \vec{B}\]
The electric field inside the resistor is given by Ohm's law:
\[V=IR, \quad E = \frac{V}{d} = \frac{IR}{d}\]
\[\implies \vec{E} = \frac{IR}{d}\hat{z}\]
The magnetic field at the surface of the resistor is given by Ampere's law:
\[\int \vec{B} \cdot d\vec{l}= \mu_0 I_{\text{enc}} \implies B = \frac{\mu_0 I}{2 \pi a}\]
\[\implies \vec{B} = \frac{\mu_0 I}{2 \pi a} \hat{\theta}\] 
Thus, the Poynting vector at the surface of the resistor is:
\[\vec{S} = \frac{1}{\mu_0} \left(\frac{IR}{d}\hat{z}\right) \times \left(\frac{\mu_0 I}{2 \pi a} \hat{\theta}\right) = \frac{I^2R}{2 \pi a d} (-\hat{r})\]
\underline{Note:} $\hat{\theta} = \left<-sin\theta,cos\theta\right>$\\\\
The power flowing into the resistor is then:
\[P = \int \vec{S} \cdot \hat{n} dA\]
\[P =-2\pi ad \left(\frac{I^2R}{2 \pi a d}\right)= -I^2 R\]
This makes sense because $P=-I^2R$ is the power entering the resistor via the electromagnetic fields. The resistor dissipates this same power as heat, and in steady state the fields are unchanged, so power in equals power out.
\newpage\noindent
\textbf{Wave Intensity}\\
By substituting the sinusoidal expression for $E = E_m sin(kx-\omega t)$ into the Poynting vector, we can find the rate of energy transfer as a function of time. However, the \textbf{intensity} $I$ (the time-averaged magnitude of the Poynting vector) is usually more useful.
\[I = S_{\text{avg}}\]
\[I =\frac{1}{c\mu_0}\left[E^2\right]_{\text{avg}}=\frac{1}{c\mu_0}\left[E_m^2 sin^2(kx-\omega t)\right]_{\text{avg}}\]
Over a full cycle, the average value of $sin^2$ is $\frac{1}{2}$, thus:
\[I = \frac{E_m^2}{2c\mu_0}\]
\[\boxed{I = \frac{1}{c\mu_0}E^2_{\text{rms}}}\]
Where $E_{\text{rms}} = E_m/\sqrt{2}$ is the root-mean-square value of the electric field.\\\\
\underline{Note:} The energy density of the electric and magnetic fields are exactly equal in an electromagnetic wave.\\\\
If we treat the source of the electromagnetic radiation as a point source that emits radiation isotropically (equally in all directions), the intensity at a distance $r$ from the source is:
\[I = \frac{P}{A} = \frac{P}{4\pi r^2}\]
Where $P$ is the total power emitted by the source.\\\\
\textbf{Radiation Pressure}\\
Assuming a beam of electromagnetic radiation hits the surface of an unconstrained object over time $\Delta t$ and the object absorbs all the radiation, then the object will gain an energy $\Delta U$ and momentum $\Delta p$ from the radiation. Since electromagnetic radiation carries momentum, the momentum transferred to the object is:
\[\Delta p = \frac{\Delta U}{c} \quad \text{(total absorption)}\]
If the radiation is reflected instead of absorbed, the momentum transferred to the object is:
\[\Delta p = \frac{2\Delta U}{c} \quad \text{(total reflection)}\]
The radiation pressure $P$ is defined as the force per unit area exerted on the object by the radiation, and the force is given by:
\[F = \frac{\Delta p}{\Delta t}\]
Note that we can express the energy $\Delta U$ in terms of the intensity $I$ of the radiation and the area $A$ of the object (where the area is perpendicular to the incident ray):
\[\Delta U = I A \Delta t\]
Thus the pressure exerted on the object is:
\[F = \frac{IA}{c} \implies P = \frac{I}{c} \quad \text{(total absorption)}\]
\[F = \frac{2IA}{c} \implies P = \frac{2I}{c} \quad \text{(total reflection)}\]
    \item Qualitatively describe how electromagnetic waves are generated. For an accelerating charge, identify the directions in which one will see the largest/smallest electric fields and describe the polarization of the electric field vector.\\\\
Consider how the electric field lines of a point charge change as it accelerates from rest from point 1 to point 2. As David Griffiths likes to say, “electromagnetic news travels at the speed of light.” For a distant observer, the electric field remains unchanged (light gray) until the disturbance in the field, the kink in the field lines, reaches them at the speed of light, shown by the expanding circle centered at point 2. Observers inside the circle see the new electric field (black) corresponding to the charge at point 2. This propagating kink in the electric field represents the radiated electromagnetic wave produced by the accelerating charge.
\begin{figure}[H]
    \centering
    \includegraphics[width=0.5\textwidth]{EM Generation.jpg}
\end{figure}
\vspace{-1em}
Notice that there is no change in the electric field along the line of motion of the charge, thus there is no electromagnetic radiation emitted in that direction. The maximum change in the electric field and thus the strongest radiation is observed in the plane perpendicular to the acceleration of the charge.
\newpage\noindent
Note that this change in electric field also induces a magnetic field according to Ampere-Maxwell law, with a direction consistent with the magnetic field produced by a moving charge (both out of the page).\\\\
\begin{minipage}{0.4\textwidth}
    \centering
    \includegraphics[width=\textwidth]{EM Wave B.jpg}
\end{minipage}
\hfill
\begin{minipage}{0.5\textwidth}
By the Ampere-Maxwell law:
\[\int \vec{B}\cdot d\vec{l} = \mu_0 \varepsilon_0 \frac{d\Phi_E}{dt}\]
There will be a magnetic field pointing out of the page just like in the case with the electromagnetic wave pulse before.
\end{minipage}\\\\
\textbf{Polarization}\\
Polarization is the property of light that describes the direction in which its electric field oscillates. Unpolarized light has electric fields oscillating randomly in all perpendicular directions, while polarized light has oscillations in a specific direction. Polarizing materials allow only one component of the electric field to pass, producing polarized light and reducing the intensity, basically filtering the light.
\end{enumerate}
\section*{Chapters 27 \& 30.9 - Circuits}
This section with primarily cover steady state and time varying circuits with resistors, capacitors, and inductors. Oscillatory and AC circuits will be covered in the next section.\\\\
\textbf{Circuit Components}
\begin{itemize}
    \item \textbf{emf device}:\\
    \begin{minipage}{0.2\textwidth}
    \includegraphics[width=\linewidth]{Battery.jpg}
    \end{minipage}
    \hfill
    \begin{minipage}{0.75\textwidth}
    A "charge pump" that does work on charges by maintaining a potential difference (voltage) between its terminals. Examples include batteries and generators.\\\\
    \underline{Note:} A non-ideal emf device has internal resistance and is modeled as an ideal emf device in series with a resistor.
    \end{minipage}    
    \item \textbf{Resistor}:\\
    \begin{minipage}{0.2\textwidth}
    \includegraphics[width=\linewidth]{Resistor.jpg}
    \end{minipage}
    \hfill
    \begin{minipage}{0.75\textwidth}
    Opposes the flow of current, converting electrical energy into heat. The voltage across a resistor is proportional to the current through it, following Ohm's law: $V=IR$.
    \end{minipage}\\\\
    \textit{Resistors in series}\\
    Two resistors are in series if the \textit{same current} flows through both resistors.
    \[\boxed{R_{\text{eq}} = R_1 + R_2 + \cdots \quad \text{(series resistors)}}\]
\newpage\noindent
    \textit{Resistors in parallel}\\
    Two resistors are in parallel if they have the \textit{same voltage} across both resistors.
    \[\boxed{\frac{1}{R_{\text{eq}}} = \frac{1}{R_1} + \frac{1}{R_2} + \cdots \quad \text{(parallel resistors)}}\]
    \item \textbf{Capacitor}:\\
    \begin{minipage}{0.2\textwidth}
    \includegraphics[width=\linewidth]{Capacitor.jpg}
    \end{minipage}
    \hfill
    \begin{minipage}{0.75\textwidth}
    Stores electrical energy in an electric field between its plates. The voltage across a capacitor is proportional to the charge stored on its plates: $V=\frac{Q}{C}$.
    \end{minipage}\\\\
    \textit{Capacitors in series}\\
    Two capacitors are in series if they have the \textit{same charge} on both capacitors.
    \[\boxed{\frac{1}{C_{\text{eq}}} = \frac{1}{C_1} + \frac{1}{C_2} + \cdots \quad \text{(series capacitors)}}\]
    \textit{Capacitors in parallel}\\
    Two capacitors are in parallel if they have the \textit{same voltage} across both capacitors.
    \[\boxed{C_{\text{eq}} = C_1 + C_2 + \cdots \quad \text{(parallel capacitors)}}\]      
    \item \textbf{Inductor}:\\
    \begin{minipage}{0.2\textwidth}
    \includegraphics[width=\linewidth]{Inductor.jpg}
    \end{minipage}
    \hfill
    \begin{minipage}{0.75\textwidth}
    Stores energy in a magnetic field created by the current flowing through it. The voltage across an inductor is proportional to the rate of change of current through it: $V=-L\frac{dI}{dt}$.
    \end{minipage}\\\\
    \textit{Inductors in Series}\\
    Inductors are in series when the \textit{same current} flows through each inductor.
    \[\boxed{L_{\text{eq}} = L_1 + L_2 + \cdots \quad \text{(series inductors)}}\]
    \textit{Inductors in Parallel}\\
    Inductors are in parallel when they share the \textit{same voltage} across their terminals. 
    \[\boxed{\frac{1}{L_{\text{eq}}} = \frac{1}{L_1} + \frac{1}{L_2} + \cdots \quad \text{(parallel inductors)}}\]    
    \underline{Note:} The equivalent inductance gets more complicated when mutual inductance is involved. There, it adds or subtracts based on the orientation of the inductors.    
\end{itemize}
\textbf{Grounding a circuit}\\
    \begin{minipage}{0.2\textwidth}
    \includegraphics[width=\linewidth]{Ground.jpg}
    \end{minipage}
    \hfill
    \begin{minipage}{0.75\textwidth}
Grounding a circuit means connecting a point in the circuit to the Earth, which is considered to be at zero potential. This provides a reference point for measuring voltages in the circuit. Note that even though voltage is zero, this doesn't mean that current will be zero. In fact, current can flow to or from ground depending on the circuit configuration.
    \end{minipage}\\\\
\textbf{Ammeter}\\
An ammeter is used to measure the current flowing through a circuit by putting it in series with the circuit element whose current is to be measured. An ideal ammeter has zero resistance so that it does not affect the current it is measuring.\\\\
Note that a non-ideal ammeter increases the total resistance of the circuit, which decreases the current flowing through the circuit according to Ohm's law.\\\\
\textbf{Voltmeter}\\
A voltmeter is used to measure the voltage across a circuit element by connecting it in parallel with that element. An ideal voltmeter has infinite resistance so that it does not draw any current from the circuit.\\\\
Note that a non-ideal voltmeter decreases the total resistance of the circuit branch it is connected to (it is in parallel), which increases the current flowing through that branch according to Ohm's law.
\begin{enumerate}
    \item Know and be able to apply the Voltage Loop Rule and the Current Node Rule (conservation of energy and conservation of charge, respectively) to analyze a circuit. A good example is to prove the rules for adding resistors in series or in parallel.\\\\
\textbf{Kirchohoff's Loop Rule (Conservation of Energy)}\\
The algebraic sum of all the potential differences (voltages) around any closed loop in a circuit must equal zero (no work is done in a complete circuit).\\\\
We can start at any point in the loop and add the voltages we encounter as we traverse the loop.\\\\
\underline{Resistance rule}\\
Moving across a resistor in the direction of the current results in a voltage drop ($-\Delta V$), moving against the current results in a voltage rise ($+\Delta V$).\\\\
\underline{emf rule}\\
When we from the negative end to the positive end of an emf device, we have a voltage rise ($+\mathcal{E}$). Moving from the positive end to the negative end results in a voltage drop ($-\mathcal{E}$).
\newpage\noindent
Consider the following example:
\vspace{-1em}
\begin{figure}[H]
    \centering
    \includegraphics[width=0.5\textwidth]{Circuit.jpg}
\end{figure}
\vspace{-1em}
The emf device is non-ideal so it has an internal resistance $r$. We can \textit{potential difference between two points} $a$ and $b$ in the circuit by traversing a loop from $a$ to $b$ and summing the voltages encountered:
\[(a \to b):\quad V_a + \mathcal{E} - Ir = V_b\]
The current can be found using the loop rule:
\[\mathcal{E}-Ir-IR = 0 \implies I = \frac{\mathcal{E}}{R+r}\]
Thus, the potential difference between points $a$ and $b$ is:
\[V_b-V_a= \mathcal{E}-Ir = \mathcal{E}-\frac{\mathcal{E}}{R+r}r\]
\[\boxed{V_b - V_a = \mathcal{E}\left(\frac{R}{R+r}\right)}\]
\underline{Note:} The direction and path taken to traverse the loop does not matter, as long as you are consistent with the signs of the voltages encountered.\\\\
We can see how the potential differences change when moving through the circuit:
\begin{figure}[H]
    \centering
    \includegraphics[width=0.9\textwidth]{Circuit voltages.jpg}
\end{figure}
The voltage drops are linear across resistors since $V=IR$, $I$ is constant, and $R= \rho \frac{L}{A}$ is proportional to length $L$.\\\\
\underline{Note:} In a circuit with multiple emf devices, we can arbitrarily assign a direction for the current. If we find that the current is negative, it simply means that the actual current flows in the opposite direction.\\\\
\textbf{Kirchhoff's Junction Rule (Conservation of Charge)}\\
At any junction (node) in an electrical circuit, the sum of currents flowing into that junction must equal the sum of currents flowing out of that junction.\\\\
Current flowing into the junction is positive, current flowing out of the junction is negative.\\\\
\underline{Note:} All the current in the circuit must equal the current from the emf device, as charge is conserved. Thus, if there are multiple branches in a circuit, the sum of the currents in each branch must equal the total current supplied by the emf device ($I_{\text{total}}=I_1 + I_2 + \cdots$). This is useful for circuits with resistors in parallel.
    \item Know the relation between current and voltage for the basic circuit elements of resistor, capacitor, and inductor.
    \item Write differential equations for RL and RC circuits. Understand qualitatively the behavior of resistors, inductors, and capacitors in DC and transient regimes. Be able to describe how electromagnetic energy is stored, supplied, and dissipated in such circuits.\\\\
\textbf{RC Circuits}\\
In the following circuit, the capacitor is initially uncharged, and it begins to charge when the switch is closed at $a$. When the switch is moved to $b$, the capacitor then discharges through the resistor.
\begin{figure}[H]
    \centering
    \includegraphics[width=0.4\textwidth]{RC.png}
\end{figure}
Using the loop rule, we can find the differential equations for the charging and discharging of the capacitor.\\\\
\textit{Charging the capacitor:} (CW from the negative terminal of the battery)
\[\mathcal{E}-IR-\frac{q}{C} = 0, \quad I = \frac{dq}{dt}\]
\underline{Note:} When moving from the negative plate to the positive plate of a capacitor, there is a voltage rise. When moving from the positive plate to the negative plate, there is a voltage drop.
\[\boxed{R\frac{dq}{dt} + \frac{q}{C} = \mathcal{E} \quad \text{(charging equation)}}\]
The solution to this differential equation is:
\[q(t) = C\mathcal{E}\left(1-e^{-\frac{t}{RC}}\right)\]
\[I(t) = \frac{\mathcal{E}}{R}e^{-\frac{t}{RC}}\]
We can define $RC$ as the \textbf{capacitive time constant} of the circuit.
\[\boxed{\tau_C = RC \quad \text{(capacitive time constant)}}\]
\textit{Discharging the capacitor} (CW from the left side of the resistor)
\[-IR - \frac{q}{C} = 0, \quad I = \frac{dq}{dt}\]
\[\boxed{R\frac{dq}{dt} + \frac{q}{C} = 0 \quad \text{(discharging equation)}}\]
The solution to this differential equation is:
\[q(t) = q_0 e^{-\frac{t}{RC}}\]
\[I(t) = -\frac{q_0}{RC} e^{-\frac{t}{RC}}\]
Note that $q_0$ is the initial charge on the capacitor as it begins to discharge ($q_0 = CV_0$). The negative sign in the current indicates that the current flows in the opposite direction during discharge.\\\\
\textbf{RL Circuits}\\
In the following circuit, the inductor initially has no current flowing through it. When the switch is closed at $a$, the current begins to increase through the inductor. When the switch is moved to $b$, the current then decreases as the energy stored in the magnetic field of the inductor is released through the resistor.\\\\
\underline{Note:} The current will approach $\mathcal{E}/R$ as $t \to \infty$ when charging the inductor as steady state is reached (no change in current, so inductor acts like a short circuit). When discharging, the current will approach $0$ as $t \to \infty$.
\begin{figure}[H]
    \centering
    \includegraphics[width=0.4\textwidth]{RL.png}
\end{figure}
\vspace{-1em}
Using the loop rule, we can find the differential equations for the charging and discharging of the inductor.
\begin{figure}[H]
    \centering
    \includegraphics[width=0.7\textwidth]{RL charging.jpg}
\end{figure}
\textit{Charging the inductor:} (CW from the negative terminal of the battery)
\[\mathcal{E}-IR-L\frac{dI}{dt} = 0\]
\[\boxed{L\frac{dI}{dt} + IR = \mathcal{E} \quad \text{(charging equation)}}\]
The solution to this differential equation is:
\[I(t) = \frac{\mathcal{E}}{R}\left(1-e^{-\frac{R}{L}t}\right)\]
We can define $L/R$ as the \textbf{inductive time constant} of the circuit.
\[\boxed{\tau_L = \frac{L}{R} \quad \text{(inductive time constant)}}\]
\textit{Discharging the inductor:} (CW from the left side of the resistor)
\[-IR - L\frac{dI}{dt} = 0\]
\underline{Note:} The voltage across the inductor \textit{in the direction of the current} is always $V_L = -L\frac{dI}{dt}$. In the charging case, the current is increasing, so $V_L$ is negative (voltage drop). In the discharging case, the current is decreasing, so $V_L$ is positive (voltage rise). We will keep the negative sign in the equation to account for the negative $dI/dt$ during discharge.
\[\boxed{L\frac{dI}{dt} + IR = 0 \quad \text{(discharging equation)}}\]
The solution to this differential equation is:
\[I(t) = I_0 e^{-\frac{R}{L}t}\]
Note that $I_0$ is the initial current through the inductor as it begins to discharge.\\\\
\textbf{Energy and power}\\
In DC and transient circuits, energy supplied by the emf device is stored in the
electric fields of capacitors and the magnetic fields of inductors, and is dissipated as thermal energy in resistors. Therefore, the change of the total energy stored must be equal to the net power supplied by the emf device minus the power dissipated in resistors.
\[\frac{dU_{\text{Tot}}}{dt} = P_{\text{emf}} - P_R\]
Where:
\begin{itemize}
    \item $U_{\text{Tot}} = U_C + U_L = \frac{1}{2}CV_C^2 + \frac{1}{2}LI^2, \quad V_C= \frac{Q}{C}$
    \item $P_{\text{emf}} = \mathcal{E}I$
    \item $P_R = I^2 R$
\end{itemize}
    \item Solve the differential equations for such circuits and identify characteristic timescales. If initial conditions are given, be able to find solutions that satisfy those initial conditions.\\\\
The time constant generally describe how quickly a circuit responds to changes. For RC circuits, the time constant $\tau_C = RC$ describes how quickly a capacitor charges or discharges. For RL circuits, the time constant $\tau_L = \frac{L}{R}$ describes how quickly the current through an inductor builds up or decays.\\\\
\textbf{RC}
\begin{itemize}
    \item Charging: After a time $t=\tau_C$, the capacitor is about 63\% charged (where $Q_{\text{max}}=C\mathcal{E}$). After $t=5\tau_C$, the capacitor is 99\% charged.
    \item Discharging: After a time $t=\tau_C$, the charge on the capacitor falls to about 37\% of its initial value ($Q_0$). After $t=5\tau_C$, the capacitor is fully discharged.
\end{itemize}
\textbf{RL}
\begin{itemize}
    \item Charging: After a time $t=\tau_L$, the current through the inductor is about 63\% of its maximum value ($I_{\text{max}}=\frac{\mathcal{E}}{R}$). After $t=5\tau_L$, the current is effectively at its maximum value.
    \item Discharging: After a time $t=\tau_L$, the current through the inductor falls to about 37\% of its initial value ($I_0$). After $t=5\tau_L$, the current is effectively zero.
\end{itemize}
\end{enumerate}
\section*{Chapter 31 - Oscillatory \& AC behavior}
Notice that the differential equation for RL circuits are second order differential equations in terms of charge $q$ (since $I=dq/dt$). Thus, combining inductors and capacitors in circuits can lead to oscillatory behavior, similar to mass-spring systems.\\\\
Consider the following LC circuit where the capacitor starts fully charged. As the circuit oscillates, energy is exchanged between the electric field of the capacitor and the magnetic field of the inductor, but the total amount is conserved.
\begin{figure}[H]
    \centering
    \includegraphics[width=1\textwidth]{LC.png}
\end{figure}
\begin{enumerate}
    \item Write differential equations for LC and RLC circuits. Understand qualitatively the behavior of resistors, inductors, and capacitors at high and low frequencies. Be able to describe how electromagnetic energy is stored, supplied, and dissipated in such circuits.\\\\
\textbf{LC Circuits}\\
Using the loop rule, we can find the differential equation for the LC circuit. (CW from the positive plate of the capacitor)
\[-\frac{q}{C} - L\frac{dI}{dt} = 0, \quad I = \frac{dq}{dt}\]
\[\boxed{L\frac{d^2 q}{dt^2} + \frac{q}{C} = 0 \quad \text{(LC oscillations)}}\]
We can also derive this differential equation using energy conservation, since the total energy in the LC circuit is conserved:
\[U_{Tot}=U_B+U_E = \frac{1}{2}LI^2 + \frac{1}{2}\frac{q^2}{C} = \text{(Constant)}\]
\[\frac{dU_{Tot}}{dt} = \frac{d}{dt}\left(\frac{1}{2}LI^2 + \frac{1}{2}\frac{q^2}{C}\right) = 0\]
\[\implies LI \frac{dI}{dt} + \frac{q}{C} \frac{dq}{dt} = 0, \quad I = \frac{dq}{dt}\]
Thus,
\[L\frac{d^2q}{dt^2} + \frac{q}{C} = 0\]
\textbf{RLC Circuits}\\
Consider the following RLC circuit where each component is in series with a time varying emf device.
\begin{figure}[H]
    \centering
    \includegraphics[width=0.6\textwidth]{RLC.jpg}
\end{figure}
\vspace{-1em}
Using the loop rule, we can find the differential equation for the RLC circuit. (CW from the negative terminal of the emf device)
\[V(t) - IR - L\frac{dI}{dt} - \frac{q}{C} = 0, \quad I = \frac{dq}{dt}\]
\[\boxed{L\frac{d^2 q}{dt^2} + R\frac{dq}{dt} + \frac{q}{C} = V(t) \quad \text{(RLC equation)}}\]
If we differentiate this equation with respect to time, we can also express it in terms of current $I$:
\[L\frac{d^2 I}{dt^2} + R\frac{dI}{dt} + \frac{I}{C} = \frac{dV(t)}{dt}\]
Notice that by introducing a driving voltage ($V(t)$), the differential equation becomes non-homogeneous.
\newpage\noindent
We can also derive this differential equation using energy, however now the total energy is not conserved due to the resistor dissipating energy as heat and the emf device supplying energy to the circuit. Thus, the \textit{change} in total energy stored must equal the net power supplied by the emf device minus the power dissipated in the resistor.
\[P_{\text{in}} = IV(t), \quad P_{\text{out}}=I^2R\]
\[\implies P_{\text{Tot}} = IV(t) - I^2R\]
Like before, the total energy in the circuit is stored in the capacitor and inductor, therefore
\[\frac{d}{dt}\left(\frac{1}{2}LI^2 + \frac{1}{2}\frac{q^2}{C}\right) = IV(t) - I^2R\]
\[\implies LI \frac{dI}{dt} + \frac{q}{C} \frac{dq}{dt} = IV(t) - I^2R, \quad I = \frac{dq}{dt}\]
Rearranging gives:
\[L\frac{d^2 q}{dt^2} + R\frac{dq}{dt} + \frac{q}{C} = V(t)\]
\textbf{The Electrical–Mechanical Analogy}\\
Notice that the differential equations for LC and RLC circuits are mathematically analogous to the differential equations for mass-spring systems where
\begin{itemize}
    \item The inductance $L$ is the inertia of the system, analogous to mass $m$.
    \item The resistance $R$ is the damping in the system, analogous to the damping coefficient.
    \item The inverse capacitance $\frac{1}{C}$ is the restoring force of the system, analogous to the spring constant $k$.
    \item The charge $q$ is analogous to the displacement $x$ of the mass.
    \item The driving voltage $V(t)$ is analogous to the driving force $F(t)$ on the mass.
\end{itemize}
    \item Solve the differential equations for such circuits and identify characteristic frequencies and resonances. If initial conditions are given, be able to find solutions that satisfy those initial conditions.\\\\
To solve the differential equations for driven and undriven LC and RLC circuits we must first understand how these circuits behave at its \textit{natural frequency}.\\\\
\textbf{Undriven LC Circuit}\\
Differential equation:
\[L\frac{d^2 q}{dt^2} + \frac{q}{C} = 0\]
Solving the characteristic equation for this second order homogeneous differential equation for the general solution:
\[Lr^2 + \frac{1}{C} = 0 \implies r^2 = -\frac{1}{LC}\]
\[r = \pm  \frac{1}{\sqrt{LC}} i, \quad \text{let } \omega = \frac{1}{\sqrt{LC}}\]
\[ \implies q(t) = c_1 \cos(\omega t) + c_2 \sin(\omega t)\]
Since $c_1$ and $c_2$ are two arbitrary constants that together form a vector in a two-dimensional solution space, they may be parameterized by a magnitude and an angle, allowing the general solution to be written as a single cosine with a phase shift.
\[Q_m \cos(\omega t + \phi) = Q_m \cos\phi \cos(\omega t) - Q_m \sin\phi \sin(\omega t)\]
Thus the general solution for the undriven LC circuit is:
\[\boxed{q(t) = Q_m \cos(\omega t + \phi)}\]
\[\boxed{I(t) = \frac{dq}{dt} = -\omega Q_m \sin(\omega t + \phi)}\]
Where $\omega$ is the natural frequency of the circuit:
\[\boxed{\omega = \frac{1}{\sqrt{LC}} \quad \text{(natural frequency of LC circuit)}}\]
\textbf{Undriven RLC Circuit}\\
Differential equation:
\[L\frac{d^2 q}{dt^2} + R\frac{dq}{dt} + \frac{q}{C} = 0\]
Solving the characteristic equation for this second order homogeneous differential equation for the general solution:
\[Lr^2 + Rr + \frac{1}{C} = 0\]
Using the quadratic formula:
\[r = \frac{-R \pm \sqrt{R^2 - \frac{4L}{C}}}{2L}\]
There are three cases to consider based on the discriminant:
\begin{itemize}
    \item \textbf{Overdamped} ($R^2 > 4\frac{L}{C}$): Two distinct real roots, leading to an exponentially decaying solution without oscillations.
    \item \textbf{Critically damped} ($R^2 = 4\frac{L}{C}$): Repeated roots, leading to the fastest return to equilibrium without oscillations.
    \item \textbf{Underdamped} ($R^2 < 4\frac{L}{C}$): Complex roots, leading to oscillatory behavior with an exponentially decaying amplitude.
\end{itemize}
Looking at the underdamped case, we can express the roots as:
\[r = \frac{-R}{2L} \pm i \sqrt{\frac{1}{LC} - \frac{R^2}{4L^2}}\]
\[\implies q(t) = e^{-\frac{R}{2L}t} \left( c_1 \cos\left(\sqrt{\frac{1}{LC} - \frac{R^2}{4L^2}} \,t\right) + c_2 \sin\left(\sqrt{\frac{1}{LC} - \frac{R^2}{4L^2}} \,t\right) \right)\]
Like before, we can express the general solution as a single cosine with a phase shift.
\[\boxed{q(t) = Q_m e^{-\frac{R}{2L}t} \cos\left(\omega' t + \phi\right)}\]
Where $\omega'$ is the damped natural frequency of the circuit:
\[\boxed{\omega' = \sqrt{\frac{1}{LC} - \frac{R^2}{4L^2}} \quad \text{(damped natural frequency of RLC circuit)}}\]
Note that as we set $R = 0$ we recover the natural frequency and general solution for the undriven LC circuit.
    \item Know or be able to derive the complex impedances for resistors, capacitors, and inductors. Know or derive the rules for combining parallel and series impedances. Use those impedances to describe the relations among currents and voltages in circuits, both in magnitude and phase.\\\\
Suppose that we have a sinusoidally varying emf device that drives an LC or RLC circuit: 
\[V(t) = V_m \sin(\omega_d t)\]
Where $\omega_d$ is the \textbf{driving frequency} of the emf device.\\\\
Under \textit{forced oscillation} from the driving emf device, charge, current, and voltages will always occur at the driving frequency $\omega_d$ instead of the natural frequency $\omega$ or damped natural frequency $\omega'$. However, there may be a phase difference between the driving voltage and the response of the circuit.\\\\
Because of this property of forced oscillations, where all quantities oscillate at the same frequency, we can use \textbf{complex impedances} to analyze the circuit. Physically, impedance describes how the circuit responds to a driving voltage at a specified driving frequency, giving the amplitude and phase of the steady-state response.\\\\
\underline{Note:} In steady state, transient behavior has decayed, and the circuit oscillates only at the driving frequency. Thus in cases such as RLC circuits, we can ignore the homogeneous solution (representing the transient behavior) and focus only on the particular solution (representing the steady-state forced oscillation).\\\\
\textbf{Reactance and Impedance}\\
The \textbf{reactance} of a circuit element describes its opposition to alternating current (AC) due to energy storage in electric or magnetic fields. Unlike resistance, reactance depends on the driving frequency and causes the voltage and current to be out of phase. Reactance has units of ohms ($\Omega$) and may be positive or negative depending on the type of circuit element.\\\\
The \textbf{impedance} of a circuit is a complex quantity that combines resistance and reactance, representing the total opposition to AC current flow. It is defined as
\[Z = R + iX\]
where $R$ is the resistance and $X$ is the reactance.\\\\
For a resistor,
\[\boxed{Z_R = R}\]
For an inductor,
\[\boxed{Z_L = i\omega_d L \implies X_L = \omega_d L}\]
For a capacitor,
\[\boxed{Z_C = -\frac{i}{\omega_d C} \implies X_C = -\frac{1}{\omega_d C}}\]
The magnitude of the impedance satisfies $|Z| = \frac{|V|}{|I|}$. Thus, a higher impedance results in a lower current for a given voltage amplitude.\\\\
This allows us to understand the behavior of capacitors and inductors at different frequencies in AC circuits:
\begin{itemize}
    \item At low frequencies
    \begin{itemize}
        \item Capacitive reactance is large ($X_C = \frac{1}{\omega_d C}$): the capacitor behaves like an open circuit, since charge has time to accumulate and oppose current flow.
        \item Inductive reactance is small ($X_L = \omega_d L$): the inductor behaves like a short circuit, since slowly varying currents induce only small opposing emf \\($\mathcal{E}=-L \frac{dI}{dt}$).
    \end{itemize}
    \item At high frequencies
    \begin{itemize}
        \item Capacitive reactance is small: the capacitor behaves like a short circuit due to rapid charge and discharge with little voltage buildup.
        \item Inductive reactance is large: the inductor behaves like an open circuit, since rapidly changing currents induce large opposing emf.
    \end{itemize}
\end{itemize}
\newpage\noindent
\textbf{Phasors}\\
We can graphically represent the relationship between voltages and currents in AC circuits using vectors in the complex plane called \textbf{phasors}. Phasors rotate counterclockwise at the driving frequency $\omega_d$, with magnitude equal to the amplitude of the alternating quantity (voltage or current) and angle equal to its phase at a time $t$. Its projection onto the real or imaginary axis gives the instantaneous value of the quantity (depending on whether sine or cosine is used).\\\\
Let's look at the phase relationships for each circuit element when driven by a sinusoidal voltage source:
\[V(t) = V_m \sin(\omega_d t)\]
\begin{itemize}
    \item \textbf{Resistive Load}: in a circuit with only a resistor, the voltage and current are in phase $\phi =0$. We see this by Ohm's law:
    \[I(t) = \frac{V_R}{R} = \frac{V_R}{R} \sin(\omega_d t)\]
    \[\implies V_R = I_R R \quad \text{(Relating the amplitudes)}\]
    \vspace{-2em}
    \begin{figure}[H]
        \centering
        \includegraphics[width=0.6\textwidth]{Resistor phasor.png}
    \end{figure}
    \item \textbf{Capacitive Load}: in a circuit with only a capacitor, the current leads the voltage by 90 degrees. Using the loop rule:
    \[v_C = V_C \sin(\omega_d t)\]
    \[v_C = \frac{q_C}{C} \implies q_C = C V_C \sin(\omega_d t)\]
    \[I(t) = \frac{dq_C}{dt} = \omega_d C V_C \cos(\omega_d t) = \omega_d C V_C \sin\left(\omega_d t + 90^\circ\right)\]
    \[I(t) = \frac{V_C}{X_C} \sin\left(\omega_d t + 90^\circ\right), \quad X_C = \frac{1}{\omega_d C}\]
    \[\implies V_C = I_C X_C \quad \text{(Relating the amplitudes)}\]
    \vspace{-2em}
    \begin{figure}[H]
        \centering
        \includegraphics[width=0.6\textwidth]{Capacitor phasor.png}
    \end{figure}
    \item \textbf{Inductive Load}: in a circuit with only an inductor, the current lags the voltage by 90 degrees. Using the loop rule:
    \[v_L = V_L \sin(\omega_d t)\]
    \[v_L = L \frac{dI}{dt} \implies \frac{dI}{dt} = \frac{V_L}{L} \sin(\omega_d t)\]
    \[I(t) = \int \frac{V_L}{L} \sin(\omega_d t)\, dt = -\frac{V_L}{\omega_d L} \cos(\omega_d t) = \frac{V_L}{\omega_d L} \sin\left(\omega_d t - 90^\circ\right)\]
    \[I(t) = \frac{V_L}{X_L} \sin\left(\omega_d t - 90^\circ\right), \quad X_L = \omega_d L\]
    \[\implies V_L = I_L X_L \quad \text{(Relating the amplitudes)}\]
    \vspace{-2em}
    \begin{figure}[H]
        \centering
        \includegraphics[width=0.6\textwidth]{Inductor phasor.png}
    \end{figure}
\end{itemize}
Note that even though we found the phase relationships using isolated circuit elements, these relationships hold true when the elements are combined in circuits as well. 
\newpage\noindent
Putting it all together in a RLC circuit:
\vspace{-1em}
\begin{figure}[H]
    \centering
    \includegraphics[width=0.6\textwidth]{RLC phasor.png}
\end{figure}
\begin{minipage}{0.4\textwidth}
\centering
\includegraphics[width=\textwidth]{Loop rule.png}
\end{minipage}
\hfill
\begin{minipage}{0.5\textwidth}
By the loop rule, the magnitude of the driving voltage $\mathcal{E}$ is the resultant of the voltages for each circuit element:
\[|\mathcal{E}| = \sqrt{\left(V_R\right)^2 + \left(V_L - V_C\right)^2}\]
\[\mathcal{E} = \sqrt{\left(IR\right)^2 + \left(IX_L - IX_C\right)^2}\]
\end{minipage}\\\\
Thus,
\[I = \frac{\mathcal{E}}{\sqrt{\left(IR\right)^2 + \left(IX_L - IX_C\right)^2}} = \frac{\mathcal{E}}{Z}\]
The angle $\phi$ between the current phasor and the voltage phasor is given by:
\[\tan\phi = \frac{V_L - V_C}{V_R} = \frac{IX_L - IX_C}{IR} = \frac{X_L - X_C}{R}\]
Note that if $X_L > X_C$, then $\phi > 0$ and the current lags behind the voltage (inductive behavior). If $X_C > X_L$, then $\phi < 0$ and the current leads the voltage (capacitive behavior). If $X_L = X_C$, then $\phi = 0$ and the current and voltage are in phase (resonance).\\\\
This makes sense!! In a capacitive circuit, current must first charge the capacitor before a voltage can build up across it, so current leads voltage. In an inductive circuit, the inductor resists changes in current, so a voltage must first be applied across it to change the current, meaning voltage leads current.\\\\
We also see that when the reactances are equal ($X_L = X_C$), the impedance is purely resistive ($Z=R$) and the circuit is in resonance. This cancellation only occurs when the driving frequency matches the natural frequency of the circuit ($\omega_d = \omega = \frac{1}{\sqrt{LC}}$). At resonance, the current is maximized because the source is not fighting against any reactance.\\\\
\textbf{Main takeaway:} Phasors allow us to understand the behavior of voltage and current across circuit elements in AC circuits. This is what gives rise to reactance and impedance, and why they let us analyze AC circuits algebraically using vector arithmetic instead of solving differential equations.
\newpage\noindent
\textbf{Solving AC Driven Circuits}\\
Finally let us discuss how to solve circuits using impedances. The rules for combining impedances in series and parallel are the same as for resistors:
\begin{itemize}
    \item \textbf{Series impedances}:
    \[\boxed{Z_{\text{eq}} = Z_1 + Z_2 + \cdots \quad \text{(series impedances)}}\]
    \item \textbf{Parallel impedances}:
    \[\boxed{\frac{1}{Z_{\text{eq}}} = \frac{1}{Z_1} + \frac{1}{Z_2} + \cdots \quad \text{(parallel impedances)}}\]
\end{itemize}
We can then use Ohm's law in phasor form to relate voltages and currents in the circuit $\widetilde{V} = Z \widetilde{I}$ where $\widetilde{V}$ and $\widetilde{I}$ are the phasor representations (complex amplitudes) of the voltage and current, respectively at an instant in time. \\\\
\underline{Note:} Just like using phasors for waves, we can convert back to time domain by taking the real part or imaginary part of $\widetilde{I}$ depending on whether we used sine or cosine for the driving voltage.\\\\
\textbf{Driven LC Circuit}\\
Consider a driven LC circuit with a sinusoidal voltage source in series with the inductor and capacitor.:
\[V(t) = V_m \cos(\omega_d t) \implies V(t) = \Re\!\left\{\widetilde{V}\, e^{i\omega_d t}\right\} =\Re\!\left\{ V_m\, e^{i\omega_d t} \right\}\]
Finding the equivalent impedance:
\[Z_{\text{eq}} = Z_L + Z_C = i\omega_d L - \frac{i}{\omega_d C} = i\left(\omega_d L - \frac{1}{\omega_d C}\right)\]
We can write this as a phasor with a magnitude and a phase $Z = ||Z||e^{i\phi}$:\\
\begin{minipage}{0.4\textwidth}
\centering
\includegraphics[width=\textwidth]{Phasor LC.jpg}
\end{minipage}
\hfill
\begin{minipage}{0.5\textwidth}
Assuming that $\omega_dL > \frac{1}{\omega_d C}$,
\[||Z|| = \sqrt{\left(\omega_d L - \frac{1}{\omega_d C}\right)^2}\]
\[\phi = \tan^{-1}\left(\frac{\omega_d L - \frac{1}{\omega_d C}}{0}\right) = +90^\circ\]
\[\implies Z_{\text{eq}} = ||Z|| e^{i\phi}\]
\end{minipage}\\\\
Using Ohm's law in phasor form to find the current phasor:
\[\widetilde{I} = \frac{\widetilde{V}}{Z_{\text{eq}}} = \frac{V_m e^{i0}}{||Z||e^{i\phi}}  \implies \widetilde{I} = \frac{V_m}{||Z||} e^{-i\phi}\]
To find current in the time domain, we take the real part, noting that the phasor rotates at the driving frequency $\omega_d$:
\[I(t) = \Re\!\left\{\widetilde{I}e^{i\omega_d t}\right\} = \Re\!\left\{\frac{V_m}{||Z||} e^{-i\phi}e^{i\omega_d t}\right\} = \frac{V_m}{||Z||} \cos(\omega_d t - \phi)\]
Since $\phi = +90^\circ$, $cos(\omega_d t - 90^\circ) = \sin(\omega_d t)$, thus:
\[I(t) = \frac{V_m}{\left(\omega_d L - \frac{1}{\omega_d C}\right)} \sin(\omega_d t)\]
\textbf{However,} this solution only describes the steady-state forced oscillation of the circuit. The complete solution must also include the free oscillation of the circuit, which is represented by the homogeneous solution we found earlier: 
\[I_h = I_0 \sin(\omega_0 t + \phi_0)\]
Where $\omega_0 = \frac{1}{\sqrt{LC}}$ is the natural frequency of the circuit, and $I_0$ and $\phi_0$ are constants determined by the initial conditions of the circuit. Note that $\phi_0$ is different from $\phi$ used in the phasor representation.\\\\
Thus, the complete solution for the driven LC circuit is ($I = I_h + I_p$):
\[\boxed{I(t) = I_0 \sin(\omega_0 t + \phi_0) + \frac{V_m}{\left(\omega_d L - \frac{1}{\omega_d C}\right)} \sin(\omega_d t)}\]
\underline{Note:} This solution is valid when the circuit is not in resonance ($\omega_d \neq \omega_0$). At resonance, the current amplitude would blow up to infinity (there is no reactance or resistance). In a real circuit, there would still be some internal resistance that would limit the current.\\\\
\textbf{Driven RLC Circuit}\\
Consider a driven RLC circuit with a sinusoidal voltage source in series with the resistor, inductor, and capacitor:
\[V(t) = V_m \cos(\omega_d t) \implies V(t) = \Re\!\left\{\widetilde{V}\, e^{i\omega_d t}\right\} =\Re\!\left\{ V_m\, e^{i\omega_d t} \right\}\]
Finding the equivalent impedance:
\[Z_{\text{eq}} = Z_R + Z_L + Z_C = R + i\omega_d L - \frac{i}{\omega_d C} = R + i\left(\omega_d L - \frac{1}{\omega_d C}\right)\]
\newpage\noindent
We can write this as a phasor with a magnitude and a phase $Z = ||Z||e^{i\phi}$:\\
\begin{minipage}{0.4\textwidth}
\centering
\includegraphics[width=\textwidth]{Phasor RLC.jpg}
\end{minipage}
\hfill
\begin{minipage}{0.5\textwidth}
Assuming that $\omega_dL > \frac{1}{\omega_d C}$,
\[||Z|| = \sqrt{R^2 + \left(\omega_d L - \frac{1}{\omega_d C}\right)^2}\]
\[\phi = \tan^{-1}\left(\frac{\omega_d L - \frac{1}{\omega_d C}}{R}\right)\]
\[\implies Z_{\text{eq}} = ||Z|| e^{i\phi}\]
\end{minipage}\\\\
Using Ohm's law in phasor form to find the current phasor:
\[\widetilde{I} = \frac{\widetilde{V}}{Z_{\text{eq}}} = \frac{V_m e^{i0}}{||Z||e^{i\phi}}  \implies \widetilde{I} = \frac{V_m}{||Z||} e^{-i\phi}\]
To find current in the time domain, we take the real part, noting that the phasor rotates at the driving frequency $\omega_d$:
\[I(t) = \Re\!\left\{\widetilde{I}e^{i\omega_d t}\right\} = \Re\left\{\frac{V_m}{||Z||} e^{-i\phi}e^{i\omega_d t}\right\} = \frac{V_m}{\sqrt{R^2 + \left(\omega_d L - 1/\omega_d C\right)^2}} \cos(\omega_d t - \phi)\]
Unlike the driven LC circuit, this solution describes the complete steady-state behavior of the driven RLC circuit because the transient free oscillations decay over time due to the resistor dissipating energy as heat. Thus, the complete solution for the driven RLC circuit is:
\[\boxed{I(t) = \frac{V_m}{\sqrt{R^2 + \left(\omega_d L - 1/\omega_d C\right)^2}} \cos(\omega_d t - \phi)}\]
\end{enumerate}
Some last things to note about oscillating circuits:\\\\
\textbf{Energy and Power}\\
Once we have found the charge and current in the circuit, we can find the energy stored in the capacitor and inductor at any time:
\[U_C = \frac{1}{2} C V_C^2 = \frac{1}{2} \frac{q^2}{C}, \quad U_L = \frac{1}{2} L I^2\]
For RLC circuits, suppose that we found that the current in the circuit is:
\[I(t) = I_m \sin(\omega_d t - \phi)\]
Since it is oscillatory, we must use the \textbf{root mean square (rms)} value of the current to find the average power dissipated in the resistor:
\[\boxed{I_{\text{rms}} = \frac{I_m}{\sqrt{2}}}\]
\[\boxed{P_{\text{avg}} = I_{\text{rms}}^2 R = \frac{I_m^2 R}{2}}\]
This gives the average power dissipated as heat in the resistor over one full cycle of oscillation.\\\\
We can also find other RMS values:
\[V_{\text{rms}} = \frac{V_m}{\sqrt{2}}, \quad \mathcal{E}_{\text{rms}} = \frac{\mathcal{E}_m}{\sqrt{2}}\]
From Ohm's law we can see that:
\[V_{\text{rms}} = I_{\text{rms}} Z\]
Thus we can redefine the average power dissipated in the resistor as:
\[P_{\text{avg}} = I_{\text{rms}} V_{\text{rms}} \cos\phi\]
Where $\cos\phi$ is called the \text{power factor} of the circuit. \\\\
\textbf{Transformers}\\
To deliver large amounts of power while minimizing losses, electricity can be transmitted at high voltage and low current through power lines, then stepped down with a transformer to increase the current for safe use in homes.
\[P_{\text{delivered}} = P_{\text{in}} - P_{\text{loss}} = IV - I^2 R\]
The ideal transformer consists of a primary coil with $N_p$ turns and a secondary coil with $N_s$ turns wound around an iron core. It operates using Faraday's law of induction with an AC voltage source connected to the primary coil, creating a time-varying magnetic flux that induces an emf in the secondary coil.
\begin{figure}[H]
    \centering
    \includegraphics[width=0.4\textwidth]{Transformer Ideal.png}
\end{figure}
Suppose the emf source is $\mathcal{E} = \mathcal{E}_{m} \sin(\omega t)$. This would cause a sinusoidal current in the primary coil, creating a time-varying magnetic flux $\Phi_B$ through both coils and thus inducing an emf in each turn.
\[\mathcal{E}_{\text{turn}} = \frac{d\Phi_B}{dt}\]
The change in magnetic flux in the primary and secondary are the \textit{same}, thus the total emf in each coil is proportional to the number of turns:
\[\mathcal{E}_{\text{turn}} = \frac{V_p}{N_p} = \frac{V_s}{N_s}\]
\[\boxed{V_s = V_p \frac{N_s}{N_p}}\]
Note that if $N_s > N_p$, then $V_s > V_p$ and the transformer is a step-up transformer. If $N_s < N_p$, then $V_s < V_p$ and the transformer is a step-down transformer.\\\\
Using the conservation of energy (ideal transformer has no losses), the power input to the primary coil must equal the power output from the secondary coil:
\[P_p = P_s \implies I_p V_p = I_s V_s\]
\[\boxed{I_s = I_p \frac{N_p}{N_s}}\]
Since the secondary current has a resistor load $R$ connected to it, the current in the secondary coil is determined by Ohm's law:
\[I_s = \frac{V_s}{R} = \frac{V_p}{R} \frac{N_s}{N_p}\]
\[\implies I_p = \frac{V_p}{R} \left(\frac{N_s}{N_p}\right)^2\]
This equation has the form $V_p = I_p R_{eq}$ where 
\[R_{eq} = R \left(\frac{N_p}{N_s}\right)^2\]
\textbf{End Final Exam}
% LORD HAVE MERCY ON ME, IT IS SO OVER.
\end{document}