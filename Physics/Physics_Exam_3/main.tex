\documentclass[12pt]{article}
\usepackage{amssymb}
\usepackage{geometry}
\usepackage{amsmath, amsfonts, bm, graphicx}
\usepackage{float,multicol}
\geometry{margin=1in}
\title{}
\date{}
\author{}

\begin{document}

\section*{Mathematical Methods}

\begin{enumerate}
    \item Use the Taylor series expansion to find approximations. The ones for sin, cos, tan, and $(1 + x)^n$ are especially useful.
\begin{flalign*}
\sin x &= \sum_{n=0}^{\infty} (-1)^n \frac{x^{2n+1}}{(2n+1)!} 
       = x - \frac{x^3}{3!} + \frac{x^5}{5!} - \frac{x^7}{7!} + \cdots & \\[6pt]
\cos x &= \sum_{n=0}^{\infty} (-1)^n \frac{x^{2n}}{(2n)!} 
       = 1 - \frac{x^2}{2!} + \frac{x^4}{4!} - \frac{x^6}{6!} + \cdots & \\[6pt]
\tan x &= \sum_{n=1}^{\infty} \frac{B_{2n} (-4)^n (1-4^n)}{(2n)!}\,x^{2n-1} 
       = x + \frac{x^3}{3} + \frac{2x^5}{15} + \frac{17x^7}{315} + \cdots & \\[6pt]
(1+x)^m &= \sum_{n=0}^{\infty} \binom{m}{n} x^n 
       = 1 + mx + \frac{m(m-1)}{2}x^2 + \frac{m(m-1)(m-2)}{6}x^3 + \cdots &
\end{flalign*}
\underline{Note:} For small x, higher order terms reduce to zero

    \item Use complex exponentials to manipulate complicated trig functions.
\[e^{ix} = \cos x + i \sin x\]
    \item Solve differential equations by substituting in trial solutions. Especially you should recognize the differential equation for a simple harmonic oscillator and be able to come up with solutions to that ODE that satisfy any initial conditions you are given.

\[\frac{d^2x}{dt^2} + \omega^2 x = 0\]

\[x(t) = A \cos(\omega t) + B \sin(\omega t)\]
Wave equation: 
\[v^2 \frac{\partial ^2 y}{\partial x^2}=\frac{\partial^2 y}{\partial t^2}\]

\item Useful integration formulas:
\[\int\frac{1}{(x^2+a^2)^\frac{3}{2}}dx=\frac{x}{a^2\sqrt{x^2+a^2}}+C\]
\[\int\frac{x}{(x^2+a^2)^\frac{3}{2}}dx=-\frac{1}{\sqrt{x^2+a^2}}+C\]
\end{enumerate}
\newpage
\section*{Chapter 28 - Magnetic Fields}
\textbf{Types of magnets}
\begin{itemize}
    \item \textbf{Current loop:} a current carrying loop of wire creates an electromagnet.
    \item \textbf{Permanent Magnet:} the magnetic fields of the electrons within the material do not cancel out, resulting in a net magnetic field.
\end{itemize}
\textit{All} magnets are \textbf{magnetic dipoles} with a \textbf{north} and \textbf{south} pole (the magnetic monopole doesn't exist, sadly). Opposite magnetic poles attract each other, and like magnetic poles repel each other. Magnetic field lines are \textit{closed loops} that exit through the North pole and enter through the South pole.
\begin{multicols}{2}
    \begin{figure}[H]
    \centering
    \includegraphics[width=0.6\textwidth]{Magnetic Field.png}
\end{figure}
\begin{figure}[H]
    \centering
    \includegraphics[width=0.34\textwidth]{Horseshoe.png}
\end{figure}
\end{multicols}\noindent
\underline{Note:} Inside the bar magnet, the magnetic field lines point from south to north, completing the closed loop.\\\\
\underline{Magnetic field lines and the magnetic field are related by:}
\begin{itemize}
    \item The direction of the magnetic field is tangent to the field lines.
    \item The spacing of the field lines represents the strength (magnitude) of the magnetic field. Closer lines = stronger field.
\end{itemize}
Also, analogous to Gauss's law for electric fields, we have \textbf{Gauss's law for magnetism:}
\[\int \vec{B} \cdot \hat{n}\,d\vec{A} = 0\]
Since there are no magnetic monopoles, the net magnetic flux through any closed surface is zero (there are no sources or sinks of magnetic field lines).\\\\\newpage
\begin{enumerate}
    \item Solve Newton's second law to determine the motion of charged particles acting under the influence of a magnetic field and any other forces (e.g., gravity, electric fields...).\\\\
Stationary charges do not interact with the magnetic field. Moving charges with a component of velocity perpendicular to the magnetic field experiences a force:
\[\boxed{\vec{F}_B = q\vec{v} \times \vec{B}}\]
\underline{Note:} This force is \textit{always} perpendicular to the velocity of the particle, so it does \textbf{no work} on the particle and cannot change its speed, only its direction.\\\\
\underline{Note:} The magnetic force is zero when the velocity is along the magnetic field lines (i.e., parallel or antiparallel) or when stationary.\\\\
 The unit for the magnetic field $\vec{B}$ is the Tesla (T):
 \[1\, \text{T} = 1\, \frac{\text{N}}{\text{C}\cdot \text{m/s}} = 1\, \frac{\text{N}}{\text{A}\cdot \text{m}}\]
\underline{Recall:} \textbf{Right hand rule}
\begin{multicols}{2}
\begin{figure}[H]
    \centering
    \includegraphics[width=0.3\textwidth]{RHR.png}
\end{figure}
\columnbreak
\begin{itemize}
    \item Point fingers in the direction of the velocity $\vec{v}$.
    \item Curl fingers toward the direction of the magnetic field $\vec{B}$, sweeping through the smaller angle.
    \item Thumb points in the direction of the force $\vec{F}_B$ for \textbf{a positive charge}. For a negative charge, the force is in the opposite direction.
\end{itemize}
\end{multicols}
\underline{Note:} When $\vec{B}$ and $\vec{v}$ are orthogonal, we can just multiply the magnitudes to find the force and use the right hand rule to find the direction.\\\\
\underline{Note:} A magnetic force exists even if there is \textit{relative velocity} between charges and a magnetic field. For example, a moving magnet will exert a magnetic force on stationary charges.\\\\
\underline{Note to self:} Bring dynamics formula sheet for kinematics equations.\\\\
The total electromagnetic force on a charged particle in both electric and magnetic fields is given by the \textbf{Lorentz force:}
\[\boxed{\vec{F} = q\vec{E} + q\vec{v} \times \vec{B}}\]
\newpage
    \item Explain the Hall effect and describe its applications.\\\\
Here are several interesting applications where both the magnetic field and electric field acts on a moving charge.\\\\
\textbf{Wien Filter (Velocity Selector)}
\begin{figure}[H]
    \centering
    \includegraphics[width=0.7\textwidth]{Wien.jpg}
\end{figure}
\vspace{-1em}
Suppose we have a source charged particles ($+q$) with random velocities. If it passes through a region with \textit{only} an $\vec{E}$ field, it will be pushed onto the negative plate, following a parabolic trajectory. However, if there is a $\vec{B}$ field in addition to the $\vec{E}$ field, then the forces will cancel for particles with a specific velocity:
\[\sum \vec{F}_y=\vec{F}_B-\vec{F}_E=0 \implies \vec{F}_B=\vec{F}_E \]
\[q\vec{v}\times \vec{B}=q\vec{E}\]
\[\boxed{v=\frac{E}{B}}\]
Thus, only particles with velocity $v=E/B$ will pass straight through the filter. \\\\
By combining the Wien filter with another region of magnetic fields, we create a mass spectrometer that can separate particles based on their charge-to-mass ratio.\\\\
\textbf{Mass Spectrometer}
\begin{figure}[H]
    \centering
    \includegraphics[width=0.4\textwidth]{Mass Spec.jpg}
\end{figure}
\vspace{-1em}
Since we know both the charge and velocity entering the magnetic field region, we can find the particle's mass by measuring the radius of its circular path:
\[\sum \vec{F}_n=\vec{F}_B=\frac{mv^2}{R} \implies qvB=\frac{mv^2}{R}\]
\[\boxed{m=\frac{BRq}{v}}\]
\newpage
\underline{Note:} Typically, the charges are accelerated through a potential difference $V$ before entering the velocity selector, so we can find their velocity using energy conservation:
\[W_{nc}=\Delta E = 0 \implies \Delta U = \Delta K\]
\[qV=\frac{1}{2}mv^2 \implies v=\sqrt{\frac{2qV}{m}}\]
The full set up looks like this:
\begin{figure}[H]
    \centering
    \includegraphics[width=0.9\textwidth]{Mass Spec full.jpg}
\end{figure}
Also note that since the magnetic force is opposite for negative charges, they will curve in the opposite direction in the magnetic field region:
\begin{figure}[H]
    \centering
    \includegraphics[width=0.35\textwidth]{Diff Charges.jpg}
\end{figure}
\newpage
Let's consider a setup similar to a Wien filter, but where the parallel plates are designed to deflect the particle beam rather than selectively filter it.\\\\
\textbf{Cathode Ray Tube}
\begin{figure}[H]
    \centering
    \includegraphics[width=0.6\textwidth]{Deflection.jpg}
\end{figure}
\vspace{-1em}
First, without a $\vec{B}$ field, the particles will be deflected by the $\vec{E}$ field:
\[\sum \vec{F}=-qE\hat{\jmath} \implies |a_y|=\frac{|q|E}{m}\]
The time spent in the field is:
\[\Delta x = vt = L \implies t = \frac{L}{v}\]
The vertical displacement upon exiting the plates is:
\[\Delta y = v_{oy}t + \frac{1}{2}a_yt^2 = 0 + \frac{1}{2}\left(\frac{|q|E}{m}\right)\left(\frac{L}{v}\right)^2\]
\[\boxed{\Delta y = \frac{|q|EL^2}{2mv^2}}\]
Now consider adding a magnetic field like this:
\begin{figure}[H]
    \centering
    \includegraphics[width=0.6\textwidth]{CRT.png}
\end{figure}
\vspace{-1em}
We know from the Wien filter that the forces will cancel when:
\[v=\frac{E}{B}\]
Thus, plugging this into our previous equation for vertical displacement:
\[\frac{m}{|q|}=\frac{BL^2}{2\Delta y E}\]
\newpage
Finally, let's talk about the Hall effect!\\\\
\textbf{Hall Effect}
\begin{figure}[H]
    \centering
    \includegraphics[width=0.6\textwidth]{Hall Effect.jpg}
\end{figure}
\vspace{-1em}
Consider a conductor with a current $I$ flowing through it in a region with a $\vec{B}$ field. The moving charges will be pushed to one side of the conductor by the magnetic force, creating a \textbf{Hall potential difference} ($\Delta V$) and an electric field ($\vec{E}$) inside the conductor. 
\[\Delta V=Ed\]
Eventually, when the electric force balances the magnetic force, the charges stop accumulating.
\[\sum \vec{F} = \vec{F}_E-\vec{F}_B=0 \implies qE=qv_dB\]
Thus, by measuring the Hall potential difference, we can find the magnetic field strength:
\[\boxed{B=\frac{\Delta V}{v_d d}}\]
We can also find the number of charge carriers per unit volume ($n$) in the conductor, letting $q=e$ for electrons and plugging in for $v_d$ from before:
\[I=nev_d A ,\quad \text{(A is cross-sectional area of conductor)}\]
\[n=\frac{IBd}{eA\Delta V}\]
\underline{Note:} It is also possible to determine the drift velocity using the Hall effect, by mechanically moving the conductor such that there is no relative velocity between the charges and the magnetic field. Therefore, there will be \textbf{zero} Hall potential difference (since there is no magnetic force).
\newpage
    \item Explain the principle of operation of a cyclotron\\\\
\textbf{Circulating Charged Particles}\\
We know that for a particle of charge $q$ moving with $\vec{v}$ in a uniform magnetic field, it will tend towards a circular path due to the magnetic force (no tangential force, only normal force).
\[\sum F_n = |q|vB=\frac{mv^2}{r} \implies r=\frac{mv}{|q|B} \quad \text{(radius)}\]
From which we can define the following quantities:
\[T=\frac{2\pi r}{v}=\frac{2 \pi}{v}\frac{mv}{|q|B}=\frac{2\pi m}{|q|B} \quad \text{(period)}\]
\[f=\frac{1}{T}=\frac{|q|B}{2\pi m} \quad \text{(frequency)}\]
\[\omega = \frac{2 \pi}{T}=2\pi f = \frac{|q|B}{m} \quad \text{(angular frequency)}\]
\underline{Note:} The quantities $T,f, \text{ and } \omega$ do not depend on the speed of the particle (as long as it isn't moving at relativistic speeds). Fast particles move in large circles and slow ones in small circles, but all particles with the same charge-to-mass ratio $|q|/m$ take the same time $T$ to complete one loop.\\\\
\textbf{Helical Paths}\\
If the velocity of the charged particle has a component parallel to the magnetic field, then the particle will follow a helical path:
\vspace{-1em}
\begin{figure}[H]
    \centering
    \includegraphics[width=0.6\textwidth]{Helical.png}
\end{figure}
\vspace{-1em}
Where the angle $\phi$ is the angle between $\vec{v}$ and $\vec{B}$.
\[v_{\parallel} = v \cos \phi, \quad v_{\perp} = v \sin \phi\]
The radius of the helical path is determined by $v_{\perp}$:
\[r=\frac{mv_{\perp}}{|q|B}=\frac{mv\sin \phi}{|q|B}\]
The pitch of the helix (distance between successive turns) is determined by $v_{\parallel}$ and the period $T$:
\[p=v_{\parallel}T=v\cos \phi \left(\frac{2\pi m}{|q|B}\right)\]
\newpage\noindent
\textbf{Cyclotron}\\
A cyclotron is a device that uses a combination of a constant magnetic field and an oscillating electric potential difference to accelerate charged particles. The magnetic field forces the particles to move in circular paths while the potential difference between the dees accelerates them each time they cross the gap.
\begin{figure}[H]
    \centering
    \includegraphics[width=0.35\textwidth]{Cyclotron.png}
\end{figure}
Suppose a proton is injected at source $S$. It will be accelerated toward the negatively charged dee and enter it. Once inside, there will be no electric field (shielded by the conducting walls of the dee), and it will move in a semicircular path due to the magnetic field. When it exits the dee, the potential difference is reversed to accelerate it again across the gap. Thus, the frequency $f$ at which the proton circulates (independent of speed) \textit{must} match the frequency of the oscillating potential difference $f_{osc}$:
\[f=f_{osc} \quad \text{(resonance condition)}\]
\[\frac{|q|B}{2\pi m}=f_{osc}\]
\textbf{Synchrotron}\\
At relativistic speeds (above 10\% of $c$), the frequency of revolution is now no longer independent of the charged particle's speed. As the speed approaches the speed of light, the frequency of revolution decreases, and is no longer in sync to the fixed $f_{osc}$. Thus a \textbf{synchrotron} is used to vary both the magnetic field and $f_{osc}$ to keep the particle in resonance as it accelerates to higher speeds. The proton also follows a circular path instead of a spiral in a synchrotron.
\newpage
    \item Determine the forces and/or torques on various arrangements of current carrying wires (straight, circular loops, square loops, etc...) located in a given magnetic field.\\\\
\textbf{Magnetic Force on a Current Carrying Wire}\\
We know that moving charges experience a magnetic force in a magnetic field. Thus, a current-carrying wire (which has moving charges) will also experience a magnetic force when placed in a magnetic field.
\begin{figure}[H]
    \centering
    \includegraphics[width=0.5\textwidth]{Wire.png}
\end{figure}
\vspace{-1em}
\underline{Note:} The motion of electrons is opposite to the direction of conventional current. However, since both the charge and velocity are negative, the magnetic force ends up being in the same direction as if we considered positive charges moving with the current.\\\\
We know that the magnetic force on a single charge is:
\[\vec{F}_B = q\vec{v} \times \vec{B}\]
Thus, for $N$ charges in a wire segment of length $L$, the total magnetic force is:
\[\vec{F}_B = Nq\vec{v}_d \times \vec{B}\]
If we rewrite $N$ in terms of the number of charge carriers per unit volume $n$ and the volume of the wire segment $AL$ (where $A$ is the cross-sectional area), we get:
\[\vec{F}_B = (nAL)q\vec{v}_d \times \vec{B}\]
Recall that current is defined as $I=qnv_dA$, so we can rewrite the magnetic force as:
\[\boxed{\vec{F}_B = I \vec{L} \times \vec{B} \quad \text{(force on a straight wire)}}\]
Where $\vec{L}$ is a vector in the direction of the conventional current with magnitude $L$.\\\\
If the wire is not straight or the field is not uniform, we can find the differential force on a small current element $I dl$ and integrate over the length of the wire:
\[\boxed{d\vec{F}_B = I d\vec{L} \times \vec{B}}\]
\underline{Note:} There is no such thing as an isolated current-carrying wire, there must always be a way to introduce current into the wire and take it out at the other end.
\newpage
\textbf{Magnetic Torque on a Current Loop}\\
A motor converts current into rotation by using magnetic forces on a current-carrying loop to generate a torque. In this case, the direction of current is reversed every half turn to keep the torque in the same direction using a commutator (not shown).
\vspace{-1em}
\begin{figure}[H]
    \centering
    \includegraphics[width=0.4\textwidth]{Torque.png}
\end{figure}
\vspace{-1em}
Let's consider the following rectangular current loop in a uniform magnetic field:
\begin{figure}[H]
    \centering
    \includegraphics[width=0.9\textwidth]{Torque 1.jpg}
\end{figure}
\begin{multicols}{2}
\begin{figure}[H]
    \centering
    \includegraphics[width=0.34\textwidth]{RHR 2.png}
\end{figure}
The orientation of the loop is defined using a normal vector $\vec{n}$ that is perpendicular to the plane of the loop and follows the right-hand rule with respect to the current direction.\\\\
Curl fingers in the direction of the current and the thumb points in the direction of $\vec{n}$. The angle $\theta$ is defined as the angle between $\vec{n}$ and $\vec{B}$.
\end{multicols}
Finding the magnetic force on each side of the loop using $\vec{F}_B = I \vec{L} \times \vec{B}=ILBsin\theta$ where $\theta$ is the angle between $\vec{L}$ and $\vec{B}$:
\[||\vec{F}_1||=||\vec{F}_3||=iaB\]
\[||\vec{F}_2||=||\vec{F}_4||=ibBsin(90^\circ-\theta)=ibBcos\theta\]
\newpage
By symmetry, the forces act in opposite directions on each side, so the net force on the loop is zero. However, there is a torque about the center of the loop due to $\vec{F}_1$ and $\vec{F}_3$ (since their lines of action do not pass through the center):
\[\tau = \vec{r}\times\vec{F}\]
\[\tau = \left(iaB \frac{b}{2}sin\theta\right)+\left(iaB \frac{b}{2}sin\theta\right)=iabBsin\theta\]
Note that $A=ab$ is the area of the loop, so we can rewrite the torque as:
\[\boxed{\tau = iABsin\theta}\]
This relation holds for any shape of current loop, as long as $A$ is the area of the loop and $\theta$ is the angle between $\vec{n}$ and $\vec{B}$.\\\\
If we have a \textit{coil} with $N$ loops of wire, we can approximate them as $N$ identical current loops stacked together in the same plane. Thus, the total torque on the coil is:
\[\boxed{\sum \tau = NIABsin\theta}\]
\underline{Note:} The current-carrying coil will tend to rotate such that $\vec{n}$ is aligned with $\vec{B}$, minimizing the potential energy of the system.
\newpage
\textbf{Magnetic Dipole Moment}\\
Similar to a bar magnet, a current-carrying loop tends to align itself with an external magnetic field. Thus, the current loop is said to be a \textbf{magnetic dipole} with a \textbf{magnetic dipole moment} $\vec{\mu}$ defined as:
\[\boxed{\vec{\mu} = NIA\hat{n}}\]
Where $N$ is the number of loops, $I$ is the current, $A$ is the area of the loop, and $\hat{n}$ is the unit normal vector to the plane of the loop.It has units of Amphere-square meter ($\text{A}\cdot \text{m}^2$). \\\\
Using $\vec{\mu}$, we can rewrite the torque on the current loop as:
\[\boxed{\vec{\tau} = \vec{\mu} \times \vec{B}}\]
Similar to the electric dipole in an electric field, the potential energy of a magnetic dipole in a magnetic field is given by:
\[\boxed{U = -\vec{\mu} \cdot \vec{B}}\]
\underline{Note:} The minimum potential energy ($-\mu B$) occurs when $\vec{\mu}$ is aligned with $\vec{B}$, and the maximum potential energy ($\mu B$) occurs when they are anti-aligned.
\begin{figure}[H]
    \centering
    \includegraphics[width=0.3\textwidth]{Potential.png}
\end{figure}
\underline{Note:} The work done by an external torque to rotate the dipole is $W_{ext}=\Delta U = U_f - U_i$. If the dipole is stationary before and after the rotation.\\\\
A bar magnet and a rotating sphere of charge are magnetic dipoles as well, and we can approximate the Earth as a big magnetic dipole. Most subatomic particles (like electrons, protons, and neutrons) also have intrinsic magnetic dipole moments due to their spin and charge. Thus, we can model their interactions with magnetic fields using the same equations as above.\\\\
Note that when a magnet exerts a magnetic force on a current-carrying wire, Newton’s third law requires that the wire exert an equal and opposite force back on the magnet. The only way the wire can interact with the magnet is through magnetic fields, so the wire must itself produce a magnetic field...
\end{enumerate}
\newpage
\section*{Chapter 29 - Magnetic Fields due to Currents}
\begin{enumerate}
    \item Use the Biot-Savart law to calculate the magnetic field due to a current-carrying wires of arbitrary (but tractable) geometry. e.g., a loop.
\begin{multicols}{2}
Like electric fields, magnetic fields obey superposition. Thus, we can find the magnetic field from a wire by summing up the $d\vec{B}$ at point $P$ produced by small current elements $Id\vec{l}$ along the wire.
\begin{figure}[H]
    \centering
    \includegraphics[width=0.5\textwidth]{Biot Savart.jpg}
\end{figure}
\end{multicols}
\vspace{-1em}
We can find $d\vec{B}$ by using the \textbf{Biot-Savart Law}:
\[\boxed{d\vec{B} = \frac{\mu_0}{4\pi} \frac{ I d\vec{l} \times \hat{r}}{r^2}\quad \text{(current carrying wire)}}\]
Where
\begin{itemize}
    \item $I d\vec{l}$ is the current element that produces the differential magnetic field $d\vec{B}$.
    \item $r$ is the distance from the current element to point $P$.
    \item $\hat{r}$ is the unit vector that points from the current element to point $P$.
    \item $\mu_0$ is the permeability of free space and has a value of:
\[\boxed{\mu_0 = 4\pi \times 10^{-7} \,\text{T}\cdot \text{m/A}}\]
\end{itemize}
\underline{Note:} $I d\vec{l} \times \hat{r} = I dl \sin \theta $ where $\theta$ is the angle between $d\vec{l}$ and $\vec{r}$.\\\\
For a moving point charge:
\[\boxed{d\vec{B} = \frac{\mu_0}{4\pi} \frac{ q \vec{v} \times \hat{r}}{r^2} \quad \text{(point charge)}}\]
We can also find the direction of the magnetic field using the right-hand rule:
\begin{multicols}{2}
\begin{figure}[H]
    \centering
    \includegraphics[width=0.2\textwidth]{RHR 3.png}
\end{figure}
\begin{itemize}
    \item Grasp the current element and point thumb in the direction of the current.
    \item Your fingers will then naturally curl around in the direction of the magnetic field lines due to that element.
\end{itemize}
\end{multicols}
\newpage
\textbf{Magnetic Field Due to a Current in a Straight Wire}
\begin{figure}[H]
    \centering
    \includegraphics[width=0.4\textwidth]{Long Wire.jpg}
\end{figure}
\vspace{-1em}
Biot-Savart law:
\[d\vec{B} = \frac{\mu_0}{4\pi} \frac{ I d\vec{l} \times \hat{r}}{r^2}\]
Vectors:
\[d\vec{l} = dy' \hat{\jmath}\]
\[\vec{r}=\left<x,0\right>-\left<0,y'\right>=\left<x,-y'\right>,\quad ||\vec{r}|| = \sqrt{x^2 + y'^{\,2}}\]
\[\hat{r}=\frac{\left<x,-y'\right>}{\sqrt{x^2 + y'^{\,2}}}\]
Cross product:
\begin{align*}
d\vec{B} &= \frac{\mu_0}{4\pi} \frac{ I dy'\hat{\jmath} \times \left<x,-y'\right>}{(x^2 + y'^{\,2})^{3/2}}\\
&= \frac{\mu_0}{4\pi} \frac{ -Ixdy'\hat{k}}{(x^2 + y'^{\,2})^{3/2}}
\end{align*}
Integrate:
\begin{align*}
\vec{B}&=-\frac{\mu_0 I}{2\pi}\int_{-L/2}^{L/2}\frac{xdy'}{(x^2 + y'^{\,2})^{3/2}} \hat{k} \\
&=-\frac{\mu_0 I}{2\pi} \frac{1}{x\sqrt{x^2+(L/2)^2}}\hat{k}
\end{align*}
Taking the limit as $L \to \infty$, letting $r=x$:
\[\boxed{||\vec{B}|| = \frac{\mu_0 I}{2\pi r} \quad \text{(infinite straight wire)}}\]
\underline{Note:} For a semi-infinite wire (from $0\to\infty$), the magnetic field is half that of an infinite wire at the same distance $r$ from the wire.
\[||\vec{B}|| = \frac{\mu_0 I}{4\pi r} \quad \text{(semi-infinite straight wire)}\]
\newpage
\textbf{Magnetic Field Due to a Current in a Circular Loop of Wire}
\begin{figure}[H]
    \centering
    \includegraphics[width=0.5\textwidth]{Loop.jpg}
\end{figure}
Biot-Savart law:
\[d\vec{B} = \frac{\mu_0}{4\pi} \frac{ I d\vec{l} \times \hat{r}}{r^2}\]
Vectors:
\[d\vec{l} = Rd\theta \, \hat{\theta}\]
\[\vec{r}=\left<0,0,z\right>-\left<Rcos\theta,Rsin\theta,0\right>=\left<-Rcos\theta,-Rsin\theta,z\right>,\quad ||\vec{r}|| = \sqrt{R^2+z^2}\]
\[\hat{r}=\frac{\left<-Rcos\theta,-Rsin\theta,z\right>}{\sqrt{R^2+z^2}}\]
Rewriting $\hat{\theta}$ in cartesian, using a unit circle:
\begin{figure}[H]
    \centering
    \includegraphics[width=0.3\textwidth]{Unit circle.jpg}
\end{figure}
\vspace{-2em}
\[\hat{\theta}=\left<-sin\theta,cos\theta,0\right>\]
Cross product:
\begin{align*}
d\vec{l}\times\vec{r}&=
\begin{vmatrix}
\hat{\imath} & \hat{\jmath} & \hat{k} \\
-Rd\theta sin\theta & Rd\theta cos\theta & 0 \\
-Rcos\theta & -Rsin\theta & z
\end{vmatrix}\\
&=(Rzcos\theta d\theta)\hat{\imath}+(Rzsin\theta d\theta)\hat{\jmath}+(R^2 sin^2\theta d\theta+R^2cos^2 \theta d\theta)\hat{k}\\
&=(Rzcos\theta d\theta)\hat{\imath}+(Rzsin\theta d\theta)\hat{\jmath}+(R^2 d\theta)\hat{k}
\end{align*}
\begin{figure}[H]
    \centering
    \includegraphics[width=0.45\textwidth]{loop 2.png}
\end{figure}
By symmetry, the $\hat{\imath}$ and $\hat{\jmath}$ components will cancel out when integrating over the full loop, so we only need to consider the $\hat{k}$ component:
\[B_z=\int_{0}^{2\pi} \frac{\mu_0}{4\pi}\frac{IR^2 \, d\theta}{(R^2+z^2)^{3/2}}=\frac{\mu_0}{4\pi}\frac{2\pi IR^2}{(R^2+z^2)^{3/2}}\]
Notice that $\mu = NIA = I\pi R^2$, Thus
\[\boxed{\vec{B}=\frac{\mu_0}{4\pi}\frac{2\mu}{(R^2+z^2)^{3/2}} \hat{k}}\]
\textbf{Magnetic Field due to a Current in a Circular Arc of Wire}
\begin{figure}[H]
    \centering
    \includegraphics[width=0.5\textwidth]{Arc.jpg}
\end{figure}
\vspace{-1em}
Biot-Savart law:
\[d\vec{B} = \frac{\mu_0}{4\pi} \frac{ I d\vec{l} \times \hat{r}}{r^2}\]
The angle between $d\vec{l}$ and $\vec{r}$ is always $90^\circ$ for a circular arc, so:
\[d\vec{B}=\frac{\mu_0}{4\pi} \frac{ I dl}{R^2} \hat{k},\quad \text{(Direction found by RHR)}\]
\[\vec{B}=\int_{0}^{\phi}\frac{\mu_0}{4\pi} \frac{ I dl}{R^2} \hat{k}\]
\[\boxed{\vec{B}=\frac{\mu_0 I\phi}{4\pi R} \hat{k} \quad \text{(circular arc of wire)}}\]
\newpage
    \item Describe Ampere's law and the conditions under which it can be used to solve for a magnetic field from a given current distribution. Use Ampere's law to find the field in those situations with sufficient symmetry to apply it. e.g., a long wire.\\\\
\textbf{Ampere's Law}\\
Ampere's Law states that the circulation of the magnetic field $\vec{B}$ around a closed \textbf{Amperian loop} is proportional to the net current $I_{\text{enc}}$ piercing the surface bounded by that loop.
\[\boxed{\int \vec{B}\cdot d\vec{l}=\mu_0 I_{\text{enc}}}\]
We may arbitrarily choose a direction for integration around the Amperian loop and initially assume that the magnetic field $\vec{B}$ is aligned with this direction. If the resulting value of $\vec{B}$ is negative, this indicates that the true direction of the magnetic field is opposite to the assumed direction.\\\\
The signs for the current enclosed can be determined using the right-hand rule:
\begin{multicols}{2}
\begin{figure}[H]
    \centering
    \includegraphics[width=0.3\textwidth]{RHR4.png}
\end{figure}
\begin{itemize}
    \item Curl fingers around the Amperian loop, with fingers pointing in the direction of integration ($d\vec{l}$).
    \item Your thumb will point in the direction of positive current piercing the surface.
\end{itemize}
\end{multicols}
\underline{Note:} Generally, for a CCW integration direction, current coming out of the page is positive and current going into the page is negative.\\\\
The magnetic field determined using Ampere's law is the superposition of the fields produced by all currents, including those outside the Amperian loop. However, similar to Gauss's law, only the current enclosed by the Amperian loop contributes to the net magnetic circulation around the closed path, while the contribution from external currents results in zero net circulation.\\\\
Ampere's law is most useful for finding magnetic fields in situations with high symmetry, such as:
\begin{itemize}
    \item Infinite straight wire (cylindrical symmetry)
    \item Infinite solenoid (translational and cylindrical symmetry)
    \item Toroidal solenoid (cylindrical symmetry)
    \item Infinite sheet of current (planar symmetry)
\end{itemize}
In these cases, the magnetic field is uniform along the Amperian loop, allowing us to take $\vec{B}$ outside the integral.\\\\
\textbf{Magnetic Field of a Long Straight Wire with Current}\\
\textbf{Outside wire:}
\begin{figure}[H]
    \centering
    \includegraphics[width=0.4\textwidth]{Ampere long wire.jpg}
\end{figure}
\vspace{-1em}
Ampere's law:
\[\int \vec{B}\cdot d\vec{l}=\mu_0 I_{\text{enc}}\]
\[B\int dl = \mu_0 I\]
\[\boxed{B=\frac{\mu_0 I}{2\pi r} \quad \text{(outside wire)}}\]
\textbf{Inside wire (with uniform current density):}
\begin{figure}[H]
    \centering
    \includegraphics[width=0.4\textwidth]{Ampere long wire 1.jpg}
\end{figure}
\vspace{-1em}
Current density:
\[J=\frac{I}{\pi R^2}\]
\[\implies I_{\text{enc}} = J \pi r^2 = I \frac{r^2}{R^2}\]
Ampere's law:
\[\int \vec{B}\cdot d\vec{l}=\mu_0 I_{\text{enc}}\]
\[B\int dl = \mu_0 I \frac{{r^2}}{R^2}\]
\[\boxed{B=\frac{\mu_0 I r}{2\pi R^2} \quad \text{(inside wire)}}\]
\newpage\noindent
\textbf{Magnetic Field of an Infinite Solenoid}\\
A solenoid is to magnetostatics as capacitors is to electrostatics. An ideal infinite solenoid produces a uniform magnetic field inside and \textbf{zero} magnetic field outside.\\\\
\textit{Why is it zero outside the infinite solenoid?}\\
Consider the magnetic field at a point off axis from a single current loop. When it is close, the current loop looks more like an infinite wire, magnetic field pointing left.
\vspace{-1em}
\begin{figure}[H]
    \centering
    \includegraphics[width=0.3\textwidth]{Solenoid near.jpg}
\end{figure}
\vspace{-1em}
However, as point $P$ moves further away, the current loop starts to look more like a magnetic dipole, magnetic field pointing right.
\vspace{-1em}
\begin{figure}[H]
    \centering
    \includegraphics[width=0.6\textwidth]{Solenoid far.jpg}
\end{figure}
\vspace{-1em}
The key idea is that, for an infinite solenoid, only a \textit{finite} set of nearby current loops contributes a strong leftward magnetic field at a given external point, while an \textit{infinite} set of more distant loops contributes a much weaker rightward field. The cumulative effect of these infinitely many weak contributions exactly cancels the finite strong contribution, yielding zero net magnetic field outside the solenoid. The full mathematical proof is given by Pathak (\textit{An elementary argument for the magnetic field outside a solenoid}).
\begin{multicols}{2}
    Note that in a \textbf{finite} solenoid, it is not long enough for complete cancellation to occur, so there will be a small magnetic field outside the solenoid (similar to the fringe electric fields in a finite capacitor).
    \begin{figure}[H]
    \centering
    \includegraphics[width=0.29\textwidth]{Ideal Solenoid.png}
\end{figure}
\end{multicols}
\newpage
Calculating the magnetic field inside an infinite solenoid using Ampere's law:
\begin{figure}[H]
    \centering
    \includegraphics[width=0.5\textwidth]{Solenoid.jpg}
\end{figure}
Suppose an Amperian loop with side length $L$ that encloses $N$ loops of wire, each loop carries current $I$
\[\int \vec{B}\cdot d\vec{l}=\mu_0 I_{\text{enc}}\]
\[\int_b B dl = \mu_0 NI \quad \text{(Zero magnetic circulation along sides $a,c,d$)}\]
\[B_{\text{inside}}=\frac{\mu_0 NI}{L}\]
\[\boxed{B_{\text{inside}}=\mu_0 n I \quad \text{(infinite solenoid)}}\]
Where $n=\frac{N}{L}$ is the number of loops per unit length.\\\\
\underline{Note:} If the Amperian loop encloses both ends of the loop, the net current enclosed is zero, so the magnetic field outside the solenoid is zero as well.\newpage\noindent
\textbf{Magnetic Field of a Toroidal Solenoid}
\vspace{-1em}
\begin{figure}[H]
    \centering
    \includegraphics[width=0.7\textwidth]{Toroid.png}
\end{figure}
\vspace{-1em}
Suppose the toroid has $N$ loops of wire, each carrying current $I$, we will integrate clockwise around an Amperian loop of radius $r$:
\[\int \vec{B}\cdot d\vec{l}=\mu_0 I_{\text{enc}}\]
\[B(2\pi r)=\mu_0 I N\]
\[\boxed{B=\frac{\mu_0 IN}{2\pi r} \quad \text{(toroidal solenoid)}}\]
\textbf{Infinite Sheet of Current}
\vspace{-1em}
\begin{figure}[H]
    \centering
    \includegraphics[width=0.5\textwidth]{Inf Sheet 1.jpg}
\end{figure}
\vspace{-1em}
Suppose the sheet carries a current density $\rho$ (current per unit length). We will integrate around a rectangular Amperian loop of width $L$ and height $H$:
\begin{multicols}{2}
\begin{figure}[H]
    \includegraphics[width=0.4\textwidth]{Inf Sheet.jpg}
\end{figure}
\[\int \vec{B}\cdot d\vec{l}=\mu_0 I_{\text{enc}}, \quad I_{\text{enc}} = \rho L\]
\[B(2L)=\mu_0 \rho L\]
\[\boxed{B=\frac{\mu_0 \rho}{2} \quad \text{(infinite sheet of current)}}\]
\underline{Note:} Due to superposition, the magnetic field above and below are uniform and point in opposite directions.
\end{multicols}
\newpage
    \item Find the magnetic force on a current carrying wire due to another current carrying wire.\\\\
\textbf{Force Between Two Parallel Current-Carrying Wires}
\begin{figure}[H]
    \centering
    \includegraphics[width=0.4\textwidth]{parallel wire.png}
\end{figure}
\vspace{-1em}
To find the force on wire b due to wire a, we first find the magnetic field produced by wire a at the location of wire b:
\[B_a = \frac{\mu_0 i_a}{2\pi d}\]
Since the distance between the wires is constant, this magnetic field is unifrom along wire b. Now we can find the force on wire b using 
\[\vec{F}_{ba} = i_b \vec{L} \times \vec{B}_a\]
\[F_{ba}=i_bLB_asin(90^\circ)=\frac{\mu_0Li_ai_b}{2\pi d}, \quad \text{($\vec{L}$ and $\vec{B}_a$ are perpendicular)}\]
Using the right-hand rule, we find that the force is toward wire a. By Newton's third law, the force on wire a due to wire b is equal in magnitude and opposite in direction.\\\\
\underline{Note:} Parallel currents attract, and anti-parallel currents repel.
    \item Describe the forces and torques on magnetic dipoles in terms of their magnetic moment.\\\\
Recall that the torque on a magnetic dipole in a magnetic field is given by:
\[\vec{\tau} = \vec{\mu} \times \vec{B} \implies U = -\vec{\mu} \cdot \vec{B}, \qquad \text{where } \vec{\mu} = NIA\hat{n}\]
We know that potential energy is related to force by:
\[\boxed{\vec{F} = -\nabla U}\]
\newpage\noindent
\underline{Ex:} Force between a bar magnet and a circular loop of current\\\\
Since both the bar magnet and the current loop produce magnetic dipoles, we can imagine that they will behave like two bar magnets where like poles repel and opposite poles attract.
\vspace{-1em}
\begin{figure}[H]
    \centering
    \includegraphics[width=0.9\textwidth]{dipole.jpg}
\end{figure}
\vspace{-1em}
Recall that the magnetic field along the axis of a magnetic dipole (current loop) is: 
\[B_z=\frac{\mu_0}{4\pi}\frac{2\mu}{(R^2+z^2)^{3/2}}\]
Taking the limit $R \ll z$ (far field approximation):
\[B_{z}=\frac{\mu_0}{4\pi}\frac{2\mu}{z^3}\]
Thus, the potential energy of the current loop in the magnetic field of the bar magnet is:
\[U=-\vec{\mu}_{loop} \cdot \vec{B}_{bar,z} = -\mu_{loop}\frac{\mu_0}{4\pi}\frac{2\mu_{bar}}{z^3}\]
Now let's find the magnitude of the force on the current loop due to the bar magnet:
\[\boxed{||\vec{F}_z||=\left|-\frac{dU}{dz}\right|=\frac{\mu_0}{4 \pi}\frac{6\mu_{bar}\,\mu_{loop}}{z^4}}\]
\begin{multicols}{2}
    We can find the direction of the force using the right-hand rule (also by intuition by looking at the poles). Note that this force is equal and opposite for the force on the bar magnet due to the current loop.
\begin{figure}[H]
    \centering
    \includegraphics[width=0.5\textwidth]{dipole 1.jpg}
\end{figure}
\end{multicols}
\textbf{Force from a Magnetic Mirror (*work in progress*)}\\
A charged particle can be confined to a region of space by a magnetic field that is stronger at the ends than in the middle. As the particle approaches the stronger field (shown by the more closely spaced field lines), a component of the magnetic force pushes it back toward the center of the region, reflecting it back and forth between the two ends.\\\\
In a magnetic mirror, a charged particle moves in a helical path with
\[\vec{v} = v_{\parallel} \hat{B} + v_{\perp} \hat{\perp}\]
With a component along the magnetic field and a component perpendicular to it.
\vspace{-1em}
\begin{figure}[H]
    \centering
    \includegraphics[width=0.5\textwidth]{Fields_in_magnetic_bottles.jpg}
\end{figure}
Magnetic moment of a positive charged particle (with an axis of rotation tangent to the local magnetic field line):
\[T=\frac{2\pi r_L}{v_{\perp}}=\frac{2\pi}{\omega},
\qquad \omega=\frac{qB}{m}\]
For a single loop \(N=1\), the magnetic dipole moment magnitude is
\[\|\mu\| = IA, \qquad I=\frac{q}{T}=\frac{q\omega}{2\pi}, \qquad A=\pi r_L^2, \qquad
r_L=\frac{m v_{\perp}}{qB}\]
Substituting,
\[\|\mu\|
= \frac{q\omega}{2\pi}\,\pi r_L^2
= \frac{q\omega}{2}\,r_L^2
= \frac{q}{2}\left(\frac{qB}{m}\right)\left(\frac{m v_{\perp}}{qB}\right)^2\]
\[\boxed{\|\mu\|=\frac{m v_{\perp}^2}{2B}}\]
Since the magnetic force does no work on a charged particle and there is no electric field, the total kinetic energy is conserved:
\[K=\frac{1}{2}m v_{\perp}^{2}+\frac{1}{2}m v_{\parallel}^{2} = \text{constant}\]
Rewriting the perpendicular kinetic energy in terms of the magnetic moment,
\[\frac{1}{2}m v_{\perp}^{2} = \mu B\]
Thus the total kinetic energy becomes
\[K = \mu B + \frac{1}{2}m v_{\parallel}^{2}
= \text{constant}\]
Taking the derivative along the magnetic field lines \(s\),
\[\frac{dK}{ds}
= \mu \frac{dB}{ds}
+ m v_{\parallel} \frac{d v_{\parallel}}{ds}=0\]
Rearranging,
\[m v_{\parallel} \frac{d v_{\parallel}}{ds}
= -\mu \frac{dB}{ds}\]
Since \( \dfrac{ds}{dt} = v_{\parallel} \), this gives
\[m \frac{d v_{\parallel}}{dt}
= -\mu \frac{dB}{ds}\]
Therefore, the parallel force on the particle is
\[\boxed{F_{\parallel}= -\mu \frac{dB}{ds}}\]
\underline{Note:} We can also use try using $\vec{F}_B=q\vec{v} \times \vec{B}$ to derive this result, using Gauss's law for magnetism $\nabla \cdot \vec{B} = 0 \implies \frac{dB}{d_x}+\frac{dB}{dy}+\frac{dB}{dz}=0$ to relate the changes in $B$ along and perpendicular to the field lines. Another way is to use the conservation of energy argument along with Faraday's law to relate the changes in $v_{\perp}$ and $v_{\parallel}$ as the particle moves into regions of different magnetic field strength.\\\\
\underline{Note:} The magnetic force only reflects the particle as a result of the changing magnetic field strength (shown by the curvature of the field lines), which creates a component of the magnetic force along the direction of motion of the particle.
\end{enumerate}
\newpage
\section*{Chapter 30 \& 31.11 - Induction and Inductance}
\begin{enumerate}
    \item Understand the meaning Faraday's law of induction and be able to use it to determine the electromotive force. Explain and apply the equivalence of Faraday's law and the Lorentz force law for motional emfs.\\\\
\textbf{Faraday's Law of Induction}\\
Faraday's law states that if an arbitrary closed loop encloses a region with a changing magnetic flux, then an electromotive force (emf) will be induced in the loop. If there happens to be a conducting wire along the loop, then an induced current will exist in the wire.
\[\boxed{\mathcal{E} = -\frac{d\Phi_B}{dt}}\]
The units for magnetic flux $\Phi_B$ are Weber (Wb), where
\[1\, \text{Wb} = 1\, \text{T}\cdot \text{m}^2\]
It can also be written in integral form as:
\[\boxed{\mathcal{E} = \int \vec{E} \cdot d\vec{l} = -\frac{d}{dt}\int \vec{B}\cdot\hat{n}dA}\]
Where
\begin{itemize}
    \item $\mathcal{E}$ is the \textbf{induced emf} (in volts), representing the work done per unit charge by the induced electric field (\(\vec{E}\)) around the faradian loop (in a CCW direction).
    \item $\hat{n}$ is the unit normal vector to the surface bounded by the faradian loop.
    \item $dA$ is the differential area element of that surface.
\end{itemize}
Another way to think about Faraday's law is that there will be circulating electric fields induced in regions of changing magnetic flux.\\\\
The induced electric field has the following properties:
\begin{itemize}
    \item Do not terminate and originate on charges
    \item Form closed loops!
    \item Is present whenever the magnetic flux is changing, even in regions where there are no conductors
    \item Is a non-conservative field, meaning that the work done by the field around a closed loop is non-zero
\end{itemize}
In the case where there is a coil of wire with $N$ turns, and the coil is tightly wound so that each turn experiences the same magnetic flux, then the total emf induced in the coil is:
\[\mathcal{E}=-N\frac{d\phi_B}{dt} \quad \text{(coil of N turns)}\]
\underline{Note:} Typically, we don't need to think too hard about the direction of the induced electric field, and just need to consider the magnitude of the induced emf. The direction can be determined using Lenz's law.\\\\
By examining the terms in Faraday's law, we can see that there are a few ways to induce an emf in a loop:
\begin{itemize}
    \item Change the magnitude of the magnetic field $B$ within the coil.
    \item Change either the total area of the coil or the portion of that area that lies within the magnetic field (ex: by expanding the coil or moving it in/out of the magnetic field).
    \item Change the orientation of the loop (angle between $\vec{B}$ and $\hat{n}$). For example, by rotating the coil within the magnetic field.
\end{itemize}
\textbf{Motional emf}\\
A motional emf is induced when there is relative motion between a conductor and a magnetic field. This can be understood using either Faraday's law or the Lorentz force law.
\begin{figure}[H]
    \centering
    \includegraphics[width=0.5\textwidth]{Motional emf.jpg}
\end{figure}
\vspace{-1em}
Consider the following circuit formed by a conductor sliding on two stationary rails in the presence of a uniform magnetic field $\vec{B}$ pointing out of the page. By the Lorentz force law, we can find the emf induced in the circuit as follows:
\[\vec{F}_B=q\vec{v}\times\vec{B} \implies F_B = qvB\]
\[F_E= qE\]
The magnetic force pushes positive charges down, creating a potential difference and electric field in the conductor. The electrons stop moving when the magnetic force is balanced by the electric force (Lorentz force equalling zero):
\[F_B = F_E\]
\[qvB = qE \implies E = vB\]
The emf induced in the circuit is then:
\[\boxed{\mathcal{E} = EL = vBL}\]
\newpage\noindent
We can also find the emf using Faraday's law:
\[\mathcal{E} = \int \vec{E} \cdot d\vec{l} = -\frac{d}{dt}\int \vec{B}\cdot\hat{n}dA\]
Only the area is changing as the conductor slides, so:
\[\mathcal{E} = -\frac{d}{dt}(BLx) = -BL\frac{dx}{dt} = -BLv\]
Taking the magnitude, we get the same result as before:
\[\boxed{|\mathcal{E}| = BLv}\]
\textbf{Power and Energy}\\
Considering the same circuit as before, when there is a current $I$ flowing through the circuit, there is a magnetic force slowing down the conductor:
\[\vec{F}_B = I\vec{L} \times \vec{B} \implies F_B = ILB\]
We can find the current using Ohm's law and the emf:
\[I = \frac{\mathcal{E}}{R} = \frac{BLv}{R}\]
\[F_B =\frac{BLv}{R}LB = \frac{B^2L^2v}{R}\]
If we want the conductor to move at a constant velocity then we need to apply an external force equal in magnitude and opposite in direction to the magnetic force:
\[F_{\text{app}} =F_B= \frac{B^2L^2v}{R}\]
The power supplied by this external force is:
\[P_{\text{app}} = F_{\text{app}} v = \frac{B^2L^2v^2}{R}\]
There is no change in kinetic energy of the conductor so all this power must be dissipated as heat in the resistor. We can verify this by calculating the power dissipated in the resistor:
\[P_R = I^2 R = \left(\frac{BLv}{R}\right)^2 R = \frac{B^2L^2v^2}{R}\]
\underline{Note:} Even though there is also current in the upper and lower rails, and thus a magnetic force, these forces cancel out since they point in opposite directions, so they do not contribute to the net force on the conductor.\\\\
Also note that as a result of Lenz's law, regardless of the direction of motion of the conductor, there will be a magnetic force opposing the motion and thus requiring the external force to do \textit{positive} work to keep the conductor moving at a constant velocity. 
\newpage\noindent
\textbf{Eddy Currents}\\
Suppose that we move a solid conducting plate out of a region with a uniform magnetic field $\vec{B}$ pointing into the page. Similar to the motional emf example, the relative motion of the field and the conductor will induce a current and thus a magnetic force that opposes the motion. However, since the conductor solid, the conduction electrons will not follow a single path, but rather will swirl about within the plate, forming eddy currents. However, we can represent the effect of these eddy currents \textit{as if} it were a single induced current loop around the edge of the plate.
\begin{figure}[H]
    \centering
    \includegraphics[width=0.5\textwidth]{Eddy.png}
\end{figure}


    \item Apply Lenz's law to determine the sign of induced currents or electromotive forces.\\\\
\textbf{Lenz's Law}\\
Lenz's law states that the induced current around a loop will flow in a direction that produces a secondary magnetic field that will \textit{oppose} the \textit{change} in the magnetic flux in the loop. Note that Lenz's law still holds if there is no conducting wire to actually produce the secondary magnetic field, the direction is the same regardless.
    \item Define inductance and be able to calculate the self or mutual inductance for various (usually simple) current distributions. Know and use the reciprocity relation for mutual inductances.
    \item Be able to calculate the energy stored in magnetic fields for a given inductor using either the formula involving inductance or the energy density of the magnetic field.
    \item Describe the principle of operation of a generator and a motor.
\end{enumerate}

\section*{Chapter 32 - Maxwell's Equations}
\begin{enumerate}
    \item Describe Gauss's law for Magnetism and its applications.
    \item Describe the Ampere-Maxwell law, the displacement current, and calculate the magnetic field in a charging or discharging capacitor.
\end{enumerate}

\section*{Chapter 33 - Electromagnetic Waves}
\begin{enumerate}
    \item Test whether a given arrangement of propagating electromagnetic fields satisfies Maxwell's Equations.
    \item Understand the principles of electromagnetic waves and their basic properties, such as phase speed, relation between $\vec{k}$, $\vec{E}$, and $\vec{B}$.
    \item Define the Poynting vector or Poynting flux and be able to use it to calculate the energy carried by electromagnetic fields.
    \item Qualitatively describe how electromagnetic waves are generated. For an accelerating charge, identify the directions in which one will see the largest/smallest electric fields and describe the polarization of the electric field vector.
\end{enumerate}

\section*{Chapters 27, 30.12, \& 31 - Circuits}
\begin{enumerate}
    \item Know and be able to apply the Voltage Loop Rule and the Current Node Rule (conservation of energy and conservation of charge, respectively) to analyze a circuit. A good example is to prove the rules for adding resistors in series or in parallel.
    \item Know the relation between current and voltage for the basic circuit elements of resistor, capacitor, and inductor.
    \item Write differential equations for RLC circuits (or RL, RC, LC). Understand qualitatively the behavior of resistors, inductors, and capacitors at high and low frequencies. Be able to describe how electromagnetic energy is stored, supplied, and dissipated in such circuits.
    \item Solve the differential equations for such circuits, identify characteristic timescales, frequencies, and resonances. If initial conditions are given, be able to find solutions that satisfy those initial conditions.
    \item Know or be able to derive the complex impedances for resistors, capacitors, and inductors. Know or derive the rules for combining parallel and series impedances. Use those impedances to describe the relations among currents and voltages in circuits, both in magnitude and phase.
\end{enumerate}
\textbf{End Final Exam}
\end{document}