\documentclass[12pt]{article}
\usepackage{amsmath, amsfonts,amssymb, bm, graphicx,geometry}
\usepackage{booktabs} 
\usepackage{multicol}
\usepackage{float}
\usepackage{wrapfig}
\usepackage{bm, xcolor}
\geometry{margin=1in}
\title{}
\date{}
\author{}

\begin{document}

\begin{center}
    \Large \textbf{Dynamics Formula Sheet} \\
    \normalsize The rest of Dynamics
\end{center}
\section*{Work and Energy of Rigid Bodies}
\textbf{Review from energy of particles:}\\
Work-Energy Principle:
\[T_1 + U_{1\to2}=T_2\]
where \(T\) is kinetic energy and \(U\) is work done by all forces from state 1 to state 2.\\\\
Conservation of Mechanical Energy:
\[T_1 + V_{g1}+V_{e1}+U^{NC}_{1\to2}=T_2 + V_{g2}+V_{e2}\]
\[U_{1\to2}=\int_{A1}^{A2} \mathbf{F} \cdot d\mathbf{r}\]
\underline{Note:} Internal forces do not do work on a rigid body.\\\\
\textbf{Forces that do zero work}\\
\begin{itemize}
    \item Force applied at a fixed point (ex: reaction forces at a pin or surface)
    \item Forces acting perpendicular to motion (ex: weight and normal force when horizontal motion only)
    \item When rolling without slipping, the friction at the contact point does no work
\end{itemize}
\textbf{Work of a moment couple}
\begin{figure}[H]
    \centering
    \includegraphics[width=0.4\textwidth]{Couple.png}
\end{figure}
\[\boxed{U_{1\to2} = \int_{\theta_1}^{\theta_2} M \, d\theta}\]
For a constant moment \(M\):
\[U_{1\to2} = M(\theta_2 - \theta_1)\]
\newpage\noindent
\textbf{Kinetic Energy of a Rigid Body}\\
The total kinetic energy of a rigid body can be expressed as the sum of the kinetic energy of its center of mass and the kinetic energy due to rotation about the center of mass.
\[\boxed{T = \frac{1}{2}m\bar{v}^2+\frac{1}{2}\bar{I}\omega^2}\]
\textbf{Non-centroidal Rotation}
\begin{figure}[H]
    \centering
    \includegraphics[width=0.4\textwidth]{Non centroid.png}
\end{figure}\noindent
For rotation about a FIXED axis $o$:
\[\boxed{T=\frac{1}{2}I_o\omega^2}\]
\textbf{Power}\\
\[P=\frac{dU}{dt}=\mathbf{F} \cdot \mathbf{v}\]
\[P=\frac{dU}{dt}=M\frac{d\theta}{dt}=M\omega\]
\newpage
\section*{Impulse and Momentum of Rigid Bodies}
Like energy, total impulse and momentum of a rigid body is equal to a linear and rotational component at the center of mass.
\begin{figure}[H]
    \centering
    \includegraphics[width=0.9\textwidth]{Impulse momentum.png}
\end{figure}
\noindent
From the impulse momentum diagram, we can write 3 equations for rigid body motion:
\[\text{X-direction:}\quad
m \bar{v}_{x,1} + \sum \int_{t_1}^{t_2} \bar{\mathbf{F}}_x \, dt = m \bar{v}_{x,2}\]
\[\text{Y-direction:}\quad
m \bar{v}_{y,1} + \sum \int_{t_1}^{t_2} \bar{\mathbf{F}}_y \, dt = m \bar{v}_{y,2}\]
\[\text{Moment about point } G:\quad
\bar{I}\omega_1+ \sum \int_{t_1}^{t_2} \mathbf{M_G} \, dt
= \bar{I}\omega_2\]
We can sum moments about any point. For a point \(P\) not at the center of mass, we have:
\[\text{Moment about point } P:\quad
\bar{I}\omega_1 + m \bar{v}_1 d_{\perp 1}
+ \sum \int_{t_1}^{t_2} \mathbf{M_P} \, dt
= \bar{I}\omega_2 + m \bar{v}_2 d_{\perp 2}\]
Where \(d_{\perp}\) is the perpendicular distance from point \(P\) to the line of action of the velocity vector \(\bar{v}\), basically the "moment" of the linear momentum about point \(P\).\\\\
\underline{Note:} The angular impulse for summing moments uses MOMENTS. Take care to convert forces to moments about the point of interest.
\newpage\noindent
\textbf{Non-centroidal Rotation}
\begin{figure}[H]
    \centering
    \includegraphics[width=0.4\textwidth]{Impulse.png}
\end{figure}
\noindent
Impulse/momentum about fixed point \(o\):
\[I_o\omega_1 + \sum \int_{t_1}^{t_2} \mathbf{M_o} \, dt = I_o\omega_2\]
\textbf{Systems of Rigid Bodies}\\
For a system of rigid bodies, we can apply the impulse-momentum equations to the entire system as a whole or to each body individually. When applying to the entire system, internal forces and moments cancel out.\\\\
\textbf{Conservation of Angular Momentum}\\
If no external applied moments act on a rigid body or system of rigid bodies, the angular momentum remains constant.\\\\
\textbf{Eccentric Impact}
\begin{figure}[H]
    \centering
    \includegraphics[width=1\textwidth]{Eccentric Impact.png}
\end{figure}
\noindent
Similar to eccentric collisions in particles, we can use the coefficient of restitution \(e\) to relate the relative velocities before and after impact.
\[\boxed{(v_B')_n-(v_A')_n=e[(v_a)_n-(v_B)_n]}\]
\newpage\noindent
Even for constrained motion like below, the same equation for coefficient of restitution applies.
\begin{figure}[H]
    \centering
    \includegraphics[width=0.6\textwidth]{Fixed.png}
\end{figure}\noindent
Problem approach:
\begin{itemize}
    \item Draw impulse-momentum diagram
    \item Write impulse-momentum equations. If there is no external impulse, use conservation of momentum. If no external moment impulse, use conservation of angular momentum.
    \item Write coefficient of restitution equation
    \item Solve the system of equations
\end{itemize}

\end{document}
