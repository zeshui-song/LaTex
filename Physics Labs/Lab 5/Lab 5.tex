\documentclass[12pt]{article}
\usepackage{amsmath, amsfonts, bm, graphicx}
\usepackage{tabularx}
\usepackage{booktabs} 
\usepackage{float}

\title{Ph291E Lab 5 -- Diffraction \& Interference}
\author{
    Zeshui Song\\
    The Cooper Union
}
\date{November 27, 2025}

\begin{document}

\maketitle

\section{Purpose}  
Using a diffraction grating, we will measure the wavelength of light from a laser diode. We will then confirm this wavelength by measuring the single slit diffraction pattern produced by the same laser diode through a known aperture width. Additionally, we will measure the width of a human hair using Babinet’s principle.
\newpage
\section{Results}
\textbf{Part A:}\\
Calculated wavelengths:
\begin{itemize}
    \item $\lambda_+ = 649.7~\mathrm{nm}$
    \item $\lambda_- = 643.0 ~\mathrm{nm}$
    \item $\lambda_{avg} = 646.4 ~\mathrm{nm}$
\end{itemize}
Uncertainty in wavelength:
\begin{itemize}
    \item Half-range uncertainty: $3.4 ~\mathrm{nm}$
    \item Error propagated uncertainty $0.1 ~\mathrm{nm}$
\end{itemize}
Best value for wavelength: $\lambda = 646.4 \pm 3.4 ~\mathrm{nm}$\\\\
The uncertainty from error propagation is significantly smaller than the half-range uncertainty, indicating that systematic errors dominate over measurement errors in this calculation.\\\\
\textbf{Part B:}\\
Calculated wavelengths for aperture width $a = 0.2 ~\mathrm{mm}$:
\begin{itemize}
    \item $\lambda_{-} = 566.8 ~\mathrm{nm}$
    \item $\lambda_{+} = 552.2 ~\mathrm{nm}$
    \item $\lambda_{2-} = 639.4 ~\mathrm{nm}$
    \item $\lambda_{2+} = 550.4 ~\mathrm{nm}$
\end{itemize}
Calculated wavelengths for aperture width $a = 0.3 ~\mathrm{mm}$:
\begin{itemize}
    \item $\lambda_{-} = 318.9 ~\mathrm{nm}$
    \item $\lambda_{+} = 578.4 ~\mathrm{nm}$
    \item $\lambda_{2-} = 405.4 ~\mathrm{nm}$
    \item $\lambda_{2+} = 675.7 ~\mathrm{nm}$
\end{itemize}
Calculated wavelengths for aperture width $a = 0.4 ~\mathrm{mm}$:
\begin{itemize}
    \item $\lambda_{-} = 398.6 ~\mathrm{nm}$
    \item $\lambda_{+} = 699.3 ~\mathrm{nm}$
    \item $\lambda_{2-} = 538.5 ~\mathrm{nm}$
    \item $\lambda_{2+} = 524.5 ~\mathrm{nm}$
\end{itemize}
Uncertainty in wavelength: $32.9 ~\mathrm{nm}$ (SDOM of all calculated wavelengths)\\\\
Best value for wavelength: $\lambda = 537.3 \pm 32.9 ~\mathrm{nm}$\\\\
\textbf{Part C:}\\
Calculated hair widths using wavelength from Part A ($\lambda = 646.4 ~\mathrm{nm}$):
\begin{itemize}
\item $a_{-} = 100.9~\mu\mathrm{m}$
\item $a_{+} = 100.9~\mu\mathrm{m}$
\item $a_{2-} = 87.4~\mu\mathrm{m}$
\item $a_{2+} = 84.8~\mu\mathrm{m}$
\item $a_{3-} = 84.5~\mu\mathrm{m}$
\item $a_{3+} = 88.1~\mu\mathrm{m}$
\end{itemize}
Uncertainty in hair width: $2.9 ~\mu\mathrm{m}$ (SDOM of all calculated hair widths)\\\\
Best value for hair width: $a = 91.6 \pm 2.9 ~\mu\mathrm{m}$
\newpage
\section{Conclusion}
The wavelength of the laser diode was calculated using the intensity patterns from a diffraction grating and a single slit aperture. The results using the diffraction grating yielded a wavelength of $646.4 \pm 3.4 ~\mathrm{nm}$, while the single slit aperture method produced a wavelength of $537.3 \pm 32.9 ~\mathrm{nm}$. Comparing to typical wavelengths for red lasers (around $620-699 ~\mathrm{nm}$), only the diffraction grating result falls within this range, suggesting it is more accurate. Additionally, the single slit aperture results had a much larger uncertainty as a result of systematic errors in estimating the minima positions, and accurately setting up the aperture widths using the vernier caliper. Therefore, the wavelength determined from the diffraction grating was used to calculate the width of a human hair using Babinet's principle, resulting in a hair width of $91.6 \pm 2.9 ~\mu\mathrm{m}$. This value is consistent with typical human hair widths (ranging from $17$ to $180 ~\mu\mathrm{m}$). Therefore, after determining the wavelength of the laser diode using a diffraction grating, we were able to successfully take measurements on the scale of tens of micrometers using diffraction principles instead of using direct measurement tools such as calipers or micrometers. This shows that taking measurements via diffraction patterns can be just as effective as direct measurement methods for small scales, provided the wavelength is known accurately.
\section{Answers to questions}
\paragraph{Question 1} \mbox{}\\
N-Slit intensity:
\[I(\delta) = I_0 \left[\frac{sin\left(\frac{N\delta}{2}\right)}{sin\left(\frac{\delta}{2}\right)}\right]^2, \quad \delta = \frac{2\pi}{\lambda}dsin\theta\]
The principal maxima occur when both the numerator and denominator are zero. Thus,
\[\frac{\delta}{2} = m\pi, \quad m = 0, \pm 1, \pm 2, \ldots\]
Since N is an integer, the numerator will also be zero at these points. Therefore, the principal maxima occur at:
\[\frac{1}{2}\left(\frac{2\pi}{\lambda}dsin\theta\right)=m\pi\]
\[\frac{d}{\lambda}sin\theta=m\]
\[\implies sin\theta = \frac{m\lambda}{d}, \quad m = 0, \pm 1, \pm 2, \ldots\]
\paragraph{Question 2} \mbox{}\\
Single slit diffraction intensity:
\[I(\beta)=I_0\left(\frac{sin\beta}{\beta}\right)^2, \quad \beta = \frac{\pi}{\lambda}asin\theta\]
Minima occur when the numerator is zero (excluding the central maximum at $\beta = 0$):
\[sin\beta = 0 \implies \beta = p\pi, \quad p = \pm 1, \pm 2, \ldots\]
Thus,
\[\frac{\pi}{\lambda}asin\theta = p\pi\]
\[\implies sin\theta = \frac{p\lambda}{a}, \quad p = \pm 1, \pm 2, \ldots\]
\end{document}
