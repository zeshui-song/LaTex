\documentclass[12pt]{article}
\usepackage{amsmath, amsfonts, bm, graphicx}
\usepackage{tabularx}
\usepackage{booktabs} 
\usepackage{float}

\title{Ph291E Lab 4 -- Bessel \& Telescope}
\author{
    Zeshui Song\\
    The Cooper Union
}
\date{November 13, 2025}

\begin{document}

\maketitle

\section{Purpose}  
Using Bessel's method to determine the focal lengths of two different converging lenses, we will construct a simple refracting telescope using these lenses and determine its magnification both experimentally and theoretically.
\section{Data}
% Approx Focal Lengths
\textbf{Approximate Focal Lengths}
\begin{align*}
f_1 &= 10 \pm 0.3 \text{ cm}\\
f_2 &= 4.6 \pm 0.6 \text{ cm}\\
\end{align*}
\newpage
\textbf{Measurements for Bessel's Method Calculations for Focal Lengths}
\begin{table}[H]
\centering
\caption{Bessel's Method Data for Lens 1}
\scriptsize 
\begin{tabularx}{\linewidth}{*{5}{>{\centering\arraybackslash}X}}
\toprule
\textbf{D (cm)} & \textbf{d (cm)}  & \textbf{Random Error D (cm)}& \textbf{Random Error d (cm)} & \textbf{Optical Bench Inst. Error (cm)}\\
\midrule
43.00 & 9.85 & & & \\
43.00 & 9.85 & & & \\
43.00 & 9.25 & & & \\
43.00 & 9.20 & & & \\
43.00 & 9.60 & & & \\
43.00 & 9.75 & 0 & 0.12 & 0.05 \\
\midrule
\multicolumn{4}{l}{\textbf{Mean D}} & \textbf{43.00 cm} \\
\multicolumn{4}{l}{\textbf{Mean d}} & \textbf{9.58 cm} \\
\bottomrule
\end{tabularx}
\end{table}
Image sharpness uncertainty (for position d):
\begin{itemize}
    \item Position 1: 0.30 cm
    \item Position 2: 0.20 cm
\end{itemize}
\begin{table}[H]
\centering
\caption{Bessel's Method Data for Lens 2}
\scriptsize 
\begin{tabularx}{\linewidth}{*{5}{>{\centering\arraybackslash}X}}
\toprule
\textbf{D (cm)} & \textbf{d (cm)}  & \textbf{Random Error D (cm)}& \textbf{Random Error d (cm)} & \textbf{Optical Bench Inst. Error (cm)}\\
\midrule
24.95 & 7.45 & & & \\
24.95 & 8.25 & & & \\
24.95 & 7.55 & & & \\
24.95 & 7.95 & & & \\
24.95 & 8.20 & & & \\
24.95 & 7.30 & 0 & 0.16 & 0.05 \\
\midrule
\multicolumn{4}{l}{\textbf{Mean D}} & \textbf{24.95 cm} \\
\multicolumn{4}{l}{\textbf{Mean d}} & \textbf{7.78 cm} \\
\bottomrule
\end{tabularx}
\end{table}
Image sharpness uncertainty (for position d):
\begin{itemize}
    \item Position 1: 0.05 cm
    \item Position 2: 0.60 cm
\end{itemize}
\newpage
\section{Calculations}
\paragraph{Sample Calculations for Distance d (Lens 1)} \mbox{}\\
\textbf{Mean Calculation:}
\[\bar{x} = \frac{1}{n} \sum_{i=1}^{n} x_i\]
\[\bar{x} = \frac{1}{6} (9.85+9.85+9.25+9.20+9.60+9.75)\]
\[\bar{x} =  \text{9.5833 cm}\]
\textbf{Standard Deviation:}
\[S_x = \sqrt{\frac{1}{n-1} \sum_{i=1}^{n} (x_i - \bar{x})^2}\]
\[\begin{aligned}
S_x &= \sqrt{\frac{1}{6-1} \Big[ (9.85-9.5833)^2 + (9.85-9.5833)^2 + \cdots + (9.75-9.5833)^2 \Big]} \\
\end{aligned}\]
\[S_x =  \text{0.29268 cm}\]
\textbf{Standard Deviation of the Mean (SDOM):}
\[\sigma_{\bar{x}} = \frac{S_x}{\sqrt{n}}\]
\[\sigma_{\bar{x}} = \frac{0.29268}{\sqrt{6}}\]
\[\sigma_{\bar{x}}=  \text{0.11948 cm}\]
The same procedure was applied to calculate the mean, standard deviation and standard deviation of the mean for the remaining measurements.
\paragraph{Bessel's Method Calculations for Lens 1} \mbox{}\\
\[f=\frac{D^2-d^2}{4D}\]
\[f=\frac{(43.00)^2-(9.58)^2}{4(43.00)}\]
\[f= \text{10.22 cm}\]
\paragraph{Error Propagation for Bessel's Method (Lens 1)} \mbox{}\\
Since the measurements of D and d are independent, we can use the independent error propagation formula:
\[\delta f = \sqrt{\left(\frac{\partial f}{\partial D} \delta D\right)^2 + \left(\frac{\partial f}{\partial d} \delta d\right)^2}\]
\[\delta f = \sqrt{\left(\frac{D^2+d^2}{4D^2} \delta D\right)^2 + \left(-\frac{d}{2D} \delta d\right)^2}\]
\textbf{Chosen uncertainties:} $\delta D = 0.05~\mathrm{cm}$ (Instrumental uncertainty, larger than random) and $\delta d = 0.30~\mathrm{cm}$ (Image sharpness uncertainty, larger than random and instrumental).
\[\delta f = \sqrt{\left(\frac{(43.00)^2+(9.58)^2}{4(43.00)^2} (0.05)\right)^2 + \left(-\frac{9.58}{2(43.00)} (0.30)\right)^2}\]
\[\delta f = \text{0.036 cm}\]
\paragraph{Angular Magnification} \mbox{}\\
Where $f_{obj}$ is the focal length of the lens 1 and $f_{eye}$ is the focal length of the lens 2.
\[m_\theta=-\frac{f_{obj}}{f_{eye}}\]
\[m_\theta=-\frac{10.216}{5.630}\]
\[m_\theta= \text{-1.814}\]
\paragraph{Error Propagation for Angular Magnification} \mbox{}\\
Since the measurements of $f_{obj}$ and $f_{eye}$ are independent, we can use the independent error propagation formula:
\[\delta m_\theta = \sqrt{\left(\frac{\partial m_\theta}{\partial f_{obj}} \delta f_{obj}\right)^2 + \left(\frac{\partial m_\theta}{\partial f_{eye}} \delta f_{eye}\right)^2}\]
\[\delta m_\theta = \sqrt{\left(-\frac{1}{f_{eye}} \delta f_{obj}\right)^2 + \left(\frac{f_{obj}}{f_{eye}^2} \delta f_{eye}\right)^2}\]
\textbf{Chosen uncertainties:} $\delta f_{obj} = 0.036~\mathrm{cm}$ (calculated) and $\delta f_{eye} = 0.095~\mathrm{cm}$ (calculated).
\[\delta m_\theta = \sqrt{\left(-\frac{1}{5.630} (0.036)\right)^2 + \left(\frac{10.216}{(5.630)^2} (0.095)\right)^2}\]
\[\delta m_\theta = \text{0.031}\]
\section{Results}
Focal Lengths:
\begin{itemize}
    \item Lens 1: $f_{obj} = 10.22 \pm 0.04~\mathrm{cm}$
    \item Lens 2: $f_{eye} = 5.63 \pm 0.10~\mathrm{cm}$
\end{itemize}
Angular Magnification:
\begin{itemize}
    \item Theoretical: $m_\theta = -1.81 \pm 0.03$
    \item Experimental approximation: $m_\theta \approx -1.36$
\end{itemize}
\section{Conclusion}
The focal lengths of the two lenses were determined using Bessel's method to be $10.22 \pm 0.04~\mathrm{cm}$ for lens 1 and $5.63 \pm 0.10~\mathrm{cm}$ for lens 2. The approximate focal length of lens 1 ($10 \pm 0.3~\mathrm{cm}$) is within the uncertainty range of the calculated focal length, while the approximate focal length of lens 2 ($4.6 \pm 0.6~\mathrm{cm}$) is slightly outside the uncertainty range of the calculated focal length. The theoretical angular magnification of the telescope was calculated to be $-1.81 \pm 0.03$, while the experimental approximation yielded a value of $-1.36$. This discrepancy could be due to systemic errors such as the lenses not being set up exactly $f_{obj} + f_{eye}$ apart, and bias in estimating the lengths of tape seen through the telescope. 
\newpage
\section{Answers to questions}
\paragraph{Question 1} \mbox{}\\
\vspace{-1em}
\begin{figure}[H]
    \centering
    \includegraphics[width=1\textwidth]{1.JPG}
\end{figure}
Thin lens equation:
\[\frac{1}{f} = \frac{1}{P} + \frac{1}{i}\]
\[\frac{1}{f}=\frac{P+i}{Pi}\]
Since $D = P + i$ and $i=D-P$,
\[\frac{1}{f}=\frac{D}{P(D-P)}\]
\[f=\frac{P(D-P)}{D}=\frac{PD-P^2}{D}\]
Putting into standard form to find the 2 P values that forms a sharp image:
\[P^2-PD+fD=0\]
Using the quadratic formula:
\[P=\frac{D \pm \sqrt{D^2-4fD}}{2}\]
Since the two values for P are real and unique, the discriminant must be non-zero:
\[\boxed{D^2-4fD > 0 \implies D > 4f}\]
\begin{figure}[H]
    \centering
    \includegraphics[width=0.8\textwidth]{2.JPG}
\end{figure}\noindent
The distance between the two positions of the lens $P_2-P_1=d$ is:
\[d=P_2-P_1=\sqrt{D^2-4fD}\]
Rearranging for f:
\[d^2=D^2-4fD\]
\[4fD=D^2-d^2\]
\[\boxed{f=\frac{D^2-d^2}{4D}}\]
\paragraph{Question 2} \mbox{}\\
Thin lens equation:
\[\frac{1}{f} = \frac{1}{P} + \frac{1}{i}\]
\[\frac{1}{f}=\frac{P+i}{Pi}\]
\[f=\frac{Pi}{P+i}=\frac{i}{1+i/P}\]
For the approximate focal length we assumed $P\rightarrow\infty$, so $i/P\rightarrow0$:
\[f_{approx}=i\]
However, $P$ is finite but large, so
\[1+i/P>1\]
\[\implies f=\frac{i}{1+i/P}<i=f_{approx}\]
\end{document}
