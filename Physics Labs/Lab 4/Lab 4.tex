\documentclass[12pt]{article}

\usepackage{amsmath, amsfonts, bm, graphicx}
\usepackage{tabularx}
\usepackage{booktabs} 
\usepackage{float}

\title{Ph291E Lab 4 -- Bessel \& Telescope}
\author{
    Zeshui Song\\
    The Cooper Union
}
\date{November 13, 2025}

\begin{document}

\maketitle

\section{Purpose}  
Using Bessel's method to determine the focal lengths of two different converging lenses, we will construct a simple refracting telescope using these lenses and determine its magnification both experimentally and theoretically.
\section{Data}
\paragraph{Part A: Pfund's Method} \mbox{}\\
% Petri Dish Thickness
\begin{table}[H]
\centering
\caption{Petri Dish Thickness}
\scriptsize 
\begin{tabular}{ccccc}
\toprule
\textbf{Thickness (mm)} & \textbf{Inst. Error (mm)} & \textbf{Random Error (mm)} \\
\midrule
2.180 & & \\
2.190 & & \\
2.185 & & \\
2.185 & & \\
2.195 & & \\
2.180 & 0.005 & 0.002 \\
\midrule
\multicolumn{3}{l}{\textbf{Mean Thickness}} & \multicolumn{2}{l}{\textbf{2.186 mm}} \\
\bottomrule
\end{tabular}
\end{table}
\section{Calculations}
\section{Results}
\section{Conclusion}
\section{Answers to questions}
\paragraph{Question 1} 
\end{document}
