\documentclass[12pt]{article}
\usepackage{amsmath, amsfonts, bm, graphicx}
\usepackage{tabularx}
\usepackage{booktabs} 
\usepackage{float}

\title{Ph291E Lab 4 -- Bessel \& Telescope}
\author{
    Zeshui Song\\
    The Cooper Union
}
\date{November 13, 2025}

\begin{document}

\maketitle

\section{Purpose}  
Using Bessel's method to determine the focal lengths of two different converging lenses, we will construct a simple refracting telescope using these lenses and determine its magnification both experimentally and theoretically.
\section{Data}
% Approx Focal Lengths
\textbf{Approximate Focal Lengths}
\begin{align*}
f_1 &= \\
f_2 &= \\
\end{align*}
\textbf{Measurements for Bessel's Method Calculations for Focal Lengths}

\begin{table}[H]
\centering
\caption{Bessel's Method Data for Lens 1}
\scriptsize 
\begin{tabularx}{\linewidth}{*{5}{>{\centering\arraybackslash}X}}
\toprule
\textbf{D (cm)} & \textbf{d (cm)}  & \textbf{Random Error D (mm)}& \textbf{Random Error d (mm)} & \textbf{Inst. Error (mm)}\\
\midrule
2.180 & & & & \\
2.190 & & & & \\
2.185 & & & & \\
2.185 & & & & \\
2.195 & & & & \\
2.180 & 0.005 & 0.002 & 0.002 & 0.0005 \\
\midrule
\multicolumn{4}{l}{\textbf{Mean D}} & \textbf{7.66 mm} \\
\multicolumn{4}{l}{\textbf{Mean d}} & \textbf{7.66 mm} \\
\bottomrule
\end{tabularx}
\end{table}
\begin{table}[H]
\centering
\caption{Bessel's Method Data for Lens 2}
\scriptsize 
\begin{tabularx}{\linewidth}{*{5}{>{\centering\arraybackslash}X}}
\toprule
\textbf{D (cm)} & \textbf{d (cm)}  & \textbf{Random Error D (mm)}& \textbf{Random Error d (mm)} & \textbf{Inst. Error (mm)}\\
\midrule
2.180 & & & & \\
2.190 & & & & \\
2.185 & & & & \\
2.185 & & & & \\
2.195 & & & & \\
2.180 & 0.005 & 0.002 & 0.002 & 0.0005 \\
\midrule
\multicolumn{4}{l}{\textbf{Mean D}} & \textbf{7.66 mm} \\
\multicolumn{4}{l}{\textbf{Mean d}} & \textbf{7.66 mm} \\
\bottomrule
\end{tabularx}
\end{table}
\textbf{Approximate Angular Magnification}

\section{Calculations}
\paragraph{Length D Sample Calculations} \mbox{}\\
\textbf{Mean Calculation:}
\[\bar{x} = \frac{1}{n} \sum_{i=1}^{n} x_i\]
\[\bar{x} = \frac{1}{6} (2.180+2.190+2.185+2.185+2.195+2.180)\]
\[\bar{x} =  \text{2.1858 mm}\]
\textbf{Standard Deviation:}
\[S_x = \sqrt{\frac{1}{n-1} \sum_{i=1}^{n} (x_i - \bar{x})^2}\]
\[
\begin{aligned}
S_x &= \sqrt{\frac{1}{6-1} \Big[ (2.180-2.1858)^2 + (2.190-2.1858)^2 + \cdots + (2.180-2.1858)^2 \Big]} \\
\end{aligned}
\]
\[S_x =  \text{0.005845 mm}\]
\textbf{Standard Deviation of the Mean (SDOM):}
\[\sigma_{\bar{x}} = \frac{S_x}{\sqrt{n}}\]
\[\sigma_{\bar{x}} = \frac{0.005845}{\sqrt{6}}\]
\[\sigma_{\bar{x}}=  \text{0.0023863 mm}\]
The same procedure was applied to calculate the mean, standard deviation and standard deviation of the mean for the remaining measurements.
\paragraph{Bessel's Method Calculations for Lens 1} \mbox{}\\
\[f=\frac{D^2-d^2}{4D}\]
\[f=\frac{(90.0)^2-(30.0)^2}{4(90.0)}\]
\[f= \text{36.67 cm}\]
\paragraph{Error Propagation for Bessel's Method (Lens 1)} \mbox{}\\
Since the measurements of D and d are independent, we can use the independent error propagation formula:
\[\delta f = \sqrt{\left(\frac{\partial f}{\partial D} \delta D\right)^2 + \left(\frac{\partial f}{\partial d} \delta d\right)^2}\]
\[\delta f = \sqrt{\left(\frac{D^2+d^2}{4D^2} \delta D\right)^2 + \left(-\frac{d}{2D} \delta d\right)^2}\]
\textbf{Chosen uncertainties:} $\delta D = 0.05~\mathrm{mm}$ (random, larger than instrumental) and $\delta d = 0.005~\mathrm{mm}$ (instrumental, larger than random).
\[\delta f = \sqrt{\left(\frac{(90.0)^2+(30.0)^2}{4(90.0)^2} (0.05)\right)^2 + \left(-\frac{30.0}{2(90.0)} (0.005)\right)^2}\]
\[\delta f = \text{0.0142 cm}\]
\paragraph{Angular Magnification} \mbox{}\\
\[m_\theta=-\frac{f_{obj}}{f_{eye}}\]
\[m_\theta=-\frac{36.67}{10.0}\]
\[m_\theta= \text{-3.667}\]
\paragraph{Error Propagation for Angular Magnification} \mbox{}\\
Since the measurements of $f_{obj}$ and $f_{eye}$ are independent, we can use the independent error propagation formula:
\[\delta m_\theta = \sqrt{\left(\frac{\partial m_\theta}{\partial f_{obj}} \delta f_{obj}\right)^2 + \left(\frac{\partial m_\theta}{\partial f_{eye}} \delta f_{eye}\right)^2}\]
\[\delta m_\theta = \sqrt{\left(-\frac{1}{f_{eye}} \delta f_{obj}\right)^2 + \left(\frac{f_{obj}}{f_{eye}^2} \delta f_{eye}\right)^2}\]
\textbf{Chosen uncertainties:} $\delta f_{obj} = 0.0142~\mathrm{cm}$ (from previous calculation) and $\delta f_{eye} = 0.05~\mathrm{cm}$ (instrumental).
\[\delta m_\theta = \sqrt{\left(-\frac{1}{10.0} (0.0142)\right)^2 + \left(\frac{36.67}{(10.0)^2} (0.05)\right)^2}\] 
\[\delta m_\theta = \text{0.0191}\]
\section{Results}
\section{Conclusion}
\section{Answers to questions}
\paragraph{Question 1} 
\end{document}
