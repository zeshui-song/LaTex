\documentclass[12pt]{article}
\usepackage{amssymb}
\usepackage{geometry}
\usepackage{amsmath, amsfonts, bm, graphicx}
\usepackage{float,multicol}
\geometry{margin=1in}
\title{}
\date{}
\author{}

\begin{document}
\begin{figure}[H]
    \centering
    \includegraphics[width=0.6\textwidth]{Kinematics.png}
\end{figure}
\textbf{Using the circle intersection method:}\\
We know the locations of the fixed points $O$, $C$, and $E$:
\[O = (0,0) \quad C = (0, l_4) \quad E = (-6,l_4)\]
We know the Cartesian coordinates for points $A$ and $B$ by defining it in terms of a vector relative their respective fixed points:
\[\vec{l_1}= \left<l_2 \cos(\theta_2), l_2 \sin(\theta_2)\right> \implies A = \left<0,0\right>+\left<l_2 \cos(\theta_2), l_2 \sin(\theta_2)\right>\]
\[\vec{l_3} = \left<l_3 \cos(\theta_3), l_3 \sin(\theta_3)\right> \implies B = \left<0,l_4\right>+\left<l_3 \cos(\theta_3), l_3 \sin(\theta_3)\right>\]
We can then define the equations of the circles centered at $O$ and $C$ with radii $l_2$ and $l_3$ respectively:
\[\]

\end{document}