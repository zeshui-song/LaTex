\documentclass[12pt]{article}
\usepackage{amssymb}
\usepackage{geometry}
\usepackage{amsmath, amsfonts, bm, graphicx}
\usepackage{float,multicol}
\geometry{margin=1in}
\title{Dynamics Final Project Report}
\date{}
\author{Zeshui Song, Roy He, Ryan Lee}

\begin{document}
\maketitle
\section*{Introduction to project}
\section*{Assumptions}
\section*{Constraints: Kinematics}
Our goal is to find the angles $\theta_2$, $\theta_3$, $\theta_4$, and $\theta_5$ given the input angle $\theta_1$ and the lengths $l_1$, $l_2$, $l_3$, $l_4$, and $l_5$. We can use two methods to solve for these angles: the vector loop method and the circle intersection method.
\vspace{-1em}
\begin{figure}[H]
    \centering
    \includegraphics[width=0.5\textwidth]{Kinematics.png}
\end{figure}
\noindent
We found that it is faster and cleaner to use geometry to find the angles. To do that, we just need to find the coordinate of point $B$.
\vspace{-1em}
\begin{figure}[H]
    \centering
    \includegraphics[width=0.3\textwidth]{Geometry.png}
\end{figure}
\vspace{-2em}
\noindent
We know the coordinates of point \(A\) and point \(C\):
\[\vec{l_1} = \langle l_1 \cos \theta_1, l_1 \sin \theta_1 \rangle, \quad \vec{l_4} = \langle 0, l_4 \rangle\]
\[A=\vec{O}+\vec{l_1}, \quad C=\vec{O}+\vec{l_4}\]
We can define the vector from point \(A\) to point \(C\) and its magnitude \(d\):
\[\vec{d} = \vec{l_4} - \vec{l_1}, \quad d = \|\vec{d}\|\]
By the pythagorean theorem, we can find the height \(h\) from point \(P\) to point \(B\):
\[h^2 = l_2^2 - a^2\]
By law of cosines for triangle \(ABP\)
\[l_2^2 = a^2 + h^2 - a h \cos \theta_p, \quad \theta_p = 90^\circ\]
\[l_2^2 = a^2 + h^2 \quad (1)\]
By law of cosines for triangle \(CPB\)
\[l_3^2 = (d - a)^2 + h^2 - (d - a) h \cos \theta_p, \quad \theta_p = 90^\circ\]
\[l_3^2 = (d - a)^2 + h^2 \quad (2)\]
Solving for \(a\) using (1) and (2)
\[l_3^2 - l_2^2 + a^2 = d^2 - 2ad + a^2\]
\[a = \frac{l_3^2 - l_2^2 - d^2}{-2d}\]
Knowing \(a\) and \(h\), we can find the coordinates of point \(B\):
\[\vec{B} = \vec{A} + a \hat{d} + h \hat{d}_\perp\]
\[\hat{d} = \frac{\vec{d}}{d}\]
\[\hat{d}_\perp = \left\langle -\frac{d_y}{d}, \frac{d_x}{d} \right\rangle\]
Finally, we can define vectors \(\vec{l_2}\) and \(\vec{l_3}\) using points \(A\), \(B\), and \(C\). From there, we can find point $D$ by numerically finding the normal vector of \(\vec{l_3}\) and using the length \(l_5\). With points \(D\) and \(E\), we have fully defined the system and can find all the angles using trigonometry.
\newpage
\end{document}