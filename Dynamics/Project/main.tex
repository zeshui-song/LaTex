\documentclass[12pt]{article}
\usepackage{amssymb}
\usepackage{geometry}
\usepackage{amsmath, amsfonts, bm, graphicx}
\usepackage{float,multicol}
\geometry{margin=1in}
\title{}
\date{}
\author{}

\begin{document}
\begin{figure}[H]
    \centering
    \includegraphics[width=0.6\textwidth]{Kinematics.png}
\end{figure}
\noindent
Our goal is to find the angles $\theta_2$, $\theta_3$, $\theta_4$, and $\theta_5$ given the input angle $\theta_1$ and the lengths $l_1$, $l_2$, $l_3$, $l_4$, and $l_5$. We can use two methods to solve for these angles: the vector loop method and the circle intersection method.\\
\textbf{Using the vector loop method:}\\
\begin{figure}[H]
    \centering
    \includegraphics[width=1\textwidth]{Vector loop.png}
\end{figure}
\noindent
By numerically solving the equations, we can find angles $\theta_2$ and $\theta_3$. Following that, we know that $l_5$ is always perpendicular to $l_3$, and thus we can find $\theta_4$ by trigonometry. By using the vector from C to D $\vec{l_5}$ we can find the point $D$. Knowing both $D$ and $E$, we can find angle $\theta_5$ using trigonometry as well.\\
\textbf{Using the circle intersection method:}\\
We found that it is faster and cleaner to use the circle intersection method to define the angles and points instead.\\
We know the locations of the fixed points $O$, $C$, and $E$:
\[O = (0,0) \quad C = (0, l_4) \quad E = (-6,l_4)\]
We know the Cartesian coordinates for points $A$ and $B$ by defining it in terms of a vector relative their respective fixed points:
\[\vec{l_1}= \left<l_2 \cos(\theta_1), l_2 \sin(\theta_1)\right> \implies A = \left<0,0\right>+\left<l_2 \cos(\theta_1), l_2 \sin(\theta_1)\right>\]
\[\vec{l_3} = \left<l_3 \cos(\theta_3), l_3 \sin(\theta_3)\right> \implies B = \left<0,l_4\right>+\left<l_3 \cos(\theta_3), l_3 \sin(\theta_3)\right>\]


\end{document}