\documentclass[12pt]{article}
\usepackage{amssymb}
\usepackage{geometry}
\usepackage{amsmath, amsfonts, bm, graphicx}
\usepackage{float,multicol}
\usepackage{pdfpages}
\geometry{margin=1in}
\title{}
\date{}
\author{}

\begin{document}
\includepdf[pages=1-2]{ME200_FinalExam_Fall2025.pdf}

\section*{Problem 1. Way Down Hadestown}
\begin{figure}[H]
    \centering
    \includegraphics[width=0.5\textwidth]{P1.png}
\end{figure}
\textbf{a)} Given:
\begin{itemize}
    \item Path $ABC$ is a straight line a distance $d = 0.5 \,m$ from $O$
    \item Path $CDA$ is an arc of radius $r = 2 \,m$
    \item When Orpheus reaches point $A$, the turntable has $\omega = 0.5 \,rad/s$ and $\alpha = 0.1 \,rad/s^2$ (counterclockwise).
    \item Orpheus starts from rest at point $A$ and walks with velocity $v=0.5 t \,m/s$ relative to the turntable, where $t$ is the time in seconds since he started walking.
\end{itemize}
Find:
\begin{itemize}
    \item His velocity and acceleration at point $B$
\end{itemize}
\begin{figure}[H]
    \centering
    \includegraphics[width=0.4\textwidth]{P1a.png}
\end{figure}
\textbf{Kinematics}\\\\
Finding time $t$ when Orpheus reaches point $B$:\\
\begin{minipage}{0.5\textwidth}
    \centering
    \includegraphics[width=0.5\textwidth]{P1a1.png}
\end{minipage}
\hfill
\begin{minipage}{0.4\textwidth}
\[x_{AB} = \sqrt{r^2-d^2} \implies x_{AB} = 1.9364m\]
\end{minipage}
\[\Delta x = x_{AB} = \int_{0}^{t} v dt'\]
\[\int_{0}^{t} 0.5 t' dt' = \left[ 0.25 t'^2 \right]_{0}^{t} = 0.25 t^2\]
\[1.9364m = 0.25 t^2 \implies t = 2.783s\]
Finding $\theta_B$ when Orpheus reaches point $B$:
\[\theta_B = \omega t + \frac{1}{2}\alpha t^2\]
\[\theta_B = 0.5(2.783) + 0.5(0.1)(2.783)^2 = 1.7787 \,rad\]
Finding $\omega$ at time $t$:
\[\omega = \omega_0 + \alpha t = 0.5 + 0.1(2.783) = 0.7783 \,rad/s\]
\newpage\noindent
\textbf{Velocity}
\[\mathbf{v_P}=\mathbf{v_B}+\mathbf{\Omega}\times\mathbf{r_{P/B}}+\left( \mathbf{{v}_{P/B}} \right)\]
Where:
\begin{itemize}
    \item $\mathbf{v_P}$: Velocity of Orpheus relative to $XYZ$
    \item $\mathbf{v_B}$: Velocity of point $B$ relative to $XYZ$
    \[\mathbf{v_B}= \left< -d\omega sin\theta_B, d\omega cos\theta_B,0  \right> = \left< -0.5(0.7783) sin(1.7787), 0.5(0.7783) cos(1.7787),0  \right>\]
    \[\mathbf{v_B}= \left< -0.3807, -0.0803,0 \right>\]
    \item $\mathbf{\Omega}$: Angular velocity of turntable
    \[\mathbf{\Omega} = \omega \mathbf{k}\]
    \item $\mathbf{r_{P/B}}$: Position of Orpheus relative to point $B$
    \[\mathbf{r_{P/B}} = \left< 0,0,0 \right>\]
    \item $\mathbf{v_{P/B}}$: Velocity of Orpheus relative to turntable (point $B$)
    \[\mathbf{v_{P/B}} = \left< 0,0.5t,0 \right>\]
\end{itemize}
Plugging in values:
\[\mathbf{v_P}=\left< -0.3807, -0.0803,0  \right>+\left< 0,1.3915,0 \right>\]
\[\boxed{\mathbf{v_P}=\left< -0.3807, 1.3112,0 \right> \,m/s}\]
\newpage\noindent
\textbf{Acceleration}
\[\mathbf{a_P}=\mathbf{a_B}+\dot{\mathbf{\Omega}}\times \mathbf{r_{P/B}}+\mathbf{\Omega}\times(\mathbf{\Omega}\times\mathbf{r_{P/B}})+2\mathbf{\Omega}\times\left(\mathbf{v_{P/B}}\right)_{xyz}+\left(\mathbf{a_{P/B}}\right)_{xyz}\]
\[\mathbf{a_P}=\mathbf{a_B}+2\mathbf{\Omega}\times\left(\mathbf{v_{P/B}}\right)_{xyz}+\left(\mathbf{a_{P/B}}\right)_{xyz}\]
Find $\mathbf{a_B}$\\
where the normal direction is $\hat{n} = \left< -cos\theta, -sin\theta \right>$ and the tangential direction is $\hat{t} = \left< -sin\theta, cos\theta \right>$:
\[a_{Bn}= r\omega^2\]
\[a_{Bt}= r\alpha\]
Note that at $t = 2.783s$, $\omega = 0.7783 \,rad/s$ and $\alpha = 0.1 \,rad/s^2$:
\[\mathbf{a_B} = (0.5)(0.7783)^2 \left< -cos\theta_B, -sin\theta_B,0 \right> + (0.5)(0.1) \left< -sin\theta_B, cos\theta_B,0 \right>\]
\[\mathbf{a_B} = 0.3028 \left< -cos(1.7787), -sin(1.7787),0 \right> + 0.05 \left< -sin(1.7787), cos(1.7787),0 \right>\]
\[\mathbf{a_B} = \left< 0.01357,-0.30659, 0 \right>\]
Cross product:
\[2\mathbf{\Omega}\times\left(\mathbf{v_{P/B}}\right)_{xyz} = 2(\omega \hat{k})\times(1.3915 \hat{\jmath}) = -2(0.7783)(1.3915 \hat{\imath}) = -2.166 \hat{\imath}\]
Find $\left(\mathbf{a_{P/B}}\right)_{xyz}$:
\[\left(\mathbf{a_{P/B}}\right)_{xyz} = \frac{d}{dt}v = \frac{d}{dt} 0.5 t = 0.5 \hat{\jmath}\]
Plugging in values:
\[\mathbf{a_P} = \left< 0.01357,-0.30659, 0 \right> + \left<-2.166,0,0 \right> + \left< 0, 0.5, 0 \right>\]
\[\boxed{\mathbf{a_P} = \left< -2.1524,0.1934,0 \right> \,m/s^2}\]
\newpage\noindent
\textbf{b)} Given:
\begin{itemize}
    \item After point $B$ Orpheus continues walking along path $CDA$ at constant speed $v=1.5 \,m/s$ relative to the turntable.
    \item At point $D$, the turntable has $\omega = 0.5 \,rad/s$ and $\alpha = 0.1 \,rad/s^2$ (counterclockwise).
\end{itemize}
Find:
\begin{itemize}
    \item His velocity and acceleration at point $D$
\end{itemize}
\begin{figure}[H]
    \centering
    \includegraphics[width=0.4\textwidth]{P1b.png}
\end{figure}
\textbf{Kinematics from point A to point D}\\
From part A, we know that the distance $AB$ is 1.9364m. Thus, the distance $AC$ is $2(1.9364) = 3.8728m$. \\\\
Also from part A, we can find the time $t_{AC}$ it takes for Orpheus to reach point $C$:
\[3.8728 \,m = 0.25 t^2 \implies t_{AC} = 3.935 \,s\]
To find the arc length $s_{CD}$ from point $C$ to point $D$, we need to find the angle in the sector $AOC$. Using the same triangle from part A:
\[\theta_{AOC} = 2 \times \theta_{OAB} = 2 \times cos^{-1}\left(\frac{d}{r}\right) = 2 \times cos^{-1}\left(\frac{0.5}{2}\right) = 2.63623 \,rad\] 
Thus the arclength $s_{CD}$ is:
\[s_{CD} = r \theta_{AOC} = 2(2.63623) = 5.2724 \,m\]
Finding the total time since Orpheus started walking from point $A$ to point $D$:
\begin{itemize}
    \item Time from $A$ to $C$: $t_{AC} = 3.935 \,s$
    \item Time from $C$ to $D$:
    \[s_{CD} = v_{CD} t_{CD} \implies t_{CD} = \frac{s_{CD}}{v_{CD}} = \frac{5.2724}{1.5} = 3.5149 \,s\]
    \item Total time from $A$ to $D$:
    \[t_{AD} = t_{AC} + t_{CD} = 3.935 + 3.5149 = 7.4499 \,s\]    
\end{itemize}
Finding $\theta_D$ when Orpheus reaches point $D$:
\begin{itemize}
    \item From $A$ to $D$:
    \[\theta_D = \omega_0 t_{AD} + \frac{1}{2}\alpha t_{AD}^2\]
    \[\theta_D = 0.5(7.4499) + 0.5(0.1)(7.4499)^2 = 6.5 \,rad\]
\end{itemize}
\textbf{Velocity}\\
Using the same velocity equation from part A:
\[\mathbf{v_P}=\mathbf{v_D}+\mathbf{\Omega}\times\mathbf{r_{P/D}}+\left( \mathbf{{v}_{P/D}} \right)\]
Where:
\begin{itemize}
    \item $\mathbf{v_D}$: Velocity of point $D$ relative to $XYZ$
    \[\mathbf{v_D} = \left< -r\omega sin\theta_D, r\omega cos\theta_D  \right>\]
    \[\mathbf{v_D} = \left< -(2)(0.5) sin(6.5), (2)(0.5) cos(6.5)  \right>\]
    \[\mathbf{v_D} = \left< -0.21511, 0.97658,0 \right>\]
    \item $\mathbf{r_{P/D}}$: Position of Orpheus relative to point $D$ 
    \[\mathbf{r_{P/D}} = \left< 0,0,0 \right>\]
    \item $\mathbf{v_{P/D}}$: Velocity of Orpheus relative to turntable (point $D$)
    \[\mathbf{v_{P/D}} = \left< 1.5 sin\theta_D, -1.5 cos\theta_D,0 \right>\]
    \[\mathbf{v_{P/D}} = \left< 1.5 sin(6.5), -1.5 cos(6.5),0 \right> = \left<0.3226, -1.4648,0 \right>\]
\end{itemize}
Plugging in values:
\[\mathbf{v_P}=\left< -0.21511, 0.97658,0 \right>+\left<0.3226, -1.4648,0 \right>\]
\[\boxed{\mathbf{v_P}=\left< 0.10748, -0.48822,0 \right> \,m/s}\]
\newpage\noindent
\textbf{Acceleration}\\
Using the same acceleration equation from part A:
\[\mathbf{a_P}=\mathbf{a_D}+\dot{\mathbf{\Omega}}\times \mathbf{r_{P/D}}+\mathbf{\Omega}\times(\mathbf{\Omega}\times\mathbf{r_{P/D}})+2\mathbf{\Omega}\times\left(\mathbf{v_{P/D}}\right)_{xyz}+\left(\mathbf{a_{P/D}}\right)_{xyz}\]
\[\mathbf{a_P}=\mathbf{a_D}+2\mathbf{\Omega}\times\left(\mathbf{v_{P/D}}\right)_{xyz}+\left(\mathbf{a_{P/D}}\right)_{xyz}\]
Where:
\begin{itemize}
    \item $\mathbf{a_D}$: Acceleration of point $D$ relative to $XYZ$\\
    where the normal direction is $\hat{n} = \left< -cos\theta, -sin\theta \right>$ and the tangential direction is $\hat{t} = \left< -sin\theta, cos\theta \right>$:
\[a_{Dn}= r\omega^2\]
\[a_{Dt}= r\alpha\]
Note that at $D$, $\omega = 0.5 \,rad/s$ and $\alpha = 0.1 \,rad/s^2$:
\[\mathbf{a_D} = (0.5)(0.5)^2 \left< -cos\theta_D, -sin\theta_D,0 \right> + (0.5)(0.1) \left< -sin\theta_D, cos\theta_D,0 \right>\]
\[\mathbf{a_D} = 0.125 \left< -cos(6.5), -sin(6.5),0 \right> + 0.05 \left< -sin(6.5), cos(6.5),0 \right>\]
\[\mathbf{a_D} = \left< -0.1328,0.021939, 0 \right>\]
    \item $\mathbf{v_{P/D}}$: Velocity of Orpheus relative to turntable (point $D$)
    \[\mathbf{v_{P/D}} = \left<0.3226, -1.4648,0 \right>\]
    \item $\mathbf{a_{P/D}}$: Acceleration of Orpheus relative to turntable (point $D$)
    \[\mathbf{a_{P/D}} = \left< 0,0,0 \right>\]
\end{itemize}
Cross product:
\[2\mathbf{\Omega}\times\left(\mathbf{v_{P/D}}\right)_{xyz} = 2(\omega \hat{k})\times \left<0.3226, -1.4648,0 \right> = (2\omega\cdot1.4648) \hat{\imath} + (2\omega\cdot0.3226) \hat{j}\]
\[2\mathbf{\Omega}\times\left(\mathbf{v_{P/D}}\right)_{xyz} = \left< 1.4648, 0.3226, 0 \right>\]
Plugging in values:
\[\mathbf{a_P} = \left< -0.1328,0.021939, 0 \right> + \left< 1.4648, 0.3226, 0 \right> + \left< 0, 0, 0 \right>\]
\[\boxed{\mathbf{a_P} = \left< 1.332,0.3445,0 \right> \,m/s^2}\]
\newpage
\section*{Problem 2. Nervous Fidgeting}
\begin{figure}[H]
    \centering
    \includegraphics[width=0.5\textwidth]{P2.png}
\end{figure}
Given:
\begin{itemize}
    \item The pencil is a uniform rod of density $\rho_p = 0.9 g/cm^3$ and the eraser is a rectangular prism of density $\rho_e = 1.4 g/cm^3$.
    \item The coefficient of restitution between the pencil and finger is $e = 0.1$.
    \item Impact is impulsive and the finger does not move, only hitting tangentially
\end{itemize}
\textbf{a)} Find the 3D center of mass of the pencil-eraser system.\\\\
Finding the masses:
\begin{itemize}
    \item Mass of pencil:
    \[v_p = \pi \left(\frac{.75}{2}\right)^2(19)= 8.3939 \, cm^3\]
    \[m_p = \rho_p v_p = (0.9)(8.3939) = 7.5545 \, g\]
    \item Mass of eraser:
    \[v_e = (1)(4)(2) = 8 \, cm^3\]
    \[m_e = \rho_e v_e = (1.4)(8) = 11.2 \, g\]
\end{itemize}
\newpage\noindent
Finding the center of mass:
\begin{figure}[H]
    \centering
    \includegraphics[width=0.8\textwidth]{P2a.png}
\end{figure}
\[\bar{x}= \frac{m_p(9.5)+m_e(20)}{m_p+m_e}\]
\[\bar{x}= \frac{7.5545(9.5)+11.2(20)}{7.5545+11.2}=15.7704 \, cm\]
\[\bar{z}= \frac{m_p(0)+m_e(0)}{m_p+m_e}=0 \, cm\]
\begin{figure}[H]
    \centering
    \includegraphics[width=0.8\textwidth]{P2a1.png}
\end{figure}
\[\bar{y}= \frac{m_p(0)+m_e(0)}{m_p+m_e}=0 \, cm\]
\[\boxed{G_{\text{sys}}=\left< 15.7704, 0, 0 \right> \, cm}\]
\newpage\noindent
\textbf{b)} Calculate the moment of inertia of the pencil and eraser system about the
relevant centroidal axis.\\\\
Axis of rotation: z-axis through center of mass\\\\
Using the parallel axis theorem:
\[I_{G_{sys}} = I_{G_p} + m_p d_p^2 + I_{G_e} + m_e d_e^2\]
Where:
\begin{itemize}
    \item $I_{G_p}$: Moment of inertia of pencil about its centroidal axis
    \[I_{G_p} = \frac{1}{12} m_p L^2 = \frac{1}{12} (7.5545) (19^2) = 227.279 \, g\cdot cm^2\]
    \item $d_p$: Distance from pencil centroid to system centroid
    \[d_p = 15.7704 - 9.5 = 6.2704 \, cm\]
    \item $I_{G_e}$: Moment of inertia of eraser about its centroidal axis
    \[I_{G_e} = \frac{1}{12} m_e (b^2+ c^2) = \frac{1}{12} (11.2) (1^2 + 2^2) = 4.666 \, g\cdot cm^2\]
    \item $d_e$: Distance from eraser centroid to system centroid
    \[d_e = 20 - 15.7704 = 4.2296 \, cm\]
\end{itemize}
Plugging in values:
\[I_{G_{sys}} = 227.279 + 7.5545(6.2704)^2 + 4.666 + 11.2(4.2296)^2\]
\[\boxed{I_{G_{sys}} = 729.334 \, g\cdot cm^2}\]
\newpage\noindent
\textbf{c)}
\begin{figure}[H]
    \centering
    \includegraphics[width=1\textwidth]{P2c.png}
\end{figure}\noindent
\textbf{d)} After the pencil has fallen 1.5 m, it hits your finger. Find the angular velocity and the velocity of the center of mass just before and just after the impact.\\\\
Finding velocity of center of mass just before impact:
\[mgh = \frac{1}{2}mv^2\]
\[v = \sqrt{2gh} = \sqrt{2(9.81)(1.5)} = 5.4249 \, m/s = 542.49 \, cm/s\]
Distance from center of mass to point of impact:
\[d = (21 - 3)cm - (15.7704)6cm = 2.2296\,cm\]
Conservation of angular momentum about point of impact: 
\[m_{\text{sys}}\bar{v}_i (2.2296) = -m_{\text{sys}}\bar{v}_f (2.2296) + \bar{I}_{\text{sys}}\omega_f\]
\[41.815\bar{v}_i = -41.815\bar{v}_f + 729.334\omega_f\]
\[22678.79 = -41.815\bar{v}_f + 729.334\omega_f\]
Coefficient of restitution:
\[v_B'-v_A' = e[v_A-v_B]\]
\[-v_A' = (0.1)v_A\]
\[\implies v_A' = -0.1 (-542.49), \quad v_A = \overline{v}\]
\[v_A' = 54.249 \hat{\jmath}\]
Kinematics:
\[v_A' = \bar{v}_f+\omega\hat{k} \times r_{A/G}\]
\[v_A' = \bar{v}_f+2.2296\omega \hat{\jmath}\]
\[\implies \bar{v}_f = 54.249 \hat{\jmath} - 2.2296\omega \hat{\jmath}\]
Plugging in $\bar{v}_f$ into the angular momentum equation:
\[22678.79 = -41.815(54.249 - 2.2296\omega) + 729.334\omega\]
\[\implies \boxed{\omega_f = 30.3285 \, rad/s}\]
Finding $\bar{v}_f$:
\[\bar{v}_f = 54.249 \hat{\jmath} - 2.2296(30.3285) \hat{\jmath}\]
\[\boxed{\bar{v}_f = -13.3714 \,(cm/s)\,\hat{\jmath}}\]
Recall that just before impact:
\[\boxed{\bar{v}_i = -542.49 \,(cm/s)\,\hat{\jmath}\quad \omega_i = 0 \, rad/s}\]
\section*{Problem 3: 41CS Elevator}
\begin{figure}[H]
    \centering
    \includegraphics[width=0.6\textwidth]{P3.png}
\end{figure}
Given:
\begin{itemize}
    \item The cable does not slip
    \item Each pulley has weight $W_p = 300 lbs$, radius $r_p = 1.25 ft$, and radius of gyration $k_G = 0.7 ft$
    \item Elevator car weighs $W_e = 4100 lbs$
    \item Counterweight weighs $W_c = 5000 lbs$
    \item One of the pulleys is driven with torque $\tau$
\end{itemize}
\textbf{a)} Find the moment of inertia of each pulley about its axle.
\[\bar{I}_P = m k_G^2\]
\[\boxed{\bar{I}_P = \frac{300}{32.2} (0.7)^2 = 4.565 \, lb \cdot ft^2}\]
\textbf{b)} Find the \textit{constant} torque required for the elevator car to have speed 10 $ft/s$ after moving up 15 $ft$ from the ground floor.\\\\
Let the datum be at the ground floor and up be positive. Position 1 is at the ground floor and position 2 is at 15 ft above the ground floor. Assuming at position 1, the elevator car and the pulleys are at rest and the counterweight is at $y_0$ ft above the ground floor.
\[T_1 + V_{g1}+V_{e1}+U^{NC}_{1\to2}=T_2+V_{g2}+V_{e2}\]
\[m_{c}gy_0 + \tau \Delta \theta =\frac{1}{2}m_e v_f ^2 +\frac{1}{2}m_c v_f^2 + 2\left(\frac{1}{2}I_p\omega_f^2 \right)+m_e g (15 \, ft) +m_cg(y_0-15 \, ft) \]
\[ \tau \Delta \theta =\frac{1}{2}m_e (10 \, ft/s) ^2+\frac{1}{2}m_c (10 \, ft/s) ^2 + I_p\omega_f^2 +m_e g (15 \, ft) +m_cg(-15 \, ft) \]
Kinematics:
\[\Delta y_e = -\Delta y_c \implies v_{e} = -v_{c}\]
\[v = 10 \, ft/s = \omega_f r \quad \text{(cable does not slip)}\]
\[\implies \omega_f = \frac{10}{1.25} = 8 \, rad/s\]
Finding $\Delta \theta$:
\[\Delta \theta = \frac{\Delta y}{r} = \frac{15 \, ft}{1.25 \, ft} = 12 \, rad\]
Plugging in values:
\[\tau (12) = 
\frac{1}{2}\left(\frac{4100}{32.2}\right)(10)^2+
\frac{1}{2}\left(\frac{5000}{32.2}\right)(10)^2+
4.565(8^2)+
(4100)(15) +
(5000)(-15)\]
\[\implies \boxed{\tau = 76.8828 \, lb \cdot ft}\]
\newpage
\section*{Problem 4: The Swinging Sticks}
\begin{figure}[H]
    \centering
    \includegraphics[width=0.5\textwidth]{P4.png}
\end{figure}
Given:
\begin{itemize}
    \item Rod BAC has mass $m_1 = 64g = 0.064 \, kg$
    \item Rod BD has mass $m_2 = 43g = 0.043 \, kg$
    \item BAC has center at C, BD has center at D
    \item At the instant, rod BAC has angular velocity $\omega_1 = 2 \, rad/s$ (CCW) and rod BD has no angular velocity $\omega_2 = 0 \, rad/s$
\end{itemize}
\textbf{a)} Full system FBD
\begin{figure}[H]
    \centering
    \includegraphics[width=0.7\textwidth]{P4aa.png}
\end{figure}
\newpage\noindent
Rod BD KD:
\begin{figure}[H]
    \centering
    \includegraphics[width=0.7\textwidth]{P4a1.png}
\end{figure}
Rod BAC KD:
\begin{figure}[H]
    \centering
    \includegraphics[width=1\textwidth]{P4a2.png}
\end{figure}
\newpage\noindent
\textbf{b)} Moments of intertia about center of mass:
\begin{itemize}
    \item Rod BD:
    \[\bar{I_2} = \frac{1}{12} m_2 L_2^2 = \frac{1}{12} (0.043)(0.2^2) = 1.4333 \times 10^{-4} \, kg \cdot m^2\]
    \item Rod BAC:
    \[\bar{I_1} = \frac{1}{12} m_1 L_1^2 = \frac{1}{12} (0.064)(0.3^2) = 4.8 \times 10^{-4} \, kg \cdot m^2\]
\end{itemize}
Rod BD force balances:
\begin{align*}
    \sum{F_x} &: B_x = m_{2}\bar{a}_{2x} \\
    \sum{F_y} &: B_y - m_{2}g = m_{2}\bar{a}_{2y} \\
    \sum{M_D} &: B_x (0.01 \, m) = \bar{I_{2}}\alpha_2 \\
\end{align*}
Rod BAC force balances:
\begin{align*}
    \sum{F_x} &: -B_x + A_x = m_{1}\bar{a}_{1x} \\
    \sum{F_y} &: -B_y +A_y - m_{1}g = m_{1}\bar{a}_{1y} \\
    \sum{M_C} &: B_y (0.15 \, m) - A_y (0.07 \, m)  = \bar{I_{1}}\alpha_1 \\
\end{align*}
\noindent
Kinematics diagram:
\begin{figure}[H]
    \centering
    \includegraphics[width=0.8\textwidth]{P4b.png}
\end{figure}
\vspace{-1em}
\noindent
Velocities:
\begin{align*}
    \mathbf{v_B}=&\mathbf{v_A}+ \omega_1 \hat{k} \times \mathbf{r_{B/A}}\\
    =& (2) \hat{k} \times (-0.08 \hat{\imath})\\
    =& -0.16 \, (m/s) \, \hat{\jmath}
\end{align*}
Since rod BD is not rotating at the instant, point D and point B have the same velocity:
\[\mathbf{v_D} = \mathbf{v_B} = -0.16 \, (m/s) \, \hat{\jmath}\]
Accelerations:
\begin{align*}
    \mathbf{a_B}=&\mathbf{a_A}+\alpha_1 \hat{k}\times \mathbf{r_{B/A}} - \omega_1^2 \mathbf{r_{B/A}}\\
    =&\alpha_1 \hat{k}\times (-0.08 \hat{\imath}) - (2)^2 (-0.08 \hat{\imath})\\
    =& \left<0.32, -0.08\alpha_1, 0 \right>\\
\end{align*}
\vspace{-3em}
\begin{align*}
    \mathbf{a_C}=&\mathbf{a_A}+\alpha_1 \hat{k}\times \mathbf{r_{C/A}} - \omega_1^2 \mathbf{r_{C/A}}\\
    =&\alpha_1 \hat{k}\times (0.07 \hat{\imath}) - (2)^2 (0.07 \hat{\imath})\\
    =& \left<-0.28, 0.07\alpha_1, 0 \right>\\
    \implies \mathbf{a_C} =&\, \mathbf{\bar{a}_1} = \left<-0.28, 0.07\alpha_1, 0 \right>
\end{align*}
\vspace{-2em}
\begin{align*}
    \mathbf{a_D}=&\mathbf{a_B}+\alpha_2 \hat{k}\times \mathbf{r_{D/B}} - \omega_2^2 \mathbf{r_{D/B}}\\
    =&\left<0.32, -0.08\alpha_1, 0 \right> + \alpha_2 \hat{k}\times (-0.01 \hat{\jmath})\\
    =& \left<0.32, -0.08\alpha_1, 0 \right>+ \left<0.01\alpha_2, 0, 0 \right>\\
    =& \left<0.32 + 0.01\alpha_2, -0.08\alpha_1, 0 \right>\\
    \implies \mathbf{a_D} =&\, \mathbf{\bar{a}_2} = \left<0.32 + 0.01\alpha_2, -0.08\alpha_1, 0 \right>\\
\end{align*}
\textbf{Rewriting system of equations using kinematics}\\
Rod BD force balances:
\begin{align}
    \sum F_x &: B_x = m_{2}(0.32 + 0.01\alpha_2) \tag{1} \\
    \sum F_y &: B_y - m_{2}g = m_{2}(-0.08\alpha_1) \tag{2} \\
    \sum M_D &: B_x (0.01 \, \text{m}) = \bar{I}_{2}\alpha_2 \tag{3}
\end{align}

Rod BAC force balances:
\begin{align}
    \sum F_x &: -B_x + A_x = m_{1}(-0.28) \tag{4} \\
    \sum F_y &: -B_y + A_y - m_{1}g = m_{1}(0.07\alpha_1) \tag{5} \\
    \sum M_C &: B_y (0.15 \, \text{m}) - A_y (0.07 \, \text{m}) = \bar{I}_{1}\alpha_1 \tag{6}
\end{align}
Unknowns: $A_x, A_y, B_x, B_y, \alpha_1, \alpha_2$ (6 unknowns and 6 equations)
\newpage\noindent
\textbf{c)} Find the velocity and acceleration of point C at this instant. Leave in terms of $\mathbf{\alpha_{BD}}$ and $\mathbf{\alpha_{BAC}}$.\\\\
Using the kinematics derived in part b, where $\alpha_1 = \alpha_{BAC}$ and $\alpha_2 = \alpha_{BD}$:
\[\boxed{\mathbf{a_C} = \left<-0.28, 0.07\alpha_{BAC}, 0 \right> \, m/s^2}\]
Velocity of point C:
\vspace{-1em}
\begin{align*}
    \mathbf{v_C}=&\mathbf{v_A}+ \omega_1 \hat{k} \times \mathbf{r_{C/A}}\\
    =& (2) \hat{k} \times (0.07 \hat{\imath})\\
    =& 0.14 \, (m/s) \, \hat{\jmath}
\end{align*}
\[\boxed{\mathbf{v_C}= \left<0, 0.14, 0 \right> \, m/s}\]
\textbf{d)} Find the velocity and acceleration of point D at this instant. Leave in terms of $\mathbf{\alpha_{BD}}$ and $\mathbf{\alpha_{BAC}}$.\\\\
Using the kinematics derived in part b, where $\alpha_1 = \alpha_{BAC}$ and $\alpha_2 = \alpha_{BD}$:
\[\boxed{\mathbf{a_D} = \left<0.32 + 0.01\alpha_{BD}, -0.08\alpha_{BAC}, 0 \right> \, m/s^2}\]
Velocity of point D:
\[\boxed{\mathbf{v_D} = \mathbf{v_B} = -0.16 \, (m/s) \, \hat{\jmath}}\]
\textbf{e)} Setting up the equations in part b into matrix form to solve for the unknowns.\\\\
Rod BD force balances:
\begin{align}
    \sum F_x &: 0A_x + 0A_y + B_x + 0B_y + 0\alpha_1 - 0.01m_{2}\alpha_2= 0.32m_{2} \tag{1} \\
    \sum F_y &: 0A_x + 0A_y + 0B_x + B_y +0.08m_{2}\alpha_1 + 0\alpha_2  = m_{2}g \tag{2} \\
    \sum M_D &: 0A_x + 0A_y + 0.01B_x +0B_y+ 0\alpha_1 -\bar{I}_{2}\alpha_2 = 0 \tag{3}
\end{align}
Rod BAC force balances:
\begin{align}
    \sum F_x &:  A_x + 0A_y -B_x +0B_y +0\alpha_1 +0\alpha_2 = -0.28m_{1} \tag{4} \\
    \sum F_y &:  0A_x + A_y + 0B_x -B_y -0.07m_{1}\alpha_1 + 0\alpha_2=  m_{1}g \tag{5} \\
    \sum M_C &: 0A_x - 0.07A_y + 0B_x +0.15B_y - \bar{I}_{1}\alpha_1 + 0\alpha_2=0  \tag{6}
\end{align}
Matrix:
\[
\begin{bmatrix}
0 & 0 & 1 & 0 & 0 & -0.01 m_2 \\
0 & 0 & 0 & 1 & 0.08 m_2 & 0 \\
0 & 0 & 0.01 & 0 & 0 & -\bar{I}_2 \\
1 & 0 & -1 & 0 & 0 & 0 \\
0 & 1 & 0 & -1 & -0.07 m_1 & 0 \\
0 & -0.07 & 0 & 0.15 & -\bar{I}_1 & 0
\end{bmatrix}
\begin{bmatrix}
A_x \\ A_y \\ B_x \\ B_y \\ \alpha_1 \\ \alpha_2
\end{bmatrix}
=
\begin{bmatrix}
0.32 m_2 \\
m_2 g \\
0 \\
-0.28 m_1 \\
m_1 g \\
0
\end{bmatrix}.
\]
\newpage\noindent
Python code to solve the matrix:
\begin{verbatim}
    import numpy as np
# --------------------------------------------------
# Constants
# --------------------------------------------------
m1 = 0.064 #rod BAC kg
m2 = 0.043 #rod BD kg
I1 = 4.8e-4 #rod BAC kg*m^2
I2 = 1.4333e-4 #rod BD kg*m^2
g = 9.81 #m/s^2

# Coefficient matrix
A = np.array([
    [0,     0,     1,     0,     0,        -0.01*m2],
    [0,     0,     0,     1,     0.08*m2,   0],
    [0,     0,     0.01,  0,     0,        -I2],
    [1,     0,    -1,     0,     0,         0],
    [0,     1,     0,    -1,    -0.07*m1,   0],
    [0,    -0.07,  0,     0.15, -I1,     0]
])

# Righ hand side
b = np.array([
    0.32*m2,
    m2*g,
    0.0,
    -0.28*m1,
    m1*g,
    0.0
])

# Solve the system
x = np.linalg.solve(A, b)

# Unpack results
Ax, Ay, Bx, By, alpha1, alpha2 = x

# Print results
print(f"A_x = {Ax:.4g} N")
print(f"A_y = {Ay:.4g} N")
print(f"B_x = {Bx:.4g} N")
print(f"B_y = {By:.4g} N")
print(f"alpha_1 = {alpha1:.4g} rad/s^2")
print(f"alpha_2 = {alpha2:.4g} rad/s^2")
\end{verbatim}
\textbf{Solutions:}
\begin{itemize}
    \item $A_x = -0.003734 \, N$
    \item $A_y = 1.04 \, N$
    \item $B_x = 0.01419 \, N$
    \item $B_y = 0.4547 \, N$
    \item $\alpha_{BAC} = \alpha_1 = -9.546 \, rad/s^2$
    \item $\alpha_{BD} = \alpha_2 = 0.9897 \, rad/s^2$
\end{itemize}









\end{document}