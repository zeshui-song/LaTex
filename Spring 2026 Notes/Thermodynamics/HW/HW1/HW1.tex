\documentclass[12pt]{article}
\usepackage{amsmath}
\usepackage{amssymb}
\usepackage{geometry}
\usepackage{amsmath, amsfonts, bm, graphicx}
\usepackage{tabularx}
\usepackage{booktabs} 
\usepackage{float}
\usepackage{enumitem}
\geometry{margin=1in}

\title{}
\date{}
\author{}

\begin{document}
\section*{1-58}
\vspace{-1em}
\begin{minipage}[t]{0.6\textwidth} \vspace{0pt}
\centering
\includegraphics[width=\textwidth]{1-58.png}
\end{minipage}
\hfill
\begin{minipage}[t]{0.4\textwidth} \vspace{0pt}
Given:
\begin{itemize}
    \item Spring force, $F_{s} = 150\,\text{N}$
    \item Mass of piston, $m = 3.2\,\text{kg}$
    \item Area of piston, $A = 35\,\text{cm}^2$
    \item Atmospheric pressure, $P_{atm} = 95\,\text{kPa}$
\end{itemize}
Find: 
\begin{itemize}
    \item Pressure of gas inside cylinder, $P_{gas}$
\end{itemize}
\end{minipage}\\\\
\textbf{Drawing a FBD of the piston:}
\begin{figure}[H]
    \centering
    \includegraphics[width=0.3\textwidth]{1-58_1.png}
\end{figure}\noindent
Using the formula for pressure: $\boxed{P = F/A}$, we can find the force exerted by the gas on the piston and write the vertical force balance:
\[\sum F_y = P_{gas}A - F_{s} - mg - P_{atm}A = 0\]
\[\implies P_{gas} = \frac{F_{s} + mg}{A} + P_{atm}\]
Plugging in the values:
\[
P_{\text{gas}}
= \frac{
[150\,\text{N}]
+ [3.2\,\text{kg}]\,[9.81\,\text{m/s}^2]\,
\left[\frac{\text{N}}{\text{kg}\cdot\text{m/s}^2}\right]
}{
[35\,\text{cm}^2]\left[\frac{(10^{-2} \text{m})^2}{\text{cm}^2}\right]
}
+ [95\,\text{kPa}]\left[\frac{10^3 \text{Pa}}{\text{kPa}}\right]
= 146826.28\,\text{Pa}
\approx \boxed{147\,\text{kPa}}
\]

\section*{1-70}
\vspace{-1em}
\begin{minipage}[t]{0.6\textwidth} \vspace{0pt}
\centering
\includegraphics[width=\textwidth]{1-70.png}
\end{minipage}
\hfill
\begin{minipage}[t]{0.4\textwidth} \vspace{0pt}
Given:
\begin{itemize}
    \item The gauge pressure of blood is measured with a column of mercury of height \[h_{Hg} = 120\,\text{mm}\]
    \item $\rho_{blood} = 1050\,\text{kg/m}^3$
    \item $\rho_{Hg} = 13600\,\text{kg/m}^3$
\end{itemize}
Find: 
\begin{itemize}
    \item Height of blood column, $h_{blood}$
\end{itemize}
\end{minipage}
\vspace{-2.5em}
\begin{figure}[H]
    \centering
    \includegraphics[width=0.3\textwidth]{1-70_1.png}
\end{figure}\vspace{-1em}\noindent
For a given gauge pressure, we can find the height/depth of a fluid column using the hydrostatic pressure formula:
\[P= \rho gh\]
Since the pressure $P$ exerted by the blood is the same regardless of whether we measure it using a column of mercury or a column of blood, we can set the two hydrostatic pressure equations equal to each other:
\[P = \rho_{Hg}gh_{Hg} = \rho_{blood}gh_{blood}\]
\[\implies h_{blood} = \frac{\rho_{Hg}h_{Hg}}{\rho_{blood}}\]
Plugging in the values:
\[h_{blood} = \frac{[13600\,\text{kg/m}^3][0.12\,\text{m}]}{[1050\,\text{kg/m}^3]} = 1.554\,\text{m} \approx \boxed{1.55\,\text{m}}\]
\underline{Note:} The pressures are gauge pressures because they are measured relative to atmospheric pressure, so we don't have to account for atmospheric pressure in our calculations.

\section*{1-90}
\vspace{-1em}
\begin{minipage}[t]{0.6\textwidth} \vspace{0pt}
\centering
\includegraphics[width=\textwidth]{1-90.png}
\end{minipage}
\hfill
\begin{minipage}[t]{0.4\textwidth} \vspace{0pt}
Given:
\begin{itemize}
    \item $m_1 = 25\,\text{kg}$
    \item $m_2 = 1900\,\text{kg}$
    \item $D_1 = 10\,\text{cm}$
\end{itemize}
Find: 
\begin{itemize}
    \item $D_2$ such that $m_1$ lifts $m_2$
\end{itemize}
\end{minipage}\\\\
\textbf{Pascal's Principle} states that the pressure applied to a confined fluid increases the pressure throughout the fluid by the same amount. Thus, the pressure applied on piston 1 is equal to the pressure applied on piston 2 ($P_1 = P_2$). Solving this equation will give us the minimum diameter $D_2$ such that $m_1$ can lift $m_2$.
\[P_1 = \frac{F_1}{A_1} = \frac{m_1 g}{\frac{1}{4}\pi D_1^2}, \quad P_2 = \frac{F_2}{A_2} = \frac{m_2 g}{\frac{1}{4}\pi D_2^2}\]
\[\implies \frac{m_1 g}{\frac{1}{4}\pi D_1^2} = \frac{m_2 g}{\frac{1}{4}\pi D_2^2}\]
\[\implies D_2 = D_1 \sqrt{\frac{m_2}{m_1}}\]
Plugging in the values:
\[D_2 = [10\,\text{cm}] \sqrt{\frac{[1900\,\text{kg}]}{[25\,\text{kg}]}} = 87.177\,\text{cm} \approx \boxed{87.2\,\text{cm}}\]

\section*{1-105}
\vspace{-1em}
\begin{minipage}[t]{0.6\textwidth} \vspace{0pt}
\centering
\includegraphics[width=\textwidth]{1-105.png}
\end{minipage}
\hfill
\begin{minipage}[t]{0.4\textwidth} \vspace{0pt}
Given:
\begin{itemize}
    \item Gauge pressure inside pressure cooker, $P_{in,gauge} = 100\,\text{kPa}$
    \item The opening area of the valve, $A = 4\,\text{mm}^2$
    \item Atmospheric pressure, $P_{atm} = 101\,\text{kPa}$
\end{itemize}
Find: 
\begin{itemize}
    \item Mass of petcock required to open at pressure $P_{in,gauge}$
\end{itemize}
Assuming that the atmospheric pressure acts on an area equal to $A$.
\end{minipage}\\\\
FBD of the petcock:
\vspace{-2em}
\begin{figure}[H]
    \centering
    \includegraphics[width=0.25\textwidth]{1-105_1.png}
\end{figure}\vspace{-1em}\noindent
\underline{Recall:} 
$P_{gauge} = P_{abs} - P_{atm} \implies P_{abs} = P_{gauge} + P_{atm}$\\\\
The vertical force balance on the petcock is:
\[\sum F_y = P_{in,abs}A - P_{atm}A - mg = 0\]
\[\implies (P_{in,gauge} + P_{atm})A - P_{atm}A - mg = 0\]
\[ \implies P_{in,gauge} A - mg = 0 \implies m = \frac{P_{in,gauge} A}{g}\]
Plugging in the values:
\[m = \frac{[100\,\text{kPa}]\left[\frac{10^3 \text{Pa}}{\text{kPa}}\right][4\,\text{mm}^2]\left[\frac{(10^{-3} \text{m})^2}{\text{mm}^2}\right]}{[9.81\,\text{m/s}^2]} = 0.04077\,\text{kg} \approx \boxed{40.8\,\text{g}}\]

\section*{1-110}
\textbf{How measuring with a manometer works:}
\begin{itemize}
    \item \textbf{Basic single fluid manometer:}
    \begin{figure}[H]
        \centering
        \includegraphics[width=0.3\textwidth]{Manometer.png}
    \end{figure}
    The pressure in a fluid does not change with horizontal position (equal depth), so the pressure at points 1 and 2 are equal:
    \[P_1 = P_2\]
    Where we use the hydrostatic pressure formula to express the pressures at points 1 and 2:
    \[P_1 = P_{gas}, \quad P_2 = P_{atm} + \rho g h\]
    \item \textbf{Pressure with multiple layers of fluid:}
    \begin{figure}[H]
        \centering
        \includegraphics[width=0.3\textwidth]{Stacked Fluid.png}
    \end{figure}
    We use the following rules to find the pressure at different points in a fluid with multiple layers:
    \begin{enumerate}
        \item The pressure change across a fluid layer is given by the hydrostatic pressure formula: $\Delta P = \rho g h$
        \item Pressure increases downward (+) and decreases upward (-).
        \item The pressure at points of equal depth in a fluid are equal (Pascal's Principle).
    \end{enumerate}
    Ex: At point 1 in the figure, $P_1 = P_{atm}+\rho_1g h_1+\rho_2 g h_2 + \rho_3 g h_3$
\end{itemize}
\textbf{Solving 1-110:}\\
\vspace{-1em}
\begin{minipage}[t]{0.6\textwidth} \vspace{0pt}
\centering
\includegraphics[width=\textwidth]{1-110.png}
\end{minipage}
\hfill
\begin{minipage}[t]{0.4\textwidth} \vspace{0pt}
Given:
\begin{itemize}
    \item $P_{gauge} = 370\,\text{kPa}$
    \item Specific gravities
    \begin{itemize}
        \item Oil: $SG_{oil} = 0.79$
        \item Mercury: $SG_{Hg} = 13.6$
        \item Gasoline: $SG_{gas} = 0.70$
    \end{itemize}
    \item Height differences:
    \begin{itemize}
        \item $h_{water} = 45\,\text{cm}$
        \item $h_{oil} = 50\,\text{cm}$
        \item $h_{Hg} = 10\,\text{cm}$
        \item $h_{gas} = 22\,\text{cm}$
    \end{itemize}
\end{itemize}
Find: 
\begin{itemize}
    \item Gauge pressure of gas line.
\end{itemize}
\end{minipage}\\\\\\
\textbf{Specific gravity} is defined as a relative density compared to water (at 4$^\circ$C) where $\rho_{water} = 1000\,\text{kg/m}^3$:
\[\boxed{SG = \frac{\rho_{fluid}}{\rho_{water}} \implies \rho_{fluid} = SG \cdot \rho_{water}}\]
We will apply the pressure rules for multiple layers of fluid to the manometer:
\begin{enumerate}
    \item We will begin at a known pressure point (gauge or atmospheric pressure) and follow the fluid layers to the unknown pressure point.
    \item Sign convention: add when going down, subtract when going up.
    \item \textbf{Horizontal jump:} We can "jump" horizontally across bends in the tube if both sides of the jump are within the \textbf{same continuous fluid}. This is because pressure is identical at the same horizontal level within a single static fluid. \textit{Any pressure decrease from moving upward is perfectly balanced by an equal pressure increase when moving back down to that same level on the other side.}
    \item Thus, we only calculate vertical heights between \textbf{interfaces}.
\end{enumerate}
Therefore:
\[P_{gauge} - \rho_{water}gh_{water} + \rho_{oil}gh_{oil} - \rho_{Hg}gh_{Hg} - \rho_{gas}gh_{gas} = P_{gas}\]
\[\implies P_{gas} = P_{gauge} + \rho_{water}g(-h_{water} + SG_{oil}h_{oil} - SG_{Hg}h_{Hg} - SG_{gas}h_{gas})\]
Plugging in the values:
\[P_{gas} = [370\,\text{kPa}]+[1000\,\text{kg/m}^3][9.81\,\text{m/s}^2]\bigg(-[0.45\,\text{m}] + [0.79][0.5\,\text{m}] - [13.6][0.1\,\text{m}] - [0.70][0.22\,\text{m}]\bigg)\]
\[P_{gas} = [370\,\text{kPa}]-
\left[15391.89\,\frac{kg}{m\cdot s^2}\right]
\left[\frac{N}{kg \cdot m/s^2}\right]
\left[\frac{Pa}{N/m^2}\right]\]
\[= [370\,\text{kPa}]-
[15.391\,\text{kPa}] = 354.609\,\text{kPa} \approx \boxed{354.6\,\text{kPa}}\]

\section*{1-118}
\vspace{-1em}
\begin{figure}[H]
    \centering
    \includegraphics[width=0.6\textwidth]{1-118.png}
\end{figure}
\textbf{List of relevant temperature conversions:}
\begin{itemize}[label=\textbullet]
    \item Absolute temperature conversions
    \begin{itemize}[label=\textopenbullet]
        \item $T_{K} = T_{^\circ C} + 273.15$
        \item $T_{^\circ R} = T_{^\circ F} + 459.67$
        \item $T_{^\circ F} = \frac{9}{5}T_{^\circ C} + 32$
        \item $T_{^\circ R} = \frac{9}{5}T_{K}$
    \end{itemize}
    \item Temperature difference conversions
    \begin{itemize}[label=\textopenbullet]
        \item $\Delta K = \Delta ^\circ C$
        \item $\Delta ^\circ R = \Delta ^\circ F$
        \item $\Delta ^\circ F = \frac{9}{5}\Delta ^\circ C$
        \item $\Delta ^\circ R = \frac{9}{5}\Delta K$
    \end{itemize}
\end{itemize}
\underline{Note:} Fahrenheit and Celsius are relative temperature scales (based on the freezing/boiling points of water), while Rankine and Kelvin are absolute temperature scales (starting at absolute zero).\\\\
Plug and chug for 1-118 gives \boxed{d)\,18\,R}
\end{document}