\documentclass[12pt]{article}
\usepackage{amsmath}
\usepackage{amssymb}
\usepackage{geometry}
\usepackage{amsmath, amsfonts, bm, graphicx}
\usepackage{tabularx}
\usepackage{booktabs} 
\usepackage{float}
\geometry{margin=1in}

\title{}
\date{}
\author{}

\begin{document}

\begin{center}
    \section*{\LARGE\underline{Ch 1: Intro and Basic Concepts}}
\end{center}

\section*{Thermodynamics and Energy}
\textit{Thermodynamics} is the science of energy, the transformation of energy, and the accompanying change in the state of matter. Classical thermodynamics methods look at end conditions or \textit{states} where these states are in equilibrium.\\\\
The \textbf{First Law of Thermodynamics} states energy is conserved and the energy of the universe stays constant.\\\\
The \textbf{Second Law of Thermodynamics} states that while the total energy of the universe remains constant, every time a process occurs (we do something with the energy, like to get work), we reduce the amount of \textit{usable} energy. \textbf{Entropy} is a measure of this loss of usable energy.

\section*{Systems and Control Volumes}
A \textit{system} is a quantity of matter or a region in space chosen for study. The \textit{surroundings} is everything outside the system. The \text{boundary} is the real or imaginary surface that separates the system from its surroundings (note that boundary is the contact surface between the system and its surroundings).\\\\
A \textbf{closed system} (or \textit{control mass}) encloses a fixed amount of matter, and no mass crosses its boundary. A \textbf{open system} (or \textit{control volume}) encloses a region in space through which mass may flow through its boundary.\\\\
\underline{Note}: In both closed and open systems, energy (in the form of heat or work) may cross the system boundary.\\\\
A \textbf{isolated system} is a special type of closed system that does not exchange either mass or energy with its surroundings.\\\\
\underline{Ex}: Piston and cylinder device (closed system) and nozzle(open system)
\begin{figure}[H]
    \centering
    \includegraphics[width=0.3\textwidth]{Piston.png}
    \hspace{0.05\textwidth}
    \includegraphics[width=0.3\textwidth]{Nozzle.png}
\end{figure}\noindent
\textbf{Control surfaces} are the boundaries of control volumes. They may be real or imaginary, fixed or moving.

\section*{State and Equilibrium}
The \textbf{state} of a system is defined by the complete set of its properties at a given moment. At a fixed state, all properties have definite values, and a change in any property results in a change of state.\\\\
Thermodynamics deals with \textbf{equilibrium states}, which are states of balance in which no unbalanced driving forces exist.
\begin{itemize}
\item \textbf{Thermal equilibrium} exists when there is a uniform temperature throughout the system. 
\item \textbf{Mechanical equilibrium} exists when there are no unbalanced forces within the system or between the system and its surroundings, typically related to pressure (it is in mechanical equilibrium if no change in pressure at any point in the system over time). 
\item \textbf{Phase equilibrium} exists when different phases of a system (solid, liquid, gas) are in equilibrium with no changes in mass. 
\item \textbf{Chemical equilibrium} exists when there are no changes in the chemical composition of a system over time (no chemical reactions).
\end{itemize}
The \textbf{state} of a system is defined by its properties, but it is not necessary to specify all properties to fix the state. Once a sufficient number of properties are specified, all remaining properties are determined automatically. This principle is expressed by the \textbf{state postulate}:

\begin{quote}
\textbf{State Postulate:} The state of a \emph{simple compressible system} is completely specified by two independent, intensive properties.
\end{quote}\noindent
A \textbf{simple compressible system} is one in which electrical, magnetic, gravitational, motion, and surface tension effects are negligible. If any of these effects are significant, an additional property must be specified for each effect.\\\\
The two properties used to fix the state must be \textbf{independent}, meaning that one property can be varied while the other is held constant. For single-phase systems, temperature and pressure are independent properties, whereas in multiphase systems they are dependent and cannot fully specify the state.

\section*{Processes and Cycles}
Any change that a system undergoes from one equilibrium state to another is called a \textbf{process}, and the series of states through which a system passes during a process is called the \textbf{path} of the process.\\\\
\underline{Note}: To fully define a process, the initial and final states must be specified, as well as the path taken between these states and the interactions with the surroundings.\\\\
A \textbf{cycle} is a special type of process in which a system returns to its initial state at the end of the process. In a cycle, all properties of the system return to their original values, and the net change in any property over the cycle is zero.\\\\
When a process occurs very slowly, allowing the system to remain in near-equilibrium states throughout, it is called a \textbf{quasi-static} or \textbf{quasi-equilibrium} process. Quasi-static processes are idealized processes that help in analyzing thermodynamic systems, as they allow the use of equilibrium thermodynamics to describe the system's behavior. \\\\
\underline{Ex}: $P-V$ diagram of a compression process
\begin{figure}[H]
    \centering
    \includegraphics[width=0.4\textwidth]{Process.png}
\end{figure}\noindent
The prefix \textit{iso-} denotes a process in which a particular property remains constant. For example, an 
\begin{itemize}
\item \textbf{Isothermal} process has constant temperature $T$

\item \textbf{Isobaric} process has constant pressure $P$

\item \textbf{Isochoric} (or \textbf{Isometric}) process has constant specific volume $V$
\end{itemize}
\newpage

\section*{The Steady-Flow Process}
\textit{Steady} means no change in time.\\\\
\textit{Unsteady} or \textit{transient} means change in time.\\\\
\textit{Uniform} means no change in space.\\\\
Many engineering devices operate for long periods under consistent conditions and are classified as \textit{steady-flow devices}. 
A \textbf{steady-flow process} is one in which a fluid flows through a control volume steadily: properties may vary throughout the region, but at any fixed point they remain constant in time.\\\\
Consequently, the volume $V$, mass $m$, and total energy $E$ of the control volume remain constant during the process.\\\\
Steady-flow conditions are closely approximated in devices intended for continuous operation, such as turbines, pumps, boilers, condensers, heat exchangers, power plants, and refrigeration systems.\\\\
\underline{Note:} Cyclic devices like reciprocating engines or compressors, where flow pulsates, can be analyzed as steady-flow by using time-averaged properties.

\section*{Pressure}
\textbf{Pressure} is defined as a normal force exerted by a fluid per unit area. In solids, it is referred to as \textit{normal stress}. Note that stress is a tensor while pressure is a scalar.\\\\
Pressure has units of \textbf{pascals} where 1 Pa = 1 $\text{N/m}^2$.
\begin{itemize}
    \item 1 bar = $1 \times 10^5$ Pa
    \item 1 atm = 101.325 kPa = 1.01325 bar = 14.696 psi
    \item 1 kgf/cm$^2$ = 9.807 N/cm$^2$ = 98.07 kPa
    \item 1 kgf/cm$^2$ = 0.96784 atm
    \item 1 kgf/cm$^2$ = 0.9807 bar
    \item 1 psi = 1 pound-force/in$^2$ = 6894.76 Pa = 6.89476 kPa
\end{itemize}
The actual pressure relative to absolute vacuum is called \textbf{absolute pressure} $P_{abs}$. Pressure measured relative to atmospheric pressure is called \textbf{gauge pressure} $P_{gauge}$. Pressure lower than atmospheric pressure is called \textbf{vacuum pressure} $P_{vacuum}$. The relationships between these pressures are:\\\\
\begin{align*}
    P_{gauge} &= P_{abs} - P_{atm} \\
    P_{vacuum} &= P_{atm} - P_{abs}
\end{align*}
\begin{figure}[H]
    \centering
    \includegraphics[width=0.8\textwidth]{Gauge pressure.png}
\end{figure}\noindent
Hydrostatic pressure is the pressure exerted by a fluid at rest due to the force of gravity. It increases with depth in the fluid and is given by the equation:\\\\
$P = P_0 + \rho g h$ or $P = P_0 + \gamma_s h$\\\\
where:
\begin{itemize}
    \item $P$ = pressure at depth $h$ (Pa)
    \item $P_0$ = pressure at the surface of the fluid (Pa)
    \item $\rho$ = density of the fluid (kg/m$^3$)
    \item $g$ = acceleration due to gravity (m/s$^2$)
    \item $h$ = depth below the surface (m)
    \item $\gamma_s = \rho g$ = specific weight of the fluid (N/m$^3$)
\end{itemize}
\underline{Note:} This can also be used to find pressure differences between two points in a static fluid by using the difference in height between the two points and replacing $P_0$ with the pressure at the higher point.\\\\
In cases where density varies with depth, the pressure difference can be calculated using the integral form:\\\\
\[P_2 - P_1 = -\int_{h_1}^{h_2} \rho g \, dh\]
The integral is negative because pressure increases with depth (as $h$ decreases).\\\\
\textbf{Pascal's Law}\\
Pascal's law states that a change in pressure applied to a confined fluid increases the pressure throughout the fluid by the same amount. Meaning that pressure is uniform regardless of location in a static fluid. 






\end{document}