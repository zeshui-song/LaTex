\documentclass[10pt, fleqn]{article}
\usepackage[a4paper,margin=0.5in]{geometry}
\usepackage{multicol}
\usepackage{amsmath, amssymb}
\usepackage{titlesec}
\usepackage{float}
\usepackage{amsfonts, bm, graphicx}
\usepackage{enumitem}
\usepackage[font=small, skip=4pt]{caption}

\setlength{\abovedisplayskip}{6pt}
\setlength{\belowdisplayskip}{6pt}
\setlength{\abovedisplayshortskip}{4pt}
\setlength{\belowdisplayshortskip}{4pt}

\setlength{\textfloatsep}{8pt plus 2pt minus 2pt}
\setlength{\intextsep}{6pt plus 2pt minus 2pt}
\setlength{\floatsep}{8pt plus 2pt minus 2pt}

\setlength{\mathindent}{0pt}

\titleformat{\section}{\large\bfseries}{}{0em}{}
\titleformat{\subsection}{\normalsize\bfseries}{}{0em}{}
% Uppercase Volume - Wrapped in \mathord to prevent shifting
\newcommand{\vol}{\mathord{\text{\ooalign{\hidewidth $V$\hidewidth\cr\kern-1.0pt--}}}}

% Lowercase Volume - Adjusted for math mode stability
\newcommand{\svol}{\mathord{\text{\ooalign{\hidewidth $v$\hidewidth\cr\kern-0.2pt\raisebox{-0.1ex}{--}}}}}
\begin{document}

\begin{center}
    \Large \textbf{Thermodynamics Formula Sheet} \\
    \normalsize Chapter 1
\end{center}

\begin{multicols}{2}

\section*{Pressure}
\textbf{Constants}
\[\begin{aligned}
1 \text{ atm} &= 101.325 \text{ kPa} \\
&= 1.01325 \text{ bar} \\
&= 760 \text{ Torr} \\
&= 760 \text{ mmHg} \\
&= 29.92 \text{ inHg} \\
&= 14.696 \text{ psi}
\end{aligned}\]
\textbf{Pressure Conversions}
\[1\,\text{Pa} = 1\,\text{N/m}^2\]
\[1\, \text{ bar} = 10^{5}\, \text{ Pa}\]
\[1\, \text{ psi} = 6.89476 \times 10^{3}\, \text{ Pa}\]
\[1\, \text{ Torr} = 133.322\, \text{ Pa}\]
\[1\, \text{ mmHg} = 133.322\, \text{ Pa}\]
\[1\, \text{ inHg} = 3.38639 \times 10^{3}\, \text{ Pa}\]
\textbf{Gauge vs Absolute Pressure}
\[P_{\text{gauge}} = P_{\text{absolute}} - P_{\text{atmospheric}}\]
\[P_{\text{vacuum}} = P_{\text{atmospheric}} - P_{\text{absolute}}\]
\underline{Note:} Generally, gauge pressures already account for atmospheric pressure, and thus reads zero when open to the atmosphere.\\
\textbf{Formulas}
\[P=F/A\]
\[P = \rho g h\quad \text{(hydrostatic pressure)}\]
\textbf{Multi-fluid Manometer}
\begin{enumerate}
    \item Begin at a known pressure point (gauge or atmospheric pressure) and follow the fluid layers to the unknown pressure point.
    \item Sign convention: add when going down, subtract when going up.
    \item \textbf{Horizontal jump:} We can "jump" horizontally across bends in the tube if both sides of the jump are within the \textbf{same continuous fluid}. This is because pressure is identical at the same horizontal level within a single static fluid. \textit{Any pressure decrease from moving upward is perfectly balanced by an equal pressure increase when moving back down to that same level on the other side.}
    \item The pressure at each end of the manometer equals the total pressure obtained by summing the contributions of all fluid columns along the vertical path.
\end{enumerate}
\textbf{Pascal's Principle}\\
The pressure applied to a confined fluid increases the pressure throughout the fluid by the same amount.
\newcolumn
\section*{Temperature}
\textbf{Temperature Conversions}
\begin{itemize}[label=\textbullet]
    \item Absolute temperature conversions
    \begin{itemize}[label=\textopenbullet]
        \item $T_{K} = T_{^\circ C} + 273.15$
        \item $T_{^\circ R} = T_{^\circ F} + 459.67$
        \item $T_{^\circ F} = \frac{9}{5}T_{^\circ C} + 32$
        \item $T_{^\circ R} = \frac{9}{5}T_{K}$
    \end{itemize}
    \item Temperature difference conversions
    \begin{itemize}[label=\textopenbullet]
        \item $\Delta K = \Delta ^\circ C$
        \item $\Delta ^\circ R = \Delta ^\circ F$
        \item $\Delta ^\circ F = \frac{9}{5}\Delta ^\circ C$
        \item $\Delta ^\circ R = \frac{9}{5}\Delta K$
    \end{itemize}
\end{itemize}
\underline{Note:} Fahrenheit and Celsius are relative temperature scales (based on the freezing/boiling points of water), while Rankine and Kelvin are absolute temperature scales (starting at absolute zero).
\section*{SI Prefixes}
\makeatletter
\@fleqnfalse
\makeatother
\[\begin{aligned}
\text{femto (f)}  &= 10^{-15} \\
\text{pico (p)}   &= 10^{-12} \\
\text{nano (n)}   &= 10^{-9} \\
\text{micro ($\mu$)}  &= 10^{-6} \\
\text{milli (m)}  &= 10^{-3} \\
\text{centi (c)}  &= 10^{-2} \\
\text{deci (d)}   &= 10^{-1} \\
\text{deca (da)}  &= 10^{1} \\
\text{hecto (h)}  &= 10^{2} \\
\text{kilo (k)}   &= 10^{3} \\
\text{mega (M)}   &= 10^{6} \\
\text{giga (G)}   &= 10^{9} \\
\text{tera (T)}   &= 10^{12} \\
\text{peta (P)}   &= 10^{15} \\
\end{aligned}\]
\makeatletter
\@fleqntrue
\makeatother
\underline{Note:} $1\text{Angstrom}\,(\text{\AA}) = 10^{-10}$ m
\end{multicols}\noindent
\underline{Note: $1\,\text{lbf} = 32.174\,\text{lbm} \cdot \text{ft/s}^2$}
\newpage
\makeatletter
\@fleqnfalse
\makeatother
\section*{System Properties}
\begin{center}
\begin{tabular}{|l|l|}
\hline
\textbf{Extensive Properties} & \textbf{Intensive Properties} \\
\hline
Temperature: $T$ & $T$ \\
Pressure: $P$ & $P$ \\
Volume: $\vol$ & Specific Volume: $\svol = \vol / m$ \\
Internal Energy: $U$ & Specific Internal Energy: $u = U / m$ \\
Entropy: $S$ & Specific Entropy: $s = S / m$ \\
\hline
\end{tabular}
\end{center}
\section*{Density and Specific Gravity}
\[\rho = \frac{m}{\vol}\quad \text{(density)}\]
\[\svol = \frac{\vol}{m} = \frac{1}{\rho} \quad \text{(specific volume)}\]
\textbf{Specific gravity} is defined as a relative density compared to water (at 4$^\circ$C) where $\rho_{water} = 1000\,\text{kg/m}^3$:
\[\boxed{SG = \frac{\rho_{fluid}}{\rho_{water}} \implies \rho_{fluid} = SG \cdot \rho_{water}}\]
\section*{Ideal Gas Law}
\[
P \, \vol = m \, \bar{R} \, T, \quad
\bar{R} = \frac{R_u}{M} \quad \text{(specific gas constant)}
\]
Where:
\begin{itemize}[label=\textbullet]
    \item $P$ = \textit{absolute} pressure (Pa)
    \item $\vol$ = volume (m$^3$)
    \item $m$ = mass of gas (kg)
    \item $T$ = \textit{absolute} temperature (K)
    \item Universal gas constant: $R_u = 8.314$ J/(mol·K)
    \item $M$ = molar mass of gas (kg/mol)
    \item $\bar{R}$ = specific gas constant (J/kg·K)
\end{itemize}

\end{document}
