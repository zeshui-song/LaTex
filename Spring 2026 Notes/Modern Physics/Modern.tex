\documentclass[12pt]{article}
\usepackage{amssymb, amsmath, amsfonts, bm, graphicx, geometry}
\usepackage{tabularx, booktabs, float, enumitem}
\geometry{margin=1in}

\title{PH-214 Modern Physics Notes}
\date{}
\author{}

\begin{document}
\maketitle
\vspace{-6em}
\section*{Useful Math}
\begin{itemize}
    \item Taylor series expansions:
\begin{flalign*}
\sin x &= \sum_{n=0}^{\infty} (-1)^n \frac{x^{2n+1}}{(2n+1)!} 
       = x - \frac{x^3}{3!} + \frac{x^5}{5!} - \frac{x^7}{7!} + \cdots & \\[6pt]
\cos x &= \sum_{n=0}^{\infty} (-1)^n \frac{x^{2n}}{(2n)!} 
       = 1 - \frac{x^2}{2!} + \frac{x^4}{4!} - \frac{x^6}{6!} + \cdots & \\[6pt]
\tan x &= \sum_{n=1}^{\infty} \frac{B_{2n} (-4)^n (1-4^n)}{(2n)!}\,x^{2n-1} 
       = x + \frac{x^3}{3} + \frac{2x^5}{15} + \frac{17x^7}{315} + \cdots & \\[6pt]
(1+x)^m &= \sum_{n=0}^{\infty} \binom{m}{n} x^n 
       = 1 + mx + \frac{m(m-1)}{2}x^2 + \frac{m(m-1)(m-2)}{6}x^3 + \cdots &
\end{flalign*}
\underline{Note:} For small x, higher order terms reduce to zero

    \item Use complex exponentials to manipulate complicated trig functions (Euler's Identity).
\[e^{ix} = \cos x + i \sin x\]
\end{itemize}
\textbf{Del Operator:}

\newpage
\section*{Unit 1: Electromagnetic Waves}
\textbf{Maxwell's Equations}
\begin{enumerate}
    \item \textbf{Gauss's Law for Electricity}\\
    Relates the electric flux through a closed 3D Gaussian surface to the total charge enclosed within that surface.
    \[\oint_S \vec{E}\cdot\,d\vec{a}=\frac{q_{\text{enc}}}{\varepsilon_0}\]
     There exists a diverging electric field if there exists a non-zero charge density at that point.
    \[\vec{\nabla}\cdot\vec{E}=\frac{\rho}{\varepsilon_0}\]

    \item \textbf{Gauss's Law for Magnetism}\\
    Any 3D Gaussian surface will have zero net magnetic flux (no magnetic monopoles).
    \[\oint_S \vec{B}\cdot\,d\vec{a}=0\]
    There exists no diverging magnetic field at any point in space because there are no magnetic monopoles.
    \[\vec{\nabla}\cdot\vec{B}=0\]

    \item \textbf{Faraday's Law of Induction}\\
    Relates the electric circulation around a closed Faradian loop to the changing magnetic flux through the surface bounded by the loop.    
    \[\oint_P \vec{E}\cdot d\vec{l}=-\frac{d\Phi_B}{dt}, \quad \Phi_B = \oint_S \vec{B}\cdot\,d\vec{a}\]
    \underline{Note:} For coils with $N$ turns, multiply the flux by $N$.\\\\
    A changing magnetic field induces a circulating (curling) electric field.
    \[\vec{\nabla}\times\vec{E}=-\frac{\partial \vec{B}}{\partial t}\]

    \item \textbf{Ampere-Maxwell Law}\\
    Relates the magnetic circulation around a closed Amperian loop to the enclosed current and changing electric flux through the surface bounded by the loop.
    \[\oint_P \vec{B}\cdot d\vec{l}= \mu_0I_{\text{enc}}+\mu_0\varepsilon_0 \frac{d\Phi_E}{dt},\quad \Phi_E=\oint_S \vec{E}\cdot\,d\vec{a}\]
    Both a changing electric field and a non-zero current density induce a circulating (curling) magnetic field.
    \[\vec{\nabla}\times\vec{B}= \mu_0\vec{J}+\mu_0\varepsilon_0 \frac{\partial \vec{E}}{\partial t}, \quad I_{\text{enc}}=\oint_S \vec{J}\cdot d\vec{a}\]
\end{enumerate}



\end{document}