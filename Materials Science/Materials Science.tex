\documentclass[12pt]{article}
\usepackage{amsmath}
\usepackage{amssymb}
\usepackage{geometry}
\usepackage{amsmath, amsfonts, bm, graphicx}
\usepackage{tabularx}
\usepackage{booktabs} 
\usepackage{float}
\usepackage{wrapfig}
\geometry{margin=1in}
\title{}
\date{}
\author{}

\begin{document}

\textbf{SI Prefixes}
\[\begin{aligned}
\text{femto (f)}  &= 10^{-15} \\
\text{pico (p)}   &= 10^{-12} \\
\text{nano (n)}   &= 10^{-9} \\
\text{micro ($\mu$)}  &= 10^{-6} \\
\text{milli (m)}  &= 10^{-3} \\
\text{centi (c)}  &= 10^{-2} \\
\text{deci (d)}   &= 10^{-1} \\
\text{deca (da)}  &= 10^{1} \\
\text{hecto (h)}  &= 10^{2} \\
\text{kilo (k)}   &= 10^{3} \\
\text{mega (M)}   &= 10^{6} \\
\text{giga (G)}   &= 10^{9} \\
\text{tera (T)}   &= 10^{12} \\
\text{peta (P)}   &= 10^{15} \\
\end{aligned}\]
\underline{Note:} $1\text{Angstrom}\,(\text{\AA}) = 10^{-10}$ m
\newpage
\section*{\LARGE\underline{Mechanical Properties}}
\section*{Stress and Strain}
\textbf{Types of loading}
\begin{figure}[H]
    \centering
    \includegraphics[width=0.8\textwidth]{types of loading.png}
\end{figure}
\newpage
\noindent \textbf{Stress (Force Normalized by Area)}\\\\
Tensile and Compression Stress
\[\sigma = \frac{F}{A_o} \quad \text{(in units of pressure)}\]
Shear Stress
\[\tau = \frac{F}{A_o}\]
\noindent \textbf{Strain (Displacement Normalized by Original Length)}\\\\
Tensile and Compression Strain
\[\epsilon = \frac{l - l_0}{l_0}\]
Shear Strain
\[\gamma = \tan\theta \quad \text{($\theta$ is the shear angle)}\]

\noindent \textbf{Normal and Shear Stress Along an Angled Plane}
\begin{figure}[H]
    \centering
    \includegraphics[width=0.8\textwidth]{geometric stress.png}
\end{figure}
\newpage

\section*{Elastic Deformation}
\textbf{Relationship Between Stress and Strain}\\\\
Tensile and Compression
\[\sigma = E\epsilon \quad \text{(E (GPa or psi) is the modulus of elasticity)}\]
Shear
\[\tau = G\gamma \quad \text{(G is the shear modulus)}\]
\underline{Note:} The modulus of elasticity (Young's modulus) is the slope of the stress - strain plot. (It describes a material's resistance to elastic deformation. Stiffer $\implies$ higher E)\\\\
\textbf{Anelasticity:} time dependent elastic strain, where deformation and recovery is not instantaneous.\\
\textbf{Viscoelastic behavior}: materials (such as polymers) with significant anelasticity\\\\

\noindent \textbf{Poisson's Ratio}
\begin{figure}[H]
    \centering
    \includegraphics[width=0.8\textwidth]{poissons.jpg}
\end{figure}
\[ \nu = -\frac{\epsilon_{\text{lateral}}}{\epsilon_{\text{longitudinal}}}\]
\underline{Note:} Lateral is perpendicular to the direction of loading and longitudinal is along the direction of loading
\newpage
\underline{Example:} Rectangular prism
\begin{figure}[H]
    \centering
    \includegraphics[width=0.4\textwidth]{poissons 2.png}
\end{figure}
\noindent If the applied stress is uniaxial (only along 1 axis) and the material is isotropic (constant properties regardless of direction), then for a $\sigma_z$, $\epsilon_x=\epsilon_y$
\[ \nu = -\frac{\epsilon_x}{\epsilon_z}= -\frac{\epsilon_y}{\epsilon_z}\]

\noindent \textbf{Relating modulus of elasticity, shear modulus and Poisson's ratio}
\[E=2G(1+\nu)\]
\underline{Note:} Some materials (like foams) expand under tension so they have a negative Poisson's ratio, these materials are called \textbf{auxetics}.
\newpage
\section*{Plastic Deformation}
\begin{wrapfigure}{l}{0.4\textwidth}
    \centering
    \includegraphics[width=0.4\textwidth]{yield.jpg}
\end{wrapfigure}
Point P is the \textbf{Proportional Limit} where the exact departure from linearity occurs and deformation becomes permanent.
\\\\
\textbf{Yield Stress ($\sigma_y$):} stress at which noticeable strain has occurred (0.002)
\\\\\\\\\\\\\\\\\\\\\\\\\\\\\\\\\\\\\\
\begin{figure}[H]
    \centering
    \includegraphics[width=0.55\textwidth]{tensile.jpg}
\end{figure}
\noindent\textbf{Tensile Strength}: Stress at the maximum point on the stress - strain plot. After this point, necking occurs and all deformation is focused at the neck until fracture (point F)

\noindent\textbf{Ductility}\\\\
As $\%$ elongation:
\[\%EL = \frac{l_f - l_0}{l_0} \times 100\]
As $\%$ reduction in area
\[\%RA = \frac{A_0 - A_f}{A_0} \times 100\]
$l_f$ and $A_f$ are length and cross-sectional area of sample at fracture respectively.\\\\
\noindent\textbf{Resilience:} capacity of a material to absorb energy when it is deformed elastically and unloaded (similar to spring potential energy)\\\\
\noindent\textbf{Modulus of Resilience}
\[U_r = \int_{0}^{\epsilon_{yield}}\sigma d\epsilon\]
Area under the stress - strain plot from 0 to yield point\\\\
Assuming a linear elastic region:
\[U_r=\frac{1}{2}\sigma_y\epsilon_y\]
\newpage
\section*{\LARGE\underline{Crystal Structures}}
\noindent\textbf{Atomic Packing Factor}
\[ APF = \frac{\text{Volume of atoms in unit cell}}{\text{Total unit cell volume}}\]
\noindent\textbf{Packing Fraction}
\[PF=\frac{\text{Total Cross Sectional Area of Atoms}}{\text{Total Area of Plane}}\]
\noindent\textbf{Number of Atoms per Unit Cell}
\[N=N_i+\frac{N_f}{2}+\frac{N_c}{8}\]
$N_i$ are interior atoms, $N_f$ are face atoms and $N_c$ are corner atoms
\section*{Simple Cubic}
\begin{figure}[H]
    \centering
    \includegraphics[width=0.6\textwidth]{SC.jpg}
\end{figure}
\[2R=a\]
\[APF=\frac{(\text{\# atoms})(\text{volume/atom})}{(\text{volume/unit cell})}\]
\[APF=\frac{(1)(\frac{4}{3}\pi(a/2)^3)}{a^3}\]
\section*{Body Centered Cubic}
\begin{figure}[H]
    \centering
    \includegraphics[width=0.6\textwidth]{BCC.png}
\end{figure}
\[\text{Triangle formed along the main diagonal and face diagonal}\]
\begin{figure}[H]
    \centering
    \includegraphics[width=0.3\textwidth]{BCC diag.png}
\end{figure}
\[4R=\sqrt{3}a\]
\[APF=\frac{(\text{\# atoms})(\text{volume/atom})}{(\text{volume/unit cell})}\]
\[APF=\frac{(2)(\frac{4}{3}\pi(\sqrt{3}/4a )^3)}{a^3}\]
\newpage

\section*{Face Centered Cubic}
\begin{figure}[H]
    \centering
    \includegraphics[width=0.6\textwidth]{FCC.png}
\end{figure}
Along the face diagonal:
\[4R=\sqrt{2}a\]
\[APF=\frac{(\text{\# atoms})(\text{volume/atom})}{(\text{volume/unit cell})}\]
\[APF=\frac{(4)(\frac{4}{3}\pi(\sqrt{2}/4a )^3)}{a^3}\]
\newpage
\section*{Hexagonal Close Packed}
\begin{figure}[H]
    \centering
    \includegraphics[width=0.6\textwidth]{HCP.jpg}
\end{figure}
\begin{figure}[H]
    \centering
    \includegraphics[width=0.4\textwidth]{Untitled.jpg}
\end{figure}
Area of base hexagon is 3 parallelograms or 6 equilateral triangles:
\[\text{Area}=\frac{3a^2\sqrt{3}}{2}\]
Given height c:
\[\text{Volume of unit cell}=\frac{3a^3\sqrt{3}}{2}\]
\newpage
\section*{Theoretical Density for Crystals}
\[\rho=\frac{(\text{atoms/unit cell})(\text{g/mol})}{(\text{vol/unit cell})(\text{atoms/mol})}=(\text{g/vol})\]
\[\rho=\frac{nA}{V_c N_A}\]
Where:
\begin{itemize}
    \item n = \# of atoms in unit cell
    \item A = atomic weight
    \item $V_c$ = volume of unit cell
    \item $N_a$ = Avogadro's number ($6.022 \times 10^{23 }\text{ atoms/mol}$)
\end{itemize}
\section*{Ceramic Crystal Structures}
Factors that determine crystal structure:
\begin{itemize}
    \item Relative sizes of ions ($\frac{r_{cation}}{r_{anion}}$)
    \item Maintenance of charge neutrality (Net charge in ceramic is zero)
\end{itemize}
\underline{Note:} As $\frac{r_{cation}}{r_{anion}}$ increases, so does coordination number
\begin{figure}[H]
    \centering
    \includegraphics[width=0.8\textwidth]{ceramic.png}
\end{figure}
\newpage
\noindent\textbf{Theoretical Density for Ceramics}
\[\rho =\frac{n'(\sum A_C+\sum A_A)}{V_c N_A}\]
Where:
\begin{itemize}
    \item n' = \# of Atoms per unit cell (For AX structures, this is equal for cations and anions)
    \item $\sum A_A$ = sum of cation molar mass
    \item $\sum A_C$ = sum of anion molar mass
    \item $V_c$ = volume of unit cell
    \item $N_A$ = Avogadro's number ($6.022 \times 10^{23} \text{ atoms/mol}$)
\end{itemize}
\section*{Rock Salt Structure}
\begin{figure}[H]
    \centering
    \includegraphics[width=0.4\textwidth]{Rocksalt.png}
\end{figure}
\underline{Note:} Cations prefer octahedral sites (in black)\\
Along the edges:
\[2R_A+2R_C=a\]
\section*{AX Crystal Structure (Cesium Chloride)}
\begin{figure}[H]
    \centering
    \includegraphics[width=0.4\textwidth]{AX.png}
\end{figure}
\underline{Note:} Cations prefer cubic sites (Body Center, in blue) \\
Across the main diagonal:
\[2R_A+2R_C=\sqrt{3}a\]
\section*{AX$_2$ Crystal Structures (Flourite)}
\begin{figure}[H]
    \centering
    \includegraphics[width=0.4\textwidth]{AX2.png}
\end{figure}
\underline{Note:} Cations prefer cubic sites (Body Center, in blue) \\
There are half as many Ca$^{2+}$ as F$^-$ (for CaF$_2$)
\section*{ABX$_3$ Crystal Structure (Perovskite)}
\begin{figure}[H]
    \centering
    \includegraphics[width=0.4\textwidth]{ABX3.png}
\end{figure}
\section*{Point Coordinates}
To find the coordinate indices (q, r, s), find the Cartesian coordinates and divide by the corresponding lattice parameter
\[(q,r,s) = \left( \frac{x}{a}, \frac{y}{b}, \frac{z}{c} \right)\]
\section*{Crystallographic Directions}
How to define:
\begin{itemize}
    \item Position vector to pass through origin
    \item Read off projections onto coordinate axes in terms of lattice parameters (a, b, c)
    \item Multiply through by common denominator
    \item Enclose in square brackets without commas, negatives go on top (ex: [$\overline{1} 2 3$])
\end{itemize}
How to read [123]:
\begin{itemize}
    \item Divide by the common denominator used previously (say 6)
    \item Vector in Cartesian: (1/6,1/3,1/2)
\end{itemize}
\section*{Crystallographic Planes}
How to define with Miller indices:
\begin{itemize}
    \item Define any origin
    \item Read intercepts of the plane with the coordinate axes in terms of lattice parameters (a, b, c)
    \item Take reciprocals of intercepts
    \item Enclose in parentheses without commas, negatives go on top (ex: ($\overline{1} 2 3$))
\end{itemize}
How to read:
\begin{itemize}
    \item Take reciprocals of plane to identify intercepts
\end{itemize}
\section*{Linear and Planer Density}
\textbf{Linear Density}
\[LD = \frac{\text{Number of atom centered on line}}{\text{Unit length of direction vector}}\]
\textbf{Planer Density}
\[PD=\frac{\text{Number of atoms centered on plane}}{\text{Area of plane}}\]
\newpage
\section*{\LARGE\underline{Point Defects}}
\begin{itemize}
    \item Vacancies: Missing atoms in the lattice
    \begin{figure}[H]
        \centering
        \includegraphics[width=0.5\textwidth]{vacancy.png}
    \end{figure}
    \item Interstitials: Extra atoms positioned between lattice sites
    \begin{figure}[H]
        \centering
        \includegraphics[width=0.5\textwidth]{intersitial.png}
    \end{figure}
\end{itemize}
\section*{Equilibrium Concentration of Vacancies}
\[ N_v = Ne^{\left( \frac{-Q_v}{kT} \right)}\]
Where:
\begin{itemize}
    \item $N_v$ = number of vacancies
    \item N = total number of atomic sites (each lattice site can be occupied by an atom or be a vacancy)
    \item $Q_v$ = activation energy required to form a vacancy (J or eV)
    \item k = Boltzmann's constant ($8.617 \times 10^{-5}$ eV/atom-K)
    \item T = temperature (K)
\end{itemize}
\newpage
\section*{Alloys}
Alloys are solid solutions of an impurity element (solute) dissolved in a base element (solvent). There are two types:
\begin{figure}[H]
    \centering
    \includegraphics[width=0.8\textwidth]{alloy.png}
\end{figure}
\noindent\textbf{Hume-Rothery rule}\\\\
Conditions for forming a solid solution:
\begin{itemize}
    \item Atomic radii must be within 15\% of each other ($\Delta r < 0.15\%$)
    \item Electronegativity values must be similar (i.e. Close together on the periodic table)
    \item Crystal structures of solvent and solute must be the same (for pure metals)
    \item Valencies should be similar (All else being equal, a metal will have a greater tendency to dissolve a metal of higher valency than one of lower valency)
\end{itemize}
\newpage
\noindent\textbf{Maximum Radius for an Interstitial Impurity}\\\\
\noindent\textbf{FCC}
\begin{figure}[H]
    \centering
    \includegraphics[width=0.7\textwidth]{FCC radius.png}
\end{figure}
\vspace{-4em}
\[2r+2R=a\]
\[r=\frac{a-2R}{2}\]
\[\boxed{r=\frac{2R\sqrt{2}-2R}{2}=0.41R}\]
\noindent\textbf{BCC}
\begin{figure}[H]
    \centering
    \includegraphics[width=0.7\textwidth]{BCC radius.png}
\end{figure}
\vspace{-2em}
\[\left(\frac{a}{2}\right)^2+\left(\frac{a}{4}\right)^2=(R+r)^2\]
\[\left(\frac{4R}{2\sqrt{3}}\right)^2+\left(\frac{4R}{4\sqrt{3}}\right)^2=R^2+2Rr+r^2\]
\[r^2+2Rr-0.667R^2=0\]
\[\Rightarrow r=\frac{-2R\pm\sqrt{(2R)^2-4(1)(-0.667R^2)}}{2(1)}\]
\[\boxed{r=0.291R}\]
\newpage
\section*{Specification of composition}
\noindent\textbf{Weight Percent}
\[C_1=\frac{m_1}{m_1+m_2}\times 100\]
Where
\begin{itemize}
    \item $C_1$ = weight percent of element 1
    \item $m_1$ = mass of element 1
    \item $m_2$ = mass of element 2 
\end{itemize}
\noindent\textbf{Atomic Percent}
\[C_1'=\frac{n_{m1}}{n_{m1}+n_{m2}}\times 100\]
Where
\begin{itemize}
    \item $C_1'$ = atomic percent of element 1
    \item $n_{m1}$ = number of moles of element 1
    \item $n_{m2}$ = number of moles of element 2
\end{itemize}
\underline{Note:}
\[n_{m1} = \frac{m_1 \text{(grams)}}{A_1 \text{(g/mol)}}\]
\noindent\textbf{Conversion between wt\% and at\%}
\[C_1'=\frac{C_1A_2}{C_1A_2 + C_2A_1}\times 100\]
Where $A_1$ and $A_2$ are the atomic weights of elements 1 and 2 respectively
\newpage
\section*{Diffusion}
\textbf{Diffusion:} The general process where atoms move from one place to another, causing mass transport within or between materials.\\[4pt]
\textbf{Interdiffusion:} Occurs when atoms from two different metals migrate into each other, forming a region where the metals mix.\\[4pt]
\textbf{Self-diffusion:} Happens in pure metals when atoms of the same kind exchange places, even though there is no concentration difference.
\section*{Diffusion Mechanisms}
For diffusion to occur: (1) an adjacent site must be vacant, and (2) the atom must have enough energy to break bonds and distort the lattice. Only a fraction of atoms at a given temperature have sufficient vibrational energy for diffusion, and this fraction increases with temperature. For metals, two main models describe this atomic motion.
\begin{itemize}
    \item \textbf{Vacancy Diffusion:} Atoms move into adjacent vacant lattice sites.
    \begin{figure}[H]
        \centering
        \includegraphics[width=0.6\textwidth]{vacancy diff.png}
    \end{figure}
    \item \textbf{Interstitial Diffusion:} Smaller atoms move through the spaces between larger atoms.
    \begin{figure}[H]
        \centering
        \includegraphics[width=0.6\textwidth]{interstitial diff.png}
    \end{figure}
\end{itemize}
\newpage
\section*{Fick's First Law}
We quantify how fast diffusion occurs by defining the rate of mass transfer or the \textbf{diffusion flux} (J). It is the amount of substance that diffuses through a unit area per unit time.
\[J=\frac{M}{At} \quad (kg/m^2\cdotp s \text{ or } atoms/m^2\cdotp s)\]
Where:
\begin{itemize}
    \item M = mass of diffusing substance (kg or atoms)
    \item A = cross-sectional area (m$^2$)
    \item t = time (s)
\end{itemize}
\textbf{Fick's first law—diffusion flux for steady-state diffusion (in one direction)}
\[J=-D\frac{dC}{dx}\]
Where:
\begin{itemize}
    \item D = diffusion coefficient($m^2/s$)\\
    The negative sign indicates that the direction of diffusion is down the concentration gradient, from a high to a low concentration.
    \item $\frac{dC}{dx}$ = concentration gradient in the x direction.
\end{itemize}
\textbf{Fick's first law} describes the diffusion of atoms through a thin metal plate when the concentrations (or pressures) on both surfaces are constant. In this case, a \textbf{steady-state diffusion} is achieved, where the diffusion flux remains constant over time and there is no net accumulation of the diffusing species within the plate.
\begin{figure}[H]
    \centering
    \includegraphics[width=0.8\textwidth]{ficks 1.png}
\end{figure}\noindent
When the concentration $C$ is plotted against position $x$ in a solid, the curve is called the \textbf{concentration profile}, and the \textbf{concentration gradient} is the slope at any point on this curve. For simplicity, the concentration profile is often assumed to be linear.
\[\text{(Concentration Gradient)}=\frac{dC}{dx}=\frac{\Delta C}{\Delta x}=\frac{C_A-C_B}{x_A-x_B}\]
The \textbf{concentration} $C$ of a diffusing species is expressed as the mass of the species per unit volume of the solid, typically in units of kg/m$^3$ or g/cm$^3$.\\\\
In diffusion, the term \textbf{driving force} refers to what compels the process to occur. When diffusion follows Fick's law, the \textbf{concentration gradient} acts as the driving force.\\\\
\underline{Note:} We can convert between weight percent and concentration using the density of the material.
\[C''_1=\frac{C_1}{\frac{C_1}{\rho_1}+\frac{C_2}{\rho_2}}\times 10^3\]
Where:
\begin{itemize}
    \item $C_1$ = weight percent of 1
    \item $C_2$ = weight percent of 2
    \item $\rho_1$ = density of 1 (g/cm$^3$)
    \item $\rho_2$ = density of 2 (g/cm$^3$)
\end{itemize}
\newpage
\section*{Fick's second law—nonsteady-state diffusion}
For nonsteady-state diffusion, the diffusion flux and the concentration gradient at some particular point in a solid vary with time, with a net accumulation or depletion of the diffusing species resulting.\\\\
\textbf{Fick's second law—diffusion equation for nonsteady-state diffusion (in one direction)}
\[\frac{\partial C}{\partial t} = D\frac{\partial^2 C}{\partial x^2}\]
Where:
\begin{itemize}
    \item $C$ = concentration of the diffusing species (kg/m$^3$ or g/cm$^3$)
    \item $D$ = diffusion coefficient (m$^2$/s)
\end{itemize}
Solutions to this differential equation are possible when physically meaningful boundary conditions are specified.\\\\
\textbf{Semi-infinite Solid}\\\\
A practically important case is diffusion into a \textbf{semi-infinite solid} where the surface concentration is held constant, often from a gas phase with constant partial pressure. The assumptions are:
\begin{enumerate}
    \item Initially, the solute atoms are uniformly distributed in the solid with concentration $C_0$.
    \item The surface is at $x = 0$ and $x$ increases into the solid.
    \item Time $t = 0$ is defined just before diffusion begins.
\end{enumerate}
\noindent
These conditions can be expressed as:

\textbf{Initial condition:} 
\[t = 0, \quad C = C_0 \quad \text{for } 0 \le x \le \infty\]

\textbf{Boundary conditions:} 
\begin{align*}
&t > 0, \quad C = C_s \quad \text{at } x = 0 \quad \text{($C_s$ is the constant surface concentration)}\\
&t > 0, \quad C = C_0 \quad \text{at } x = \infty
\end{align*}\newpage\noindent
This results in the following solution:
\[\boxed{\frac{C_x-C_0}{C_s - C_0} = 1-\text{erf}\left(\frac{x}{2\sqrt{Dt}}\right)}\]
Where:
\begin{itemize}
    \item $C_x$ = concentration at distance x after time t
    \item $C_0$ = initial concentration inside solid
    \item $C_s$ = surface concentration
    \item $D$ = diffusion coefficient
    \item $x$ = distance into solid
\end{itemize}
\underline{Note:} The concentrations can be in any consistent units (wt\%, at\%, kg/m$^3$, etc.) since they are in a ratio.\\\\
\textbf{Target concentration ($C_1$):} \\
    If you want the solute in the alloy to reach a certain concentration $C_1$, you can express it in a normalized form:
    \[
    \frac{C_1 - C_0}{C_s - C_0} = \text{constant}
    \]
    This ratio is constant for a given point in the material at a fixed time.\\\\
\textbf{Relation to diffusion distance and time:} \\
    \[    \frac{x}{2\sqrt{Dt}} = \text{constant} \quad \text{or equivalently} \quad \frac{x^2}{Dt} = \text{constant}    \]
    This is constant for fixed target concentration $C_1$ and initial and surface concentrations $C_0$ and $C_s$.\\\\
\underline{Note:} \textbf{Carburizing} is the process by which the surface carbon concentration of a ferrous alloy is increased by diffusion from the surrounding environment. ($C_0$ is zero for pure iron)
\newpage\noindent
\underline{Note:} Values for the error function can be found for various $\frac{x}{2\sqrt{Dt}}$ inputs can be found can be found in tables. If the value does not match exactly, do a linear approx using the two closest values.
\begin{figure}[H]
    \centering
    \includegraphics[width=1\textwidth]{error function.png}
\end{figure}
\section*{Factors that influence diffusion}
The diffusion coefficient D refers to the rate at which atoms diffuse. The diffusing species and the host material influence the diffusion coefficient.\\\\
\textbf{Temperature Dependence}
\[D=D_o e^{\left(-\frac{Q_d}{RT}\right)}\]
Where:
\begin{itemize}
    \item $D_o$ = a temperature-independent preexponential ($m^2/s$)
    \item $Q_d$ = activation energy for diffusion (J/mol or eV/atom)
    \item R = universal gas constant ($8.314$ J/mol$\cdot$K)
    \item T = absolute temperature (K)
\end{itemize}
\begin{figure}[H]
    \centering
    \includegraphics[width=1\textwidth]{diffusion co.png}
\end{figure}
Rewriting the equation:
\[\ln D = \ln D_o - \frac{Q_d}{R}\left(\frac{1}{T}\right)\]
This is the equation of a straight line with slope $-\frac{Q_d}{R}$ and y-intercept $\ln D_o$. Thus, a plot of $\ln D$ versus $\frac{1}{T}$ yields a straight line, from which $D_o$ and $Q_d$ can be determined.
\newpage
\section*{Diffusion in semiconducting materials}

\newpage
\section*{\LARGE\underline{Line Defects}}
Dislocations are line defects. It occurs when there is slipping between crystal planes when dislocations move, and they cause permanent (plastic) deformation. There are two main types:
\begin{itemize}
    \item \textbf{Edge Dislocation}
    \begin{figure}[H]   
        \centering
        \includegraphics[width=0.8\textwidth]{Edge dislocation.png}
    \end{figure}
    This occurs when an extra half plane of atoms is inserted in a crystal and moves when shear stress is applied. The direction of movement is parallel to the applied shear stress.
    \item \textbf{Screw Dislocation}
    \begin{figure}[H]
        \centering
        \includegraphics[width=0.8\textwidth]{screw dislocation.png}
    \end{figure}
    The direction of movement is perpendicular to the applied shear stress.\\\\
Plastic deformation occurs due to many dislocations. The process by which plastic deformation is produced by dislocation motion is termed \textbf{slip}; the crystallographic plane along which the dislocation line traverses is the \textbf{slip plane}.\\\\
All metals and alloys contain dislocations formed during solidification. The \textbf{solidification density} is the number of dislocations per unit volume in a material.
\end{itemize}
\section*{Characteristics of dislocations}
Due to the presence of an etra half plane of atoms, there exists regions of compressive, tensile, and shear \textbf{lattice strains} around the dislocation line. \\\\
\underline{Ex:}
\begin{figure}[H]
    \centering
    \includegraphics[width=0.4\textwidth]{lattice strain.png}
\end{figure}
These lattice strains interact in the following ways:
\begin{figure}[H]
    \centering
    \includegraphics[width=0.5\textwidth]{interaction.png}
\end{figure}
\noindent
The attractive forces will result in the half planes merging into a complete plane, eliminating the dislocation. The repulsive forces will make it more difficult for dislocations to move. The number of dislocations increases dramatically during plastic deformation because of multiplication of existing dislocations, grain boundaries, internal defects, and surface irregularities.
\section*{Slip Systems}
The preferred slip plane and slip direction are those with the densest atomic packing (highest planer density) and highest linear density respectively. The combination of slip plane and slip direction is called a \textbf{slip system}.\\\\
\underline{Ex:} FCC $\{111\}\langle110\rangle$ slip system
\begin{figure}[H]
    \centering
    \includegraphics[width=0.6\textwidth]{slip system FCC.png}
\end{figure}
\textbf{Table of Slip Systems}
\begin{figure}[H]
    \centering
    \includegraphics[width=0.9\textwidth]{table.png}
\end{figure} \noindent
\underline{Note:} More slip systems = more ductile material\\\\
The \textbf{Burgers vector} has the same direction as the dislocation motion (slip direction) and its magnitude is equal to the interatomic spacing in that direction (unit slip distance). 
\underline{Ex:} Burgers vector for FCC and BCC
\[\vec{b}_{FCC} = \frac{a}{2}\langle110\rangle\]
\[\vec{b}_{BCC} = \frac{a}{2}\langle111\rangle\]
\section*{Slip in single crystals}
For an applied shear stress $\sigma$, the resolved shear stress represents the magnitude of the shear stress acting on the slip plane in the slip direction. It is given by: 
\[\tau_R=\sigma \cos \phi \cos \lambda\]
\begin{figure}[H]
    \centering
    \includegraphics[width=0.4\textwidth]{resolved.png}
\end{figure}\noindent
A metal single crystal has a number of different slip systems, each with its own value of resolved shear stresses. However, there will be one slip system with the maximum resolved shear stress, denoted $\tau_{R}(\text{max})$.
\[\tau_{R}\text{(max)}=\sigma (\cos \phi \cos \lambda)_{max}\]
In this slip system with the maximum resolved shear stress, slip occurs when the resolved shear stress reaches a critical value (the critical resolved shear stress, $\tau_{CRSS}$). This is the minimum shear stress required to initiate slip and thus determines when yielding will begin (when $\tau_{R}\text{(max)}=\tau_{CRSS}$).
\[\sigma_y = \frac{\tau_{CRSS}}{(\cos\phi \cos\lambda)_{max}} \quad \text{(yield stress)}\]
This yield stress is minimized when $\phi=\lambda =45^\circ$, maximizing $\cos\phi \cos\lambda = 0.5$. Thus,
\[\sigma_{y,min} = 2\tau_{CRSS}\]
\newpage \noindent
\underline{Ex:} Consider a single crystal of BCC iron oriented such that a tensile stress is applied along a [010] direction. Compute the resolved shear stress along a (110) plane and in a [$\overline{1}$11] direction when a tensile stress of 52 MPa (7500 psi) is applied.
\begin{figure}[H]
    \centering
    \includegraphics[width=0.6\textwidth]{Ex slip.jpg}
\end{figure}
Recall that we can find the angle by using the projection:
\[\cos\theta=\frac{\vec{a}\cdot\vec{b}}{|\vec{a}||\vec{b}|}\]
Where:
\begin{itemize}
    \item Applied tensile direction $[010] \implies \vec{l}=(0,1,0)$
    \item Slip plane $(110) \implies \vec{n}=(1,1,0)$ (normal vector to plane)
    \item Slip direction $[\overline{1}11] \implies \vec{d}=(-1,1,1)$
\end{itemize}
Thus,
\[
\begin{aligned}
\phi &= \cos^{-1}\!\left(\frac{\vec{n}\cdot\vec{l}}{|\vec{n}||\vec{l}|}\right)\\
     &= \cos^{-1}\!\left(\frac{(1,1,0)\cdot(0,1,0)}{\sqrt{1^2+1^2+0^2}\sqrt{0^2+1^2+0^2}}\right)\\
     &= \cos^{-1}\!\left(\frac{1}{\sqrt{2}}\right)=45^\circ
\end{aligned}
\qquad
\begin{aligned}
\lambda &= \cos^{-1}\!\left(\frac{\vec{d}\cdot\vec{l}}{|\vec{d}||\vec{l}|}\right)\\
     &= \cos^{-1}\!\left(\frac{(-1,1,1)\cdot(0,1,0)}{\sqrt{(-1)^2+1^2+1^2}\sqrt{0^2+1^2+0^2}}\right)\\
     &= \cos^{-1}\!\left(\frac{1}{\sqrt{3}}\right)=54.74^\circ
\end{aligned}
\]
\[\tau_R=(52\text{MPa}) \cos (45^\circ) \cos (54.74^\circ)=21.3\text{MPa}\]
\newpage
\section*{Deformation by Twinning}
In addition to slip, plastic deformation in some metallic materials can occur by the
formation of mechanical twins, or twinning. Twinning occurs when a shear force produces atomic displacements such that on one side of a plane (the twin boundary), atoms are located in mirror-image positions of atoms on the other side.
\begin{figure}[H]
    \centering
    \includegraphics[width=1\textwidth]{twinning.png}
\end{figure}
Twinning is different from slip in several ways:
\begin{itemize}
    \item Twinning produces a reorientation of the crystal lattice, where slip crystallographic orientation is the same before and after deformation
    \item Slip occurs in distinct atomic spacing multiples, whereas the atomic displacement for twinning is less than the interatomic separation.
\end{itemize}
\begin{figure}[H]
    \centering
    \includegraphics[width=0.6\textwidth]{diff.png}
\end{figure}
Mechanical twinning occurs in metals with BCC and HCP crystal structures at low temps and high rates of loading where there are fewer slip systems. Twinning is important because it may place new slip systems in orientations that are favorable relative to the stress axis such that the slip process can now take place.
\newpage
\section*{Strengthening by grain size reduction}
The strength and hardness of metals depend on how easily dislocations can move. Plastic deformation occurs through dislocation motion, so if dislocations move freely, the metal is soft and weak. If their motion is restricted, more force is needed to deform the metal, making it harder and stronger. Thus, nearly all strengthening methods work by hindering dislocation movement.\\\\
Grain size affects a metal’s mechanical properties because grain boundaries hinder dislocation motion. When a dislocation moves from one grain to another, it encounters two obstacles:
\begin{itemize}
    \item The change in crystal orientation makes it harder for the dislocation to continue moving.
    \item The atomic disorder at the grain boundary disrupts slip-plane continuity.
\end{itemize}
As a result, smaller grains (more boundaries) strengthen the metal by restricting dislocation motion.
\begin{figure}[H]
    \centering
    \includegraphics[width=0.6\textwidth]{grain bound.png}
\end{figure}\noindent
The relationship between yield strength and grain size is given by the \textbf{Hall-Petch equation}:
\[\sigma_y=\sigma_o+k_y d^{-1/2}\]
Where:
\begin{itemize}
    \item $\sigma_y$ = yield stress
    \item $\sigma_o$ = friction stress (stress required to move dislocations, a material constant)
    \item $k_y$ = strengthening coefficient (a material constant)
    \item d = average grain diameter
\end{itemize}
\newpage
\section*{Solid-solution strengthening}
By creating alloys with impurity atoms in either a substitutional or interstitial solid solution, the strength of a metal can be increased. This process is called \textbf{solid-solution strengthening}. High-purity metals are almost always softer and weaker than alloys composed of the same base metal. This is because the impurity atoms impose lattice strains that hinder dislocation motion.
\begin{figure}[H]
    \centering
    \includegraphics[width=1\textwidth]{solid solution.png}
\end{figure}\noindent
\section*{Strain hardening}
\textbf{Strain hardening} is the phenomenon by which a ductile metal becomes harder and stronger as it is plastically deformed. Sometimes it is also called work hardening, or \textbf{cold working}.\\\\
We can define the amount of plastic deformation as \textbf{percent cold work} instead of strain:
\[\%CW=\frac{A_o - A_d}{A_o} \times 100\]
Where:
\begin{itemize}
    \item $A_o$ = original cross-sectional area
    \item $A_d$ = cross-sectional area after deformation
\end{itemize}
\section*{Recovery}
When a polycrystalline metal is plastically deformed at low temperatures, its grains change shape, dislocation density increases, and it undergoes strain hardening. Some deformation energy is stored as strain energy, and properties such as electrical conductivity and corrosion resistance may change.\\\\
\noindent
These effects can be reversed by \textbf{annealing}, a heat treatment involving \textbf{recovery} and \textbf{recrystallization} (sometimes followed by grain growth). During \textbf{recovery}, atomic diffusion allows dislocations to rearrange and reduce internal strain energy without external stress. This lowers dislocation density and restores properties such as electrical and thermal conductivity to their original states.
\section*{Recrystallization}
\textbf{Recrystallization} replaces deformed grains with new, strain-free, equiaxed ones that have low dislocation density. This process, driven by stored strain energy, restores the metal’s softness and ductility. It depends on time and temperature; the \textbf{recrystallization temperature} (temp for completion in 1~h) is typically $0.3$–$0.5\,T_m$ for metals and up to $0.7\,T_m$ for alloys ($T_m$ is the melting temperature). Greater cold work lowers this temperature, while impurities raise it by slowing grain boundary motion.
\section*{Grain Growth}
\textbf{Grain growth} occurs when strain-free grains enlarge at elevated temperatures, even without prior recovery or recrystallization. As grain boundaries migrate, large grains grow at the expense of smaller ones, reducing total boundary area and energy. The driving force is the decrease in grain boundary energy, and boundary motion occurs through short-range atomic diffusion.
\begin{figure}[H]
    \centering
    \includegraphics[width=0.4\textwidth]{grain growth.png}
\end{figure}\noindent
The grain diameter varies with time according to the relationship:
\[d^n-d^n_o=Kt\]
Where:
\begin{itemize}
    \item $d$ = grain diameter at time t
    \item $d_o$ = initial grain diameter
    \item n = grain growth exponent (typically 2)
    \item K = temperature-dependent rate constant
\end{itemize}
\end{document}