\documentclass[12pt]{article}
\usepackage{amsmath, amsfonts, bm, graphicx,amssymb,geometry,tabularx,booktabs,float,wrapfig}
\geometry{margin = 1in}
\title{}
\date{}
\author{}

\begin{document}
\noindent
\textbf{SI Prefixes}
\[\begin{aligned}
\text{femto (f)}  &= 10^{-15} \\
\text{pico (p)}   &= 10^{-12} \\
\text{nano (n)}   &= 10^{-9} \\
\text{micro ($\mu$)}  &= 10^{-6} \\
\text{milli (m)}  &= 10^{-3} \\
\text{centi (c)}  &= 10^{-2} \\
\text{deci (d)}   &= 10^{-1} \\
\text{deca (da)}  &= 10^{1} \\
\text{hecto (h)}  &= 10^{2} \\
\text{kilo (k)}   &= 10^{3} \\
\text{mega (M)}   &= 10^{6} \\
\text{giga (G)}   &= 10^{9} \\
\text{tera (T)}   &= 10^{12} \\
\text{peta (P)}   &= 10^{15} \\
\end{aligned}\]
\underline{Note:} $1\text{Angstrom}\,(\text{\AA}) = 10^{-10}$ m
\newpage
\begin{center}
    \section*{\LARGE\underline{Mechanical Properties}}
\end{center}
\section*{Stress and Strain}
\textbf{Types of loading}
\begin{figure}[H]
    \centering
    \includegraphics[width=0.8\textwidth]{types of loading.png}
\end{figure}
\newpage
\noindent \textbf{Stress (Force Normalized by Area)}\\\\
Tensile and Compression Stress
\[\sigma = \frac{F}{A_o} \quad \text{(in units of pressure)}\]
Shear Stress
\[\tau = \frac{F}{A_o}\]
\noindent \textbf{Strain (Displacement Normalized by Original Length)}\\\\
Tensile and Compression Strain
\[\epsilon = \frac{l - l_0}{l_0}\]
Shear Strain
\[\gamma = \tan\theta \quad \text{($\theta$ is the shear angle)}\]

\noindent \textbf{Normal and Shear Stress Along an Angled Plane}
\begin{figure}[H]
    \centering
    \includegraphics[width=0.8\textwidth]{geometric stress.png}
\end{figure}
\newpage

\section*{Elastic Deformation}
\textbf{Relationship Between Stress and Strain}\\\\
Tensile and Compression
\[\sigma = E\epsilon \quad \text{(E (GPa or psi) is the modulus of elasticity)}\]
Shear
\[\tau = G\gamma \quad \text{(G is the shear modulus)}\]
\underline{Note:} The modulus of elasticity (Young's modulus) is the slope of the stress - strain plot. (It describes a material's resistance to elastic deformation. Stiffer $\implies$ higher E)\\\\
\textbf{Anelasticity:} time dependent elastic strain, where deformation and recovery is not instantaneous.\\
\textbf{Viscoelastic behavior}: materials (such as polymers) with significant anelasticity\\\\

\noindent \textbf{Poisson's Ratio}
\begin{figure}[H]
    \centering
    \includegraphics[width=0.8\textwidth]{poissons.jpg}
\end{figure}
\[ \nu = -\frac{\epsilon_{\text{lateral}}}{\epsilon_{\text{longitudinal}}}\]
\underline{Note:} Lateral is perpendicular to the direction of loading and longitudinal is along the direction of loading
\newpage
\underline{Example:} Rectangular prism
\begin{figure}[H]
    \centering
    \includegraphics[width=0.4\textwidth]{poissons 2.png}
\end{figure}
\noindent If the applied stress is uniaxial (only along 1 axis) and the material is isotropic (constant properties regardless of direction), then for a $\sigma_z$, $\epsilon_x=\epsilon_y$
\[ \nu = -\frac{\epsilon_x}{\epsilon_z}= -\frac{\epsilon_y}{\epsilon_z}\]

\noindent \textbf{Relating modulus of elasticity, shear modulus and Poisson's ratio}
\[E=2G(1+\nu)\]
\underline{Note:} Some materials (like foams) expand under tension so they have a negative Poisson's ratio, these materials are called \textbf{auxetics}.
\newpage
\section*{Plastic Deformation}
\begin{wrapfigure}{l}{0.4\textwidth}
    \centering
    \includegraphics[width=0.4\textwidth]{yield.jpg}
\end{wrapfigure}
Point P is the \textbf{Proportional Limit} where the exact departure from linearity occurs and deformation becomes permanent.
\\\\
\textbf{Yield Stress ($\sigma_y$):} stress at which noticeable strain has occurred (0.002)
\\\\\\\\\\\\\\\\\\\\\\\\\\\\\\\\\\\\\\
\begin{figure}[H]
    \centering
    \includegraphics[width=0.55\textwidth]{tensile.jpg}
\end{figure}
\noindent\textbf{Tensile Strength}: Stress at the maximum point on the stress - strain plot. After this point, necking occurs and all deformation is focused at the neck until fracture (point F)

\noindent\textbf{Ductility}\\\\
As $\%$ elongation:
\[\%EL = \frac{l_f - l_0}{l_0} \times 100\]
As $\%$ reduction in area
\[\%RA = \frac{A_0 - A_f}{A_0} \times 100\]
$l_f$ and $A_f$ are length and cross-sectional area of sample at fracture respectively.\\\\
\noindent\textbf{Resilience:} capacity of a material to absorb energy when it is deformed elastically and unloaded (similar to spring potential energy)\\\\
\noindent\textbf{Modulus of Resilience}
\[U_r = \int_{0}^{\epsilon_{yield}}\sigma d\epsilon\]
Area under the stress - strain plot from 0 to yield point\\\\
Assuming a linear elastic region:
\[U_r=\frac{1}{2}\sigma_y\epsilon_y\]
\newpage
\begin{center}
    \section*{\LARGE\underline{Crystal Structures}}
\end{center}
\noindent\textbf{Atomic Packing Factor}
\[ APF = \frac{\text{Volume of atoms in unit cell}}{\text{Total unit cell volume}}\]
\noindent\textbf{Packing Fraction}
\[PF=\frac{\text{Total Cross Sectional Area of Atoms}}{\text{Total Area of Plane}}\]
\noindent\textbf{Number of Atoms per Unit Cell}
\[N=N_i+\frac{N_f}{2}+\frac{N_c}{8}\]
$N_i$ are interior atoms, $N_f$ are face atoms and $N_c$ are corner atoms
\section*{Simple Cubic}
\begin{figure}[H]
    \centering
    \includegraphics[width=0.6\textwidth]{SC.jpg}
\end{figure}
\[2R=a\]
\[APF=\frac{(\text{\# atoms})(\text{volume/atom})}{(\text{volume/unit cell})}\]
\[APF=\frac{(1)(\frac{4}{3}\pi(a/2)^3)}{a^3}\]
\section*{Body Centered Cubic}
\begin{figure}[H]
    \centering
    \includegraphics[width=0.6\textwidth]{BCC.png}
\end{figure}
\[\text{Triangle formed along the main diagonal and face diagonal}\]
\begin{figure}[H]
    \centering
    \includegraphics[width=0.3\textwidth]{BCC diag.png}
\end{figure}
\[4R=\sqrt{3}a\]
\[APF=\frac{(\text{\# atoms})(\text{volume/atom})}{(\text{volume/unit cell})}\]
\[APF=\frac{(2)(\frac{4}{3}\pi(\sqrt{3}/4a )^3)}{a^3}\]
\newpage

\section*{Face Centered Cubic}
\begin{figure}[H]
    \centering
    \includegraphics[width=0.6\textwidth]{FCC.png}
\end{figure}
Along the face diagonal:
\[4R=\sqrt{2}a\]
\[APF=\frac{(\text{\# atoms})(\text{volume/atom})}{(\text{volume/unit cell})}\]
\[APF=\frac{(4)(\frac{4}{3}\pi(\sqrt{2}/4a )^3)}{a^3}\]
\newpage
\section*{Hexagonal Close Packed}
\begin{figure}[H]
    \centering
    \includegraphics[width=0.6\textwidth]{HCP.jpg}
\end{figure}
\begin{figure}[H]
    \centering
    \includegraphics[width=0.4\textwidth]{Untitled.jpg}
\end{figure}
Area of base hexagon is 3 parallelograms or 6 equilateral triangles:
\[\text{Area}=\frac{3a^2\sqrt{3}}{2}\]
Given height c:
\[\text{Volume of unit cell}=\frac{3a^3\sqrt{3}}{2}\]
\newpage
\section*{Theoretical Density for Crystals}
\[\rho=\frac{(\text{atoms/unit cell})(\text{g/mol})}{(\text{vol/unit cell})(\text{atoms/mol})}=(\text{g/vol})\]
\[\rho=\frac{nA}{V_c N_A}\]
Where:
\begin{itemize}
    \item n = \# of atoms in unit cell
    \item A = atomic weight
    \item $V_c$ = volume of unit cell
    \item $N_a$ = Avogadro's number ($6.022 \times 10^{23 }\text{ atoms/mol}$)
\end{itemize}
\section*{Ceramic Crystal Structures}
Factors that determine crystal structure:
\begin{itemize}
    \item Relative sizes of ions ($\frac{r_{cation}}{r_{anion}}$)
    \item Maintenance of charge neutrality (Net charge in ceramic is zero)
\end{itemize}
\underline{Note:} As $\frac{r_{cation}}{r_{anion}}$ increases, so does coordination number
\begin{figure}[H]
    \centering
    \includegraphics[width=0.8\textwidth]{ceramic.png}
\end{figure}
\newpage
\noindent\textbf{Theoretical Density for Ceramics}
\[\rho =\frac{n'(\sum A_C+\sum A_A)}{V_c N_A}\]
Where:
\begin{itemize}
    \item n' = \# of Atoms per unit cell (For AX structures, this is equal for cations and anions)
    \item $\sum A_A$ = sum of cation molar mass
    \item $\sum A_C$ = sum of anion molar mass
    \item $V_c$ = volume of unit cell
    \item $N_A$ = Avogadro's number ($6.022 \times 10^{23} \text{ atoms/mol}$)
\end{itemize}
\section*{Rock Salt Structure}
\begin{figure}[H]
    \centering
    \includegraphics[width=0.4\textwidth]{Rocksalt.png}
\end{figure}
\underline{Note:} Cations prefer octahedral sites (in black)\\
Along the edges:
\[2R_A+2R_C=a\]
\section*{AX Crystal Structure (Cesium Chloride)}
\begin{figure}[H]
    \centering
    \includegraphics[width=0.4\textwidth]{AX.png}
\end{figure}
\underline{Note:} Cations prefer cubic sites (Body Center, in blue) \\
Across the main diagonal:
\[2R_A+2R_C=\sqrt{3}a\]
\section*{AX$_2$ Crystal Structures (Flourite)}
\begin{figure}[H]
    \centering
    \includegraphics[width=0.4\textwidth]{AX2.png}
\end{figure}
\underline{Note:} Cations prefer cubic sites (Body Center, in blue) \\
There are half as many Ca$^{2+}$ as F$^-$ (for CaF$_2$)
\section*{ABX$_3$ Crystal Structure (Perovskite)}
\begin{figure}[H]
    \centering
    \includegraphics[width=0.4\textwidth]{ABX3.png}
\end{figure}
\section*{Point Coordinates}
To find the coordinate indices (q, r, s), find the Cartesian coordinates and divide by the corresponding lattice parameter
\[(q,r,s) = \left( \frac{x}{a}, \frac{y}{b}, \frac{z}{c} \right)\]
\section*{Crystallographic Directions}
How to define:
\begin{itemize}
    \item Position vector to pass through origin
    \item Read off projections onto coordinate axes in terms of lattice parameters (a, b, c)
    \item Multiply through by common denominator
    \item Enclose in square brackets without commas, negatives go on top (ex: [$\overline{1} 2 3$])
\end{itemize}
How to read [123]:
\begin{itemize}
    \item Divide by the common denominator used previously (say 6)
    \item Vector in Cartesian: (1/6,1/3,1/2)
\end{itemize}
\section*{Crystallographic Planes}
How to define with Miller indices:
\begin{itemize}
    \item Define any origin
    \item Read intercepts of the plane with the coordinate axes in terms of lattice parameters (a, b, c)
    \item Take reciprocals of intercepts
    \item Enclose in parentheses without commas, negatives go on top (ex: ($\overline{1} 2 3$))
\end{itemize}
How to read:
\begin{itemize}
    \item Take reciprocals of plane to identify intercepts
\end{itemize}
\section*{Linear and Planer Density}
\textbf{Linear Density}
\[LD = \frac{\text{Number of atom centered on line}}{\text{Unit length of direction vector}}\]
\textbf{Planer Density}
\[PD=\frac{\text{Number of atoms centered on plane}}{\text{Area of plane}}\]
\newpage
\begin{center}
    \section*{\LARGE\underline{Point Defects}}
\end{center}
\begin{itemize}
    \item Vacancies: Missing atoms in the lattice
    \begin{figure}[H]
        \centering
        \includegraphics[width=0.5\textwidth]{vacancy.png}
    \end{figure}
    \item Interstitials: Extra atoms positioned between lattice sites
    \begin{figure}[H]
        \centering
        \includegraphics[width=0.5\textwidth]{intersitial.png}
    \end{figure}
\end{itemize}
\section*{Equilibrium Concentration of Vacancies}
\[ N_v = Ne^{\left( \frac{-Q_v}{kT} \right)}\]
Where:
\begin{itemize}
    \item $N_v$ = number of vacancies
    \item N = total number of atomic sites (each lattice site can be occupied by an atom or be a vacancy)
    \item $Q_v$ = activation energy required to form a vacancy (J or eV)
    \item k = Boltzmann's constant ($8.617 \times 10^{-5}$ eV/atom-K)
    \item T = temperature (K)
\end{itemize}
\newpage
\section*{Alloys}
Alloys are solid solutions of an impurity element (solute) dissolved in a base element (solvent). There are two types:
\begin{figure}[H]
    \centering
    \includegraphics[width=0.8\textwidth]{alloy.png}
\end{figure}
\noindent\textbf{Hume-Rothery rule}\\\\
Conditions for forming a solid solution:
\begin{itemize}
    \item Atomic radii must be within 15\% of each other ($\Delta r < 0.15\%$)
    \item Electronegativity values must be similar (i.e. Close together on the periodic table)
    \item Crystal structures of solvent and solute must be the same (for pure metals)
    \item Valencies should be similar (All else being equal, a metal will have a greater tendency to dissolve a metal of higher valency than one of lower valency)
\end{itemize}
\newpage
\noindent\textbf{Maximum Radius for an Interstitial Impurity}\\\\
\noindent\textbf{FCC}
\begin{figure}[H]
    \centering
    \includegraphics[width=0.7\textwidth]{FCC radius.png}
\end{figure}
\vspace{-4em}
\[2r+2R=a\]
\[r=\frac{a-2R}{2}\]
\[\boxed{r=\frac{2R\sqrt{2}-2R}{2}=0.41R}\]
\noindent\textbf{BCC}
\begin{figure}[H]
    \centering
    \includegraphics[width=0.7\textwidth]{BCC radius.png}
\end{figure}
\vspace{-2em}
\[\left(\frac{a}{2}\right)^2+\left(\frac{a}{4}\right)^2=(R+r)^2\]
\[\left(\frac{4R}{2\sqrt{3}}\right)^2+\left(\frac{4R}{4\sqrt{3}}\right)^2=R^2+2Rr+r^2\]
\[r^2+2Rr-0.667R^2=0\]
\[\Rightarrow r=\frac{-2R\pm\sqrt{(2R)^2-4(1)(-0.667R^2)}}{2(1)}\]
\[\boxed{r=0.291R}\]
\newpage
\section*{Specification of composition}
\noindent\textbf{Weight Percent}
\[C_1=\frac{m_1}{m_1+m_2}\times 100\]
Where
\begin{itemize}
    \item $C_1$ = weight percent of element 1
    \item $m_1$ = mass of element 1
    \item $m_2$ = mass of element 2 
\end{itemize}
\noindent\textbf{Atomic Percent}
\[C_1'=\frac{n_{m1}}{n_{m1}+n_{m2}}\times 100\]
Where
\begin{itemize}
    \item $C_1'$ = atomic percent of element 1
    \item $n_{m1}$ = number of moles of element 1
    \item $n_{m2}$ = number of moles of element 2
\end{itemize}
\underline{Note:}
\[n_{m1} = \frac{m_1 \text{(grams)}}{A_1 \text{(g/mol)}}\]
\noindent\textbf{Conversion between wt\% and at\%}
\[C_1'=\frac{C_1A_2}{C_1A_2 + C_2A_1}\times 100\]
Where $A_1$ and $A_2$ are the atomic weights of elements 1 and 2 respectively
\newpage
\section*{Diffusion}
\textbf{Diffusion:} The general process where atoms move from one place to another, causing mass transport within or between materials.\\[4pt]
\textbf{Interdiffusion:} Occurs when atoms from two different metals migrate into each other, forming a region where the metals mix.\\[4pt]
\textbf{Self-diffusion:} Happens in pure metals when atoms of the same kind exchange places, even though there is no concentration difference.
\section*{Diffusion Mechanisms}
For diffusion to occur: (1) an adjacent site must be vacant, and (2) the atom must have enough energy to break bonds and distort the lattice. Only a fraction of atoms at a given temperature have sufficient vibrational energy for diffusion, and this fraction increases with temperature. For metals, two main models describe this atomic motion.
\begin{itemize}
    \item \textbf{Vacancy Diffusion:} Atoms move into adjacent vacant lattice sites.
    \begin{figure}[H]
        \centering
        \includegraphics[width=0.6\textwidth]{vacancy diff.png}
    \end{figure}
    \item \textbf{Interstitial Diffusion:} Smaller atoms move through the spaces between larger atoms.
    \begin{figure}[H]
        \centering
        \includegraphics[width=0.6\textwidth]{interstitial diff.png}
    \end{figure}
\end{itemize}
\newpage
\section*{Fick's First Law}
We quantify how fast diffusion occurs by defining the rate of mass transfer or the \textbf{diffusion flux} (J). It is the amount of substance that diffuses through a unit area per unit time.
\[J=\frac{M}{At} \quad (kg/m^2\cdotp s \text{ or } atoms/m^2\cdotp s)\]
Where:
\begin{itemize}
    \item M = mass of diffusing substance (kg or atoms)
    \item A = cross-sectional area (m$^2$)
    \item t = time (s)
\end{itemize}
\textbf{Fick's first law—diffusion flux for steady-state diffusion (in one direction)}
\[J=-D\frac{dC}{dx}\]
Where:
\begin{itemize}
    \item D = diffusion coefficient($m^2/s$)\\
    The negative sign indicates that the direction of diffusion is down the concentration gradient, from a high to a low concentration.
    \item $\frac{dC}{dx}$ = concentration gradient in the x direction.
\end{itemize}
\textbf{Fick's first law} describes the diffusion of atoms through a thin metal plate when the concentrations (or pressures) on both surfaces are constant. In this case, a \textbf{steady-state diffusion} is achieved, where the diffusion flux remains constant over time and there is no net accumulation of the diffusing species within the plate.
\begin{figure}[H]
    \centering
    \includegraphics[width=0.8\textwidth]{ficks 1.png}
\end{figure}\noindent
When the concentration $C$ is plotted against position $x$ in a solid, the curve is called the \textbf{concentration profile}, and the \textbf{concentration gradient} is the slope at any point on this curve. For simplicity, the concentration profile is often assumed to be linear.
\[\text{(Concentration Gradient)}=\frac{dC}{dx}=\frac{\Delta C}{\Delta x}=\frac{C_A-C_B}{x_A-x_B}\]
The \textbf{concentration} $C$ of a diffusing species is expressed as the mass of the species per unit volume of the solid, typically in units of kg/m$^3$ or g/cm$^3$.\\\\
In diffusion, the term \textbf{driving force} refers to what compels the process to occur. When diffusion follows Fick's law, the \textbf{concentration gradient} acts as the driving force.\\\\
\underline{Note:} We can convert between weight percent and concentration using the density of the material.
\[C''_1=\frac{C_1}{\frac{C_1}{\rho_1}+\frac{C_2}{\rho_2}}\times 10^3\]
Where:
\begin{itemize}
    \item $C_1$ = weight percent of 1
    \item $C_2$ = weight percent of 2
    \item $\rho_1$ = density of 1 (g/cm$^3$)
    \item $\rho_2$ = density of 2 (g/cm$^3$)
\end{itemize}
\newpage
\section*{Fick's second law—nonsteady-state diffusion}
For nonsteady-state diffusion, the diffusion flux and the concentration gradient at some particular point in a solid vary with time, with a net accumulation or depletion of the diffusing species resulting.\\\\
\textbf{Fick's second law—diffusion equation for nonsteady-state diffusion (in one direction)}
\[\frac{\partial C}{\partial t} = D\frac{\partial^2 C}{\partial x^2}\]
Where:
\begin{itemize}
    \item $C$ = concentration of the diffusing species (kg/m$^3$ or g/cm$^3$)
    \item $D$ = diffusion coefficient (m$^2$/s)
\end{itemize}
Solutions to this differential equation are possible when physically meaningful boundary conditions are specified.\\\\
\textbf{Semi-infinite Solid}\\\\
A practically important case is diffusion into a \textbf{semi-infinite solid} where the surface concentration is held constant, often from a gas phase with constant partial pressure. The assumptions are:
\begin{enumerate}
    \item Initially, the solute atoms are uniformly distributed in the solid with concentration $C_0$.
    \item The surface is at $x = 0$ and $x$ increases into the solid.
    \item Time $t = 0$ is defined just before diffusion begins.
\end{enumerate}
\noindent
These conditions can be expressed as:

\textbf{Initial condition:} 
\[t = 0, \quad C = C_0 \quad \text{for } 0 \le x \le \infty\]

\textbf{Boundary conditions:} 
\begin{align*}
&t > 0, \quad C = C_s \quad \text{at } x = 0 \quad \text{($C_s$ is the constant surface concentration)}\\
&t > 0, \quad C = C_0 \quad \text{at } x = \infty
\end{align*}\newpage\noindent
This results in the following solution:
\[\boxed{\frac{C_x-C_0}{C_s - C_0} = 1-\text{erf}\left(\frac{x}{2\sqrt{Dt}}\right)}\]
Where:
\begin{itemize}
    \item $C_x$ = concentration at distance x after time t
    \item $C_0$ = initial concentration inside solid
    \item $C_s$ = surface concentration
    \item $D$ = diffusion coefficient
    \item $x$ = distance into solid
\end{itemize}
\underline{Note:} The concentrations can be in any consistent units (wt\%, at\%, kg/m$^3$, etc.) since they are in a ratio.\\\\
\textbf{Target concentration ($C_1$):} \\
    If you want the solute in the alloy to reach a certain concentration $C_1$, you can express it in a normalized form:
    \[
    \frac{C_1 - C_0}{C_s - C_0} = \text{constant}
    \]
    This ratio is constant for a given point in the material at a fixed time.\\\\
\textbf{Relation to diffusion distance and time:} \\
    \[    \frac{x}{2\sqrt{Dt}} = \text{constant} \quad \text{or equivalently} \quad \frac{x^2}{Dt} = \text{constant}    \]
    This is constant for fixed target concentration $C_1$ and initial and surface concentrations $C_0$ and $C_s$.\\\\
\underline{Note:} \textbf{Carburizing} is the process by which the surface carbon concentration of a ferrous alloy is increased by diffusion from the surrounding environment. ($C_0$ is zero for pure iron)
\newpage\noindent
\underline{Note:} Values for the error function can be found for various $\frac{x}{2\sqrt{Dt}}$ inputs can be found can be found in tables. If the value does not match exactly, do a linear approx using the two closest values.
\begin{figure}[H]
    \centering
    \includegraphics[width=1\textwidth]{error function.png}
\end{figure}
\section*{Factors that influence diffusion}
The diffusion coefficient D refers to the rate at which atoms diffuse. The diffusing species and the host material influence the diffusion coefficient.\\\\
\textbf{Temperature Dependence}
\[D=D_o e^{\left(-\frac{Q_d}{RT}\right)}\]
Where:
\begin{itemize}
    \item $D_o$ = a temperature-independent preexponential ($m^2/s$)
    \item $Q_d$ = activation energy for diffusion (J/mol or eV/atom)
    \item R = universal gas constant ($8.314$ J/mol$\cdot$K)
    \item T = absolute temperature (K)
\end{itemize}
\underline{Note:} $K=\text{ }^\circ C+273$
\begin{figure}[H]
    \centering
    \includegraphics[width=1\textwidth]{diffusion co.png}
\end{figure}\noindent
Rewriting the equation:
\[\ln D = \ln D_o - \frac{Q_d}{RT}\]
This is the equation of a straight line with slope $-\frac{Q_d}{R}$ and y-intercept $\ln D_o$. Thus, a plot of $\ln D$ versus $\frac{1}{T}$ yields a straight line, from which $D_o$ and $Q_d$ can be determined.
\newpage
\begin{center}
\section*{\LARGE\underline{Line Defects}}
\end{center}
Dislocations are line defects. It occurs when there is slipping between crystal planes when dislocations move, and they cause permanent (plastic) deformation. There are two main types:
\begin{itemize}
    \item \textbf{Edge Dislocation}
    \begin{figure}[H]   
        \centering
        \includegraphics[width=0.8\textwidth]{Edge dislocation.png}
    \end{figure}
    This occurs when an extra half plane of atoms is inserted in a crystal and moves when shear stress is applied. The direction of movement is parallel to the applied shear stress.
    \item \textbf{Screw Dislocation}
    \begin{figure}[H]
        \centering
        \includegraphics[width=0.8\textwidth]{screw dislocation.png}
    \end{figure}
    The direction of movement is perpendicular to the applied shear stress.\\\\
Plastic deformation occurs due to many dislocations. The process by which plastic deformation is produced by dislocation motion is termed \textbf{slip}; the crystallographic plane along which the dislocation line traverses is the \textbf{slip plane}.\\\\
All metals and alloys contain dislocations formed during solidification. The \textbf{solidification density} is the number of dislocations per unit volume in a material.
\end{itemize}
\section*{Characteristics of dislocations}
Due to the presence of an etra half plane of atoms, there exists regions of compressive, tensile, and shear \textbf{lattice strains} around the dislocation line. \\\\
\underline{Ex:}
\begin{figure}[H]
    \centering
    \includegraphics[width=0.4\textwidth]{lattice strain.png}
\end{figure}
These lattice strains interact in the following ways:
\begin{figure}[H]
    \centering
    \includegraphics[width=0.5\textwidth]{interaction.png}
\end{figure}
\noindent
The attractive forces will result in the half planes merging into a complete plane, eliminating the dislocation. The repulsive forces will make it more difficult for dislocations to move. The number of dislocations increases dramatically during plastic deformation because of multiplication of existing dislocations, grain boundaries, internal defects, and surface irregularities.
\section*{Slip Systems}
The preferred slip plane and slip direction are those with the densest atomic packing (highest planer density) and highest linear density respectively. The combination of slip plane and slip direction is called a \textbf{slip system}.\\\\
\underline{Ex:} FCC $\{111\}\langle110\rangle$ slip system
\begin{figure}[H]
    \centering
    \includegraphics[width=0.6\textwidth]{slip system FCC.png}
\end{figure}
\textbf{Table of Slip Systems}
\begin{figure}[H]
    \centering
    \includegraphics[width=0.9\textwidth]{table.png}
\end{figure} \noindent
\underline{Note:} More slip systems = more ductile material\\\\
The \textbf{Burgers vector} has the same direction as the dislocation motion (slip direction) and its magnitude is equal to the interatomic spacing in that direction (unit slip distance). 
\underline{Ex:} Burgers vector for FCC and BCC
\[\vec{b}_{FCC} = \frac{a}{2}\langle110\rangle\]
\[\vec{b}_{BCC} = \frac{a}{2}\langle111\rangle\]
\section*{Slip in single crystals}
For an applied shear stress $\sigma$, the resolved shear stress represents the magnitude of the shear stress acting on the slip plane in the slip direction. It is given by: 
\[\tau_R=\sigma \cos \phi \cos \lambda\]
\begin{figure}[H]
    \centering
    \includegraphics[width=0.4\textwidth]{resolved.png}
\end{figure}\noindent
A metal single crystal has a number of different slip systems, each with its own value of resolved shear stresses. However, there will be one slip system with the maximum resolved shear stress, denoted $\tau_{R}(\text{max})$.
\[\tau_{R}\text{(max)}=\sigma (\cos \phi \cos \lambda)_{max}\]
In this slip system with the maximum resolved shear stress, slip occurs when the resolved shear stress reaches a critical value (the critical resolved shear stress, $\tau_{CRSS}$). This is the minimum shear stress required to initiate slip and thus determines when yielding will begin (when $\tau_{R}\text{(max)}=\tau_{CRSS}$).
\[\sigma_y = \frac{\tau_{CRSS}}{(\cos\phi \cos\lambda)_{max}} \quad \text{(yield stress)}\]
This yield stress is minimized when $\phi=\lambda =45^\circ$, maximizing $\cos\phi \cos\lambda = 0.5$. Thus,
\[\sigma_{y,min} = 2\tau_{CRSS}\]
\newpage \noindent
\underline{Ex:} Consider a single crystal of BCC iron oriented such that a tensile stress is applied along a [010] direction. Compute the resolved shear stress along a (110) plane and in a [$\overline{1}$11] direction when a tensile stress of 52 MPa (7500 psi) is applied.
\begin{figure}[H]
    \centering
    \includegraphics[width=0.6\textwidth]{Ex slip.jpg}
\end{figure}
Recall that we can find the angle by using the projection:
\[\cos\theta=\frac{\vec{a}\cdot\vec{b}}{|\vec{a}||\vec{b}|}\]
Where:
\begin{itemize}
    \item Applied tensile direction $[010] \implies \vec{l}=(0,1,0)$
    \item Slip plane $(110) \implies \vec{n}=(1,1,0)$ (normal vector to plane)
    \item Slip direction $[\overline{1}11] \implies \vec{d}=(-1,1,1)$
\end{itemize}
Thus,
\[
\begin{aligned}
\phi &= \cos^{-1}\!\left(\frac{\vec{n}\cdot\vec{l}}{|\vec{n}||\vec{l}|}\right)\\
     &= \cos^{-1}\!\left(\frac{(1,1,0)\cdot(0,1,0)}{\sqrt{1^2+1^2+0^2}\sqrt{0^2+1^2+0^2}}\right)\\
     &= \cos^{-1}\!\left(\frac{1}{\sqrt{2}}\right)=45^\circ
\end{aligned}
\qquad
\begin{aligned}
\lambda &= \cos^{-1}\!\left(\frac{\vec{d}\cdot\vec{l}}{|\vec{d}||\vec{l}|}\right)\\
     &= \cos^{-1}\!\left(\frac{(-1,1,1)\cdot(0,1,0)}{\sqrt{(-1)^2+1^2+1^2}\sqrt{0^2+1^2+0^2}}\right)\\
     &= \cos^{-1}\!\left(\frac{1}{\sqrt{3}}\right)=54.74^\circ
\end{aligned}
\]
\[\tau_R=(52\text{MPa}) \cos (45^\circ) \cos (54.74^\circ)=21.3\text{MPa}\]
\newpage
\section*{Deformation by Twinning}
In addition to slip, plastic deformation in some metallic materials can occur by the
formation of mechanical twins, or twinning. Twinning occurs when a shear force produces atomic displacements such that on one side of a plane (the twin boundary), atoms are located in mirror-image positions of atoms on the other side.
\begin{figure}[H]
    \centering
    \includegraphics[width=1\textwidth]{twinning.png}
\end{figure}
Twinning is different from slip in several ways:
\begin{itemize}
    \item Twinning produces a reorientation of the crystal lattice, where slip crystallographic orientation is the same before and after deformation
    \item Slip occurs in distinct atomic spacing multiples, whereas the atomic displacement for twinning is less than the interatomic separation.
\end{itemize}
\begin{figure}[H]
    \centering
    \includegraphics[width=0.6\textwidth]{diff.png}
\end{figure}
Mechanical twinning occurs in metals with BCC and HCP crystal structures at low temps and high rates of loading where there are fewer slip systems. Twinning is important because it may place new slip systems in orientations that are favorable relative to the stress axis such that the slip process can now take place.
\newpage
\section*{Strengthening by grain size reduction}
The strength and hardness of metals depend on how easily dislocations can move. Plastic deformation occurs through dislocation motion, so if dislocations move freely, the metal is soft and weak. If their motion is restricted, more force is needed to deform the metal, making it harder and stronger. Thus, nearly all strengthening methods work by hindering dislocation movement.\\\\
Grain size affects a metal’s mechanical properties because grain boundaries hinder dislocation motion. When a dislocation moves from one grain to another, it encounters two obstacles:
\begin{itemize}
    \item The change in crystal orientation makes it harder for the dislocation to continue moving.
    \item The atomic disorder at the grain boundary disrupts slip-plane continuity.
\end{itemize}
As a result, smaller grains (more boundaries) strengthen the metal by restricting dislocation motion.
\begin{figure}[H]
    \centering
    \includegraphics[width=0.6\textwidth]{grain bound.png}
\end{figure}\noindent
The relationship between yield strength and grain size is given by the \textbf{Hall-Petch equation}:
\[\sigma_y=\sigma_o+k_y d^{-1/2}\]
Where:
\begin{itemize}
    \item $\sigma_y$ = yield stress
    \item $\sigma_o$ = friction stress (stress required to move dislocations, a material constant)
    \item $k_y$ = strengthening coefficient (a material constant)
    \item d = average grain diameter
\end{itemize}
\newpage
\section*{Solid-solution strengthening}
By creating alloys with impurity atoms in either a substitutional or interstitial solid solution, the strength of a metal can be increased. This process is called \textbf{solid-solution strengthening}. High-purity metals are almost always softer and weaker than alloys composed of the same base metal. This is because the impurity atoms impose lattice strains that hinder dislocation motion.
\begin{figure}[H]
    \centering
    \includegraphics[width=1\textwidth]{solid solution.png}
\end{figure}\noindent
\section*{Strain hardening}
\textbf{Strain hardening} is the phenomenon by which a ductile metal becomes harder and stronger as it is plastically deformed. Sometimes it is also called work hardening, or \textbf{cold working}.\\\\
We can define the amount of plastic deformation as \textbf{percent cold work} instead of strain:
\[\%CW=\frac{A_o - A_d}{A_o} \times 100\]
Where:
\begin{itemize}
    \item $A_o$ = original cross-sectional area
    \item $A_d$ = cross-sectional area after deformation
\end{itemize}
\section*{Recovery}
When a polycrystalline metal is plastically deformed at low temperatures, its grains change shape, dislocation density increases, and it undergoes strain hardening. Some deformation energy is stored as strain energy, and properties such as electrical conductivity and corrosion resistance may change.\\\\
\noindent
These effects can be reversed by \textbf{annealing}, a heat treatment involving \textbf{recovery} and \textbf{recrystallization} (sometimes followed by grain growth). During \textbf{recovery}, atomic diffusion allows dislocations to rearrange and reduce internal strain energy without external stress. This lowers dislocation density and restores properties such as electrical and thermal conductivity to their original states.
\section*{Recrystallization}
\textbf{Recrystallization} replaces deformed grains with new, strain-free, equiaxed ones that have low dislocation density. This process, driven by stored strain energy, restores the metal’s softness and ductility. It depends on time and temperature; the \textbf{recrystallization temperature} (temp for completion in 1~h) is typically $0.3$–$0.5\,T_m$ for metals and up to $0.7\,T_m$ for alloys ($T_m$ is the melting temperature). Greater cold work lowers this temperature, while impurities raise it by slowing grain boundary motion.
\section*{Grain Growth}
\textbf{Grain growth} occurs when strain-free grains enlarge at elevated temperatures, even without prior recovery or recrystallization. As grain boundaries migrate, large grains grow at the expense of smaller ones, reducing total boundary area and energy. The driving force is the decrease in grain boundary energy, and boundary motion occurs through short-range atomic diffusion.
\begin{figure}[H]
    \centering
    \includegraphics[width=0.4\textwidth]{grain growth.png}
\end{figure}\noindent
The grain diameter varies with time according to the relationship:
\[d^n-d^n_o=Kt\]
Where:
\begin{itemize}
    \item $d$ = grain diameter at time t
    \item $d_o$ = initial grain diameter
    \item n = grain growth exponent (typically 2)
    \item K = temperature-dependent rate constant
\end{itemize}
\newpage
\begin{center}
\section*{\LARGE\underline{Failure}}
\end{center}
\section*{Fracture}
\textit{Simple fracture} is when a body breaks into two or more pieces under static stress. For metals, fracture can be classified as either \textbf{brittle} or \textbf{ductile}.\\\\
\textbf{Brittle fracture} occurs with little to no plastic deformation and is characterized by a rapid crack propagation.\\\\
\textbf{Ductile fracture} occurs with extensive plastic deformation and is characterized by significant necking before fracture.
\begin{figure}[H]
    \centering
    \includegraphics[width=0.4\textwidth]{Fracture.png}
\end{figure}\noindent
Fractures involves two steps: crack formation and propagation. 
\section*{Ductile Fracture Mechanism}
\vspace{-2em}
\begin{figure}[H]
    \centering
    \includegraphics[width=1\textwidth]{Ductile.png}
\end{figure}\noindent
After necking begins, small cavities form in the center of the necked down region. The cavities grow and coalesce to form a single crack. The crack propagates toward the outer edge of the sample, resulting in fracture. Ductile materials show almost 100\% reduction in area at fracture. Ductile fractures usually resemble a cup-and-cone shape (crack propagates at 45$^\circ$ to the applied load).
\section*{Brittle Fracture}
In brittle fracture, cracks form in the center perpendicular to the direction of applied load. The crack then propagates rapidly. Brittle materials show very little reduction in area at fracture. The fracture surface is usually flat and perpendicular to the applied load. The fracture surface usually has V-shaped "chevron" markings that point in the direction of crack propagation.
\vspace{-2em}
\begin{figure}[H]
    \centering
    \includegraphics[width=0.6\textwidth]{Chevron.png}
\end{figure}\noindent
\textbf{Transgranular Fracture}\\
For most brittle materials, crack propagation corresponds to the breaking of atomic bonds along specific crystallographic planes, known as \textit{cleavage}. When the fracture surface passes through grains, it is called \textbf{transgranular fracture}.
\begin{figure}[H]
    \centering
    \includegraphics[width=0.8\textwidth]{Transgranular.png}
\end{figure}\newpage\noindent
\textbf{Intergranular Fracture}\\
In some cases, cracks propagate along grain boundaries, resulting in \textbf{intergranular fracture}.
\begin{figure}[H]
    \centering
    \includegraphics[width=0.8\textwidth]{Intergranular.png}    
\end{figure}\newpage
\section*{Principles of fracture mechanics}
\textbf{Stress concentration}\\
\vspace{-2em}
\begin{wrapfigure}{l}{0.5\textwidth}
    \centering
    \includegraphics[width=0.5\textwidth]{Stress concentrator.png}
    \vspace{-6em}
\end{wrapfigure}
The measured fracture strength of a material is often much lower than the theoretical strength calculated based on atomic bond strengths. This is due to the presence of microscopic flaws (cracks, voids, inclusions) that act as stress concentrators. The stress at the tip of a crack can be much higher than the applied stress, leading to premature fracture.\\\\
For internal cracks that are elliptical, the maximum stress at the crack tip is given by:
\[\boxed{\sigma_m=2\sigma_o\left(\frac{a}{\rho_t}\right)^{1/2}=K_t\sigma_o}\]
\[K_t = \frac{\sigma_m}{\sigma_o}\]
Where
\begin{itemize}
    \item $\sigma_m$ = maximum stress at crack tip
    \item $\sigma_o$ = applied stress
    \item a = length of a \textit{surface} crack, or half of the length of an \textit{internal} crack
    \item $\rho_t$ = radius of curvature at crack tip
    \item $K_t$ = stress concentration factor
\end{itemize}
\underline{Note:} Sharper cracks (smaller $\rho_t$) lead to higher stress concentrations. \\\\
\underline{Note:} Exterior fillets also act as stress concentrators (surface cracks).
\newpage\noindent\\
\textbf{Critical stress}\\
Cracks propagate when the stress at the crack tip reaches a critical value, $\sigma_c$, (ie. $\sigma_m = \sigma_c$ or $K_t>K_c$). The critical stress for crack propagation is given by:
\[\boxed{\sigma_c = \left(\frac{2E\gamma_s}{\pi a}\right)^{1/2}}\]
Where:
\begin{itemize}
    \item E = modulus of elasticity
    \item $\gamma_s$ = specific surface energy
    \item a = length of a \textit{surface} crack, or half of the length of an \textit{internal} crack
\end{itemize}
For ductile fracture, replace $\gamma_s$ with $\gamma_s + \gamma_p$, where $\gamma_p$ is the plastic deformation energy per unit area.\\\\
\underline{Note:} Larger cracks (larger a) lead to lower critical stresses for fracture.\\\\
\textbf{Fracture Toughness}\\
Fracture toughness, $K_C$, is a measure of a material's resistance to fracture in the presence of a crack. It represents the critical value of the stress intensity factor at which crack propagation becomes unstable. It is defined as:
\[\boxed{K_C = Y\sigma_c \sqrt{\pi a}}\]
Where:
\begin{itemize}
    \item $Y$ = geometric correction factor (dimensionless)
    \item $\sigma_c$ = critical applied stress for crack propagation
    \item $a$ = crack length (half-length for an internal crack)
\end{itemize}
$K_C$ has units of MPa$\cdot$m$^{1/2}$ or psi$\cdot$in$^{1/2}$. In general, $K_C$ may depend on specimen thickness and geometry. A higher $K_C$ value indicates greater resistance to crack propagation.
\\\\
\textbf{Plane Strain Fracture Toughness}\\
For specimens with thickness much greater than the crack dimensions and the crack-tip plastic zone, $K_C$ becomes independent of thickness. Under these conditions, plane strain exists (no strain perpendicular to the front and back faces). The fracture toughness under plane strain conditions is denoted $K_{IC}$ for Mode I crack opening.
\[\boxed{K_{IC} = Y\sigma \sqrt{\pi a}}\]
$\sigma$ is the applied stress. Under valid plane strain conditions, $K_{IC}$ is a true material property and represents the minimum fracture toughness of the material.\\\\
\underline{Note:} The fracture condition is met when $\boxed{K_C \geq K_{IC}}$.
\newpage
\begin{figure}[H]
    \centering
    \includegraphics[width=1\textwidth]{Mode I.png}
\end{figure}\noindent
\textbf{Design using fracture mechanics}\\
Typically when it comes to design, the fracture toughness $K_C$ or $K_{IC}$ is known for a material, and if we know either the applied stress $\sigma$ or the crack length $a$, we can solve for the other variable to ensure that fracture does not occur. Thus, rearranging the equation we see that:
\[\boxed{\sigma_c=\frac{K_{IC}}{Y\sqrt{\pi a}} \quad \text{(Max stress)}}\]
\[\boxed{a_c= \frac{1}{\pi}\left(\frac{K_{IC}}{Y\sigma}\right)^2 \quad \text{(Max crack length/ defect size)}}\]
\newpage
\section*{Fracture toughness testing}
There are two tests, the \textbf{Charpy} and \textbf{Izod}, that measure the \textbf{impact energy} (energy absorbed during fracture) of a material. Both tests involve breaking a notched specimen with a swinging pendulum hammer. The difference is in how the specimen is held:
\begin{figure}[H]
    \centering
    \includegraphics[width=0.7\textwidth]{Impact test.png}
\end{figure}\noindent
\textbf{Ductile-to-brittle transition}\\
Some materials exhibit a transition from ductile behavior at high temperatures to brittle behavior at low temperatures. This phenomenon is known as the ductile-to-brittle transition. The transition is characterized by a sharp drop in impact energy over a narrow temperature range.\\\\
The temperature at which the material changes from ductile to brittle fracture behavior is called the ductile-to-brittle transition temperature (DBTT). Above the DBTT, materials absorb a large amount of energy before fracturing (ductile). Below the DBTT, materials absorb very little energy (brittle). The Charpy V-notch impact test is commonly used to determine the ductile-to-brittle transition temperature (DBTT) of materials.
\vspace{-1em}
\begin{figure}[H]
    \centering
    \includegraphics[width=0.6\textwidth]{DTBT.png}
\end{figure}
\newpage
\section*{Cyclic stresses}
\textbf{Fatigue} is failure that occurs due to cyclic or fluctuating stresses, typically well below the static tensile or yield strength of the material. Most engineering failures occur by fatigue.\\\\
Fatigue failure proceeds in three stages:
\begin{itemize}
    \item Crack initiation (usually at surface defects, notches, scratches)
    \item Crack propagation (incremental crack advance per cycle)
    \item Final sudden fracture when remaining cross-section cannot support load
\end{itemize}
The applied stress may be axial (tension-compression), flexural (bending), or torsional (twisting) in nature. In general, three different fluctuating stress–time
modes are possible.\\\\
Reversed stress cycle, in which the stress alternates from a maximum tensile stress ($+$) to a maximum compressive stress ($-$) of equal magnitude:
\begin{figure}[H]
    \centering
    \includegraphics[width=0.7\textwidth]{Cyclic stress.png}
\end{figure}\noindent
Repeated stress cycle, in which maximum and minimum stresses are asymmetrical relative to the zero-stress level; mean stress $\sigma_m$, range of stress $\sigma_r$, and stress amplitude $\sigma_a$ are indicated:
\begin{figure}[H]
    \centering
    \includegraphics[width=0.7\textwidth]{Cyclic stress 2.png}
\end{figure}\newpage\noindent
Random stress cycle:
\begin{figure}[H]
    \centering
    \includegraphics[width=0.5\textwidth]{Cyclic stress 3.png}
\end{figure}
We can characterize cyclic stresses using the following parameters:
\begin{itemize}
    \item Mean stress for cyclic loading
    \[\boxed{\sigma_m = \frac{\sigma_{max} + \sigma_{min}}{2}}\]
    \item Range of stress of cyclic loading
    \[\boxed{\sigma_r = \sigma_{max} - \sigma_{min}}\]
    \item Stress amplitude of cyclic loading
    \[\boxed{\sigma_a =\frac{\sigma_r}{2}= \frac{\sigma_{max} - \sigma_{min}}{2}}\]
    \item Stress ratio of cyclic loading
    \[\boxed{R = \frac{\sigma_{min}}{\sigma_{max}}}\]
\end{itemize}
\section*{The S-N curve}
Fatigue behavior is commonly represented using an S–N curve, which plots stress amplitude vs. number of cycles to failure.
\begin{figure}[H]
    \centering
    \includegraphics[width=1\textwidth]{SN curve.png}
\end{figure}\newpage\noindent
For some metals (primarily steels), the S–N curve levels off at a plateau called the: \textbf{fatigue limit}. Below this limit, the material can theoretically endure infinite cycles without failure. For other metals (such as aluminum and copper), there is no distinct fatigue limit; instead, the S–N curve continues to slope downward, indicating that failure will eventually occur at any stress level given enough cycles. For these metals, \textbf{fatigue strength} is defined as the stress amplitude at a specified number of cycles (\textbf{fatigue life}).\\\\
\textbf{Fatigue Lifetime (Basquin's Law)}\\
To determine the number of cycles to failure for a given stress amplitude, or to determine the stress amplitude for a desired fatigue life, we use the mathematical form of the S--N curve.\\
For many metals in the high-cycle fatigue region, the S--N curve follows a power-law relationship known as \textbf{Basquin's Law}:
\[\boxed{\sigma_a = A N^b}\]
Where:
\begin{itemize}
    \item $\sigma_a$ = stress amplitude
    \item $N$ = number of cycles to failure
    \item $A$ = material constant (intercept)
    \item $b$ = fatigue exponent (slope), typically negative
\end{itemize}

Taking the logarithm of both sides gives a linear relationship:
\[\boxed{\log \sigma_a = \log A + b \log N}\]
Thus, when plotted on a log--log graph, the S--N curve becomes a straight line.\\\\
\textbf{Determining $A$ and $b$ From Data}\\
Given two points on the S--N curve, $(\sigma_{a1}, N_1)$ and $(\sigma_{a2}, N_2)$, we can compute the constants:
\[\boxed{b = \frac{\log \sigma_{a2} - \log \sigma_{a1}}{\log N_2 - \log N_1}}\]
\[\boxed{A = \sigma_{a1} N_1^{-b}}\]
Once $A$ and $b$ are known, we may solve for:
\[\boxed{N = \left(\frac{\sigma_a}{A}\right)^{1/b}}\qquad\text{or}\qquad\boxed{\sigma_a = A N^b}\]
\section*{Crack initiation and propagation}
Cracks associated with fatigue failure almost always initiate (or nucleate) on the surface of a component at stress concentrators such as: surface scratches, sharp fillets, keyways, threads, dents, etc. Cyclic loading can also produce microscopic surface cracks from dislocation slip steps. Once a fatigue crack forms, it grows incrementally every cycle.\\\\
Fracture surfaces from fatigue failures often exhibit features such as
\begin{itemize}
    \item \textbf{Benchmarks} (macroscopic): pattern formed by intermittent loading
    \item \textbf{Striations} (microscopic): each marks one cycle of crack growth
\end{itemize}
\begin{center}
\section*{\LARGE\underline{Phase Diagrams}}
\end{center}
\section*{Definitions and basic concepts}
\begin{itemize}
    \item \textbf{Component}: A chemically distinct species in a system (ex: solute or solvent).
    \item \textbf{System}: A specific body of material under consideration or it can relate to the series of possible alloys that is composed of the same components but varying in composition (ex: iron-carbon system).
    \item \textbf{Solubility limit}: The maximum concentration of solute that can dissolve in a solvent at a given temperature to form a solid solution.
\end{itemize}
Before the solubility limit, there is only a single phase solution. Beyond the solubility limit, a second phase will form (usually solid). The two phases will coexist in equilibrium. The compositions of the two phases are given by the phase diagram at that temperature.
\begin{figure}[H]
    \centering
    \includegraphics[width=0.7\textwidth]{Solubility.png}
\end{figure}
\section*{Phases}
A phase is a physically and chemically homogeneous portion of a system that is separated from other portions by distinct boundaries. Note that differences in crystal structure also define different phases.\\\\
\textbf{Types of phase systems}:
\begin{itemize}
    \item \textbf{Single phase system}: usually termed homogeneous (ex: pure metals, solid solutions)
    \item \textbf{Two phase system}: usually termed heterogeneous systems or mixtures (ex: metallic alloys)
\end{itemize}
\section*{Microstructure}
Microstructure refers to the spatial arrangement of phases within a material. Microstructures directly influences material properties and can consist of single-phase grains, two-phase mixtures, layered structures, etc.
\section*{Phase Equilibria}
\textbf{Free energy} is a function of the internal energy of a system and the entropy of the system (randomness of the atoms). A system is at \textbf{equilibrium} when its free energy is minimized under a given set of conditions (temperature, pressure, composition).\\\\
\textbf{Phase equilibrium} in a system with multiple phases occurs when the free energy of each phase is equal. At equilibrium, there is no net change in the amount or composition of each phase over time.\\\\
\textbf{Metastable} states are non-equilibrium states that can persist for long periods. Such as in the case of in solid systems, where a metastable state or microstructure may persist indefinitely.
\section*{One-component (or unary) phase diagrams}
Phase diagrams or equilibrium diagrams graphically represent the phases present in a system at equilibrium as a function of variables such as temperature, pressure, and composition.\\\\
For a one component phase diagram (ex: water, carbon), the diagram typically plots temperature vs. pressure. The diagram shows the regions of stability for different phases (solid, liquid, gas) and the lines or curves that separate these regions represent the conditions under which two phases coexist in equilibrium (phase boundaries).
\begin{figure}[H]
    \centering
    \includegraphics[width=1\textwidth]{Single component.png}
\end{figure}\noindent
At the triple point $O$, all three phases exist in equilibrium. The critical point $C$ represents the temperature and pressure above which the liquid and gas phases become indistinguishable (supercritical fluid).
\section*{Binary isomorphous systems}
A binary phase diagram shows the equilibrium relationships between phases in a two-component system as a function of temperature and composition, keeping pressure constant (usually at 1 atm).\\\\
Composition is typically expressed in weight percent (wt\%) or atomic percent (at\%). In a binary phase diagram, the x-axis represents the composition of one component (from 0 to 100 wt\% or at\%), while the y-axis represents temperature.\\\\
Many microstructures develop during phase transformations (typically during cooling).\\\\
The liquid region is a liquid solution of both copper and nickel. The $\alpha$ phase is a substitutional solid of both Cu and Ni atoms in a FCC crystal structure. This system is called an \textbf{isomorphous system} because the two components are completely soluble in each other in both liquid and solid states (and have the same structure).
\begin{figure}[H]
    \centering
    \includegraphics[width=0.6\textwidth]{Binary phase.png}
\end{figure}\noindent
For metallic alloys, solid solutions are typically given in lowercase Greek letters (e.g., $\alpha$, $\beta$) while liquids are given in uppercase Roman letters (e.g., L). The line separating the liquid phase from the liquid + solid phase is called the \textbf{liquidus line}, while the line separating the solid phase from the liquid + solid phase is called the \textbf{solidus line}. 
\section*{Interpretation of phase diagrams}
For a binary system of known composition and temperature at equilibrium, we can determine:\\\\
\textbf{Phases present}\\
Locate the point on the phase diagram corresponding to the given composition and temperature. The region in which this point lies indicates the phases present.
\newpage\noindent
\textbf{Determination of phase compositions}\\
If the point lies in a one phase region, the composition of that phase is equal to the overall composition. If the point lies in a two-phase region, draw a horizontal \textbf{tie line} at the given temperature. The intersections of this tie line with the phase boundaries give the compositions of each phase.
\begin{figure}[H]
    \centering
    \includegraphics[width=0.7\textwidth]{Tieline.png}
\end{figure}\noindent
Where $C_L$ is the composition of the liquid phase and $C_\alpha$ is the composition of the solid phase, they make up the alloy composition $C_0$ at temperature T.\\\\
\underline{Note:} The closer $C_0$ is to $C_L$, the more liquid phase there is. The closer $C_0$ is to $C_\alpha$, the more solid phase there is.\\\\
\textbf{Determination of phase amounts}\\
In a two-phase region, the relative amounts of each phase can be determined using the \textbf{lever rule}. It calculates the mass fraction of each phase based on the overall composition and the compositions of each phase at equilibrium.
\begin{figure}[H]
    \centering
    \includegraphics[width=0.75\textwidth]{Lever.png}
\end{figure}\noindent
The lever rule essentially just uses the distances along the tie line between $C_0$, $C_L$, and $C_\alpha$ to find the mass fractions:
\[\boxed{W_L = \frac{C_\alpha - C_0}{C_\alpha - C_L}\quad \text{(liquid mass fraction)}}\]
\[\boxed{W_\alpha = \frac{C_0 - C_L}{C_\alpha - C_L}\quad \text{($\alpha$-phase mass fraction)}}\]
For an alloy consisting of $\alpha$ and $\beta$ phases, the volume fraction of the $\alpha$ phase, $V_{\alpha}$, is defined as
\[V_{\alpha} = \frac{v_{\alpha}}{v_{\alpha} + v_{\beta}}\]
Where $v_{\alpha}$ and $v_{\beta}$ are the volumes of the respective phases in the alloy. An analogous expression exists for $V_{\beta}$, and, for an alloy consisting of two phases, $V_{\alpha} + V_{\beta} = 1$.\\\\
If we want to convert between mass fractions and volume fractions:\\\\
\textbf{Conversion of mass fractions of $\alpha$ and $\beta$ phases to volume fractions}
\[V_{\alpha} = \frac{\frac{w_{\alpha}}{\rho_{\alpha}}}{\frac{w_{\alpha}}{\rho_{\alpha}} + \frac{w_{\beta}}{\rho_{\beta}}}\]
\[V_{\beta} = \frac{\frac{w_{\beta}}{\rho_{\beta}}}{\frac{w_{\alpha}}{\rho_{\alpha}} + \frac{w_{\beta}}{\rho_{\beta}}}\]
\textbf{Conversion of volume fractions of $\alpha$ and $\beta$ phases to mass fractions}
\[W_{\alpha} = \frac{V_{\alpha}\rho_{\alpha}}{V_{\alpha}\rho_{\alpha} + V_{\beta}\rho_{\beta}}\]
\[W_{\beta} = \frac{V_{\beta}\rho_{\beta}}{V_{\alpha}\rho_{\alpha} + V_{\beta}\rho_{\beta}}\]
\section*{Development of microstructure in isomorphous alloys}
\textbf{Equilibrium cooling}\\
In the case where cooling occurs very slowly, the alloy remains in equilibrium at all times. The microstructure that develops during equilibrium cooling consists of grains of the solid solution phase ($\alpha$) that form as the liquid phase solidifies. The composition of the solid solution phase changes as cooling progresses, following the solidus line on the phase diagram, however the overall composition of the alloy remains constant.
\begin{figure}[H]
    \centering
    \includegraphics[width=0.4\textwidth]{Equilibrium cooling.png}
\end{figure}
\vspace{-1em}
\noindent
\textbf{Nonequilibrium cooling}\\
In the case of equilibrium cooling, the composition of the solid solution phase changes continuously as cooling progresses. However, if cooling occurs rapidly, there may not be enough time for diffusion to occur and for the composition of the solid solution phase to adjust to the equilibrium values given by the phase diagram. This results in a microstructure that is not in equilibrium, with regions of different compositions within the solid solution phase.
\vspace{-1em}
\begin{figure}[H]
    \centering
    \includegraphics[width=0.58\textwidth]{Nonequlibrium.png}
\end{figure}
\begin{itemize}
    \item \textbf{Formation of Coring:} During nonequilibrium cooling, diffusion is too slow to homogenize the grain. The initial $\alpha$ phase retains its original composition, while new layers form with changing compositions, creating a concentration gradient.
    \item \textbf{Shifted Solidus Line:} The standard equilibrium solidus line defines the \textit{outer edge} composition of the grain. A new nonequilibrium solidus line represents the \textit{average} composition of the entire $\alpha$ grain.
\end{itemize}
\section*{Mechanical properties of isomorphous alloys}
When in a solid phase, isomorphous alloys experience a solid-solution strengthening. On average, when tensile strength increases with increasing solute concentration, ductility decreases.
\section*{Binary eutectic systems}
In the binary eutectic phase diagram for the copper-silver system, there are 3 single-phase regions (liquid, $\alpha$, and $\beta$). The $\alpha$ phase is a solid solution rich in copper with nickel solute. The $\beta$ phase is a solid solution rich in nickel with copper solute. Pure copper and nickel are also considered to be $\alpha$ and $\beta$ phases, respectively.
\begin{figure}[H]
    \centering
    \includegraphics[width=0.8\textwidth]{Binary eutectic.png}
\end{figure}
\begin{itemize}
    \item \textbf{Solvus Line}: The boundary line separating a solid solution phase (like $\alpha$) from a mixed solid region (like $\alpha + \beta$), representing the solid solubility limit.
    \item \textbf{Solidus Line}: The boundary separating a solid phase from a liquid-plus-solid region (e.g., $\alpha + L$); the horizontal isotherm ($BEG$) is also considered a solidus line representing the lowest temperature a liquid can exist in equilibrium.
    \item \textbf{Liquidus Line}: The temperature boundary above which the alloy is entirely liquid; this line decreases in temperature as solute is added to the solvent.
    \item \textbf{Invariant Point ($E$)}: The specific point where the two liquidus lines meet, defined by the eutectic composition ($C_E$) and eutectic temperature ($T_E$).
\end{itemize}
Note that like before, the tie line can be used to determine phase compositions in two-phase regions, and the lever rule can be used to determine phase amounts.\\\\
\textbf{Eutectic reaction}\\
A reaction in which, upon cooling, a liquid phase transforms isothermally (at constant temperature) and reversibly into two intimately mixed solid phases. This occurs at the eutectic composition and temperature, where the liquid phase solidifies into a mixture of $\alpha$ and $\beta$ phases. The term "eutectic" means "easily melted".\\\\
\underline{General Equation}:
    $$\boxed{L(C_E) \underset{\text{heating}}{\overset{\text{cooling}}{\rightleftharpoons}} \alpha(C_{\alpha E}) + \beta(C_{\beta E})} $$
This reaction is termed an \textbf{invariant reaction} because it occurs at a specific temperature and composition where three phases coexist in equilibrium.\\\\
This reaction tells us the concentrations of each phase that is formed. These concentrations can be used in the lever rule to determine the amounts of each phase present after solidification.
\newpage
\section*{Development of microstructure in eutectic alloys}
\textbf{Case I}\\
Single-phase solid solution alloy forms when the solute concentration is less than the solubility limit of the solvent (to the left of the solvus line). The microstructure consists of grains of the $\alpha$ phase with a uniform composition.
\begin{figure}[H]
    \centering
    \includegraphics[width=0.5\textwidth]{Case 1.png}
\end{figure}\newpage\noindent
\textbf{Case II}\\
When the alloy composition is between the room-temperature solubility limit and the maximum solubility limit, a microstructure consisting of $\alpha$ polycrystal with fine $\beta$-phase inclusions forms.
\begin{figure}[H]
    \centering
    \includegraphics[width=0.5\textwidth]{Case 2.png}
\end{figure}\noindent
\textbf{Case III}\\
When the alloy composition is at the eutectic composition, the microstructure consists of alternating layers of $\alpha$ and $\beta$ phases that form simultaneously from the liquid during the eutectic reaction.
\begin{figure}[H]
    \centering
    \includegraphics[width=0.5\textwidth]{Case 3.png}
\end{figure}\noindent
This microstructure is called a \textbf{eutectic structure} and has a lamellar (layered) appearance. The spacing between the layers depends on the cooling rate; faster cooling results in finer lamellae.\\\\
\textbf{Case IV}\\
When the alloy composition is greater than the maximum solubility limit of the solute (to the right of the solvus line), and crosses the eutectic temperature during cooling, a microstructure consisting of primary $\alpha$ phase grains surrounded by a eutectic mixture of $\alpha$ and $\beta$ phases forms.
\vspace{-1em}
\begin{figure}[H]
    \centering
    \includegraphics[width=0.7\textwidth]{Case 4.png}
\end{figure}
\vspace{-1em}
\begin{figure}[H]
    \centering
    \includegraphics[width=0.6\textwidth]{HyperHypo.png}
\end{figure}\noindent
\textbf{Microconstituent}\\
A \textbf{microconstituent} is a distinct phase or combination of phases that can be identified in the microstructure of an alloy. Examples include single-phase regions (like $\alpha$ or $\beta$) and two-phase mixtures (like eutectic structures). We can use the lever rule to determine the amounts of each microconstituent present in a given alloy composition at a specific temperature.
\vspace{-1em}
\begin{figure}[H]
    \centering
    \includegraphics[width=0.7\textwidth]{Microconstituent.png}
\end{figure}
\vspace{-1em}
\noindent
Since the eutectic microstructures come from the liquid phase, we can use the lever rule to determine the amounts of the eutectic microconstituent by:
\[W_e = W_L = \frac{P}{P+Q}=\frac{C\alpha -C_0}{C_\alpha - C_L}\]
\underline{Note:} For the lever rule, we always use the lever arm on opposite side of the fulcrum (P for $W_L$ and Q for $W_\alpha$).\\\\
The primary grains of the $\alpha$ phase that form before the eutectic reaction can also be determined using the lever rule:
\[W_{\alpha'}=\frac{Q}{P+Q}=\frac{C_0-C_L}{C_\alpha - C_L}\]
We can also use the full tie line below the eutectic temperature to determine the TOTAL amounts of each phase (not just the eutectic microconstituent):
\[W_{\alpha}=\frac{Q+R}{P+Q+R}\]
\[W_{\beta}=\frac{P}{P+Q+R}\]
Remember that we use the opposite lever arms for each phase.\\\\
\underline{Note:} We can find the eutectic $\alpha$ phase fraction by doing $W_{\alpha e}=W_\alpha - W_{\alpha'}$.
\newpage
\section*{Equilibrium diagrams having intermediate phases or compounds}
The phase diagrams for many binary alloy systems are much more complex. They may contain \textbf{terminal solid solutions}, which are solid solutions that exist at the concentration extremes of the phase diagram. They may also contain \textbf{intermediate solid soltuions} or phases, which are found between the terminal solid solutions.
\begin{figure}[H]
    \centering
    \includegraphics[width=0.6\textwidth]{Intermediate.png}
\end{figure}\noindent
The composition of an intermediate phase can still be found using a tie line, and the amounts of each phase can be found using the lever rule.\\\\
\textbf{Intermetallic compounds}\\
Unlike solid solutions, which exist over a range of compositions, some systems form \textit{discrete intermediate compounds} with distinct chemical formulas. These compounds (called intermetallic compounds) form at specific, stoichiometric ratios. (Ex: The compound $\text{Mg}_2\text{Pb}$ exists specifically at a composition of $19\text{ wt\% Mg}$ and $81\text{ wt\% Pb}$ ($33\text{ at\% Pb}$)). Because the compound has no compositional range (finite width), it is represented as a vertical line on the phase diagram.\\\\
Thus, we can treat it as two phase diagrams joined at the intermetallic compound line. We can still use tie lines and the lever rule to determine phase compositions and amounts.\newpage
\begin{figure}[H]
    \centering
    \includegraphics[width=0.6\textwidth]{Intermetallic.png}
\end{figure}
\vspace{-1em}
\section*{Eutectoid and peritectic reactions}
In addition to the eutectic, other invariant reactions involving three different phases are found for some alloy systems. Upon cooling, a solid $\sigma$-phase cools into $\gamma$ and $\epsilon$ solid phases according to the \textbf{eutectoid reaction}
\[\boxed{\delta \underset{\text{heating}}{\overset{\text{cooling}}{\rightleftharpoons}} \gamma + \epsilon}\]
Eutectoid means eutectic-like. The difference between a \textit{eutectoid} from \textit{eutectic} is that one solid phase instead of a liquid transforms into two other solid phases at a single temperature.\\\\
The \textbf{peritectic reaction} is where, upon heating, one solid phase will convert into a liquid phase and another solid phase. 
\[\boxed{\delta + L \underset{\text{heating}}{\overset{\text{cooling}}{\rightleftharpoons}} \epsilon}\]
\vspace{-1em}
\begin{figure}[H]
    \centering
    \includegraphics[width=0.65\textwidth]{Invariant stuff.png}
\end{figure}\noindent
$E$ is the eutectoid and $P$ is the peritectic points respectively.
\newpage\noindent
In summary,
\begin{figure}[H]
    \centering
    \includegraphics[width=0.8\textwidth]{Summary Invariant.png}
\end{figure}
\section*{Congruent phase transformations}
Phase transformations in which there are no changes in composition are called \textbf{congruent phase transformations}. In these transformations, the composition of the phase remains constant throughout the transformation. For example, the transformation of a liquid phase into a solid phase at a specific temperature and composition without any change in composition is a congruent transformation.\\\\
There is a congruent transformation when the $\gamma$ solid phase transforms into a liquid phase at the point below:
\begin{figure}[H]
    \centering
    \includegraphics[width=0.5\textwidth]{Congruent.png}
\end{figure}
\newpage
\section*{The Gibbs phase rule}
The \textbf{Gibbs phase rule} governs phase equilibria in materials systems. It is given by:
\[P + F = C + N\]
Where:
\begin{itemize}
    \item \(P\) = number of phases
    \item \(F\) = degrees of freedom (independent variables)
    \item \(C\) = number of components
    \item \(N\) = number of non-compositional variables (e.g., temperature, pressure)
\end{itemize}
For binary systems at constant pressure (\(N=1\)):
\[F = 3 - P\]
\begin{itemize}
    \item For a single-phase region: \(F = 2\) (both temperature and composition can vary independently).
    \item For two-phase coexistence: \(F = 1\) (only temperature or phase composition can vary independently).
    \item For three-phase equilibrium: \(F = 0\) (all conditions are fixed, e.g., eutectic isotherm).
\end{itemize}
The rule helps analyze equilibrium and identify non-equilibrium conditions in phase diagrams.
\newpage
\section*{The iron-iron carbide phase diagram}
\begin{figure}[H]
    \centering
    \includegraphics[width=0.7\textwidth]{IronIron.png}
\end{figure}
Pure iron, upon heating, experiences two changes in crystal structures before it melts. At room temperature, iron has a stable BCC structure, called \textbf{ferrite} or $\alpha$-iron. At 912$^\circ$C, it transforms to a FCC structure called \textbf{austenite} or $\gamma$-iron. At 1394$^\circ$C, it transforms back to BCC structure called \textbf{$\delta$-ferrite} or $\delta$-iron. The melting point of pure iron is 1538$^\circ$C. We can track these changes along the y-axis of the iron-iron carbide phase diagram.\\\\
The phase diagram only goes up to 6.7 wt\% C because that is the composition of the compound \textbf{cementite} or $\text{Fe}_3\text{C}$. Cementite is an intermediate compound and is represented as a vertical line on the diagram. 
\begin{itemize}
    \item \textbf{Austenite (\(\gamma\)-Fe)} is stable only above 727°C.
    \item \textbf{Cementite (Fe$_3$C)} is metastable at room temperature.
    \item Over long periods at 650--700°C, cementite slowly decomposes into \(\alpha\)-iron and graphite.
\end{itemize}
One eutectic exists for the iron-iron carbide system, at 4.30 wt\% C and 1147°C (2097°F):
\[L\underset{\text{heating}}{\overset{\text{cooling}}{\rightleftharpoons}} \gamma + \text{Fe}_3\text{C}\]
A eutectoid reaction also exists at 0.76 wt\% C and 727°C (1341°F):
\[\gamma (0.76 \text{wt\% C}) \underset{\text{heating}}{\overset{\text{cooling}}{\rightleftharpoons}} \alpha(0.022 \text{wt\% C}) + \text{Fe}_3\text{C}(6.7 \text{wt\% C})\]
\underline{Note:} Ferrous alloys with less than 2.11 wt\% C are called \textbf{steels}, while those with more than 2.11 wt\% C are called \textbf{cast irons}.
\newpage
\section*{Development of microstructure in iron-carbon alloys}















































\end{document}