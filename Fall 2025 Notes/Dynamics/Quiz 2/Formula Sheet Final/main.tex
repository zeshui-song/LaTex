\documentclass[10pt, fleqn]{article}
\usepackage[a4paper,margin=0.5in]{geometry}
\usepackage{multicol}
\usepackage{amsmath, amssymb}
\usepackage{titlesec}
\usepackage{float}
\usepackage{amsfonts, bm, graphicx}
\usepackage[font=small, skip=4pt]{caption}

\setlength{\abovedisplayskip}{6pt}
\setlength{\belowdisplayskip}{6pt}
\setlength{\abovedisplayshortskip}{4pt}
\setlength{\belowdisplayshortskip}{4pt}

\setlength{\textfloatsep}{8pt plus 2pt minus 2pt}
\setlength{\intextsep}{6pt plus 2pt minus 2pt}
\setlength{\floatsep}{8pt plus 2pt minus 2pt}

\setlength{\mathindent}{0pt}

\titleformat{\section}{\large\bfseries}{}{0em}{}
\titleformat{\subsection}{\normalsize\bfseries}{}{0em}{}

\begin{document}

\begin{center}
    \Large \textbf{Dynamics Formula Sheet} \\
    \normalsize Quiz II
\end{center}

\begin{multicols}{2}

\section*{Impulse and Momentum}
\[\vec{p}=m\vec{v}\qquad \Delta\vec{p}=\mathbf{Imp}_{1\to2}=\int_{t_1}^{t_2} \mathbf{F} \, dt\]
Momentum changes by:
\[m\mathbf{v_1}+\mathbf{Imp}_{1\to2}=m\mathbf{v_2}\]
\begin{figure}[H]
    \centering
    \includegraphics[width=0.4\textwidth]{Imp2.png}
\end{figure}
\vspace{-2em}
\section*{Collisions}
\[e=\frac{v_B'-v_A'}{v_A-v_B}\]
\begin{itemize}
    \item \textbf{Perfectly elastic impact:} $e=1$\\
Momentum \textit{and} energy is conserved.
    \item \textbf{Perfectly plastic impact:} $e=0$\\
Particles stick together after impact. Only momentum is conserved.
\end{itemize}
\textbf{Oblique Central Impact}
\begin{figure}[H]
    \centering
    \includegraphics[width=0.4\textwidth]{Oblique.png}
\end{figure}\noindent
No impulsive force in $t$: \[v_{A,t}'=v_{A,t},\quad v_{B,t}'=v_{B,t}\]
Conservation of momentum in $n$: \[m_Av_{A,n}+m_Bv_{B,n} = m_Av_{A,n}'+m_Bv_{B,n}'\]
Relate velocities using $e$: \[v_{B,n}'-v_{A,n}'=e(v_{A,n}-v_{B,n})\]
\section*{Rigid Body Rotation}
\textbf{Kinematics}\\
\[\mathbf{v}=\mathbf{\omega} \times \mathbf{r},\quad \mathbf{a}=\mathbf{\alpha} \times \mathbf{r} - \omega^2 \mathbf{r}\]
\[\omega=\frac{d\theta}{dt},\quad \alpha=\frac{d\omega}{dt}=\frac{d^2\theta}{dt^2},\quad \alpha=\frac{d\theta}{dt}\frac{d\omega}{d\theta}=\omega \frac{d\omega }{d\theta}\]
\newcolumn\\
\textbf{Constant Angular Acceleration}
\[\theta=\theta_0+\omega_0 t + \frac{1}{2}\alpha t^2\]
\[\omega=\omega_0+\alpha t\]
\[\omega^2=\omega_0^2 + 2\alpha(\theta - \theta_0)\]
\section*{General Planar Motion}
\[\mathbf{v_B}=\mathbf{v_A}+\mathbf{v_{B/A}}\]
\[\boxed{\mathbf{v}_B=\mathbf{v}_A+\mathbf{\omega}\hat{k} \times \mathbf{r}_{B/A}}\]
\[\boxed{\mathbf{a}_B=\mathbf{a}_A+\mathbf{\alpha}\hat{k} \times \mathbf{r}_{B/A} - \omega^2 \mathbf{r}_{B/A}}\]
\underline{Note:} $\mathbf{r}_{B/A}$ is the position vector pointing from A to B 
\section*{Rotating Reference Frames}
Frame xyz is rotating relative to frame XYZ (the absolute frame) with angular velocity $\mathbf{\omega}$.
\begin{figure}[H]
    \centering
    \includegraphics[width=0.4\textwidth]{Rotating Ref.png}
\end{figure}
\begin{figure}[H]
    \centering
    \includegraphics[width=0.4\textwidth]{Rot Vel.png}
\end{figure}
\begin{figure}[H]
    \centering
    \includegraphics[width=0.55\textwidth]{Rot acc.png}
\end{figure}\noindent
\underline{Note:} We have to use rotating frames if we are given quantities relative to a rotating body.
\end{multicols}
\newpage
\begin{multicols}{2}
\section*{Center of Mass}
Continuous Body:
\[\bar{x}=\frac{\int x \, dW}{W},\quad \bar{y}=\frac{\int y \, dW}{W}\]
Discrete Masses:
\[x_c=\frac{\sum x_{ci}A_i}{\sum A_i}, \qquad \frac{\sum x_{ci}W_i}{\sum W_i}\]
\[y_c=\frac{\sum y_{ci}A_i}{\sum A_i},\qquad \frac{\sum y_{ci}W_i}{\sum W_i}\]
\section{Moment of Inertia}
Area Moments:
\[I_x=\int y^2 \, dA,\quad I_y=\int x^2 \, dA\]
Polar Moment:
\[J_O=\int r^2 \, dA = I_x + I_y\]
Mass Moments:
\[I_x=\int (y^2+z^2) \, dm,\  I_y=\int (x^2+z^2) \, dm,\  I_z=\int (x^2+y^2) \, dm\]
Radius of Gyration:
\[I_x = k_x^2 A, \quad I_y = k_y^2 A,\quad J_O = k_o^2 A\]
\section*{Rigid Body Kinetics}
Sum of forces and moments about the center of mass:
\[\sum \mathbf{F}=m\mathbf{\bar{a}},\quad \sum \mathbf{M_{G}}=\mathbf{\dot{H}_G}=\bar{I}\mathbf{\alpha}\]
Summing moments about a point other than $\mathbf{G}$:
\[\sum M_P=\bar{I}\alpha+m\bar{a}d\quad\]
\underline{Note:} Parallel axis theorem only applies for fixed rotation. ($I=\bar{I}+md^2$)
\section*{Rolling Motion}
There are 3 cases of rolling motion:
\begin{enumerate}
    \item Rolling without slipping:
    \[F_f\leq \mu_sN\qquad \bar{a}=r\alpha\]
    \item Rolling, slipping impeding
    \[F_f=\mu_sN\qquad \bar{a}= r\alpha\]
    \item Rolling and slipping:
    \[F_f=\mu_kN\qquad\bar{a}\text{ and }\alpha\text{ becomes decoupled and independent}\]
\end{enumerate}
\underline{Note:} If unsure about which case, assume rolling without slipping first and calculate $F_f$. If $F_f>\mu_sN$, then it is slipping, and we need to adjust our equation for $F_f$.\\
\columnbreak
\section*{Normal and Tangential}
\[\vec{v}=v\hat{u}_t\]
\[\vec{a}=\frac{dv}{dt}\hat{u}_t+\frac{v^2}{\rho}\hat{u}_n\]
\section*{Cylindrical (Radial and Transverse)}
\[\vec{v}=\dot{r}\hat{u}_r+r\dot{\theta}\hat{u}_{\theta}\]
\[\vec{a}=\left[\ddot{r}-r\dot{\theta}^2\right]\hat{u}_r+\left[r\ddot{\theta}+2\dot{r}\dot{\theta}\right]\hat{u}_{\theta}\]
\textbf{Asymmetric Rolling}
\begin{multicols}{2}
    \begin{figure}[H]
    \centering
    \includegraphics[width=0.2\textwidth]{Asym.png}
\end{figure}
\vspace{-2em}
\[\bm{\bar{a}} = \bm{a_G} = \bm{a_O} + \bm{a_{G/O}}\]
\[\bm{a_G}= \bm{a_O} + (\bm{a_{G/O}})_t + (\bm{a_{G/O}})_n\]
\[\bm{a_G}= \bm{a_O} + \bm{\alpha} \times \bm{r_{G/O}} - \omega^2 \bm{r_{G/O}}\]
\end{multicols}
\section*{Instantaneous Center}
\begin{figure}[H]
    \centering
    \includegraphics[width=0.4\textwidth]{IC1.png}
\end{figure}
\begin{figure}[H]
    \centering
    \includegraphics[width=0.3\textwidth]{IC2.png}
\end{figure}
\begin{figure}[H]
    \centering
    \includegraphics[width=0.4\textwidth]{IC3.png}
\end{figure}
\end{multicols}
\end{document}
